% Options for packages loaded elsewhere
\PassOptionsToPackage{unicode}{hyperref}
\PassOptionsToPackage{hyphens}{url}
%
\documentclass[
]{book}
\usepackage{amsmath,amssymb}
\usepackage{lmodern}
\usepackage{ifxetex,ifluatex}
\ifnum 0\ifxetex 1\fi\ifluatex 1\fi=0 % if pdftex
  \usepackage[T1]{fontenc}
  \usepackage[utf8]{inputenc}
  \usepackage{textcomp} % provide euro and other symbols
\else % if luatex or xetex
  \usepackage{unicode-math}
  \defaultfontfeatures{Scale=MatchLowercase}
  \defaultfontfeatures[\rmfamily]{Ligatures=TeX,Scale=1}
\fi
% Use upquote if available, for straight quotes in verbatim environments
\IfFileExists{upquote.sty}{\usepackage{upquote}}{}
\IfFileExists{microtype.sty}{% use microtype if available
  \usepackage[]{microtype}
  \UseMicrotypeSet[protrusion]{basicmath} % disable protrusion for tt fonts
}{}
\makeatletter
\@ifundefined{KOMAClassName}{% if non-KOMA class
  \IfFileExists{parskip.sty}{%
    \usepackage{parskip}
  }{% else
    \setlength{\parindent}{0pt}
    \setlength{\parskip}{6pt plus 2pt minus 1pt}}
}{% if KOMA class
  \KOMAoptions{parskip=half}}
\makeatother
\usepackage{xcolor}
\IfFileExists{xurl.sty}{\usepackage{xurl}}{} % add URL line breaks if available
\IfFileExists{bookmark.sty}{\usepackage{bookmark}}{\usepackage{hyperref}}
\hypersetup{
  pdftitle={Filosofía económica: Entendiendo la complejidad},
  pdfauthor={Jeshua Romero Guadarrama},
  hidelinks,
  pdfcreator={LaTeX via pandoc}}
\urlstyle{same} % disable monospaced font for URLs
\usepackage{longtable,booktabs,array}
\usepackage{calc} % for calculating minipage widths
% Correct order of tables after \paragraph or \subparagraph
\usepackage{etoolbox}
\makeatletter
\patchcmd\longtable{\par}{\if@noskipsec\mbox{}\fi\par}{}{}
\makeatother
% Allow footnotes in longtable head/foot
\IfFileExists{footnotehyper.sty}{\usepackage{footnotehyper}}{\usepackage{footnote}}
\makesavenoteenv{longtable}
\usepackage{graphicx}
\makeatletter
\def\maxwidth{\ifdim\Gin@nat@width>\linewidth\linewidth\else\Gin@nat@width\fi}
\def\maxheight{\ifdim\Gin@nat@height>\textheight\textheight\else\Gin@nat@height\fi}
\makeatother
% Scale images if necessary, so that they will not overflow the page
% margins by default, and it is still possible to overwrite the defaults
% using explicit options in \includegraphics[width, height, ...]{}
\setkeys{Gin}{width=\maxwidth,height=\maxheight,keepaspectratio}
% Set default figure placement to htbp
\makeatletter
\def\fps@figure{htbp}
\makeatother
\setlength{\emergencystretch}{3em} % prevent overfull lines
\providecommand{\tightlist}{%
  \setlength{\itemsep}{0pt}\setlength{\parskip}{0pt}}
\setcounter{secnumdepth}{5}
\usepackage{amsthm}
\usepackage{float}
\usepackage{rotating, graphicx}
\usepackage{multirow}
\usepackage{tabularx}

% new command for pretty oversets with \sim
\newcommand\simcal[1]{\stackrel{\sim}{\smash{\mathcal{#1}}\rule{0pt}{0.5ex}}}

\newcommand{\comma}{,\,}

\floatplacement{figure}{H}

\PassOptionsToPackage{table}{xcolor}

\usepackage{tcolorbox}

\definecolor{kcblue}{HTML}{D7DDEF}
\definecolor{kcdarkblue}{HTML}{2B4E70}

\makeatletter
\def\thm@space@setup{%
  \thm@preskip=8pt plus 2pt minus 4pt
  \thm@postskip=\thm@preskip
}
\makeatother

% \makeatletter % undo the wrong changes made by mathspec
% \let\RequirePackage\original@RequirePackage
% \let\usepackage\RequirePackage
% \makeatother

\newenvironment{rmdknit}
    {\begin{center}
    \begin{tabular}{|p{0.9\textwidth}|}
    \hline\\
    }
    {
    \\\\\hline
    \end{tabular}
    \end{center}
    }

\newenvironment{rmdnote}
    {\begin{center}
    \begin{tabular}{|p{0.9\textwidth}|}
    \hline\\
    }
    {
    \\\\\hline
    \end{tabular}
    \end{center}
    }

\newtcolorbox[auto counter, number within=section]{keyconcepts}[2][]{%
colback=kcblue,colframe=kcdarkblue,fonttitle=\bfseries, title=Key Concept~#2, after title={\newline #1}, beforeafter skip=15pt}
\ifluatex
  \usepackage{selnolig}  % disable illegal ligatures
\fi
\usepackage[]{natbib}
\bibliographystyle{apalike}

\title{Filosofía económica: Entendiendo la complejidad}
\author{Jeshua Romero Guadarrama}
\date{2021-08-09}

\begin{document}
\maketitle

{
\setcounter{tocdepth}{1}
\tableofcontents
}
\hypertarget{prefacio}{%
\chapter*{Prefacio}\label{prefacio}}
\addcontentsline{toc}{chapter}{Prefacio}

Publicado por Jeshua Romero Guadarrama en colaboración con JeshuaNomics:

{ Git Hub}
{ Facebook}
{ Twitter}
{ Linkedin}
{ Vkontakte}
{ Tumblr}
{ YouTube}
{ Instagram}

Jeshua Romero Guadarrama es economista y actuario por la Universidad Nacional Autónoma de México, quien ha construido el presente proyecto en colaboración con JeshuaNomics, ubicado en la Ciudad de México, se puede contactar mediante el siguiente correo electrónico: \href{mailto:jeshuanomics@gmail.com}{\nolinkurl{jeshuanomics@gmail.com}}.
Última actualización el lunes 09 del 08 de 2021

El presente texto nace al calor de las exigencias pedagógicas de entender los Fundamentos de la FIlosofía Económica, así como los Fundamentos y aplicaciones de la economía de la complejidad

\hypertarget{reconocimiento}{%
\subsubsection*{Reconocimiento}\label{reconocimiento}}
\addcontentsline{toc}{subsubsection}{Reconocimiento}

A mi alma máter: Universidad Nacional Autónoma de México (Facultad de Economía y Facultad de Ciencias). Por brindarme valiosas oportunidades que coadyuvaron a mi formación.

El libro muestra el fundamento ideológico del trabajo del economista, y las perspectivas ideológicas son las que han prevalecido en gran medida en los últimos dos siglos: liberalismo, nacionalismo y socialismo. Sobre la base y la fuerza de estas ideologías se han construido los sistemas de economía política. Roselli explora las conexiones entre teoría y juicios de valor para identificar las premisas filosóficas detrás del razonamiento económico de economistas tan diversos como Smith, Ricardo, Marx, Pareto, Keynes, Hayek, entre otros.

El liberalismo se inclinó originalmente hacia un laissez-faire sin trabas, luego hacia un papel más amplio del Estado en el sistema económico, bajo la influencia de la ideología socialista, luego nuevamente se ha apoyado en un enfoque individualista de las cuestiones de producción y distribución de riqueza; Más recientemente, la irrealidad de este enfoque ha sido revelada por las crisis sistémicas, lo que sugiere nuevas reflexiones e incertidumbres sobre la coherencia del razonamiento económico con la idea liberal: una perspectiva institucional e histórica puede abrir nuevos espacios para la comprensión de una economía liberal y capitalista.

Se examinan las vicisitudes del nacionalismo económico, sus rasgos estatistas y proteccionistas, su declive y resurgimiento reciente, no quedando claro qué forma está tomando actualmente desde el punto de vista económico y político. Esto es particularmente oscuro en el caso de esa forma específica de nacionalismo llamada populismo.

El declive y la caída del materialismo histórico de Marx no pueden ocultar el contraste inherente de intereses entre los dos lados de un contrato laboral. El legado duradero del socialismo es la relevancia duradera y multiforme, desde una fuerza laboral acobardada hasta cuestiones ambientales, de los temas sociales en las economías modernas.

Este libro presenta un estudio de los aspectos de la complejidad económica, con un enfoque en ideas fundamentales e interdisciplinarias. El tan esperado seguimiento de su volumen de 2011 Complex Evolutionary Dynamics in Urban-Regional and Ecologic-Economic Systems: From Catastrophe to Chaos and Beyond , este volumen reúne los hilos del trabajo anterior de Rosser sobre la teoría de la complejidad y sus amplias aplicaciones en economía y economía. una lista ampliada de disciplinas relacionadas.

El libro comienza con una descripción completa de las categorías más amplias de complejidad en economía (dinámica, computacional, jerárquica y estructural) antes de pasar a un análisis más detallado. Los dos capítulos siguientes abordan problemas asociados con la complejidad computacional, especialmente los de computabilidad, y discuten el Teorema de incompletitud de Godel con un enfoque en la reflexividad. Los capítulos intermedios discuten la relación entre la entropía, la econofísica, la evolución y la complejidad económica, respectivamente, con aplicaciones en la dinámica urbana y regional, la economía ecológica, la teoría del equilibrio general y la dinámica del mercado financiero. El capítulo final trabaja para reunir estos temas en un marco más amplio y exponer algunos de los límites relacionados con el análisis de cuestiones fundamentales más profundas.

Con aplicaciones en todas las disciplinas caracterizadas por sistemas adaptativos no lineales interconectados, este libro es apropiado para estudiantes graduados, profesores y profesionales de la economía y disciplinas relacionadas como ciencias regionales, matemáticas, física, biología, ciencias ambientales, filosofía y psicología.

Palabras clave:

\begin{itemize}
\tightlist
\item
  Filosofía económica
\item
  Liberalismo clásico
\item
  Nacionalismo económico
\item
  Socialismo marxista
\item
  Filosofía social
\item
  Economía heterodoxa
\item
  Filosofia politica
\item
  Teoría económica clásica
\item
  Iluminación
\item
  Economía política
\item
  Historicismo económico alemán
\item
  Liberalismo económico
\item
  El liberalismo de Keynes
\item
  Corporativismo
\item
  Historia de las ideas
\item
  Complejidad económica
\item
  Teoría económica
\item
  Teorema de incompletitud de Gödel
\item
  Reflexividad
\item
  Dinámica del mercado
\item
  Teoría del equilibrio
\item
  Complejidad computacional
\item
  Econofísica
\end{itemize}

\hypertarget{contenido}{%
\chapter*{Contenido}\label{contenido}}
\addcontentsline{toc}{chapter}{Contenido}

Parte I Teoría elemental de números

Capítulos:

\begin{itemize}
\tightlist
\item
  Congruencias
\end{itemize}

\hypertarget{uxedndice-de-contenido}{%
\section*{Índice de contenido}\label{uxedndice-de-contenido}}
\addcontentsline{toc}{section}{Índice de contenido}

Capítulo 1. Congruencias

\begin{itemize}
\item
  Introducción a las congruencias
\item
  Sistemas de residuos y función \(\phi\) de Euler
\item
  Congruencias lineales
\item
  El teorema del resto chino
\item
  Teoremas de Fermat, Euler y Wilson
\end{itemize}

\hypertarget{part-siglo-xix}{%
\part{Siglo XIX}\label{part-siglo-xix}}

\hypertarget{ideologuxedas-y-economuxeda-poluxedtica-en-el-siglo-xix}{%
\chapter*{Ideologías y economía política en el siglo XIX}\label{ideologuxedas-y-economuxeda-poluxedtica-en-el-siglo-xix}}
\addcontentsline{toc}{chapter}{Ideologías y economía política en el siglo XIX}

Según Schumpeter, las raíces de la economía se encuentran en la filosofía social y en la experiencia empresarial concreta de la vida diaria, particularmente en la Gran Bretaña del siglo XVIII. La discusión sobre la actividad económica toma diferentes giros en Gran Bretaña y Alemania: en el primero, entremezclando esas dos raíces, se basa en un punto de vista individualista; en el segundo, sobre el Estado como centro de la vida social y económica. Adam Smith, filosóficamente motivado por la Ilustración Moderada, es visto como el fundador de la Escuela Clásica de economía política y la encarnación de un liberalismo económico ``cosmopolita'', válido en cualquier momento y lugar. Más adelante en el siglo XIX, la economía política está fuertemente influenciada por el positivismo y los fenómenos económicos se estudian de manera similar a las ciencias naturales. La atención del economista neoclásico se centra en la utilidad marginal del individuo, y todo el equilibrio del sistema económico se explica en términos matemáticos, como con Walras y Pareto, apoyando implícitamente una visión conservadora de la sociedad. En Alemania, el papel central del Estado encuentra una base hegeliana y la economía política se caracteriza por un historicismo acentuado, a través de los trabajos de la Lista proteccionista, y de los exponentes de la Escuela Histórica de Economía, en una especie de ``nación económica''. edificio". La centralidad y el desarrollo de la nación, y una insistencia en la reforma social en un enfoque no marxista, se enfatizan dentro de un marco de dependencia del camino. La misma raíz hegeliana se puede encontrar en Marx y su materialismo histórico. Las clases sociales, ya presentes en el análisis de la Escuela Clásica y luego descartadas por los neoclásicos, son vistas en una perspectiva confrontacional totalmente diferente, resultado de la estructura capitalista de la sociedad: una estructura que será superada determinísticamente por la revolución proletaria o por la caída. de la tasa de ganancia del capitalista.

Palabras clave

\begin{itemize}
\tightlist
\item
  Escuela clasica
\item
  Escuela neoclásica
\item
  Escuela histórica alemana
\item
  Materialismo histórico marxista
\end{itemize}

\hypertarget{schumpeter-en-el-origen-de-la-economuxeda-poluxedtica}{%
\section*{Schumpeter: en el origen de la economía política}\label{schumpeter-en-el-origen-de-la-economuxeda-poluxedtica}}
\addcontentsline{toc}{section}{Schumpeter: en el origen de la economía política}

Joseph Alois Schumpeter ha sido una de las pocas mentes críticas que ha intentado llegar al núcleo de la ciencia económica. En un ensayo que escribió hace más de un siglo, observó que ``la ciencia de la economía, ya que tomó forma hacia el final de la 18 ª siglo, había crecido a partir de dos raíces que deben ser claramente diferenciados unos de otros''. La primera raíz se originó en el estudio de la filosofía, específicamente en esa vertiente de la filosofía que consideraba las actividades sociales como el problema fundamental, como el elemento esencial de la visión del mundo. La otra raíz reflejaba los puntos de vista de ``personas de diversos tipos'' cuyo interés se centraba en cuestiones reales y prácticas de su vida diaria. 1

En cuanto a la primera corriente de pensamiento, el mundo social ---hasta entonces aceptado como evidente en sí mismo y, por lo tanto, no merecedor de una atención especial, o como un misterio explicable sólo en términos religiosos sobrenaturales--- se veía desde una perspectiva diferente, como un problema intelectual que debía plantearse. trataba de métodos naturales, no sobrenaturales, basados \hspace{0pt}\hspace{0pt}en la observación empírica y el análisis fáctico. 2Para una verdadera comprensión del mundo social, tenía que explicarse en términos racionales, es decir, por medio de una relación causa-efecto en el comportamiento humano. La filosofía moral, como unidad que resultó de estas reflexiones, incluyó la Teología, la Ética, la Jurisprudencia y la Economía. ``En esta unidad orgánica un elemento afecta a todos los demás, casi todos los pensamientos son importantes también para la Economía''. Y en este punto Schumpeter nombra a Locke y Hume: ``Nunca más la filosofía fue hasta tal punto una ciencia social como en este período''. 3

En cuanto a la otra raíz, el interés por lo práctico y cotidiano hizo que, a diferencia de la primera, no viera la actividad humana como, per se, problemático. Los pensadores pertenecientes a esta segunda corriente, por un lado ricos en experiencia empresarial, por otro lado carecían de formación científica y eran reacios a plantear cuestiones filosóficas. Se puede entender por qué ---añade Schumpeter--- algunos comienzos excelentes no tuvieron un seguimiento significativo, porque, si el problema práctico inmediato se había resuelto, no se sintió la necesidad de una reflexión adicional y más profunda. Este tipo de literatura reveló su frescura y frutos en la observación directa del mundo social, pero resultó infructuosa más allá de eso. Sin embargo, esta economía ``vulgar'' (el adjetivo es de Schumpeter y fue utilizado anteriormente por Marx, aunque con un significado ligeramente diferente 4).), basada como estaba en la realidad de la vida empresarial, aportó una importante contribución al auge de la economía política. Con el tiempo, principalmente en Inglaterra, la experiencia de la vida práctica comenzó a ser fertilizada por un hábito mental de tipo científico: por ejemplo, ``Se logró un gran progreso cuando se abandonó la concepción `bullionista' y la gente se dio cuenta, en cambio, de que los tipos de cambio y la balanza comercial estaban correlacionados''. 5

Este tipo de fertilización cruzada de comportamiento práctico y pensamiento teórico asumió un carácter diferente en diferentes países. En Inglaterra, las condiciones políticas relacionadas con el parlamentarismo favorecieron un debate abierto en la opinión pública y una urgente necesidad de análisis económico. En otros lugares, los gobiernos autocráticos desalentaron estas discusiones sobre economía política. En Alemania, el bajo nivel de una discusión racional sobre economía fue el resultado de años de guerras religiosas y reflejó una falta de discusión libre. Allí, la adopción de modelos extranjeros obstaculizó cualquier desarrollo original de la ciencia económica. Por otro lado, en ningún país como en Alemania el Estado se convirtió en objeto de un interés inagotable: el Estado como factor esencial del proceso de civilización.

Para los británicos, su historia trataba de liberar a la sociedad de un monarca opresor, mientras que para los alemanes significaba la afirmación de un Estado fuerte a partir de un feudalismo retrógrado. El derecho administrativo ocupó en Alemania el mismo lugar que la economía política había ocupado en Inglaterra. En Inglaterra, los comerciantes discutían entre ellos de la misma manera que los funcionarios públicos discutían en Alemania. Si en Inglaterra discutieron sobre economía, esta se convirtió en teoría económica; mientras que en Alemania esto tomó la forma de ciencia política de la economía. Schumpeter cita una obra del neocamerista alemán Johann von Justi (1756), cuyo plan y objetivo son los mismos que en La riqueza de las naciones de Adam Smith (1776): no tan distante en términos de tiempo, pero dos obras están separadas por los trabajos de un siglo ''.6Tan valioso es el libro de von Justi en el campo de la técnica de la administración, en materia económica carece de enfoques y métodos que ya estaban bien disponibles. Los juicios prácticos de Von Justi revelan sentido común, pero la estructura analítica de su trabajo es defectuosa. Werner Plumpe ha observado recientemente que von Justi compartía con Smith la opinión de que el bienestar común depende del buen funcionamiento del mercado, pero von Justi pensaba que era responsabilidad del Estado canalizar adecuadamente el interés privado de manera que que el bienestar puede ser alcanzado por el mercado. A diferencia de la Ilustración escocesa, en la semántica alemana el papel del Estado es de fundamental importancia. Mientras que, según Smith,7 En Alemania, Law adquirió la misma posición que tenía la economía en Gran Bretaña; y una doctrina de economía de Estado significaba que ``el problema individual nunca es objeto de tratamiento por sí mismo, sino sólo como parte del todo''. 8 Este enfoque sistemático ha caracterizado la economía en Alemania ``hasta el día de hoy {[}de Schumpeter{]}'', escribe. Uno puede preguntarse si esto es cierto hasta nuestros días.

Schumpeter vio bien las raíces filosóficas de la ciencia económica, que es el marco ideológico que influye en las elecciones teóricas del economista, su ``filosofía mundana'', para usar las palabras de Robert Heilbroner. 9

Lo que se acaba de decir sobre la diferente actitud que adoptó la discusión sobre asuntos económicos en Gran Bretaña y Alemania, tuvo una influencia decisiva en el desarrollo de la disciplina de la economía. La primera actitud se centra en un punto de vista individualista, centrado en el individuo como agente racional y, por tanto, asumiendo que cualquier acción ``irrazonable'' debe ser vista como una curiosidad filosófica, si no como una aberración poco interesante; el segundo se vuelve hacia el Estado y ve el interés público como no necesariamente coincidente ---a veces en oposición--- al privado.

Estas filosofías son la premisa, a menudo no revelada, de todo razonamiento económico que va más allá de lo que Schumpeter define como la economía ``vulgar''. 10

Sin embargo, si en su origen filosófico común hacemos una distinción entre los dos, es fácil notar que el primero, la visión centrada en el individuo racional, se encontró fructíferamente con la economía vulgar, basada como estaba en el interés propio inmediato. en acciones humanas; mientras que el segundo complicó cualquier relación entre el Estado racional y el individuo. De hecho, si ponemos a la sociedad, o al Estado como su representante, como encarnación de la racionalidad, las acciones individuales no merecen ser puestas en el centro de la atención del analista. En el extremo, en una especie de clasificación, el individuo siempre ocupa el segundo lugar, después del Estado. Desde esta perspectiva,

Nos ocuparemos primero del enfoque basado en el individuo racional (Secciones 1.2 - 1.5 ), y luego del enfoque centrado en el Estado racional (Secciones 1.6 - 1.11 ). Nuestro período de referencia en este capítulo es principalmente el siglo XIX.

\hypertarget{ilustraciuxf3n-radical-y-moderada-adam-smith-y-david-ricardo}{%
\section*{Ilustración radical y moderada: Adam Smith y David Ricardo}\label{ilustraciuxf3n-radical-y-moderada-adam-smith-y-david-ricardo}}
\addcontentsline{toc}{section}{Ilustración radical y moderada: Adam Smith y David Ricardo}

El liberalismo del siglo XIX puede verse como centrado en el individuo y exigiendo un nuevo estándar que gobierne la relación entre el individuo y el Estado. Esta norma se basó en tres requisitos principales: no intrusión, no exclusión, no obstrucción. 11 El primer requisito implicaba asegurar al individuo y su propiedad, sus derechos de propiedad, principalmente a través del sistema legal, de la interferencia del Estado, el mercado, la sociedad; la segunda, mayoritariamente moral y originariamente arraigada en la religión, se fundamentaba en el reconocimiento de la dignidad humana, pero conocía ---cuando se trasladó a la política--- algunos límites no irrelevantes; el tercero proporcionó una palanca económica para el pleno desarrollo de las capacidades humanas, que se mantuvo libre de obstáculos de cualquier tipo.

El individualismo como pieza central del liberalismo fue un denominador común, que unificó a los liberales contra el autoritarismo, las reglas dictadas por la tradición e invocando la primacía de la libertad y la responsabilidad del hombre. Pero el liberalismo tomó formas específicas, inclinándose a veces hacia la igualdad y la solidaridad, otras veces hacia ver al hombre liberal ya la sociedad como un ``club-si-quieres-únete-si-quieres'' (Fawcett). Este punto puede entenderse mejor si tenemos en cuenta que el antecedente inmediato del liberalismo es la Ilustración, a través de su insistencia en la libertad individual y en el poder restringido del soberano.

En el campo de la disciplina económica, la Ilustración muestra los fructíferos resultados del entremezclado de la filosofía centrada en el individuo racional, con la mirada vulgar del individuo que busca la satisfacción de sus necesidades cotidianas. ``Adam Smith {[}y otros pensadores{]} fueron indiscutiblemente los pioneros clave de esta nueva ciencia, pero estudiar sus ideas económicas de forma aislada de su filosofía general, ideas morales y conceptos sociales, como es habitual, corre el riesgo de reducir el surgimiento de la economía a algo extraño y extraño. desprendido de su edad''. 12 Jonathan Israel observa que ``para ser comprendida adecuadamente en su contexto histórico, la economía clásica debe situarse en el contexto de la lucha entre el pensamiento radical y el pensamiento de la Ilustración moderada''. 13

Un trasfondo común entre los pensadores radicales y moderados se puede encontrar en su convicción compartida de que una sociedad liberal y comercial ofrece una forma superior de libertad: la libertad bajo la ley. Pero la sustancia de la base moral sobre la que descansa la sociedad es ---en estos dos enfoques--- diferente.

En la visión radical de una cultura política democrática, incluso republicana, tal como la expresan autores tan diversos como Diderot, d'Holbach, Helvétius, Condorcet y otros, la tolerancia y la búsqueda incondicionada de la libertad de pensamiento y expresión están íntimamente ligadas a la concepto de igualdad social y política. La revolución no violenta, en nombre de la libertad, que perseguían, por ejemplo, Diderot y d'Holbach, tenía como objetivo hacer de la igualdad el principio moral supremo de la organización social.

Por lo tanto, se subvirtió la concepción teológica de Locke de la igualdad, que consideraba a los individuos espiritualmente iguales ante Cristo pero no iguales en estado civil: una sociedad clasificada por diferentes clases, 14 incluso admitiendo la esclavitud, en una especie de dualismo filosófico que distinguía entre cuerpo y alma. Locke fue ``un filósofo que claramente favorecía la jerarquía social marcadamente estratificada y la amplia desigualdad de propiedad; era innegable que había un elemento de vacilación, incluso quizás de contradicción, en sus comentarios sobre la esclavitud''. 15Locke, después de haber definido a un esclavo como un cautivo tomado en una guerra justa, escribe: ``como esclavo ha perdido todos sus bienes, y como esclavo no es capaz de tener ninguna propiedad; por lo que no puede ser considerado en su condición de esclavo como parte de la sociedad civil, cuyo propósito principal es la preservación de la propiedad''. 16 Los filósofos escoceses de la Ilustración Moderada promoverían una visión similar, aunque menos extrema, de la moralidad, que incluye la ``preservación de la propiedad'', pero los pensadores radicales menospreciarán a estos filósofos como ``sentido moral'' teólogos, que restringieron el alcance de la razón. ``La concepción anglo-escocesa de la ética fue {[}por ellos{]} rotundamente rechazada''. 17

En esta búsqueda de la igualdad, los pensadores radicales del período de la revolución francesa observaron que el ``gran vicio de nuestro sistema social \ldots{} es la monstruosa desigualdad de fortunas''. ``Los ricos comprenden el resentimiento que esto les causa pero no tolerarán una república genuinamente democrática, sabiendo que tarde o temprano les privará de parte de sus riquezas''. `` Le capitaliste fue {[}por ellos{]} identificado como antisocial, egoísta y dañino, capaz de subvertir al gobierno en su propio interés. Montagnards 18 \ldots{} también pensó en términos de imponer nuevas normas de estilo de vida igualitario a través de la educación y la instrucción pública ''. 19

Una inspiración para los Montagnards fue el suizo Jean Jacques Rousseau y su corriente de pensamiento radicalmente igualitaria. De hecho, entre los pensadores radicales, él ocupa un lugar específico y extremo. Su doctrina de la ``Voluntad General'', según la cual no se puede delegar la soberanía popular, se opondría incluso a otros exponentes radicales, que abogaban por un régimen de representación parlamentaria.

Este régimen parlamentario fue visto por ellos, no solo por los pensadores de la Ilustración francesa, sino también por los padres fundadores estadounidenses, como un paso revolucionario que eliminaría el acceso hereditario o privilegiado a cualquier asamblea nacional, tomaría el control de las finanzas públicas en interés de los Estados Unidos. Estado, y obstaculizar los intentos del soberano de utilizarlos para fines personales: una democracia representativa. 20

¿Qué, por el contrario, escribió el republicano Rousseau? ``{[}L{]} a actividad de interés privado, la inmensidad de los estados, la conquista y el abuso de gobierno, sugirió el método de tener diputados o representantes del pueblo en las asambleas nacionales\ldots{} {[}pero{]} la soberanía, siendo nada menos que el ejercicio de la Voluntad General, nunca puede ser enajenada, y\ldots{} el Soberano, que no es menos que un ser colectivo, no puede ser representado sino por él mismo 21 \ldots{} {[}L{]} aquí no hay posibilidad intermedia. Los diputados del pueblo, por tanto, no son ni pueden ser sus representantes: son simplemente sus agentes y no pueden arreglar nada por sí mismos. Cualquier `ley' que el pueblo no haya ratificado en persona es nula y sin valor, no es una ley''. 22Y Rousseau agrega: ``El pueblo de Inglaterra se considera libre, pero está muy equivocado: es libre solo durante la elección de los miembros del parlamento. Tan pronto como son elegidos, la esclavitud se apodera de ella''. 23 Pocos años después, Voltaire escribiría un libro lleno de admiración por la constitución británica. 24

En su Discurso que sigue al contrato social , Rousseau explora algunas ideas sobre economía política, con fuertes acentos igualitarios. Destaca la educación pública organizada por el Estado, necesaria para llegar a la conformidad en la Voluntad General, y un sistema de tributación muy progresiva que también desalentará el lujo: ``el pueblo estaría dispuesto a adorar a un Ministro que fuera a pie al Concejo, porque él vendió sus carruajes para suplir una necesidad urgente del Estado''. 25

La ausencia de representación parlamentaria, el predominio de la voluntad general y las opiniones igualitarias de Rousseau se analizan con más detalle en el capítulo 4 de este ensayo, al tratar de la ideología populista.

En la misma línea, el pensador político italiano Melchiorre Gioja veía la monarquía como equivalente a la ignorancia y la estupidez, creía en una república libre para su país, con una constitución democrática y, siguiendo las ideas económicas de la Ilustración radical, criticaba las doctrinas del libre comercio de Adam Smith ( ver más abajo), alegando que ``la filosofía ha declarado la guerra a la desigualdad'' y favoreciendo una escrupulosa regulación estatal de la industria y el comercio. 26

El pensamiento económico clásico toma forma principalmente a través de la filosofía de la Ilustración Moderada. Se basa en una reconciliación de la ética y el utilitarismo, y en la preservación del orden social para mejorar el crecimiento económico. La validez de este pensamiento se considera universal, independiente del tiempo y el lugar. Institucionalmente, su doctrina económica se basa en un mercado autorregulado y en un sistema monetario sólido; su diseño político, en un Estado liberal que permita florecer y desarrollarse al sector privado de la economía.

Los filósofos escoceses ``cimentaron su pensamiento moral en lo que en última instancia es una postura teológica y socialmente deferente''. Hay un ``entrelazamiento inextricable de la filosofía moral de {[}Adam{]} Smith y, más tarde, la economía con la noción de la providencia divina y su defensa (y la de Hume) del orden social existente''. 27 Smith, en su Wealth of Nations , no dedica muchas palabras a la monarquía. Según él: ``Aunque los monarcas, al hacer tratados, actúan como individuos al negociar, hay una gran diferencia con respecto a su adhesión al contrato. Los individuos actúan bajo el control de la ley y, por lo tanto, están obligados a cumplir con sus compromisos; pero los monarcas no las conservan más de lo que les conviene''. 28La postura deferente de Smith redujo el alcance de la razón y tendió a priorizar el sentimiento y la tradición. Defendiendo la desigualdad como condición necesaria para el orden social, el pensador irlandés Edmund Burke rechazaría entonces las ``igualaciones obligatorias''. Derriban lo que está arriba. Nunca levantan lo que está abajo. Y deprimen alto y bajo juntos por debajo del nivel de lo que originalmente era el más bajo '' 29 : una afirmación que, curiosamente, encontramos casi sin cambios en el pensamiento`` marginalista ''de un siglo después (véanse las Secciones 1.4 y 1.5 ).

¿Cuál puede ser la base ``moral'' de esta desigualdad? Adam Smith encuentra esta base en la necesidad de preservar el orden social. En La riqueza de las naciones, el supuesto de Smith es el reconocimiento de diferentes clases sociales, cada una con su propio papel: los capitalistas, los terratenientes, los trabajadores; y en la Teoría de los sentimientos morales (que se publicó en 1759, antes de The Wealth), observa: ``La paz y el orden de la sociedad, es más importante que incluso el alivio de los miserables \ldots{} Los moralistas nos advierten contra la fascinación de la grandeza. Esta fascinación, de hecho, es tan poderosa que los ricos y los grandes son preferidos con demasiada frecuencia a los sabios y virtuosos. La naturaleza ha juzgado sabiamente que la distinción de rangos, la paz y el orden de la sociedad, se basarían más firmemente en la clara y palpable diferencia de nacimiento y fortuna, que en la invisible y a menudo incierta diferencia de sabiduría y virtud''. 30

La Ilustración Moderada proporcionó el terreno más fértil en el que el liberalismo económico y la economía política clásica del libre mercado pudieron florecer y desarrollarse. Este desarrollo no podría ocurrir sin vincular la moralidad a una perspectiva utilitaria pronunciada.

Aquí hay que tener en cuenta que ese concepto de ``utilidad'' es uno de los más utilizados y abusados \hspace{0pt}\hspace{0pt}en todo el cuerpo de la economía política. Siempre que este término sea fundamental para un escritor de economía política, debemos vincularlo necesariamente a la ideología de su creador.

La visión de la utilidad de los economistas clásicos está bastante lejos de la utilidad calculada ``científicamente'' del sucesivo pensamiento neoclásico. Este punto necesita una aclaración. A finales del siglo XVIII, la formulación más completa del concepto se encuentra en los Principios de moral y legislación de Jeremy Bentham.(1789). ¿Qué quiere decir Bentham con el ``principio de utilidad''? Es ``el principio que aprueba o desaprueba toda acción según la tendencia que parezca tener a aumentar o disminuir la felicidad de la persona o grupo cuyo interés se cuestiona''. La búsqueda inmediata de este objetivo diferencia, según Bentham, el utilitarismo del ascetismo, que pertenece a los moralistas (que se mueven por la esperanza, es decir, la perspectiva del placer) y a los ``religiosos'' (que se mueven por el miedo, que es por una perspectiva de dolor). Sin embargo, el comportamiento moral o religioso puede, por sí mismo, dar lugar a la felicidad. Por tanto, la felicidad no es necesariamente de naturaleza física, bien puede ser de naturaleza política o moral (cuando es estimulada por personas conectadas al individuo en una disposición espontánea, no coercitiva),31 En resumen, el utilitarismo no debe identificarse necesariamente con la búsqueda de la felicidad material, solo significa que el individuo tiene su propia escala de valores que no debe estar sujeta a restricciones externas. 32 Estamos muy lejos de la ``utilidad marginal'' matemáticamente mensurable de la escuela neoclásica.

Dada esta definición de utilidad como instrumento para perseguir la felicidad, que, debería repetirse, bien puede ser de naturaleza política, moral e incluso religiosa, Bentham escribe que la ética es ``el arte de dirigir la acción de los hombres hacia la producción de los mayores cantidad posible de felicidad, por parte de aquellos cuyo interés está en la mira''. Este es el ``arte del autogobierno o la ética privada''. Es importante destacar que si la felicidad es el fin (propósito) de la ética privada, ``la legislación no puede tener otro''. El deber del legislador es simplemente favorecer la ética privada. ``El arte de legislar enseña cómo una multitud de hombres, que componen una comunidad, pueden estar dispuestos a seguir ese camino que, en conjunto, es el más conducente a la felicidad de toda la comunidad''. 33 Aquí está el vínculo entre la felicidad del individuo y la comunidad.

En la Ilustración británica, cualquier relación dialéctica entre la utilidad, la ética y el orden social se resuelve, por tanto, considerando al individuo como perseguidor de su propio interés y, al mismo tiempo, como un sujeto moralmente motivado en una sociedad. Es una moral que se centra más en la honestidad de la persona y quizás en la caridad (que depende de la iniciativa individual, a diferencia de las iniciativas de bienestar público), que en discutir el orden social existente. La Ilustración británica se basa más en sentimientos de riqueza impulsados \hspace{0pt}\hspace{0pt}individualmente, y en un sistema institucional adecuado, que en los requisitos de igualdad, como lo hace la Ilustración francesa. 34``Los filósofos de la Ilustración temprana dotaron a la ética de una base nueva y, con suerte, más sólida en psicología. La moralidad se había presentado tradicionalmente como un sistema objetivo de leyes divinas \ldots{} cada vez más, la virtud se reconfiguraba como una cuestión de prestar atención a los impulsos internos: la bondad residía \ldots{} en aprovechar los motivos \ldots{} las pasiones eran naturalmente benignas \ldots{} y el placer tenía que derivarse de la simpatía''. 35 ``Al dar una conferencia a los jóvenes escoceses, Smith elevó el ego del hombre comercial por encima de las virtudes cívicas del republicano clásico, insistiendo particularmente en la riqueza, la libertad y la sabiduría política necesarias para sostener una política comercial''. 36 Existe, en la economía política de Smith, un sentimiento de simpatía que vincula el comportamiento humano y hace de la confianza la base de las relaciones sociales y económicas. El concepto de confianza es recurrente en suRiqueza : una confianza que debe partir del legislador: ``Pero la ley siempre debe confiar en las personas con el cuidado de sus propios intereses, ya que en sus situaciones locales generalmente deben poder juzgar mejor que el legislador''. 37

Sobre la base de la libertad natural y del principio utilitario como impulsor de las acciones de los hombres, la proposición central de la Ilustración Moderada de Smith es que la sociedad puede corregir sus propios desequilibrios y progresar si el mercado puede funcionar sin obstáculos; y que esto puede suceder sin destruir los principios de jerarquía, la clasificación de clase, por la que se rige la sociedad.

El ranking de clases es relevante y la clase capitalista está en el centro del sistema económico. El progreso es el resultado del buen funcionamiento de un mercado libre, que permite generar ganancias. ``El crecimiento económico está asegurado por el excedente, o producto neto, que se convierte en el motor que genera más riqueza al brindar los medios a través de los cuales se aumenta la producción, se refina la técnica, se estimula el comercio''. 38 La clase capitalista es, por tanto, el verdadero motor del crecimiento económico, a diferencia de los rentistas ociosos, que son sólo consumidores, y de los trabajadores, demasiado pobres para ahorrar o invertir (la aversión por la clase improductiva de los rentistas es una constante de los economistas clásicos , pero también es compartida por Keynes, y obviamente por Marx).

No hay mejor síntesis del significado de la riqueza de las nacionesque el escrito hace varios años por el economista político Herbert Stein, quien dice: ``Comenzar el tratado con la descripción simple y hogareña de una fábrica de alfileres fue un golpe brillante. Al principio uno se pregunta qué está haciendo eso allí. Pero luego queda claro que Smith nos está llevando a comprender y apreciar la división del trabajo. Y la división del trabajo nos lleva inexorablemente a la idea del intercambio como la forma natural y eficiente de organizar una economía. En ese momento, la batalla está a mitad de camino: el resto es sacar las implicaciones del hecho de que una economía moderna es un sistema de intercambio \ldots{} La riqueza de las naciones está llena de frases bien redactadas. El más famoso \ldots{} es probablemente: 'No es de la benevolencia del carnicero, el cervecero o el panadero, que esperamos nuestra cena, sino de la consideración de su propio interés'39 \ldots{} Las personas que usan corbata de Adam Smith no lo hacen para honrar al genio literario. Lo están haciendo para hacer una declaración de su devoción a la idea de mercado libre y gobierno limitado \ldots{} {[}a pesar de que Smith{]} estaba dispuesto a aceptar o proponer salvedades a esa política en los casos específicos en los que juzgó que su efecto neto sería beneficioso y no socavaría el carácter básicamente libre del sistema''. 40

Solo hay una cosa que agregar: al citar a los comerciantes, Adam Smith da evidencia de la fructífera mezcla de la raíz filosófica con la raíz ``vulgar'' de la economía política. Con Adam Smith, ``el filósofo político podría retirarse a favor del hombre de negocios, porque este último podría alcanzar el summum bonum del filósofo simplemente persiguiendo su propio beneficio privado''. 41

En cuanto a la universalidad del pensamiento de los economistas clásicos, debería enmarcarse dentro de la perspectiva general de la Ilustración. El discurso de Smith es claramente una expresión de esa perspectiva. Él ve en una cierta estructura organizativa del sistema económico, la encarnación de un principio de libertad individual: perfecto en sus características esenciales y, por lo tanto, no necesita ningún cambio. Las proposiciones de los economistas clásicos son válidas, como Ricardo subraya más tarde, ``en todos los países y en todos los tiempos'' 42: este es el ``cosmopolitismo'' y la a-temporalidad de la ciencia económica. De hecho, la armonía de un equilibrio natural fue el concepto inspirador de la Ilustración en cualquier rama del pensamiento, no solo en la economía. Su validez universal hace que la teoría económica clásica esté esencialmente separada de la experiencia histórica real y de los modos de organización productiva (un punto que será fuertemente cuestionado por la Escuela Histórica Alemana así como por Marx, véanse las Secciones 1.10 y 1.11 ). De hecho, con la excepción de muchas muestras específicas tomadas del pasado, mencionadas instrumentalmente por Smith, para dar evidencia y más énfasis a sus propias tesis, de validez universal, 43 su análisis histórico se limita a unas pocas páginas en el Libro V de la riqueza, donde hace una descripción estilizada, tentativa y conjetural de la evolución del desarrollo económico del hombre enumerando sucintamente estados sucesivos y diferentes de la sociedad, desde cazadores hasta pastores y campesinos, hasta el hombre dedicado a la manufactura y el comercio. 44

Si dirigimos nuestra atención a los aspectos internacionales y al comercio exterior, se evidencia el ``cosmopolitismo'' de los economistas clásicos (diríamos ``globalismo''), y por tanto su ataque al proteccionismo y las políticas mercantilistas, a favor del libre comercio. Consideremos el siguiente pasaje del ensayo de Hume sobre la balanza comercial: ``Nuestros celos y odio hacia Francia no tienen límites, y el sentimiento anterior, al menos, debe reconocerse como muy razonable y bien fundamentado. Estas pasiones han ocasionado innumerables barreras y obstáculos al comercio \ldots{} Pero, ¿qué hemos ganado con el trato? Perdimos el mercado francés para nuestra fabricación de lana, y trasladamos el comercio de vino a España y Portugal, donde compramos licor mucho peor a un precio más alto \ldots{} Cada nuevo acre de viñedo plantado en Francia, para abastecer de vino a Inglaterra, exigiría que los franceses tomaran el producto de un acre inglés, sembrado en trigo y cebada, para subsistir; y es evidente que, por lo tanto, tenemos el control del mejor producto''.45

Cincuenta años después, Ricardo retoma este tema y crea un ``modelo'': su conocido caso de estudio del comercio anglo-portugués (formalizado en la ``teoría de la ventaja comparativa'') muestra una situación de equilibrio que, una vez alcanzada, es la más económicamente eficiente y estable para ambos países 46: si la producción de telas en Inglaterra requiere el trabajo de 100 hombres durante un año, contra 120 hombres necesarios para producir vino, y si la viticultura en Portugal requiere solo el trabajo de 80 hombres, mientras que para la producción de telas se necesitan 90 hombres, será en Portugal. interés en producir solo vino (aunque la producción de telas cuesta menos que en Inglaterra) e importar telas de Inglaterra: con el trabajo de 160 hombres, todos dedicados a la viticultura, Portugal habrá obtenido vino y tela, frente a los 170 necesarios en para producir ambos en su mercado interno. 47Esta situación, que traspone a nivel internacional lo que Smith había escrito sobre la fábrica de alfileres, es, como se mencionó, eficiente y bien equilibrada, y no necesita ningún cambio (a menos que, hay que agregar, nos preocupe el hecho de que Portugal seguirá siendo un enorme viñedo e Inglaterra quedará asfixiada por el humo de sus fábricas de telas). El economista clásico elogia el libre comercio porque permite que cada nación maximice su propio producto, dados ciertos recursos y capacidades productivas. La Friedrich List alemana objetará este modelo al observar que este equilibrio evita que la economía menos avanzada cambie su estructura productiva y, por lo tanto, aumente sus ingresos a largo plazo.

Un sistema monetario estable, alejado de los caprichos del soberano, es una condición previa para que el libre mercado autorregulado funcione sin problemas. Los fundamentos teóricos del dinero estable son proporcionados por David Hume, en su ensayo sobre el dinero. 48 Es una enunciación magistral y concisa de la ``teoría cuantitativa del dinero'': ``los precios de todo dependen de la proporción entre las mercancías y el dinero, y \ldots{} cualquier alteración considerable de cualquiera de ellos tiene el mismo efecto de aumentar o disminuir los precios \ldots{} Es sólo el excedente {[}de una mercancía{]}, en comparación con la demanda, lo que determina el valor''. 49

Si el stock de dinero no solo está compuesto de oro, sino también de papel crédito, y si el crecimiento del componente papel es excesivo, la estabilidad monetaria se ve comprometida. Para limitar el aumento del crédito en papel (circulación del papel), provocado por la búsqueda de beneficios del sistema bancario, debe hacerse obligatoria la convertibilidad del papel en oro. 50 El requisito de convertibilidad priva al soberano del poder de modificar a su discreción el stock de dinero y encaja perfectamente con el funcionamiento de un mercado autorregulado. 51 La teoría cuantitativa del dinero se integró así en la corriente principal de la conducta monetaria ortodoxa, formando el núcleo central del análisis y la política monetaria clásica del siglo XIX.

Institucionalmente, algunas leyes proporcionaron los pasos necesarios para hacer de Gran Bretaña un país de libre comercio y patrón oro, implementando así el liberalismo en la forma más avanzada: la Ley para la reanudación de los pagos en efectivo de 1819 (después de las guerras napoleónicas que habían obligado al Reino Unido suspender la convertibilidad del oro) y el Bank Charter Act de 1844, para una promulgación completa del patrón oro; y el Bill of Repeal (Importation Act) de 1846, que abrió el mercado británico al abolir la protección asegurada por las Corn Laws.

En la economía clásica de Smith, Ricardo o el francés JB Say, la maximización del producto no puede verse obstaculizada por la falta de demanda, porque cualquier actividad económica genera ingresos, en forma de salarios, rentas, ganancias, que son iguales al valor del producto. . Las recesiones económicas no se deben a una caída de la demanda, sino a factores exógenos al sistema económico, a ``externalidades'', como guerras o interferencias (del gobierno, por ejemplo) que perturban el libre funcionamiento de las fuerzas del mercado. Esta es una afirmación que será negada por economistas tan diversos como Marx y Keynes.

Cualquier bien tiene demanda por su utilidad. Pero, ¿cómo se evalúa su precio en el mercado? El valor tiene dos significados diferentes, a veces expresados \hspace{0pt}\hspace{0pt}por la utilidad de ese bien, y a veces por el poder de comprar otros bienes que la posesión de ese bien en particular transmite: por lo tanto, de cualquier bien, el valor en uso debe distinguirse del valor de cambio. . 52

El valor de uso de un bien está relacionado con la utilidad que una persona obtiene al usar ese bien; mientras que el valor a cambio está vinculado a su precio. El primer tipo de valor, la utilidad de los bienes, explica los impulsores económicos de la sociedad; el segundo, el valor de cambio, explica que cualquier sociedad solo puede funcionar gracias al sistema de precios.

Smith y Ricardo están, con distintos acentos, de acuerdo sobre cómo se determina el valor de intercambio de una mercancía. Smith escribe: ``{[}Su valor{]} es igual a la cantidad de trabajo que permite {[}a una persona{]} comprar \ldots{} El trabajo, por lo tanto, es la medida real del valor de cambio de todas las mercancías \ldots{} Lo que todo vale realmente para el hombre quien lo ha adquirido, y quien quiere deshacerse de él o cambiarlo por otra cosa, es el trabajo y la molestia que puede ahorrarse y que puede imponer a otras personas''. ``Parece evidente que el trabajo es la única medida universal de valor, así como la única exacta''. 53Y Ricardo: ``Si la cantidad de trabajo realizado en mercancías regula su valor de cambio, todo aumento en la cantidad de trabajo debe aumentar el valor de la mercancía, como toda disminución debe bajarlo''. 54 Esta ``teoría del valor trabajo'' será el punto de partida de la reflexión y la crítica de Marx y, por razones opuestas, será criticada como ``no científica'' por los economistas ``marginalistas'' (o neoclásicos) de la segunda mitad del siglo XX. Siglo xix. Si el valor de cambio de una mercancía se explica en términos de una teoría de costos, como el valor de la fuerza de trabajo empleada en el proceso de producción, ¿cómo explicar el origen de la renta y la ganancia?

De hecho, la proporcionalidad entre la variación en la cantidad de trabajo y la del valor de una mercancía existiría en ausencia de maquinaria empleada en su producción. 55 Por tanto, el valor de cambio de una mercancía depende no sólo del trabajo empleado, sino también del capital empleado en la producción y de la tierra sobre la que insiste el capital. El capital y la tierra deben incluirse en el proceso de determinación del valor de cambio de una mercancía. 56Al respecto, Smith dice que este valor de cambio incluye los precios ``naturales'' de los tres factores de producción ---tierra, trabajo y capital, lo que significa rentas, salarios y ganancias--- que compensan el costo de producción de un mercancía. El precio ``natural'' puede diferir del precio de ``mercado'', pero solo por la razón de que el precio de mercado tiene en cuenta las desviaciones accidentales y temporales del primero. 57 Pero los economistas clásicos no adoptaron un criterio para determinar el precio natural del capital y la tierra. 58Con referencia al sector agrícola, a la renta de la tierra, Ricardo, quien mantiene una visión de las clases sociales (factores de producción) antagónica, a diferencia de la visión complementaria de Smith, observa que la tierra tiene rendimientos disminuidos (como demanda). aumenta, se cultivan tierras menos fértiles), de modo que capitalistas y rentistas compiten para quitarles a los trabajadores una parte cada vez mayor del valor de la mercancía. Este contraste fundamental entre capital y trabajo abre el camino a la crítica radical marxista: la tendencia histórica de la tasa de ganancia a disminuir, a menos que se explote más intensamente el trabajo (véase la sección 1.11 más adelante y el capítulo 3 ).

Sin embargo, en lo que respecta a las ganancias, si abandonamos el concepto de precio natural de los factores de producción, queda indeterminado cómo se puede dividir el valor de una mercancía entre ellos y, en particular, entre ganancias y salarios. No podemos ir más allá de la afirmación general de que el precio de una mercancía es más alto cuando se necesita más trabajo para producirla. ``Este fue el problema que desconcertó a Ricardo''. 59 Este tema recibirá más atención de Karl Marx y más tarde de Piero Sraffa. Sraffa es considerado como el economista que pudo reconciliar a Marx y Ricardo, los puntos de vista marxista y clásico (ver Capítulo 3 ).

\hypertarget{el-positivismo-y-john-stuart-mill}{%
\section*{El positivismo y John Stuart Mill}\label{el-positivismo-y-john-stuart-mill}}
\addcontentsline{toc}{section}{El positivismo y John Stuart Mill}

En esta y las siguientes Sectas. ( 1,3 - 1,5) nos ocuparemos de la tendencia a considerar la disciplina económica como una ``ciencia'', desconectada de cualquier base o trasfondo ``filosófico''. Relacionamos esta tendencia con la afirmación generalizada del positivismo: en sí mismo ---debe subrayarse--- una corriente filosófica de pensamiento bien definida. La afirmación de esta tendencia es coherente con algunos desarrollos amplios que caracterizaron el siglo XIX: un período prolongado de paz internacional después de las guerras napoleónicas (las guerras fueron pocas y circunscritas), estabilidad monetaria, grandes avances en las ciencias naturales y físicas, fuerte progreso tecnológico. En suma, una época relativamente tranquila que dio a la actividad económica unos rasgos de constancia y consistencia similares a los observados en el mundo natural y, por tanto, susceptibles de ser formalizados en ``leyes'' científicas. 60 Surgió una ciencia de la economía, que afirmó estar totalmente desconectada de las cuestiones éticas, políticas, sociales y más bien puesta al mismo nivel que las ciencias naturales.

Una lectura diferente lleva a considerar este enfoque científico como coherente con las estructuras sociales existentes, bien consolidadas en esa época ``tranquila''. Estas estructuras eran las de una sociedad individualista, y el criterio científico significaba la afirmación de los valores burgueses del liberalismo económico.

Según la definición de Hobsbawm, el positivismo es un ``hijo tardío de la Ilustración del siglo XVIII''. 61Queda por ver qué de la Ilustración conservan los positivistas y qué se deja de lado. Ciertamente mantuvieron la ruptura intelectual masiva con el pasado que había sido el foco de los pensadores ilustrados. Sin embargo, la relación de causa-efecto en el comportamiento humano fue vista por los positivistas con el mismo enfoque seguido por el científico en ciencias naturales. En las ciencias naturales, cualquier visión hermética de un universo espiritual había sido ``finalmente reemplazada por modelos de la naturaleza vista como materia en movimiento, gobernada por leyes capaces de expresión matemática. Esta entronización de la filosofía matemática, a su vez, sancionó la nueva afirmación del derecho del hombre sobre la naturaleza, tan sobresaliente para el pensamiento ilustrado''.62 Se trata de una visión que asumió un paradigma interpretativo basado en la comprobabilidad de cualquier hipótesis. El positivismo introdujo conceptos y métodos propios de las ciencias naturales en la investigación social.

Pero la Ilustración tuvo también otro lado, basado en hipótesis y teorías que no tienen el estatus no controvertido de hechos verificables o falsables o proposiciones matemático-lógicas: el lado ``ideológico''. Existía la convicción ``de que existían ciertas metas humanas objetivamente reconocidas que todos los hombres \ldots{} buscaban, a saber, la felicidad, el conocimiento, la justicia, la libertad y lo que se describía un tanto vagamente pero que se entendía bien como virtud; que estos objetivos eran comunes a todos los hombres \ldots{} Además, la naturaleza humana era fundamentalmente la misma en todos los tiempos y lugares; las variaciones locales e históricas no eran importantes''. 63 Como hemos visto, esta fue la esencia de la Ilustración Moderada en la base del liberalismo del siglo XIX.

El hombre económico, el homo oeconomicus, guiado por una visión utilitarista estricta y científicamente medida, el homo que investigan los economistas positivistas y neoclásicos, no parece encajar plenamente en la visión más amplia del hombre social, como la concibió Adam. Herrero. Paradójicamente, el positivismo parece privar a las ciencias sociales, en particular a la disciplina económica, de cualquier trasfondo filosófico. Pero, como veremos, las inferencias de la filosofía social que pueden extraerse de ella están lejos de ser insignificantes.

Entonces, es legítimo preguntarse cómo la filosofía del positivismo llegó a influir en la economía y hasta qué punto fue relevante esta influencia. La disciplina que abrió un vínculo entre esta nueva filosofía y la economía fue la sociología, y el hombre que inició la sociología fue Auguste Comte. El filósofo y economista político que, por así decirlo, tomó el toro por los cuernos fue John Stuart Mill. El Cours de Philosophie Positive de Comte se publicó en 1835, su Système de politique Positive en 1851-1854, y Auguste Comte and Positivism ---la respuesta de John Stuart Mill al positivismo--- siguió en 1865 .

Mill se siente realmente atraído por el enfoque de la filosofía de Comte, pero sigue siendo un discípulo de la Ilustración, y esto crea una considerable incomodidad para aceptar sus conclusiones. Veamos la clara explicación de Mill de los puntos principales del positivismo y las razones de su disensión parcial pero decisiva. Según la filosofía positiva, somos incapaces de conocer la esencia de ningún hecho, nuestro único conocimiento se limita a los fenómenos, a lo que nos aparece como ``hecho'', y este conocimiento es relativo, en el sentido de que de cualquier hecho solo conocemos su relación con otros hechos, en forma de sucesión o semejanza. Estas relaciones, si son constantes, ya sea por secuencia o semejanza, revelan la causa de los hechos y se denominan ``leyes''. La teoría se formula después de que se han observado los hechos; no debe crearse una teoría para observar hechos. Esto significa una prevalencia del razonamiento inductivo de la experiencia sobre la deducción de paradigmas, o incluso más de postulados. Un ejemplo interesante de investigación protopositivista lo da el economista clásico Robert Malthus enPrincipios de la población : en realidad parte de los postulados (con respecto a los elementos esenciales de la vida humana), pero su propósito es explicar cuánto está incrustada la esfera social en los sistemas biofísicos y ambientales. sesenta y cinco

Podríamos agregar que en la economía clásica, como se describe en la Sección anterior, el concepto de ``leyes'' no estaba conectado a ningún ``naturalismo''. Los adjetivos ``natural'' o ``normal'' se usaron en realidad, pero solo para significar algo definido en las condiciones más simples posibles, o simplemente ``evidente por sí mismo'' o ``habitual'', de todos modos sin ninguna referencia a las ciencias naturales. Comte, en cambio, quería usar el término en su significado ``apropiado'': también las humanidades tienen leyes naturales definidas en términos de causalidad natural.

Las ciencias ---como investigación de relaciones constantes entre hechos, es decir de ``leyes'' de las que deben depender todos los fenómenos--- son clasificadas por Comte en un orden ascendente, en el que cada ciencia representa un avance en la especialidad o un aumento en la complejidad con respecto a la ciencia precedente en la serie. Son seis, a partir de las matemáticas, 66 y derivada hasta la sociología, o ciencias sociales, que se define como la ciencia, los fenómenos de los que dependen de, y no se pueden entender sin, las principales verdades de todas las otras ciencias.

Cada ciencia se mueve, a su vez, en tres etapas: teológica, metafísica y positiva. Según Comte, sólo en el estado positivo, la mente humana, reconociendo la imposibilidad de alcanzar conceptos absolutos, abandona la búsqueda de las causas internas de los fenómenos y se limita al descubrimiento, a través de la razón y la observación combinadas, de leyes reales. que gobiernan la sucesión y semejanza de fenómenos.

De hecho, la teología y la metafísica son consideradas por Comte como no científicas, porque no miran las causas, es decir, las relaciones entre los hechos mismos, sino que ven los hechos como adscritos a causas ``celestiales'' u ``ordenanzas divinas'' ( como en teología), oa ``abstracciones realizadas'' (como en metafísica). La etapa positiva, según Comte, ha sido alcanzada ocasionalmente en sociología a veces en el pasado (por autores tan diversos como Montesquieu, Maquiavelo, Adam Smith, Bentham), pero esta etapa positiva en sociología aún no se ha desarrollado completamente. La ciencia social se ha desarrollado hasta ahora solo hasta la etapa metafísica, y tiene que ser ``actualizada'' a la etapa positiva, según él.

Debe observarse que esas ``abstracciones realizadas'' parecen reacias a dejarse de lado. En otras palabras, la etapa positiva de la sociología conduce a una fuerte mezcla de ciencia y filosofía. Herbert Spencer acuñó el término de ``supervivencia del más apto''. 67 Utilizando el lenguaje de la biología, mezcló la felicidad utilitaria benthamita, el progreso humano y la necesidad de que el gobierno se adapte al interés propio del hombre en una especie de ciencia que equipara lo ``correcto'' con lo ``natural''. Esto significa que el liberalismo se confunde con la biología ''. 68 ``Los pontífices del positivismo, Comte y Spencer, se revuelcan en la metafísica, pensando estar fuera de ella''. 69

A diferencia del carácter ahistórico de las obras de los economistas clásicos, Comte atribuye una importancia decisiva al método histórico. ``La comparación histórica de varios estados consecutivos de la humanidad no es solo la principal herramienta científica de la nueva filosofía política: desarrollada, demostrará ser la base misma de la ciencia. Es aquí donde la ciencia sociológica se distingue claramente de la ciencia biológica''. Este llamado a la historia puede parecer bastante extraño, cuando pensamos en el enfoque muy diferente seguido por la casi contemporánea Escuela Histórica de Economía Alemana (ver Sección 1.10). Más adelante en el mismo texto, sin embargo, Comte parece alejarse de este pasaje cuando escribe que el método histórico es ``equivalente al de la comparación zoológica en el estudio de la vida individual''. ``La secuencia necesaria de varios estados sociales corresponde exactamente, desde el punto de vista científico, a la coordinación gradual de varios organismos, teniendo en cuenta las diferencias de las dos ciencias: la serie social \ldots{} no puede ser ni menos real ni menos útil que la serie de animales''. 70 De nuevo, hay una sequitur de las ciencias naturales a las sociales.

En este punto, Mill entra en escena, y observa que debemos explicar la filosofía de la ciencia, a diferencia de la ciencia misma: una tarea que Comte ---dice Mill--- no ha cumplido. Por filosofía entendemos el conocimiento científico del Hombre como ser intelectual, moral y social, es decir, la ciencia misma considerada no en cuanto a sus resultados, sino en cuanto a los procesos mediante los cuales la mente los alcanza: la lógica de la ciencia. 71 El hombre no puede ser visto como una ``pieza de maquinaria'' y estudiado mientras estudiamos la producción de fenómenos físicos. Es cierto ---admite Mill--- que, para el Hombre, la regla del deber ha sido durante mucho tiempo dictada por una autoridad divina (etapa teológica) o, más recientemente, considerada como un corolario de algunos Derechos Naturales, como en Rousseau, a quien Mill ignora. (etapa metafísica). 72Pero ``Comte niega resueltamente el derecho moral de todo ser humano\ldots{} a erigirse como juez de las cuestiones más intrincadas que pueden ocupar el intelecto humano''. ``Todo lo que pasa por los diferentes nombres de revolucionario, radical, democrático, liberal, librepensador, escéptico\ldots{} todo pasa con él {[}Comte{]} bajo la denominación de metafísico\ldots{} sin validez permanente como verdad social''. Según Mill, ``hay una doctrina positiva\ldots{} que reivindica la participación directa de los gobernados en su propio gobierno, no como un derecho natural, sino como un medio para fines importantes, en las condiciones y las limitaciones que esos fines imponen''. 73

Para enfatizar este punto, Mill da un ejemplo: ``Tomemos por ejemplo la doctrina que niega al gobierno cualquier iniciativa en el progreso social, restringiéndolos a la función de preservar el orden \ldots{} una opinión que, en tanto, fundamentada en los llamados derechos del individuo, él {[}Comte{]} justamente lo considera puramente metafísico; pero no reconoce que también se sostiene ampliamente como una inferencia de las leyes de la naturaleza humana y los asuntos humanos y, por lo tanto, sea verdadera o falsa, como una doctrina positiva''. En otras palabras, Mill identifica --- horribile dictu para los oídos de un positivista --- doctrina metafísica y positiva. Walras y Pareto afrontarán el mismo problema, con una visión ``positiva'' opuesta (véase la sección 1.5 ).

¿Cómo aborda Mill, al mismo tiempo un gran economista político y un defensor del liberalismo, el concepto utilitario? ``Considero la utilidad como el máximo atractivo en todas las cuestiones éticas; pero debe ser utilidad en el sentido más amplio, fundada en los intereses permanentes del hombre como ser progresista''. 74 Apoya el libre comercio, pero ---agrega--- ``el libre comercio es un acto social\ldots{} se apoya en un terreno diferente al principio de libertad individual''. Como límites al libre comercio cita: ``qué cantidad de control público es admisible para la prevención del fraude por adulteración; hasta qué punto deben aplicarse a los empleadores las precauciones sanitarias o los arreglos para proteger a los trabajadores empleados en ocupaciones peligrosas. Tales preguntas involucran consideraciones de libertad, solo en la medida en que dejar a la gente sola es siempre mejor,ceteris paribus , que controlarlos: pero que puedan ser legítimamente controlados para estos fines, es en principio innegable''. 75

Estos comentarios de Mill suenan incompatibles con la conclusión de Comte sobre la etapa positiva de la sociología: ``no hay libertad de conciencia \ldots{} en astronomía, en física, en química, incluso en fisiología'', es decir, en otras ciencias; y ¿por qué debería haber tal libertad en sociología? Cuando la política llegue a una etapa positiva y se encuentren nuevas doctrinas, habrá una ``opinión establecida'' y no se necesitará libertad. ``El incompetente tribunal de la opinión común es radicalmente irracional, y cesará y debe cesar una vez que la humanidad haya vuelto a decidirse por un sistema de doctrina''. Una filosofía diferente es ``no sólo incapaz de ayudar a la necesaria reorganización de la sociedad, sino un serio impedimento para la misma''. 76

El razonamiento de Comte acaba siendo una especie de negación del liberalismo. Finalmente admite que no se necesita libertad cuando prevalece una opinión científica bien establecida (cuando se alcanza una etapa positiva en las ciencias sociales), de modo que el libre pensamiento será solo un obstáculo innecesario para la reorganización de la sociedad. Este tipo de conclusión puede que no haya tenido un gran número de seguidores en el campo de la disciplina económica, pero ese enfoque ``científico'' se convirtió, y desde cierto punto de vista sigue siendo hoy, ``una parte del valor comercial de la economía''. 77

La filosofía positivista es una brecha entre dos etapas de evolución de la disciplina económica en la segunda mitad del siglo XIX. Prepara el terreno sobre el que florecerá la escuela neoclásica entrante a principios del siglo XIX, al centrarse en el individuo y considerar su actividad social y económica como gobernada por ``leyes'', para ser descubierta científicamente, de manera similar a las metodologías. de las ciencias naturales. En este proceso, la ``economía política'' de la Escuela Clásica da paso a la nueva ciencia de la ``economía'' de la Escuela Neoclásica. ¿Fue el proceso coherente con la idea del liberalismo? ¿E implicó una ideología diferente, específicamente un conservadurismo político y social implícito? ¿Cómo fue evaluado este cambio de rumbo por observadores de diferentes orígenes?

Maurice Dobb, el economista marxista, escribió que es necesario un componente de clase para explicar la afirmación repentina y casi sin oposición del pensamiento neoclásico: ``No pasaron muchos años después de la publicación de ' Das Kapital' antes de que surgiera una teoría del valor rival. levantarse y con notable poca resistencia para conquistar el campo. Esta era la teoría de la utilidad {[}neoclásica, marginal{]}, que parece haber germinado simultáneamente en varias mentes''. 78 La nueva teoría se enmarcó directamente para proporcionar una respuesta sustituta a las preguntas que Karl Marx había planteado en su Capital . ``Aunque sólo por el efecto de la negación, la influencia de Marx en la teoría económica de finales del siglo XIX parecería haber sido mucho más profunda de lo que está de moda admitir''.79 La teoría neoclásica centrada individualmente habría sido ---según esta interpretación--- un intento de dar una respuesta ``científica'' a las contradicciones de clase social marxista, el ``socialismo científico'' de Marx.

Una severa crítica a Comte vino ---en el lado opuesto--- de un libertario como Hayek, quien calificó a Comte incluso más antiliberal que el prototipo del exponente del Estado ético: ``No hay en Hegel tales fulminaciones contra la libertad ilimitada de conciencia como encontramos a través de las obras de Comte, y el intento de Hegel de utilizar la maquinaria del Estado prusiano para imponer una doctrina oficial parece muy dócil en comparación con el plan de Comte para una nueva `` religión de la humanidad '\,' y todos sus otros esquemas de reglamentación completamente antiliberales. que incluso su antiguo admirador JS Mill finalmente tildó de liberticida''. 80

Se ha argumentado que en su uso del término ``ley'', Comte confunde descripción con prescripción 81 : una ``confusión'' que parece frecuente con los científicos sociales y particularmente con los economistas.

Robert Heilbroner pensaba que este desarrollo intelectual, de la economía política a la economía, tenía que ver con la evolución del sistema capitalista: un sistema caracterizado por ``clases bien delimitadas {[}como{]} condición natural y necesaria para cualquier orden social estable'' (lo que él llama ``la visión política aristocrática'') fue desplazada por un sistema que refleja una perspectiva cada vez más democrática y minimiza, o incluso niega, la presencia de clases sociales. 82

El esquema neoclásico, basado en la teoría de la utilidad marginal, desencarnado de la estructura de la sociedad, hizo que esta estructura fuera teóricamente irrelevante. La división de clases, esencial para comprender tanto la escuela clásica como el marxismo, desapareció en una configuración microscópica de la sociedad económica; esa misma configuración que, más tarde, Keynes atacaría a través de su visión macroeconómica.

Se podría argumentar que el positivismo, al asumir un enfoque mecanicista del funcionamiento del sistema social y económico, donde los agentes individuales persiguen su utilidad óptima interactuando con otros, sin solidaridad ni comunalidad de intereses, termina favoreciendo a la clase de los burguesía individualista (esto será muy evidente con Walras y Pareto). El positivismo favorece la conservación y fortalecimiento de las posiciones sociales actuales. Dentro del liberalismo, deberíamos esperar las convulsiones del siglo XX para ver desafíos a las estructuras sociales preexistentes y diferentes enfoques filosóficos por parte de pensadores ``liberales''. Con esto, no queremos restar importancia al hecho de que el sistema económico explicado por los economistas neoclásicos y puesto en práctica mediante políticas consistentes con este marco intelectual,Una estructura de la sociedad no lo hubiera permitido.

\hypertarget{utilidad-marginal-jevons-y-marshall-estamos-en-el-campo-del-liberalismo}{%
\section*{Utilidad marginal: Jevons y Marshall: ¿estamos en el campo del liberalismo?}\label{utilidad-marginal-jevons-y-marshall-estamos-en-el-campo-del-liberalismo}}
\addcontentsline{toc}{section}{Utilidad marginal: Jevons y Marshall: ¿estamos en el campo del liberalismo?}

Por tanto, el paso de la teoría clásica a la neoclásica puede verse como un cambio de visión del orden social; el primer orden caracterizado por una sociedad estructurada en tres clases -trabajadores, terratenientes, capitalistas- y el segundo considerado como un lugar de interacción de agentes económicos únicos, personas y empresas, atomísticamente consideradas como máquinas racionales: una interacción que determina transacciones recíprocas destinadas a maximizar su utilidad individual.

La utilidad se analiza y mide, mientras que otras motivaciones del comportamiento del hombre son irrelevantes, desde un punto de vista económico. Mientras que los economistas clásicos habían considerado el valor como una característica objetiva de las mercancías, los economistas neoclásicos centran su atención en las propiedades subjetivas de las mercancías, que se refieren al consumo y la demanda. La medida de la utilidad se realiza con el método de ``pequeños incrementos'', es decir, observando el ``principio marginal'': la demanda de un determinado bien se incrementa hasta el punto en que un pequeño incremento adicional de ese bien trae al comprador más pérdidas. que la ganancia de satisfacción. Los individuos que toman sus decisiones libremente deberían lógicamente llegar a la conclusión de que la libre competencia conduce a la maximización de la utilidad para ambas partes involucradas en el intercambio. La maximización de la utilidad alcanzada en un intercambio libre se extiende mediante procesos matemáticos al bienestar máximo para toda la economía, en una situación de equilibrio. ``La teoría de la utilidad marginal\ldots{} pone el énfasis principal en un complejo de problemas que los economistas clásicos pasaron por alto con demasiada ligereza, a saber, la base para la determinación del valor y el precio''.84

De hecho, es necesario otro supuesto para completar la teoría basada en la utilidad: el mercado donde se compran y venden bienes y servicios debe operar en competencia perfecta. Si el mercado está abierto a cualquier participante, el número de empresas que operan en el mercado es tal que ninguna empresa por ningún cambio en la producción dentro de su capacidad puede afectar el precio de mercado de un producto básico.

Por tanto, el modelo neoclásico se basa en dos conceptos principales: la maximización de la utilidad en cualquier transacción libre y una estructura de mercado basada en la competencia perfecta. Lo que está fuera de este modelo no le interesa al economista, es un problema de justicia ``distributiva'' que incide negativamente en la perfección teórica de un mercado de justicia ``conmutativa'', que es un mercado que asegura la maximización de la utilidad para las partes involucradas. en la transacción.

Podemos seguir atribuyendo a los economistas del cambio de siglo, la etiqueta general de ``liberales'', pero parece que las preocupaciones éticas se limitan al backstage; Los aspectos políticos e institucionales se dan por sentados y no merecen una mención especial. Como se mencionó anteriormente, este enfoque es consistente con un conservadurismo social implícito. La visión de la Ilustración centrada en el individuo racional permanece, pero el Hombre es más una máquina de calcular que una persona con preocupaciones morales y utilitarias. La cuestión de si puede haber una oportunidad justa para liberar al mundo de los dolores de la pobreza no puede ser respondida por la ciencia económica, aunque la respuesta dependa de hechos e inferencias que son difíciles de ignorar por el economista moral.

William S. Jevons comienza criticando la teoría anterior de la economía política, sistematizada por Ricardo, y completada en sus detalles por JS Mill: ``No había nada en las Leyes del Valor que quedara {[}para Mill{]} o cualquier otro escritor futuro para aclarar arriba''. 85 Este tipo de agotamiento surge de ``la importancia exclusiva atribuida en Inglaterra a la Escuela Ricardiana'', y ha llevado al ``actual estado caótico de la Economía'', 86 de modo que ``Muchos se alegrarían si la supuesta ciencia colapsara y se convirtiera en un cuestión de historia, como la astrología, la alquimia y las ciencias ocultas en general''. 87

El cambio de rumbo de Jevons se deriva de su opinión de que, si bien las opiniones predominantes hacen del trabajo en lugar de la utilidad el origen o la causa del valor \ldots, la reflexión y la investigación repetidas me han llevado \ldots{} a la opinión algo novedosa de que el valor depende enteramente de la utilidad. . 88 La utilidad puede medirse en términos cuantitativos, y ``como {[}la economía{]} se ocupa de las cantidades, debe ser una ciencia matemática en la materia, si no en el lenguaje''. 89 La economía se acerca a la mecánica estadística (véase más arriba la misma terminología utilizada por Comte).

De hecho, el utilitarismo no es tan novedoso como afirma Jevons, ya que él mismo cita a Bentham como su defensor más asertivo: ``No dudo en aceptar la teoría utilitarista de la moral \ldots{} la felicidad de la humanidad como criterio de lo que está bien o mal''. 90La novedad relevante del pensamiento de Jevons es que, dado el carácter cuantitativo de la cuestión observada, él piensa, como otros economistas neoclásicos, que la utilidad tiene solo un significado físico (placer por adquirir y dolor por evitar), de modo que se puede expresar en formas cuantitativas, más apropiadamente en formas matemáticas. Estrictamente conectado a la ``utilidad'' está el concepto de ``felicidad''. Dejemos la palabra al propio Jevons: ``el objetivo de la Economía es maximizar la felicidad comprando placer\ldots{} al menor costo del dolor\ldots{} He intentado tratar la Economía como un Cálculo de Placer y Dolor''. 91 Más tarde se observó, burlonamente, con referencia a la economía de Maffeo Pantaleoni, 92que en su obra ``el principio hedonista se discute como más apropiado para un libro de cocina o para un Kama Sutra que para la economía política''. 93 Jevons admite que hay cuestiones de la mayor importancia, como la seguridad de una nación, o el bienestar de grandes poblaciones, pero que ``no es mi propósito investigar aquí''. 94

Si pensamos en Alfred Marshall como el prototipo de esta nueva visión, la presencia del positivismo de Comte es bien visible. De manera similar a Keynes, quien, como veremos más adelante, extrae sus ``notas de filosofía social'' al final de su Teoría general , Marshall dedica el Apéndice C de su obra magna, Principles of Economics , 95al ``Alcance y método de la economía'', y comienza citando a Auguste Comte. Si bien se distancia de él al enfatizar que la economía debe mantener un ``papel distintivo'' de la sociología (``toda la gama de la acción del hombre en la sociedad es demasiado amplia y variada para ser analizada y explicada por un solo esfuerzo intelectual''), Marshall permanece en un modo positivo al enfatizar que las fuerzas económicas se combinan mecánicamente y que la economía es una rama de la biología interpretada de manera amplia. Y concluye:96

Alfred Marshall se centra en la ``demanda'' como el principal determinante del valor de cambio. La demanda de una mercancía está relacionada con su utilidad para el individuo que la compra, y la utilidad disminuye marginalmente a medida que aumenta la disponibilidad de la mercancía. Marshall, en el otro extremo, no ve el lado de la ``oferta'', es decir, el costo de producción, en particular los costos laborales, como el criterio del valor. Mientras que los clásicos observaron que existe un solo precio ``natural'' de una mercancía, esencialmente derivado de su costo, de modo que cualquier precio de ``mercado'' no puede ser otro que una divergencia temporal del natural, con Marshall cualquier distinción entre natural y de mercado el precio desaparece; el precio ---el valor de cambio--- de una mercancía está determinado por el cruce de las curvas de su oferta y demanda en el mercado.97 Marshall observa que las condiciones de la demanda tienen mayor importancia en la determinación del precio de un bien, particularmente en el período corto, cuando las condiciones de su oferta no pueden cambiarse; mientras que solo a largo plazo, cuando se pueden realizar más o menos inversiones, las condiciones de oferta tienen mayor relevancia, porque pueden ajustarse a cambios en la demanda: ``la influencia del costo de producción sobre el valor no se manifiesta claramente excepto en períodos relativamente largos''. 98

Como hemos subrayado anteriormente, el sistema económico neoclásico sólo puede funcionar sobre la base de la competencia perfecta, donde la formación de precios no se ve obstaculizada por obstáculos que limitan la oferta y la demanda de un bien. Marshall parece tocar el lado ético cuando dedica algo de espacio a la ``competencia''. Admite que se puede ver bajo diferentes perspectivas: como resultado del egoísmo, adquiriendo así un ``mal sabor''; o como resultado de la deliberación, que es tan esencial para el mantenimiento de la energía y la espontaneidad. Quiere considerar que el término no implica ninguna cualidad moral, sino que simplemente pone en evidencia el hecho indiscutible de que los negocios y la industria modernos se caracterizan por hábitos autosuficientes, previsión, deliberada y libre elección;99

Se observará de paso que fue en los lejanos Estados Unidos donde se introdujo la primera ley de disciplina de la competencia con la Sherman Antitrust Act de 1890, el año de publicación de los Principios de Marshall .

La transición al positivismo o la aceptación del positivismo por parte de los economistas neoclásicos se sufrió, en cierta medida: por un lado, su mismo rechazo del término ``economía política'' en favor de ``economía'', casi para implicar que el primero término indicaría una subordinación de la economía a la política, significa intentar enmarcar la disciplina como una ciencia positiva libre de juicios de cualquier otro tipo; En el otro extremo, particularmente con Alfred Marshall, los aspectos éticos, expulsados \hspace{0pt}\hspace{0pt}por la puerta, vuelven a entrar por la ventana a través del convencimiento de que el objetivo primordial del hombre es el impulso de una mejora constante del propio carácter y de las relaciones intrapersonales. . 100

\hypertarget{la-economuxeda-como-ciencia-pura-luxe9on-walras-y-vilfredo-pareto}{%
\section*{La economía como ciencia pura: Léon Walras y Vilfredo Pareto}\label{la-economuxeda-como-ciencia-pura-luxe9on-walras-y-vilfredo-pareto}}
\addcontentsline{toc}{section}{La economía como ciencia pura: Léon Walras y Vilfredo Pareto}

Léon Walras, investigando el problema de las raíces del ``valor'', ataca frontalmente la visión de los economistas clásicos (como la de Smith, Ricardo, Mc Culloch) según la cual el trabajo es el origen del valor, porque esta teoría no atribuye valor a las cosas que , de hecho, tienen valor; pero tampoco está satisfecho con la identificación de valor con utilidad, que es una definición demasiado amplia. El valor de un bien, y por tanto su precio, viene dado no solo por su utilidad sino también por su escasez ( rareté ). 101

Su teoría pura de la economía estudia el valor ---definido como antes--- en las relaciones de intercambio, y aborda el problema del ``valor en el intercambio'' como un fenómeno natural, sujeto a las ``leyes'' del intercambio. Por tanto, el método de las ciencias naturales es útil para la economía. Además, estos fenómenos son mensurables y la economía pura debería ser una rama de las matemáticas. La ciencia económica, que de todos modos debe mantenerse separada de la ciencia social (de acuerdo con Marshall), no tiene una connotación moral, no tiene un verdadero interés en la ética. 102

Walras perfecciona el marginalismo hasta el nivel de un equilibrio general del sistema económico, expresado matemáticamente como una visión sinóptica de las operaciones interdependientes del sistema en un régimen hipotético de competencia totalmente libre. Su complejo conjunto de ecuaciones es al mismo tiempo una descripción del sistema y una receta de cómo debe organizarse: como hemos visto, un enfoque no infrecuente del positivismo.

La definición de intercambio de Walras es totalmente coherente con la perspectiva utilitarista: ``El intercambio de dos mercancías entre sí en un mercado perfectamente competitivo es una operación mediante la cual todos los poseedores de una o de las dos mercancías, pueden obtener la mayor cantidad posible de beneficios. posible satisfacción de sus deseos de acuerdo con la condición de que los dos productos se compren y vendan al mismo tipo de cambio en todo el mercado''. 103 De esta manera, se logra un máximo relativo de utilidad social, con la condición de que el mercado donde se realiza el intercambio sea un mercado perfectamente competitivo, con un precio único para la mercancía en cuestión, y organizado de tal manera que no exista ningún impedimento. al libre flujo de compradores y vendedores.

A pesar de todo esto, se ha observado que el equilibrio estático de múltiples ecuaciones del modelo de Walras parece ``profundamente moralista, al menos en términos de la perspectiva moral individualista y burguesa característica de la cultura europea del siglo XIX''. El equilibrio de Walras es ``no solo una idea analítica, sino también una idea ética, que constituye un pilar indispensable de la justicia social''. 104 Si la justicia a cambio (una ``justicia conmutativa'') es la única forma de justicia que un economista puede concebir, y cualquier corrección a la distribución de la riqueza está fuera de sus límites, el proceso libre de determinación del precio único es la forma más eficiente de lograr justicia. Se puede ver cómo se puede estirar el concepto de ética.

Si el enfoque de Walras se considera correcto, se siguen pocos corolarios: los precios y las cantidades de bienes producidos en libre competencia y los precios uniformes son los mejores que se pueden obtener; Se descarta cualquier posibilidad de que un comerciante se beneficie del intercambio a expensas de su contraparte. 105

Por supuesto, sería posible que el comprador obtuviera una mayor utilidad a través de un precio más favorable y más bajo, y que el vendedor obtuviera de manera similar una mayor utilidad a través de un precio más alto, pero este resultado, por supuesto, requeriría múltiples precios. Según Walras, de esta manera se podría lograr un aumento efectivo de la utilidad social, pero debemos suponer que los vendedores ricos tendrán que renunciar a algunos lujos, mientras que los compradores pobres podrán pagar las necesidades: un problema que queda fuera de la economía pura. , ya que tiene que ver con la distribución de la riqueza y la ética social.

El tema de la distribución de la riqueza es, por tanto, relevante, pero pertenece a disciplinas que no coinciden con la economía. De los dos problemas fundamentales que tiene que afrontar cualquier doctrina económica, la producción de riqueza y su distribución, esta última tiene dificultades para abrirse camino a través del pensamiento neoclásico. Habrá que esperar la evolución de la disciplina económica, y las metamorfosis del liberalismo, más adelante en el siglo XX cuando nuevas instancias sociales traerán nuevamente a primer plano el ``tema distributivo''.

Pareto es un ejemplo destacado de positivismo aplicado a la economía. Al igual que Jevons, su crítica de la Escuela Clásica --- nada más que un ``género de literatura'' - carece de atractivo: ``Estos economistas literarios, aunque han compuesto obras de gran valor, hasta ahora no han podido persuadir a la mayoría de sus lectores y, lejos de ganar terreno, lo pierden día a día. Con la excepción de Inglaterra, el reino del libre comercio principalmente porque es del interés de ciertos empresarios, el resto de países civilizados se inclina cada vez más hacia el proteccionismo. El socialismo de Estado y el socialismo en general avanzan día a día. Quizás la ciencia económica sea prácticamente tan inútil como la economía política literaria: en realidad no puede ser más inútil que eso, y merece, al menos, el mérito de comprender las verdaderas causas de los fenómenos''.106

Su positivismo descansa sobre tres pilares: analogía entre ciencias sociales y ciencias naturales; definición de ``valor'' solo en términos relativos, lo que significa que el valor de algo solo puede definirse en relación con otras cosas, un punto ya enfatizado por Comte cuando habla de la esencia de los ``hechos'' (ver arriba); ausencia de consideraciones metaeconómicas ---o, para usar sus palabras, ideologías--- en su Weltanshauung . A pesar de esta ausencia, lo que surge de sus reflexiones es un determinismo implícito, casi panglosiano, que termina como una forma de conservadurismo: la adopción de un enfoque ``científico'', lejos de significar neutralidad en la filosofía económica, esconde fuertes implicaciones políticas.

El positivismo de Pareto está claramente definido en su Corso di economia politica , donde afirma que la dependencia recíproca de los fenómenos económicos y el equilibrio general de un sistema económico presentan analogías sorprendentes con el equilibrio de un sistema mecánico. 107 Como la mecánica racional se dedica al estudio, en abstracto, del equilibrio de fuerzas y su movimiento (mientras que la mecánica aplicada, acercándose a la realidad, estudia el mismo objeto pero en determinadas condiciones concretas: de ahí las ciencias físico-químicas), la pura política La economía se dedica al estudio, en abstracto, del homo oeconomicus, una entidad que actúa solo por motivaciones utilitarias: sobre una base de ofelimidad, como dice Pareto (mientras que la economía política aplicada se dedica a seres que se aproximan al hombre real, actuando bajo diferentes motivaciones). Pero, lo que es más importante, sería un error suponer que el hombre real puede escapar a las leyes de la economía pura.

En cuanto a la utilidad, como base sobre la que actúa el homo oeconomicus , Pareto es positivista cuando va incluso más allá de otros pensadores neoclásicos, atacando el concepto neoclásico de utilidad marginal. Walras se inclinó hacia el concepto de rarité (ver arriba); Pareto se basa en el concepto ``ordinal'' de ofelimidad: coherentemente con la filosofía de Comte, cualquier objeto puede ser valorado no per se , atribuyéndole un número cardinal absoluto (el valor de esta pluma es, para mí, 3; y este valor de silla es, para mí, 5), pero solo en relación con los demás, es decir de manera ordinal (esta silla es para mí más valiosa que esta pluma, y \hspace{0pt}\hspace{0pt}no se puede agregar nada más). 108 Siguiendo este razonamiento, y mediante índices ordinales de ofelimidad, construye ``curvas de indiferencia'' que conducen al equilibrio económico general.

Mientras que otros economistas neoclásicos prefirieron evitar el tema de la distribución de la riqueza, como ``no científico'', y mientras Walras identificó el equilibrio estático general de su sistema como asegurando la ``justicia conmutativa'', si no la ``justicia distributiva''. Pareto aborda más abiertamente el tema de la distribución, que ganará cada vez más peso en el pensamiento económico, durante el siglo XX, con respecto a la producción de riqueza. 109Su ``enfoque científico'' parte de una observación estadística inductiva: la distribución de la riqueza no cambia sustancialmente, independientemente de las diferentes regiones, períodos de tiempo, organizaciones, incluso teniendo en cuenta factores desconocidos (peligro) que pueden influir en la distribución de la riqueza en ambos sentidos. En términos matemáticos, si en un sistema de ejes cartesianos informamos sobre los niveles de ingreso en abscisas, y en ordenadas el número de personas cuyos ingresos exceden un cierto nivel (Pareto usa escalas logarítmicas), y si se dibuja una curva, es una línea recta, y esta línea tiene, para todos los países interesados, la misma inclinación hacia las abscisas, de alrededor de 56 grados. La inferencia de esta observación es que, a pesar de cualquier peligro, esta distribución constante de la riqueza depende de la naturaleza del hombre.110 Esta drástica afirmación está perfectamente en línea con el enfoque científico (positivista): las ciencias sociales en la cima de las ciencias naturales.

La inferencia de Pareto a partir de esta ``ley'' es que ``a largo plazo ---como norma y en promedio--- {[}una disminución de la desigualdad de ingresos{]} es imposible \ldots{} Para obtener, en promedio, una disminución de la desigualdad de ingresos de forma general, permanente así, es absolutamente necesario un aumento de los ingresos totales, en relación a la población''. 111

Unos años más tarde, sin embargo, Pareto se volvería menos rígido: ``estas conclusiones no pueden extenderse más allá de esos límites {[}basados \hspace{0pt}\hspace{0pt}en datos del siglo XIX, con respecto a poblaciones civilizadas{]}. Es sólo una inferencia más o menos probable de que, en otras épocas y poblaciones, quizás podamos observar formas más o menos parecidas a la que hemos encontrado''. 112

La inevitabilidad de tal distribución del ingreso, con el determinismo que implica, ha sido criticada desde varios sectores. Pigou observa: (1) que incluso una pequeña diferencia en la inclinación del ángulo puede tener consecuencias importantes en términos de distribución del ingreso; y (2) Que con el tiempo la pendiente de la curva ha disminuido (como en Prusia), con una mayor igualdad en la distribución del ingreso. ``Construir sobre {[}las comparaciones de Pareto{]} cualquier ley cuantitativa precisa de distribución es claramente injustificable''. 113

Einaudi, alrededor de medio siglo después, escribirá que ``la norma constante de distribución de la riqueza solo es válida dentro de sociedades donde hay una falta de instituciones que estén conscientemente dispuestas a cambiar esa distribución''. 114

Y Schumpeter volvió dos veces a esta ``ley'': en Diez grandes economistas , observó que ``Dado que hasta tiempos muy recientes la distribución de los ingresos por paréntesis se ha mantenido notablemente estable, ¿qué podemos inferir de esto? Este problema nunca ha sido atacado con éxito''. 115Y luego, con bastante sentido común, escribió: ``Independientemente de lo que se pueda pensar de las medidas estadísticas diseñadas para {[}la distribución del ingreso{]}, esto es cierto: que la estructura de la pirámide de ingresos, expresada en términos de dinero, no ha cambiado mucho durante el período cubierto {[}Reino Unido en el siglo XIX, en su caso{]}, y que la proporción relativa de sueldos más sueldos también ha sido sustancialmente constante a lo largo del tiempo \ldots{} La medida de distribución de ingresos (o de desigualdad de ingresos) ideada de Vilfredo Pareto está abierto a objeciones. Pero el hecho en sí mismo es independiente de sus defectos que él asumió''. 116

Más recientemente, Picketty escribe que ``el juicio de Pareto estaba claramente influenciado por prejuicios políticos: era sobre todo cauteloso con los socialistas y lo que él consideraba sus ilusiones redistributivas \ldots{} El caso de Pareto es interesante porque ilustra la poderosa ilusión de la estabilidad eterna, a la que el El uso acrítico de las matemáticas a las ciencias sociales conduce a veces''. 117

Podemos agregar que sería más sencillo mencionar la distribución invariable del ingreso como una evidencia histórica, un caso específico, y no llamarlo una ``ley''. 118

Más allá de cualquier determinismo en la distribución del ingreso, como se ha descrito hasta ahora, Pareto agrega que cualquier intento de alcanzar una distribución diferente sería ineficaz en términos de bienestar total. El bienestar colectivo aumenta, según Pareto, solo si se puede mejorar la situación de alguien sin empeorar la situación de nadie.

La ``optimización'' de Pareto no ayuda a elegir entre diferentes asignaciones óptimas de Pareto en las que las distribuciones de ingresos son diferentes. Para hacer esta elección, necesitamos algunos principios que son juicios de valor, que no se pueden deducir del conocimiento objetivo sobre la ``naturaleza'' del mundo.

La ``eficiencia de Pareto'' u ``optimalidad'' representa, de hecho, la extensión al bienestar económico social del concepto de ofelimidad. Si el ingreso total se mantiene sin cambios, la pérdida que sufren los altos ingresos debido a la redistribución del ingreso es mayor que la ganancia obtenida por los de bajos ingresos. Pareto da el ejemplo de Prusia: ``Si los ingresos superiores a 4.800 marcos se redujeran a esa cantidad, y la diferencia se repartiera entre quienes reciben un ingreso inferior a 4.800 marcos, cada uno de ellos no recibiría nada más que unos cien marcos''. Ninguna acción de política pública puede ser una mejora de Pareto. ``El objetivo de lograr un óptimo de Pareto es intrínsecamente muy conservador \ldots{} {[}Desvía{]} la atención de la cuestión de si la distribución actual de la riqueza es tan desigual que debería cambiarse''. 119

Su conclusión de filosofía política es: ``El socialismo de Estado es más útil para los políticos, pero sus consecuencias económicas consisten en un derroche de riqueza y, de tal forma, empeoran, más que mejoran, las condiciones de las personas''. 120 Cuanto más se concentran los ingresos, mayor es la pérdida para los que tienen altos ingresos que la ganancia para los de bajos ingresos. La optimalidad de Pareto termina siendo una visión profundamente antiliberal. 121

Desde el punto de vista de Pareto, esta conclusión no solo sería aceptable, sino la única correcta: sería la conclusión ``científica''. De hecho, en una ciencia que ha alcanzado su etapa positiva, una teoría sólo puede formularse sobre la base de lo que puede demostrarse lógica y experimentalmente. Si consideramos verdadero lo que está de acuerdo con el sentimiento de uno, estamos fuera de la ciencia y entramos en el campo de la falsa teoría o ideología. La ideología es como un programa ético-político disfrazado de teoría científico-filosófica, como un juicio de valor transformado en un enunciado fáctico. La ideología puede ser eficaz si responde a los propios propósitos; o útil, si responde a determinadas necesidades sociales. Pero la verdad, la eficacia y la utilidad no se pueden mezclar ni entrelazar recíprocamente, porque solo la primera se basa en la lógica y la experimentación, mientras que los otros dos se basan en opiniones religiosas o metafísicas. Una teoría ``verdadera'' desafía cualquier juicio de eficacia y utilidad.122

Hemos mencionado anteriormente el encanto ambiguo de la teoría del valor trabajo sobre Karl Marx. Pero como hemos hecho al conectar la economía clásica y, en cierta medida, la neoclásica con la Ilustración, no podemos pasar a Marx sin partir del historicismo, de la centralidad del Estado y de las ideas económicas que, más o menos explícitamente, derivan de esta Weltanschauung.

\hypertarget{historicismo-nacionalismo-econuxf3mico-y-socialismo-marxista}{%
\section*{Historicismo: nacionalismo económico y socialismo marxista}\label{historicismo-nacionalismo-econuxf3mico-y-socialismo-marxista}}
\addcontentsline{toc}{section}{Historicismo: nacionalismo económico y socialismo marxista}

Ahora podemos volver a la primera sección de este capítulo y pasar de la corriente de pensamiento que ve al individuo racional en el centro de la atención del economista, a la otra línea filosófica que ve a la sociedad y, para ella, al Estado. ---Como personificación del orden racional.

La perspectiva del razonamiento económico es, de hecho, muy diferente si su filosofía se basa en la asunción de la racionalidad del Estado. La visión teórica más completa de la idea del Estado como máxima expresión de la racionalidad se puede encontrar en Georg Hegel. La idea de Estado tiene sus raíces en la historia. A través de su evolución histórica, el Estado encarna progresivamente la idea de libertad. Según Hegel, la libertad se realiza objetiva y positivamente solo por el Estado: es a través del Estado que el individuo disfruta de su libertad. La voluntad arbitraria y subjetiva del soltero no es, de hecho, Libertad. Todo lo que es el hombre, se lo pertenece al Estado: sólo en el Estado el individuo individual encuentra la razón de su existencia. La racionalidad del proceso histórico, que se desarrolla en forma dialéctica de tesis, antítesis y síntesis: está dirigido teleológicamente a la plena puesta en práctica del concepto de libertad. Hay una especie de astucia de la razón, que progresiva, providencialmente, trabaja a lo largo de la Historia: no debemos mirar las acciones y los hechos simplemente como aparecieron para quienes fueron sus protagonistas y, como tales, vinculados a sus intereses y pasiones particulares: Hegel degrada las motivaciones personales a meros accidentes de un proceso esencial y necesario.123

Hegel ayudó a establecer el Estado moderno como un objeto privilegiado de investigación y reflexión. No es solo el lugar de la soberanía y el poder; es el motor que hace la historia, o incluso la encarnación de la historia misma. Este tipo de ``idealismo estatista'' habría estado bien en la mente de los políticos alemanes, así como de los historiadores y economistas: Bismarck escribió en 1882: ``La rotación de los individuos es irrelevante \ldots{} El Estado y sus instituciones sólo son posibles si se los imagina como personalidades idénticas permanentes''. 124

La visión hegeliana es, por tanto, determinista, porque no admite desviaciones de un camino que conducirá, finalmente, a arreglos políticos y administrativos que realizarán plenamente esa libertad. ``La Historia del Mundo, con todos los escenarios cambiantes que presentan sus anales, es este proceso de desarrollo y realización del Espíritu; esta es la verdadera Teodicea, la justificación de Dios en la Historia''.

Es notable la distancia entre la visión de la Ilustración, que pone en su centro la racionalidad del individuo, y el historicismo, que ve al Estado en su evolución como la encarnación de la racionalidad. Esta sistematización determinista de la idea de Estado deja de lado el aparente caos de los derechos democráticos del individuo y favorece el principio abrumador según el cual la libertad no puede existir sin la organización del Estado. En el Estado todos los componentes del cuerpo político están conectados, y sólo dentro del Estado tiene sentido la libertad de que disfruta el individuo. El Estado es la culminación del individuo como entidad acabada. La idea abstracta y ahistórica del hombre libre es producto de la Ilustración, que surgió mucho después del Estado.

Se ha debatido si se puede considerar a Hegel como un economista político y si se puede inferir una doctrina económica de sus escritos. 125 No es nuestra intención abordar este tema, pero sería difícil negar que la centralidad del papel del Estado a lo largo de su evolución histórica tuvo un peso considerable en quienes sintieron la influencia del pensamiento de Hegel, también en el campo de la economía política. Dentro de la disciplina económica, seguir un enfoque hegeliano significó la adopción de una perspectiva bastante alejada de la seguida por los economistas clásicos. Además, la adopción de una perspectiva historicista acaba por considerar la doctrina clásica de la economía, paradójicamente, como historicista en sí misma, ``bajo la apariencia de sus abstracciones y lenguaje matemático''. 126Desde este punto de vista, el ``mercado'', punto de referencia constante para los economistas clásicos, lejos de responder a un esquema lógico-deductivo (propio de la Ilustración), es el resultado de un largo proceso histórico, es decir de las leyes. de una sociedad capitalista, tal como surgieron históricamente. Considerar el mercado como una entidad desvinculada del proceso histórico puede deberse al hecho de que, para usar la palabra de Marx, el economista ``vulgar'' no siempre es consciente de la visión filosófica que hay detrás de su propio pensamiento. Por ejemplo, ``Ricardo nunca reflexionó históricamente sobre su propio pensamiento \ldots{} Nunca toma una perspectiva histórica \ldots{} y ve como naturales e inmutables las leyes de la sociedad en la que vive \ldots{} Ricardo no era un utilitarista, no porque tuviera otra filosofía, sino porque no tenía''. 127

Podemos ver cómo la sistematización hegeliana está en el origen tanto del nacionalismo económico como del socialismo marxista. Ambos se centran en la centralidad del Estado y en el papel de la investigación histórica como necesaria para comprender sus estructuras económicas, así como sociales y políticas (una investigación que a menudo toma en los hegelianos un giro determinista); ambos desprecian (ninguna otra palabra es apropiada) el ``cosmopolitismo'' (globalismo, en el lenguaje moderno) de la Ilustración y las teorías económicas clásicas. Ambos comparten el carácter teleológico de la historia, que se encamina ---como fin último--- hacia la armonía de todas las naciones, la libertad de comercio entre iguales y la paz universal (en opinión de los nacionalistas), y hacia la emancipación del hombre (en opinión de los socialistas). Y ambos, una circunstancia que no es teóricamente relevante,

Sin duda, debemos evitar cualquier confusión de premisas filosóficas y razonamiento económico. El nacionalismo económico y el socialismo marxista no pueden compararse con el pensamiento filosófico hegeliano, del mismo modo que la escuela clásica de economía no puede identificarse íntegramente con la filosofía de la Ilustración. El problema económico al que se enfrentan las tres corrientes de pensamiento es, en el primer caso, la explicación del funcionamiento de un mercado libre (y, como subproducto, el surgimiento de Inglaterra como potencia hegemónica); en el segundo caso, el análisis histórico del crecimiento económico (teniendo especialmente en cuenta el atraso de Alemania); en el tercer caso, la investigación sobre la condición subordinada de la clase obrera, sólo removible volcando el orden social existente.

La Escuela Histórica Alemana está profundamente impregnada de nacionalismo económico; y de particular interés ---se cita con frecuencia en los debates actuales sobre el resurgimiento del nacionalismo, o si preferimos el ``soberanismo'' - es la figura de Friedrich List. La historia económica de las naciones, desde su crecimiento hasta su decadencia, que está sustancialmente ausente en el trabajo de los economistas clásicos, ocupa el escenario central en el análisis de List. No es difícil ver la huella de Hegel (aunque List nunca cita a Hegel, al menos en su Sistema Nacional de Economía Política) en varios aspectos de su obra: en el papel central del Estado, que es el principal motivo de oposición a los economistas británicos; en el análisis histórico que impregna su investigación; en el reconocimiento de que el objetivo último de todas las naciones es su unión, la paz perpetua y la libertad universal de comercio. 128

``Friedrich List - escribió el economista e historiador italiano Marcello De Cecco - es el opuesto intelectual de Smith y Ricardo. Estos últimos intentan establecer la economía política como un ejercicio de lógica, un estudio de la consistencia interna de los sistemas lógicos formulados de manera abstracta; el primero intenta sumergirse en la realidad de la historia económica y extraer de ella las lecciones más importantes. Su trabajo, mucho más que el de Smith, es una investigación sobre las causas reales de la riqueza de las naciones. Para él, la economía es una de las artes del arte de gobernar \ldots{} {[}List{]} es un académico que entiende que el libre comercio no es una verdad revelada, sino solo una forma de política económica \ldots{} Si {[}otros{]} países quisieran modernizar sus economías, convertirse en tan políticamente poderoso como Gran Bretaña, pensó que deberían combinar proteccionismo y corporativismo,129

Aquí nuevamente notamos dos polaridades en el pensamiento económico; como observa Lunghini: ``Walras {[}el economista neoclásico{]} muestra un marco teórico que, por primera vez en la historia de la ciencia económica, abarca toda la estructura lógica de la interdependencia de las cantidades económicas. Toda la concepción y la técnica son rigurosamente estadísticas {[}pero{]} son \hspace{0pt}\hspace{0pt}válidas solo para un estado estacionario. {[}La otra polaridad es{]} la teoría del desarrollo económico, que rechaza la idea de que solo las externalidades pueden explicar el cambio de un sistema económico de un equilibrio a otro''. 130

Uno puede preguntarse si la evaluación de De Cecco, de una irrelevancia sustancial de las ideas de List en la disciplina de la economía, es correcta o debería ser algo matizada. En este sentido, parece oportuno distinguir el papel de List en la evolución de la economía dominante, y su papel en influir en las políticas económicas y comerciales de países importantes y, en primer lugar, de Alemania. Desde el primer punto de vista, encontramos en la obra de List ---y de otros economistas, particularmente de nacionalidad alemana, que vinieron después de List--- ese aspecto tan agudamente observado por Schumpeter: la dificultad de conciliar una visión centrada en el Estado y en el análisis histórico con la construcción de modelos teóricos basados \hspace{0pt}\hspace{0pt}en la racionalidad del comportamiento individual. La Escuela Histórica Alemana nunca pudo ---ni quiso--- apuntar a esa perfección teórica abstracta que combina, en una lógica aparentemente rigurosa, todos los factores que influyen en cualquier transacción económica en una determinada sociedad. Esta Escuela carecía de una teoría comprensiva coherente, y esto fue un obstáculo para quienes quieren encontrar una explicación racional de todo. Por lo tanto, estaba en desventaja con respecto a la economía clásica o incluso a la economía marxista. La visión opuesta, la clásica, o mejor, la visión neoclásica del libre mercado, claramente prevaleció, se convirtió en la corriente principal, dejando poco espacio para aquellos que querían seguir rutas alternativas de investigación. Desde esta perspectiva, el comentario de De Cecco es absolutamente apropiado. Esta Escuela carecía de una teoría comprensiva coherente, y esto fue un obstáculo para quienes quieren encontrar una explicación racional de todo. Por lo tanto, estaba en desventaja con respecto a la economía clásica o incluso a la economía marxista. La visión opuesta, la clásica, o mejor, la visión neoclásica del libre mercado, claramente prevaleció, se convirtió en la corriente principal, dejando poco espacio para aquellos que querían seguir rutas alternativas de investigación. Desde esta perspectiva, el comentario de De Cecco es absolutamente apropiado. Esta Escuela carecía de una teoría comprensiva coherente, y esto fue un obstáculo para quienes quieren encontrar una explicación racional de todo. Por lo tanto, estaba en desventaja con respecto a la economía clásica o incluso a la economía marxista. La visión opuesta, la clásica, o mejor, la visión neoclásica del libre mercado, claramente prevaleció, se convirtió en la corriente principal, dejando poco espacio para aquellos que querían seguir rutas alternativas de investigación. Desde esta perspectiva, el comentario de De Cecco es absolutamente apropiado. dejando un pequeño espacio para aquellos que querían seguir rutas alternativas de investigación. Desde esta perspectiva, el comentario de De Cecco es absolutamente apropiado. dejando un pequeño espacio para aquellos que querían seguir rutas alternativas de investigación. Desde esta perspectiva, el comentario de De Cecco es absolutamente apropiado.

Más adelante, mencionaremos la influencia de List y la Escuela Histórica en las políticas económicas de diversos países, y nuestras conclusiones serán algo diferentes. Pero para ver la importancia de List, y del socialismo marxista, en el discurso político y económico del siglo XIX y más allá, es necesario mencionar la situación sociopolítica de Alemania hacia mediados de ese siglo.

\hypertarget{alemania-y-gran-bretauxf1a-en-el-siglo-xix}{%
\section*{Alemania y Gran Bretaña en el siglo XIX}\label{alemania-y-gran-bretauxf1a-en-el-siglo-xix}}
\addcontentsline{toc}{section}{Alemania y Gran Bretaña en el siglo XIX}

Alemania tiene un papel primordial en los análisis tanto de la doctrina socialista de Engels y Marx como de las teorías de List y de la Escuela Histórica. Y una gran relevancia tiene también el papel de Gran Bretaña, la potencia hegemónica del siglo XIX, con la que se compara enfáticamente a Alemania.

La principal diferencia entre Alemania y Gran Bretaña, una diferencia a menudo enfatizada en los escritos de List, particularmente cuando critica a Adam Smith y su ``escuela'', a la que llama " escuela cosmopolita ( Kosmopolitische )", se puede encontrar en el hecho de que Alemania fue aún en busca de su propia identidad nacional, a pesar de que Prusia apareció como la principal fuerza impulsora hacia la unidad, mientras que Gran Bretaña, para entonces bien consolidada como una entidad política única, había sido capaz de crear un entorno propicio para una burguesía individualista e ingeniosa, y a la revolución industrial entrante.

Vale la pena recordar cómo la burguesía británica miraba al pueblo alemán hacia mediados del siglo XIX: ``Gran Bretaña estaba acostumbrada a ver a Alemania como una especie de pariente pobre, atrasado y bastante cómico \ldots{} Henry Mayhew, el cofundador de Punch, solo podía esperar hacer la pobreza y el atraso de Alemania en general imaginables para sus lectores comparándola con el más irremediablemente miserable de los países, Irlanda \ldots{} Karl Marx, él mismo un solicitante de asilo alemán en Londres, {[}escribió sobre la importancia de los alemanes trabajadores{]}: el propósito de esta importación es el mismo que el de la importación de culis indios a Jamaica, es decir, la perpetuación de la esclavitud \ldots{} nadie sufriría más que los propios trabajadores alemanes, que constituyen en Gran Bretaña un número mayor que los trabajadores de todas las demás naciones continentales. Y los trabajadores recién importados,131

Más allá de las observaciones específicas, la evidencia estadística parece indicar que, en 1820, el producto nacional de Gran Bretaña era un 38\% más alto que el de Alemania (incluso siendo difícil comparar los dos países, Alemania todavía estaba fragmentada en estados más independientes, de muy diferente tamaño, población y economía). Más aún, sobre una base per cápita, el producto de Gran Bretaña era más de un 60\% más grande que el de Alemania. 132 Pero la educación científica y técnica proporcionada por los gobiernos alemanes con visión de futuro llevó, en Gran Bretaña, a la ansiedad y el interés en ese país, que hasta 1870, el año de la victoria alemana en la guerra con Francia, había sido ignorado en gran medida. 133Cuando estalló la Primera Guerra Mundial, el PIB alemán ya había alcanzado y superado al británico. Esta dinámica habla en voz alta de la ambición alemana de ser el primero en Europa, y uno puede preguntarse si List entendió bien de antemano los términos del tema en juego y sugirió la política económica adecuada.

Por lo tanto, es útil comparar los enfoques teóricos de List y Smith, que están influenciados por condiciones sociales muy diversas. Smith mira a una sociedad bien consolidada en su red existente de relaciones económicas y sociales. Explica a sus lectores cómo funciona esta sociedad; y si los componentes individuales de esta sociedad pueden explotar su potencial sin ser obstaculizados por obstáculos de ningún tipo, no se necesitan cambios importantes en la estructura social y económica y el cuerpo político.

El estado se limita, en la obra de Adam Smith, a un papel bastante secundario, está casi ausente salvo en circunstancias particulares; pero su presencia no es necesaria. En Smith hay un optimismo fundamental que ciertamente es de naturaleza ilustrada, pero al mismo tiempo es diferente del francés. Egalitè no es un tema que merezca una atención abrumadora, y quizás el gran crítico de la revolución francesa, el conservador Edmund Burke, 134 habría estado de acuerdo con la visión de Smith de lo que debería ser la Ilustración.

El trabajo de List nos lleva a la introducción del historicismo pleno en la investigación del economista, y es necesario situar su enfoque en el contexto político y social en la base de su libro más importante: El sistema nacional de economía política . 135 Parece apropiado dedicar las siguientes dos secciones a una sinopsis de los principales conceptos del libro, teniendo en cuenta lo lejos que están de la corriente principal de la economía actual y, al mismo tiempo, lo cerca que están de los amplios debates económicos y políticos de hoy.

\hypertarget{la-economuxeda-poluxedtica-como-sistema-de-economuxeda-nacional}{%
\section*{La economía política como sistema de ``economía nacional''}\label{la-economuxeda-poluxedtica-como-sistema-de-economuxeda-nacional}}
\addcontentsline{toc}{section}{La economía política como sistema de ``economía nacional''}

Es útil comenzar diciendo que List nació intelectualmente como liberal, y que su enfoque nacionalista tiene su origen en su observación de las políticas económicas del país del que fue invitado temporal: Estados Unidos. 136 Allí, Alexander Hamilton, el primer secretario del Tesoro de la Confederación, siguió una política intervencionista y protectora para deshacerse del antiguo país de origen, Gran Bretaña. De ahí la atención de List a los pequeños estados alemanes pertenecientes a la Zollverein (unión aduanera), cuya finalización fue fuertemente defendida por él; y, de ahí, su esperanza de que se introdujeran derechos de protección para estimular tanto el crecimiento como el libre mercado dentro del Zollverein. 137

List es crítico con Smith, porque no reconoce que otros ``sistemas de economía política'', basados \hspace{0pt}\hspace{0pt}en el concepto de ``nación'', pueden reemplazar la eficiencia de una economía globalizada. Según Smith, escribe, la mayoría de las regulaciones gubernamentales para promover el bienestar son innecesarias. La visión de Smith significa que la nación no es más que una invención léxica que solo existe en la mente de los políticos. La ``economía nacional'' es, en cambio, según List, la ciencia que, al comprender correctamente los intereses y las circunstancias que involucran concretamente a naciones específicas en un momento determinado, enseña cómo una sola nación puede llegar a ese estado de desarrollo industrial que, una vez alcanzada, también permitirá la unión con otras naciones en igualdad de condiciones, mientras que la libertad de comercio internacional será posible y útil como consecuencia.

El interés nacional británico es que, habiendo alcanzado Gran Bretaña, como nación bien diferenciada e independiente, un alto grado de desarrollo industrial, la libertad de comercio es para ellos una oportunidad. El excedente de capital que tienen disponible los impulsa a exportar sus leyes y actividades económicas a países lejanos. Toda Inglaterra se está convirtiendo en una inmensa ciudad industrial. Gran Bretaña está civilizando a Asia, África y Australia, y se crean nuevos estados siguiendo el modelo británico. Los franceses, españoles, portugueses son ``razas improductivas'' y acaban por dar sus mejores vinos a los británicos, guardándose lo peor para ellos {[}la teoría ricardiana de la ventaja comparativa viene a la mente{]}. En el tipo óptimo de arreglo, según los británicos, Francia mantendría algunas fábricas, Alemania exportaría a Gran Bretaña juguetes, relojes de cuco, escritos filosóficos y quizás algunas tropas para ser asesinadas en los desiertos de Asia o África. Entonces es necesario que las naciones menos desarrolladas surjan ``artificialmente'' {[}es decir, por sus propios esfuerzos, para ser protegidas adecuadamente{]}, al nivel que ya ha alcanzado Gran Bretaña.138

La doctrina de Smith ---escribe List--- se hunde en el materialismo, el particularismo, el individualismo. La idea smithiana del ``valor de cambio'' de una mercancía debe ser reemplazada por el concepto de ``capacidad productiva''. En sentido amplio, el gasto en educación, promoción de la justicia, defensa de la nación, contribuye a esa ``capacidad'', al formar el ``capital mental del género humano''. La ``escuela popular'' smithiana nos hace creer que el Estado, el organismo público, no debe ser tomado en consideración por la economía política. Según esa escuela, el hombre que cría cerdos es un miembro productivo de la comunidad, pero el que educa a los hombres es un mero improductivo. 139La nación debe sacrificar parte de su riqueza material con el fin de obtener ganancias en cultura, capacidad profesional y habilidades organizativas. Pero una nueva potencia industrial no puede surgir si no está protegida. Si una pérdida de valor se deriva de los derechos de protección, esta pérdida se compensará con una mayor capacidad productiva que proporcionará una gran cantidad de bienes materiales e independencia en caso de guerra. 140

El ejemplo, dado por Smith, de la fábrica de alfileres como modelo de división del trabajo, tiene, según List, el significado opuesto: no es una división, sino una unión de energías, inteligencias y capacidades para un objetivo común de producción. . La razón para trabajar juntos en esa fábrica no es la división del trabajo, sino su cooperación y unidad. Y esto es cierto no solo para una sola fábrica, sino para toda la potencia industrial y agrícola y para toda la economía de la nación. La industria y la agricultura deben unirse en una sola confederación, bajo la égida del Estado. 141 {[}Esta mención de los intereses compartidos de todos los que participan en el proceso de producción (del capital y del trabajo), un interés sometido al interés superior de la nación, parece una anticipación de la doctrina corporativa, que es atractiva para los regímenes autoritarios y también presente , no por casualidad, en algunos debates actuales: un punto sobre el que volveremos más adelante{]}.

Si el interés individual debe estar sujeto al interés de la nación, no hay lugar para políticas de laissez-faire, laissez-passer (una expresión -observa List- que no suena menos agradable a los ladrones, estafadores y ladrones que al comerciante , y por eso es dudoso que pueda adoptarse como máxima), 142 y por la idea de que la política debe mantenerse alejada del ámbito económico. 143

El interés nacional se mantiene entre los intereses del individuo y de la humanidad entera, y la nación existe en oposición a otras naciones que tienen el mismo grado de libertad. Sin embargo, la nación no puede ser pequeña, debe crearse a través de alianzas, como ha sido el caso de Gran Bretaña o Estados Unidos, o el Zollverein alemán., la unión aduanera. El proteccionismo, más o menos intenso según las diversas industrias, se adopta correctamente si el desarrollo económico de un país se ve obstaculizado por presiones competitivas provenientes de países más avanzados. Con respecto a Alemania, List observa que protegerá su propia industria, y esta protección solo se reducirá cuando el país haya alcanzado un nivel de crecimiento que le permita hacer frente a la competencia externa, aunque sea cuidadosamente contenida. List agrega con desdén que la teoría de la libertad de comercio internacional es correcta sólo para aquellos países, como Portugal, {[}El Reino de{]} Nápoles, Turquía y otros ``bárbaros y medio civilizados'' {[}!{]} - que son lo suficientemente ``tontos'' como para no perseguir la industria desarrollo a través de un nivel adecuado de protección.

La economía de un determinado pueblo se vuelve coincidente con la de una nación cuando el Estado abarca a toda la nación, y el grado de independencia de una nación puede medirse sobre la base del tamaño de su población, territorio, riqueza y poder, la relevancia de sus instituciones y el nivel de su civilización. Solo así se puede establecer una nación estable y políticamente influyente.

\hypertarget{lista-proteccionistas-mercantilistas-fisiuxf3cratas-y-la-idea-de-europa}{%
\section*{Lista: proteccionistas, mercantilistas, fisiócratas y la idea de Europa}\label{lista-proteccionistas-mercantilistas-fisiuxf3cratas-y-la-idea-de-europa}}
\addcontentsline{toc}{section}{Lista: proteccionistas, mercantilistas, fisiócratas y la idea de Europa}

Si dejamos de lado el lenguaje a veces intemperante de List (típico, sin embargo, de su propio ímpetu político ), su ``sistema de economía política'' se basa en un análisis histórico bien elaborado de las políticas y doctrinas económicas, que parece estar en armonía con el muy posterior enfoque seguido por Schumpeter, mencionado al comienzo de este ensayo.

List piensa que el proteccionismo es una necesidad, pero también está condicionado a circunstancias específicas determinadas históricamente. Y aquí notamos una Lista ``liberal''. La protección solo se justifica para aumentar la actividad manufacturera de una nación, y solo hasta el punto en que la nación, gracias a un territorio extendido, una gran población, recursos naturales importantes, agricultura avanzada e instituciones políticas bien desarrolladas, pueda competir en al mismo nivel que otros países desarrollados. Por tanto, el proteccionismo es fundamental. Puede consistir en cuotas o derechos de importación. 144

Todo el sistema listiano está conectado con la doctrina mercantilista, y contrasta obviamente con la fisiocrática, que, a su vez, tuvo una influencia notable en Adam Smith. El crecimiento de las grandes monarquías europeas estimuló la producción y el comercio nacionales, gracias a los aranceles sobre los bienes importados, y este desarrollo de una industria nacional fue acompañado por la consolidación de las libertades nacionales. Se reforzaron las instituciones políticas, se incrementó la recaudación de impuestos, así como la población y el poder militar. List dice que este modelo de crecimiento ---el ``sistema industrial'' - fue teorizado para Inglaterra por James Steuart y para Venecia por Antonio Serra, pero encontró su promulgación más completa en Francia, con JB Colbert, el ministro de finanzas de Luis XIV. List se queja, sin embargo, de que el mercantilismo ignoró el lado internacional,145

Los capítulos finales de El Sistema Nacional están dedicados a la cuestión europea. Lo que la gente llama ``el mantenimiento del equilibrio de poder europeo {[}una referencia implícita a Gran Bretaña{]} no tiene nada que ver con los intentos de las naciones menos poderosas de imponer un freno a la invasión de los más poderosos''. Si consideramos el interés abrumador que las naciones continentales tienen en común, el de oponerse a la supremacía marítima de los británicos, estaremos convencidos de que nada es tan necesario para estas naciones como una unión, y nada más ruinoso que las guerras continentales.

El sistema continental de Napoleón era demasiado franco-céntrico; buscó la humillación de otras naciones europeas en beneficio de Francia, en lugar de buscar su elevación e igualación; destruyó el comercio entre los fabricantes europeos y los países tropicales, obligando a los primeros al uso de artículos sustitutivos. Mientras tanto, sin embargo, es posible mirar hacia adelante en una unión más estrecha, tanto comercial como política, de Alemania, Holanda, Bélgica y Suiza. Este poderoso organismo nacional podría fusionar instituciones y dinastías, con Alemania como el ``punto central'' de una alianza continental duradera y garante de una paz duradera. Nuevamente, estamos bastante cerca de los debates actuales. 146

Vale la pena observar que incluso los economistas del libre mercado no podrían descartar fácilmente el argumento de List a favor del proteccionismo. List encontró oídos comprensivos en Alfred Marshall. Marshall, crítico, como hemos visto, de la escuela clásica, escribió que los economistas alemanes tenían razón al criticar la ``estrechez insular y la autoconfianza de la escuela ricardiana. En particular, les molestaba la forma en que los defensores ingleses del libre comercio asumían tácitamente que una propuesta que se había establecido con respecto a un país manufacturero, como Inglaterra, podía trasladarse sin modificaciones a los países agrícolas. El genio brillante y el entusiasmo nacional de List derrocó esta presunción; y demostró que los ricardianos habían tenido muy poco en cuenta los efectos indirectos del libre comercio'' 147: una admisión implícita, por Marshall, de que la economía clásica no puede ser válida en todo momento y lugar.

\hypertarget{historicismo-econuxf3mico-alemuxe1n}{%
\section*{Historicismo económico alemán}\label{historicismo-econuxf3mico-alemuxe1n}}
\addcontentsline{toc}{section}{Historicismo económico alemán}

Hay varias características que hacen que el trabajo de List sea excéntrico con respecto a la Escuela Histórica de Economía Alemana. Schumpeter, que coloca a List en el sistema clásico y no en la Escuela Histórica, parece por un lado menospreciar la contribución de List, quejándose de la falta de análisis riguroso y de su lenguaje periodístico; por otro lado, para subrayar el aspecto innovador de su trabajo, más arraigado en la sociología económica que en la economía propiamente dicha. Con List, ``el conjunto de hechos del crecimiento nacional, tan olvidado por los `clásicos', emerge en una formulación sumamente acertada y fue aplicado por primera vez de manera concreta que incluso los empresarios modernos, que no tenían uso del misticismo romántico , podría comprender, especialmente en los campos de la política arancelaria \ldots{} En este contexto, la contribución de List a la sociología económica es de primera importancia:148

Si recordamos la observación de De Cecco de que la influencia de List en los estudiantes de economía fue tan débil como la de Smith y Ricardo fue fuerte, esto debería matizarse porque el robusto movimiento teórico que se desarrolló en Alemania en la segunda mitad del siglo XIX tomó fuerza de las ideas de List; pero también es cierto que este movimiento languidecía desde el inicio del nuevo siglo y puede considerarse prácticamente extinguido tras la Primera Guerra Mundial. En este período, la opinión académica imperante en Alemania adoptó doctrinas y métodos del pensamiento económico neoclásico, dando vida a nuevos desarrollos con la Escuela Austriaca y luego con el ordoliberalismo alemán (Capítulo 2 ).

La Escuela Histórica de Economía Alemana se fortaleció en paralelo con el nacimiento del Estado unitario alemán y el establecimiento del Reich. Como se mencionó anteriormente, List siguió siendo una figura excéntrica, porque su trabajo precedió a la Escuela al menos por un par de décadas, y también por su insistencia específica en la protección de la economía nacional como instrumento necesario para hacerla competitiva internacionalmente. Pero fue el principal representante de la Escuela Alemana, Gustav von Schmoller, quien dedicó a List su atención y elogio: ``Friedrich List fue el primer economista que reunió, con gran estilo, los desarrollos económicos en Europa y América, investigaciones históricas con observación, extrayendo de sus hallazgos una teoría importante de la evolución socioeconómica \ldots{} Aunque básicamente siguió siendo un gran agitador,149

Posteriormente, Schumpeter dedicó algunas páginas a la Escuela Histórica; y, más recientemente, las vicisitudes de la Escuela han sido reevaluadas por un historiador de la Alemania moderna, Erik Grimmer-Solem. Ambos comparten la idea de una relativa vaguedad de su alcance y heterogeneidad de los escritores que se consideran pertenecientes a la Escuela, por lo que la Escuela misma sigue siendo vista como un ``enigma''. Su nacimiento se explica por el alto nivel alcanzado por la historiografía en la vida intelectual de Alemania; allí, la importancia de la historiografía era aún mayor en comparación con otras ciencias sociales, mientras que, por otro lado, la economía teórica, tal como la expresaba la Escuela Clásica, nunca había echado raíces. Una línea de demarcación entre las dos escuelas, dentro de los límites en los que se pueden trazar, se encuentra en su metodología: inductiva, y basada en la observación, recopilación y análisis de hechos históricamente determinados, en el caso de la Escuela Histórica; y deductivo, basado en premisas generales de validez universal, en el otro caso. La Escuela Alemana considera en cambio estas premisas de carácter dudoso, fundamentalmente precientíficas y destinadas a ser reemplazadas por una seria investigación de los hechos; para ser más específicos, son premisas que, si bien reflejan situaciones históricamente determinadas, reciben de la economía clásica una validación general, atemporal. fundamentalmente precientíficas y destinadas a ser reemplazadas por una seria investigación de los hechos; para ser más específicos, son premisas que, si bien reflejan situaciones históricamente determinadas, reciben de la economía clásica una validación general, atemporal. fundamentalmente precientíficas y destinadas a ser reemplazadas por una seria investigación de los hechos; para ser más específicos, son premisas que, si bien reflejan situaciones históricamente determinadas, reciben de la economía clásica una validación general, atemporal.

Si recordamos la distinción básica, mencionada al comienzo de este ensayo, entre asumir al individuo o al Estado como el motor primario racional de la vida social y económica, las siguientes palabras de Schmoller no dejan dudas sobre dónde se posiciona la Escuela Histórica: ``La idea de que la vida económica ha sido siempre un proceso que depende principalmente de la acción individual, una idea basada en la impresión de que se trata simplemente de métodos para satisfacer las necesidades individuales, es errónea con respecto a todas las etapas de la civilización humana''. 150

Las dos escuelas se odiaban (hemos mencionado anteriormente los ataques frontales lanzados por List contra la ``escuela cosmopolita''). Schmoller era propenso a caricaturizar la otra escuela como una doctrina egoísta disfrazada de ciencia económica; en el lado opuesto, en opinión del economista neoclásico Carl Menger, la Escuela Histórica era ``un amorfo objeto de burla'' (Grimmer-Solem).

Sin embargo, el rechazo de la economía clásica va acompañado de una oposición significativa a las teorías socialistas, en particular a las de Marx y Lassalle. ``Marx considera al hombre como un autómata de las condiciones tecnoeconómicas; en realidad, es el hombre quien determina estas condiciones según ideas y propósitos superiores. Cualquier modo de producción, cualquier relación de clase, cualquier forma de propiedad, aunque dependa de la técnica, no puede explicarse más que por referencia a causas espirituales y morales''. 151Odiar el marxismo es aparentemente sorprendente, dado su rechazo de la economía clásica y el fuerte componente social de sus ideas. ``Aunque simpatizaron con la descripción socialista de la injusticia, desde el principio se sorprendieron por la falta de fundamento empírico de sus teorías y la impracticabilidad de sus programas políticos. Contrarrestar a Lassalle \ldots{} fue especialmente urgente porque profetizó la desaparición de Mittelstand {[}empresas medianas y burguesía media, cuyo papel estos economistas consideraban central en la política y la economía alemanas{]}. Un compromiso con el empirismo, la filosofía moral y el reformismo liberal fue claramente visible en los escritos de los economistas históricos en la década de 1860 y principios de la de 1870''. 152

Schumpeter enumera los criterios de investigación seguidos por la Escuela Alemana.

Relatividad, según la cual es insostenible la idea de que existen reglas prácticas generalmente válidas en el campo de la política económica.

Unidad de la vida social, para la que existe una correlación inseparable entre todos sus elementos: en consecuencia, la Escuela tiene un desprecio por los economistas que nunca se inclinan hacia el próximo campo, permaneciendo aislados en su propio dominio.

El antirracionalismo, que ve una multiplicidad de motivaciones en el comportamiento humano, y atribuye una importancia relativamente menor a una percepción meramente lógica en lo que respecta a este comportamiento {[}¿es este el ``comportamiento irracional'' de las teorías más recientes?{]}.

La evolución, un criterio ---observa Schumpeter--- no desconocido para Marx: según este enfoque, no es gratificante aislar fenómenos y reconstruir condiciones efectivas sobre una base meramente intelectual, más bien es necesario mirar las correlaciones individuales, es decir: no en general. causas de los eventos sociales, sino en las causas concretas de los eventos específicos en los que estamos interesados.

Punto de vista orgánico: la economía no se puede dividir en una aglomeración de individuos económicos independientes, los eventos económicos no son simplemente el resultado de componentes individuales. 153

Los escritores más jóvenes de la Escuela ---entre ellos, no sólo Gustav Schmoller, sino también Lujo Brentano, Adolf Held y Georg Knapp, por nombrar algunos--- estaban de hecho más influenciados por el método estadístico que por las ideas del movimiento romántico y el Filosofía hegeliana. En este sentido, mientras los principiantes de la Escuela (la llamada ``vieja'' Escuela Histórica) todavía estaban imbuidos de la ``filosofía de la historia'' (la de Giambattista Vico, por ejemplo), los economistas antes mencionados estaban más alejados de la influencia hegeliana, mirando más directamente a desarrollos fácticos como la rápida urbanización, la ola de industrialización (en la que Alemania fue un segundo poderoso, después de la primera ola liderada por Inglaterra), el crecimiento de los sindicatos y el socialismo. Insatisfecho por las respuestas dadas por las doctrinas económicas de la ortodoxia clásica, sometieron estas doctrinas a verificaciones empíricas, combinando los instrumentos históricos y estadísticos. ``No hay duda de que la centralidad de la `cuestión social' en los asuntos públicos alemanes significó que la economía en Alemania seguía siendo una economía política que abarcaba una amplia gama de fenómenos sociales y cuestiones políticas. Pero esto tampoco fue particularmente novedoso para la supuesta''Escuela Histórica``. Después de todo, la economía clásica, marxista y nacionalista estaba ligada a programas políticos discretos: la economía de Smith, Ricardo, Marx y List eran bases analíticas sobre las que se construían sus respectivos programas de cambio político''. ``No hay duda de que la centralidad de la `cuestión social' en los asuntos públicos alemanes significó que la economía en Alemania seguía siendo una economía política que abarcaba una amplia gama de fenómenos sociales y cuestiones políticas. Pero esto tampoco fue particularmente novedoso para la supuesta''Escuela Histórica``. Después de todo, la economía clásica, marxista y nacionalista estaba ligada a programas políticos discretos: la economía de Smith, Ricardo, Marx y List eran bases analíticas sobre las que se construían sus respectivos programas de cambio político''. ``No hay duda de que la centralidad de la `cuestión social' en los asuntos públicos alemanes significó que la economía en Alemania seguía siendo una economía política que abarcaba una amplia gama de fenómenos sociales y cuestiones políticas. Pero esto tampoco fue particularmente novedoso para la supuesta''Escuela Histórica``. Después de todo, la economía clásica, marxista y nacionalista estaba ligada a programas políticos discretos: la economía de Smith, Ricardo, Marx y List eran bases analíticas sobre las que se construían sus respectivos programas de cambio político''.154

Con la Escuela Histórica surgen dos ideas centrales, que se unifican por la relevancia que se le da al análisis histórico y la centralidad del Estado: la urgencia de una reforma social, por alejada que sea de las ideas revolucionarias marxistas, y una revalorización de las políticas mercantilistas, incluyendo proteccionismo.

En cuanto a la reforma social, a diferencia de los economistas clásicos, cuyo programa político era minimalista (la ``mano invisible'', la eliminación progresiva de las barreras comerciales), la ética social pedía una mano visible, fundada en la sociabilidad natural y la acción moral constructiva del hombre. ``La cuestión principal del día era, en opinión de Schmoller, una cuestión de justicia: cómo superar las crecientes desigualdades, fortalecer a los medios y crear una mayor movilidad entre clases, una cuestión que no era únicamente económica sino también moral y cultural''. 155 Hizo hincapié en el interés común de trabajadores y capitalistas; los trabajadores bien pagados hubieran sido más confiables; también luchó por el reconocimiento oficial de los sindicatos; y creyó en la pequeña empresa, que más fácilmente habría realizado esa comunión de propósitos.156 Es notable que haya, en List, la misma relevancia de un interés compartido de las clases sociales, aludiendo al corporativismo (ver arriba, Sección 1.6 ).

Las políticas de reforma social pueden explicarse, al menos en parte, por la presión del movimiento socialista en vigoroso ascenso. Por un lado, el autoritarismo de la Corona alemana y del gobierno (mayoritariamente de Bismarck) llevó a la Ley Antisocialista de 1878, según la cual el Partido Socialista debía cesar cualquier actividad y las asociaciones socialistas debían ser disueltas y sus fondos confiscados. . Por otro lado, entre 1883 y 1889 una serie de leyes crearon un complejo sistema de seguro social que incluía una ley de seguro médico y un plan de compensación para trabajadores; en caso de discapacidad, o después de haber alcanzado cierta edad, se proporcionaba la pensión (los costos relacionados tenían que ser cubiertos, en diferentes casos, por los empleadores, los empleados o el Estado). En 1891, se promulgó una ley de protección para los trabajadores,157 La reforma social se extendió a las clases medias industriales, que tenían derecho a alguna protección (para confirmar la atención del gobierno aMittelstand). Como veremos más adelante, es comprensible por qué Marx consideró esta aparente colusión entre la Corona y las clases trabajadoras una mina contra la revolución social: una forma reaccionaria de ``socialismo feudal'', en sus propias palabras.

Con referencia al mercantilismo, como se señaló anteriormente, List había elogiado las políticas mercantilistas, mencionando en particular al ministro francés Colbert, como una fuerte afirmación del Estado central sobre los localismos y contra el ``cosmopolitismo'' de la doctrina de Smith. Con List, en el contexto alemán esta visión implicó la adopción de políticas proteccionistas, al menos en la medida en que son necesarias para poner a los países en pie de igualdad y permitir una competencia justa entre ellos. Schmoller, como List, elogia a Colbert porque ``su administración fue, principalmente, una lucha contra las autoridades municipales y provinciales'', pero su visión del proteccionismo es más matizada que la de List. Esencialmente, en su obra El sistema mercantil, identifica el mercantilismo con el Estado nacional: el mercantilismo es ``hacer Estado y hacer economía nacional al mismo tiempo\ldots{} La esencia del sistema no radica en alguna doctrina del dinero o de la balanza comercial; no en barreras arancelarias, derechos de protección o leyes de navegación; pero en algo mucho mayor: - a saber, en la transformación total de la sociedad y su organización''. La esencia del mercantilismo consiste en ``arrojar el peso del poder del Estado en la balanza de la balanza en la forma que demanden en cada caso los intereses nacionales''. Por lo tanto, según Schmoller, los términos de la relación entre libre comercio y proteccionismo deben contextualizarse: ``El libre comercio tiene un sesgo favorable especialmente hacia los intereses de los consumidores, el proteccionismo hacia los intereses de los productores; las industrias de exportación prefieren la primera, el segundo es el preferido por las empresas que aún tienen participación de mercado que explotar. La parte de la agricultura, que puede exportar, es de libre comercio; la otra, abrumada por las importaciones agrícolas, es proteccionista. Los comerciantes prevalecen por el libre comercio, son cosmopolitas; los artesanos son más bien proteccionistas. La mente abstractamente liberal se inclina hacia el optimismo, la proteccionista hacia el pesimismo. Las actitudes de libre comercio siempre tienden a prevalecer en las fases de crecimiento, las actitudes proteccionistas en períodos de estancamiento y decadencia económica. El libre comerciante confía en la división internacional del trabajo, el proteccionista en el desarrollo de las fuerzas nacionales; el primero quiere abandonar las ramas productivas más débiles, teniendo la certeza de que unas producciones nacionales más sanas sustituirán a las otras, mientras que el proteccionista se muestra tímido, quiere actuar de inmediato y defender el statu quo. El libre comercio y el proteccionismo son tendencias antitéticas que existen en todas las economías nacionales en desarrollo''.158

Estas frases, sin importar cómo se puedan valorar, suenan extremadamente reales hoy en día.

La influencia de las teorías antes mencionadas es visible en las políticas económicas alemanas, primero con la creación del Zollverein de los pequeños Estados alemanes anteriores a la unificación, y luego con el nacimiento del Reich. El Zollverein fue creado inicialmente por un acuerdo entre algunos estados, incluidos Bayern, Würtemberg, Baden, en 1820, y se completó bajo la hegemonía de Prusia en 1832. 159Cuando comenzó la segunda revolución industrial alrededor de 1870, centrada en Alemania (y Estados Unidos), su punto de apoyo estaba representado por la nueva tecnología y la industria pesada, como el acero, los productos químicos, la electricidad y los productos eléctricos, y Alemania protegió a estas industrias incipientes detrás de aranceles elevados. paredes. En general, después de la guerra victoriosa contra Francia en 1870 y el nacimiento del Reich en 1871, bajo Bismarck un fuerte Estado intervencionista dio lugar a la creación de un sistema de ``capitalismo organizado''. Hemos mencionado el seguro social y un núcleo del estado del bienestar, pero también cabe mencionar la negociación colectiva en los contratos laborales, la cartelización de la industria y la nacionalización de los ferrocarriles. En cuanto al proteccionismo, algunos historiadores también dan importancia al cese de los enormes pagos de reparaciones de guerra por parte de Francia.160

Pero conviene recordar que, tras la guerra victoriosa, el Reich unificó la moneda e inmediatamente adoptó el patrón oro: un paso de la mayor importancia para el nacimiento de un sistema monetario internacional, cuando solo un país, Gran Bretaña, lo había introducido formalmente. . Esta decisión significó una especie de consagración de Alemania como potencia mundial. El crédito de este movimiento se atribuye a un estadista y economista liberal, Ludwig Bamberg: un reconocimiento de que también existía una disposición liberal en la Alemania estatista, lo que refleja una orientación liberal occidental que nunca había desaparecido por completo (Pierenkemper-Tilly).

\hypertarget{socialismo-marxista}{%
\section*{Socialismo marxista}\label{socialismo-marxista}}
\addcontentsline{toc}{section}{Socialismo marxista}

El propio Marx quiere aclarar su relación con Hegel. ``Por lo tanto, me reconocí abiertamente como alumno de ese poderoso pensador, e incluso aquí y allá, en el capítulo sobre la teoría del valor, coqueteé con los modos de expresión que le son propios. La mistificación que sufre la dialéctica en manos de Hegel, de ninguna manera le impide ser el primero en presentar su modo general de trabajar de manera comprensiva y consciente. Con él se pone de cabeza. Debe ponerse boca arriba de nuevo, si quiere descubrir el núcleo racional dentro de la cáscara mística''. 161De ahí la idea materialista de Marx de que la historia es la negación, la antítesis de la concepción idealista hegeliana de la historia: esto significa que las ideas nacen como un reflejo de las condiciones materiales, y no al revés. ¿Qué queda, en la obra de Marx, del pensamiento de Hegel? Un poco, observa Schumpeter: el método dialéctico, que explica cualquier desarrollo real con el desarrollo conceptual; el método histórico {[}que aproxima a Marx a la Escuela Histórica{]}; la forma de expresar sus conceptos con esa oscuridad que es propia de algunas frases de su maestro. 162

La influencia en Marx de la filosofía hegeliana de la historia, de la Escuela Histórica de Economía Alemana y de la Escuela Clásica fue bien reconocida por Maurice Dobb: ``Su análisis de la sociedad capitalista fue abordado desde el punto de vista de una filosofía general de la historia, por la cual se puede decir que se combinaron el énfasis descriptivo y clasificatorio de la escuela histórica y el énfasis analítico y cuantitativo de la Economía Política abstracta''. 163

Pero la interpretación económica de la historia pertenece a Marx, no a Hegel; el único hegelismo latente de Marx se encuentra en su historicismo dialéctico, en la centralidad del Estado, y en el carácter teleológico de la historia, hacia la emancipación del hombre. Marx y Engels, ``habían estado encantados por la dialéctica hegeliana'' 164 , pero la economía política de Marx tiene como evidencia una impronta ricardiana, en sus referencias a la teoría del valor trabajo y la división fundamental de la sociedad en diferentes clases sociales. Marx no reconoce la contribución del capital a la creación del producto, que define como ``plusvalía'', porque el único valor de un producto lo da el trabajo empleado para crearlo.

Por lo tanto, la explicación de la ganancia, según Marx, no radica en ningún costo de la actividad productiva aportada por el capitalista, sino en la estructura de clases de la sociedad. La relación entre el propietario de los medios de producción y el trabajador depende de esa estructura, determinada históricamente. En una sociedad que admite la esclavitud, el amo toma todo el producto, por encima de la mera subsistencia del esclavo: no existe una cuestión de plusvalía. En una sociedad capitalista, en cambio, el trabajador es formalmente libre, ninguna ley o costumbre lo obliga a trabajar para un maestro. Sin embargo, dado que el proletario carece de medios de producción, debe vender en el mercado su trabajo, que para él es necesario para producir su subsistencia: la fuerza de trabajo es vendida por el trabajador en el mercado al capitalista como si fuera un mercancía, adquiriendo un valor. El capitalista vende el producto a un valor mayor que el valor de la fuerza de trabajo. Como consecuencia, la ganancia del capitalista es el resultado de la estructura de clases de la sociedad capitalista, basada en la distinción fundamental entre capitalista y trabajador.165

Como se mencionó anteriormente, según los economistas marxistas, la escuela neoclásica y la teoría de la utilidad marginal surgieron como una reacción a la doctrina socialista. Una de las primeras críticas a Marx proviene de Alfred Marshall: ``No es cierto que el hilado de hilo en una fábrica\ldots{} sea producto del trabajo de los operarios. Es el producto de su trabajo, junto con el del empleador y los gerentes subordinados, y del capital empleado; y que el capital mismo es el producto del trabajo y la espera: y por lo tanto, el hilado es el producto del trabajo de muchas clases y de la espera. Si admitimos que es producto únicamente del trabajo, y no del trabajo y la espera, sin duda una lógica inexorable nos puede obligar a admitir que no hay justificación para el interés, la recompensa de la espera .; porque la conclusión está implícita en la premisa, \ldots{} Marx de hecho afirma audazmente la autoridad de Ricardo para {[}su{]} premisa; pero en realidad se opone a su afirmación explícita y al tenor general de su teoría del valor, como lo es al sentido común''. 167

La concepción materialista de la historia no significa que toda acción humana sea el resultado de motivaciones económicas, o que el surgimiento de la sociedad socialista sea inevitable. En su Manifiesto del Partido Comunista , Marx y Engels se desvinculan de la izquierda hegeliana, a la que incluso antes habían pertenecido, y de un determinismo que da con certeza el advenimiento del Estado socialista; sin la autoconciencia del proletariado, nada podría impedir la explotación de los asalariados por parte de sus amos. 168 ``El propio Marx pasó gran parte de su tiempo tratando de organizar un movimiento político revolucionario, en lugar de sentarse y esperar hasta que las presuntamente férreas leyes de la historia le entregaran a la humanidad una sociedad socialista''. 169

En El Capital , 170 Marx se inclina a creer que, independientemente de la conciencia de la clase trabajadora de su condición social como requisito previo para una revolución exitosa, ``una ley de la tendencia a la caída de la tasa de ganancia'' puede formularse como una predicción de la necesidad histórica de un colapso del capitalismo. Esta ``ley'' se puede explicar dividiendo una inversión capitalista en dos categorías:

\begin{quote}
\begin{itemize}
\item
  Medios de producción ---materias primas, máquinas y otros dispositivos--- que Marx llama ``capital constante'';
\item
  El trabajo, que a su vez se divide en dos componentes: salario, la cantidad de trabajo que se realiza para la producción de la remuneración de los trabajadores, que Marx llama ``capital variable'' 171 ; y plusvalía, la cantidad de trabajo que se apropia el capitalista. En palabras de Marx, ``el trabajo excedente de la fuerza de trabajo es el trabajo gratuito realizado para el capital y, por lo tanto, forma una plusvalía para el capitalista, un valor que no le cuesta ningún rendimiento equivalente''. 172
\end{itemize}
\end{quote}

La tasa de plusvalía es la relación entre la plusvalía y el capital variable y da una medida de la intensidad de la explotación del trabajo por parte del capitalista.

En cambio, la tasa de ganancia es la relación entre la plusvalía y el capital total, constante y variable. La competencia insta al capitalista a hacer su fábrica más eficiente con nueva tecnología, lo que requiere más inversión de capital, y esto eleva la cantidad de capital inflexible o constante en relación con la cantidad flexible o variable (trabajo). Esto significa que, incluso si el nivel de explotación laboral ---la plusvalía--- permanece igual, la tasa de ganancia tiende a caer.

Marx da un ejemplo: supongamos que el trabajador trabaja tantas horas para sí mismo como para el capitalista, es decir, la relación entre la plusvalía y el salario es del 100\%. Sin embargo, la tasa de ganancia depende de la cantidad de capital total empleado en la producción. Si los salarios ---capital variable--- son iguales a 100, y la plusvalía es igualmente 100, y el capital constante empleado es 50, la tasa de ganancia es 100/150 = 66,6\%. Pero, como se acaba de mencionar, la competencia insta al capitalista a aumentar la cantidad de capital constante, digamos, a 100. Incluso si la tasa de explotación laboral, es decir, la tasa de plusvalía, sigue siendo la misma (100\%), el la tasa de beneficio disminuye: 100/200 = 50\%, y esta disminución continúa a medida que se tiene que invertir una cantidad adicional de capital constante. 173

La ley de la tendencia a la baja de la tasa de ganancia puede contrastarse mediante una disminución de los salarios, es decir, una explotación más intensa del trabajador: un aumento de la plusvalía. En otras palabras, esa ``ley'' sería defectuosa si el capitalista tuviera la capacidad de reducir los salarios en la cantidad necesaria para mantener constante su razón de ganancia, es decir, para aumentar su plusvalía.

Es lógicamente imposible discutir cuál de las dos tendencias prevalecería. Es posible que Marx, en el Vol. III de El Capital , haya dejado el tema sin resolver: su método histórico puede haberlo inducido a pensar que cualquier solución dependería de la interacción entre el progreso tecnológico y la configuración de las relaciones de clases sociales en un momento y una etapa determinados. . 174 Pero, si los salarios ya están en un nivel de subsistencia, cualquier disminución adicional sería impracticable. De todos modos, esa ``ley'', o ``tendencia'', es evidencia de un determinismo hegeliano y, al mismo tiempo, un recordatorio de la influencia de Ricardo en Marx. 175

Marx y Engels observan el estrecho vínculo entre las fuerzas de producción (tecnología, máquinas, capacidades humanas), que están en continua evolución, y el marco legal que las encapsula (derechos de propiedad, relaciones laborales, división del trabajo). La causalidad va del primero al segundo, no al revés (es decir, los modos de producción no están condicionados por las instituciones). La burguesía, que legalmente posee los instrumentos de producción, utiliza los poderes del Estado ---de su Estado--- para validar la distribución del producto que aprueba, siendo funcional a su interés, y niega a la clase obrera todo el producto al que esta la clase tiene derecho. La burguesía confía en la autoridad de su Estado para que la distribución de productos que aprueba sea aceptada como regla general y cree un sistema de valores: políticos, jurídicos,176 ) --cuya observancia general garantiza la distribución del producto impuesto por los capitalistas a la clase obrera.

Pero, ``la burguesía no puede existir sin revolucionar constantemente los instrumentos de producción, y con ello las relaciones de producción y con ellas todas las relaciones de la sociedad''. 177 Por tanto, surge una brecha entre los modos de producción y el marco institucional de la sociedad. Este marco, formado por aquellos valores definidos por la clase capitalista, se vuelve obsoleto. Con el nacimiento del socialismo, ese vacío se llena y se define una nueva organización de las fuerzas productivas; Dentro de esta nueva organización, la clase proletaria, que más ha sufrido esa obsolescencia, pone las cosas en su lugar correcto, destruyendo la institucionalidad capitalista, totalmente inadecuada. El nuevo conjunto de instituciones es de hecho el que introduce la propiedad pública, en una sociedad sin clases.

La revolución constante de los instrumentos de producción y por ende de las relaciones de producción, la expansión constante y la explotación de los mercados mundiales que se relacionan con el carácter ``cosmopolita'' de la producción y el consumo, son fuente de situaciones recurrentes de sobreproducción ``epidémica''. ---Una demanda insuficiente, en la terminología de Keynes - (``como el hechicero que ya no es capaz de controlar los poderes de ninguno de los dos mundos a quien ha invocado con sus hechizos'' - Manifiesto ). La burguesía supera esas crisis mediante la destrucción forzada de una masa de fuerzas productivas y la conquista de nuevos mercados ( Manifiesto). Paradójicamente, la clase dominante se ve inducida a ``restringir aún más el costo de producción de un trabajador, casi en su totalidad, a los medios de subsistencia que requiere para su mantenimiento y para la propagación de su raza'', mientras que los estratos más bajos de la clase media ---Pequeños comerciantes, tenderos, artesanos y campesinos--- se van hundiendo poco a poco en el proletariado. 178

El traspaso de la propiedad de los medios de producción de la burguesía al Estado, es decir al proletariado organizado como clase dominante, se producirá de diferentes formas según los distintos países. En el Manifiesto , Marx y Engels ven una transición gradual en los países más avanzados, que en un principio constará de las siguientes medidas 179 :

\begin{itemize}
\tightlist
\item
  Abolición de la propiedad de la tierra, cuya renta se aplica a fines públicos;
\item
  Un fuerte impuesto progresivo sobre la renta;
\item
  Abolición de derechos sucesorios;
\item
  Confiscación de la propiedad de todos los emigrantes {[}personas acomodadas que se trasladan al extranjero{]};
\item
  Centralización y nacionalización del crédito;
\item
  Lo mismo para comunicaciones y transporte;
\item
  Ampliación de fábricas e instrumentos de producción de propiedad estatal;
\item
  Obligación de trabajar;
\item
  Combinación de agricultura e industria y armonización de las condiciones de trabajo entre ciudades y campos;
\item
  Educación gratuita y abolición del trabajo infantil en las fábricas; educación técnica con fines industriales.
\end{itemize}

Un programa que suena no tan lejos de algunos partidos socialistas o comunistas en el mundo occidental, en particular después de la Segunda Guerra Mundial (El Partido Laborista Británico escribió, en 1948: ``Nuestras propias ideas han sido diferentes de las del socialismo continental que surgió más directamente de Marx, pero también nosotros hemos sido influenciados de cien maneras por pensadores y luchadores europeos y, sobre todo, por los autores del Manifiesto'').

En el socialismo marxista, la desaparición del Estado solo debe entenderse como la abolición de una institución burguesa. El Estado se convierte en una institución proletaria, desprovista de cualquier estructura jerárquica, vertical. Marx acepta la visión de Smith y Ricardo de una sociedad dividida en clases. Pero, si bien según los dos economistas clásicos el mantenimiento de clases diferentes es el requisito previo para la creación de nueva riqueza, Marx piensa que solo su abolición puede privar a la clase trabajadora de la subordinación impuesta por la burguesía. En el Estado proletario, ``el proletariado organizado como clase dominante'', 180sus órganos deben ser democráticos, no destinados a mantener el capitalismo como modo de producción. En la sociedad burguesa el Estado es el último protector de la estructura social: ``Sólo bajo el amparo del magistrado civil - escribe Smith, y el Manifiesto repite, con un sentido obviamente opuesto - que el dueño de esa valiosa propiedad \ldots{} puede dormir una sola noche en seguridad''. 181

La ``dictadura del proletariado'' no es, en términos dialécticos, la antítesis de la democracia; su antítesis es la ``dictadura de la burguesía'', mientras la propiedad de los medios de producción permanezca en manos de la clase media. Su significado es que ``en la dictadura proletaria la sociedad está organizada para que el poder del Estado esté en manos de la clase obrera, que utiliza toda la fuerza necesaria para evitar que sea arrebatada por la clase que antes ejercía su autoridad''. 182

El socialismo marxista se distancia tanto del socialismo reaccionario como del socialismo burgués. Dentro del primero, se hace una distinción adicional entre socialismo feudal y pequeño burgués: ambos son reaccionarios, porque miran al pasado.

En cuanto al socialismo feudal, Marx y Engels se oponen a la izquierda hegeliana, a la que, como se mencionó, ambos habían pertenecido anteriormente. Hay que reconocer, dicen, que la emancipación de los trabajadores no puede ocurrir en todos los países de la misma manera. El capitalismo es una etapa del desarrollo del hombre. El marxismo afirma la naturaleza histórica del problema económico. La izquierda hegeliana, que se calificó como el ``verdadero'' socialismo alemán, es incapaz de comprender las diversas condiciones históricas de los diferentes países: mientras que, en Francia, el socialismo bien puede tener como objetivo el ataque a la burguesía ya en el poder, porque Francia ya está más allá de la sociedad feudal y aristocrática, en Alemania (que tanto Marx como Engels miran con apasionada atención como su propio país) la clase obrera es todavía inmadura para la revolución, porque el capitalismo alemán aún no está desarrollado hasta el punto de convertir a esa clase en un ``proletariado''. El gobierno alemán contempla una alianza entre la Corona y la clase trabajadora a través de medidas de bienestar (ver Sect. 1.10 ), aparentemente a expensas de la burguesía, pero sustancialmente a expensas del proletariado. 183 En Alemania, la burguesía acaba de empezar a luchar contra la aristocracia feudal y la monarquía absoluta. Luchar por el socialismo en estas condiciones significa retrasar el éxito de la revolución liberal burguesa, asustándolos con la amenaza de un ataque proletario, cuyas condiciones previas aún son inmaduras. 184 En la Alemania del siglo XIX, sólo una revolución liberal consumada contra un sistema feudal puede ser el preludio de una revolución proletaria inmediata y posterior. 185 El ``verdadero'' socialismo de los hegelianos de izquierda es utópico y abstractamente filosófico: de hecho está infectado, según Marx, por el romanticismo alemán y, por tanto, por un nacionalismo derivado de Hegel y Fichte, que veían la monarquía prusiana como coincidente con el fin último. del absoluto, el fin de la historia.

También reaccionario es el socialismo pequeñoburgués. En este tipo de socialismo mira una clase social que ya está desapareciendo bajo el empuje de la revolución burguesa, que es la clase de los pequeños comerciantes, arrojados al proletariado por la acción de la competencia; ven el momento: el Manifiestosubraya: cuándo desaparecerán por completo como una sección independiente de la sociedad moderna; la misma opinión tiene la clase de pequeños propietarios campesinos, que sufren por la concentración de la tierra en unas pocas manos. Las ``últimas palabras'' de ambos son: gremios corporativos para la manufactura, relaciones patriarcales en la agricultura. Estas dos clases, en la medida en que todavía existen, siguen siendo una fracción de la clase media, son conservadoras, no revolucionarias, quieren diferenciarse del proletariado y, por lo tanto, son potencialmente reaccionarias. 186

Al mismo tiempo, el socialismo marxista rechaza las ideas de los socialistas conservadores o burgueses, 187como los owenistas en Inglaterra y los fourieristas en Francia, que apuntan a una burguesía sin proletariado, es decir, a mantener todas las ventajas del modo de producción y de la sociedad burguesa, sin los peligros y las luchas que de ellas resultan. Es una locura cualquier doctrina del socialismo que se base en la buena voluntad burguesa como fuente de cambio. El economista que es un socialista conservador trabaja por mejorar la educación técnica, por la participación en los beneficios, por los subsidios al desempleo causado por los desarrollos tecnológicos. Quiere mitigar las condiciones más duras del capitalismo, sin interferir con la organización y estructura de propiedad del capitalismo. Este tipo de socialismo cree que el avance de la clase trabajadora significa un cambio en las condiciones materiales de vida sin revolución;

El tema del proteccionismo se ve desde la misma perspectiva: aparentemente es una forma de defensa de la industria nacional y por ende del trabajo; en realidad, se basa en una armonía ficticia de intereses entre ellos: una armonía ya demostrada por Smith y Mill como falaz; pero exaltado por List y todavía utilizado para desalentar y reprimir el crecimiento de los sindicatos. 188

En este capítulo hemos demostrado que la economía neoclásica de principios del siglo XIX, en nombre de la ``ciencia'', había perdido el valor ético que había sostenido la visión de Adam Smith, según la cual, como se observó recientemente, ``intencional La actividad humana de todo tipo se consideraba incrustada en la ética: como conducta, es decir, no meramente como conducta''. 189 Smith había abogado por una sociedad que exaltara la libertad del individuo, convencido de que su acción beneficiaría a los suyos y, en última instancia, al bienestar general, al mismo tiempo manteniendo un marco social donde diferentes clases pudieran coexistir en sus diferentes formas. roles. Este era, esencialmente, el tipo de sociedad liberal según los economistas de la escuela clásica.

La Escuela Neoclásica, y sus teorías de la utilidad marginal que reaccionaron a la Escuela Clásica, y llegaron entonces a prevalecer, querían ir más allá de la economía clásica de Adam Smith, David Ricardo o John Stuart Mill. Los economistas neoclásicos enfatizaron los mercados libres competitivos como un requisito previo para alcanzar una posición de equilibrio. Pero hemos distinguido dos enfoques. El primer enfoque se basó en el ``descubrimiento'' del precio de mercado correcto a través de la nueva relevancia otorgada a la ``demanda'' de bienes (Marshall). El segundo sobre la construcción de un equilibrio general del sistema económico, matemáticamente formulado (Walras). 190 En cualquier caso, el ``valor de las cosas'' pasaría de un criterio objetivo (los costos de producción, y el componente laboral en particular) a uno subjetivo (la preferencia racional individual, expresada por una función de utilidad marginal).

Al estar influenciados por la nueva filosofía positiva, ambos enfoques, el walrasiano y el marshalliano, se definieron a sí mismos como ``científicos'', distinguiendo implícita o explícitamente la ``verdad'' (declaraciones empíricamente verificables y posiblemente sistematizadas en ``leyes'') de la ``ideología'' (que implican valores). más allá de la verificación empírica como ética, conciencia, confianza\ldots). Según el padre del positivismo, Comte, si se identifican científicamente ciertas ``leyes'' como explicación de un sistema económico regido por el principio de utilidad, ninguna libertad de pensamiento sería más necesaria para abordar los problemas económicos de la sociedad.

Los títulos de las principales obras marginalistas incluían a menudo el adjetivo ``puro'', y los prefacios de estos libros se apresuraron a enfatizar su propósito ``puramente científico'', simplemente para definir su contenido como desprovisto de cualquier componente relacionado con conceptos exógenos al utilitarismo. En particular, la ética fue expulsada del campo de la investigación en economía, a menos que se la viera como una especie de embellecimiento o componente cuantitativamente menor de utilidad, medido numéricamente. En el mejor de los casos, la justicia conmutativa sustituye a cualquier tipo de justicia distributiva. Según diferentes escritores, o las preocupaciones morales deberían estar fuera de la vista de la teoría del economista, o la moralidad tiene que identificarse con la utilidad personal o la búsqueda de la felicidad material.

Pero una filosofía social implícita no podía eliminarse, y la ideología positiva sonaba como un soporte intelectual de la conservación de las estructuras sociales y económicas existentes. Marx escribió: ``{[}l{]} a economía vulgar \ldots{} busca explicaciones plausibles de los fenómenos más intrusivos, para el uso cotidiano burgués, pero para el resto, se limita a sistematizar, de manera pedante, y proclamar verdades eternas, las ideas trilladas sostenidas por burguesía autocomplaciente con respecto a su propio mundo, para ellos el mejor de todos los mundos posibles''. 191

Si tomamos el punto de observación de un académico, o un hacedor de políticas, hacia fines del siglo XIX, preocupado como podría haber estado por los abrumadores temas del crecimiento económico, la producción y distribución de la riqueza, la independencia y la fuerza de una nación, malestar social generalizado en un mundo cada vez más industrializado, tres grandes corrientes de pensamiento ---un liberalismo individualista, nacionalismo, socialismo--- estarían presentes en su mente. Habían madurado y luchado entre sí durante gran parte del siglo. Si bien no vinieron de cero y más bien reflejaron el desarrollo de nuevas condiciones económicas y sociales, estas tres corrientes marcaron un corte decisivo con respecto al pasado. Las principales obras de Adam Smith y Marshall, abanderados de las escuelas clásica y neoclásica, se publicaron respectivamente en 1776 y 1890.fin - de - siécle environment, salió en 1900, los principales libros de List y Schmoller se publicaron en 1846 y 1890, Marx's Capital apareció en 1867.

La influencia de esas filosofías en el pensamiento económico del siglo XX es el objeto de los dos capítulos siguientes, mientras que el cuarto nos llevará a cuestiones de actualidad inmediata. El quinto y último capítulo tratará de responder si, y en qué medida, el liberalismo, el nacionalismo y el socialismo ---como ideologías económicas--- aún pueden proporcionar una guía para las reflexiones y decisiones de académicos y políticos.

\hypertarget{notas}{%
\section*{Notas}\label{notas}}
\addcontentsline{toc}{section}{Notas}

\begin{enumerate}
\def\labelenumi{\arabic{enumi}.}
\item
  Schumpeter (1954, Capítulo I).
\item
  Esta nueva visión ``sacude el chiste de la autoridad, acostumbra a los hombres a pensar por sí mismos, da nuevas pistas, que los hombres geniales pueden llevar más allá y, por la misma oposición, ilustran puntos , donde nadie antes sospechaba alguna dificultad'': Hume (1740). El texto fue reimpreso y precedido por John Maynard Keynes y Piero Sraffa, Cambridge University Press, 1938, ver p.~4 de esta edición.
\item
  Schumpeter (1954, pag. 23).
\item
  Marx da esta definición de la economía vulgar : ``{[}Se{]} ocupa sólo de las apariencias, rumia sin cesar sobre el material proporcionado desde hace mucho tiempo por la economía científica, y luego busca explicaciones plausibles de los fenómenos más intrusivos, para el uso cotidiano burgués, pero para el el descanso se limita a sistematizar de manera pedante, y proclamar para verdades eternas, la trillada idea que tiene la burguesía autocomplaciente con respecto a su propio mundo, para ellos el mejor de los mundos posibles''. La referencia de Marx fue principalmente a la naciente escuela neoclásica. Véase Marx (nd {[}1867{]}, vol.~Yo, p.~58).
\item
  Schumpeter (1954, pag. 26). Una política económica bullionista se centra en la acumulación de reservas de metales preciosos. El bullionismo puede verse como un mercantilismo temprano. En términos actuales, se traduce en una política orientada a un fuerte superávit en la balanza comercial.
\item
  Schumpeter (1954, pag. 35).
\item
  Plumpe2016, pag. 43).
\item
  Schumpeter (1954, pag. 33).
\item
  Heilbroner1988).
\item
  El filósofo italiano Benedetto Croce (1951, págs. 275-277) (publicado originalmente en La Critica , 36, 1938).
\item
  Estoy adoptando esta convincente partición de Fawcett (2014, págs. 120-124).
\item
  Israel2010, pag. 106).
\item
  págs. 106-107.
\item
  Como veremos, una sociedad estructurada en diferentes clases es un punto central en el pensamiento de Smith.
\item
  Israel2006, pag. 604).
\end{enumerate}

dieciséis.
Locke (1690, Capítulo 7, 85).

\begin{enumerate}
\def\labelenumi{\arabic{enumi}.}
\setcounter{enumi}{16}
\item
  Israel2010, pag. 179).
\item
  Un grupo político durante la revolución francesa.
\item
  Israel2014, págs.285 y 368).
\item
  Israel2010, págs. 63-65).
\item
  Rousseau1913, págs.83 y 22).
\item
  pag. 83.
\item
  pag. 83.
\item
  de Voltaire1770).
\item
  Rousseau1913, págs.255, 260, 269, 273, 280).
\item
  Israel2014, págs. 644-645). Véase también Gioja (1831, pag. 27).
\item
  Israel2010, pag. 181).
\item
  Herrero (1811, vol.~2, pág. 50). Es justo agregar que esta redacción es una interpretación del pensamiento de Smith por su editor, William Playfair.
\item
  Burke1800, pag. 14).
\item
  Herrero (1853, Parte VI, Sect. II).
\item
  Bentham1823, págs. 2, 9, 24-25).
\item
  Un tema bien enfatizado por Hayek, más de un siglo después (ver Capítulo 2 ).
\item
  Bentham1823, págs.310, 313, 323).
\item
  El contraste entre la Ilustración británica, impulsada por la riqueza, y la francesa, impulsada por la igualdad, es quizás el mismo que encontramos en los impulsores de la Revolución Gloriosa Inglesa de 1688 y la Revolución Francesa de 1789.
\item
  Porter (2000, pag. 261).
\item
  Porter (2000, pag. 202).
\item
  Herrero (1811, vol.~2, págs. 19-20).
\item
  Israel2010, pag. 107).
\item
  Herrero (1811, pag. 11). Esta frase concuerda perfectamente con lo que escribe David Ricardo: ``{[}la{]} búsqueda de la ventaja individual está admirablemente conectada con el bien universal del conjunto'' (2004, p.~81).
\item
  Stein1994).
\item
  Keynes (1926, pag. 11).
\item
  Ricardo (2004, pag. 76).
\item
  Véase, por ejemplo, el caso de metayers ( coloni partiarii ) en Francia, vol.~1, pág. 276.
\item
  Herrero (1811, vol.~2, págs. 152-155).
\item
  Hume (nd {[}1770{]}) Discurso V.
\item
  Ricardo (2004, págs. 82-87).
\item
  Ricardo (2004, pag. 82).
\item
  Hume (nd {[}1770{]}) Discurso III.
\item
  Hume (nd {[}1770{]}, pag. 35). Este énfasis en la demanda en la determinación de precios será descuidado por los economistas clásicos y, más adelante en el siglo XIX, será reafirmado por el contrario por los economistas neoclásicos (ver más abajo).
\item
  Hume había invocado un " banco público" que podría destruir el exceso de crédito en papel (Discurso III).
\item
  Ricardo (1810).
\item
  Ricardo (2004, pag. 5).
\item
  Herrero (1811, vol.~2, Libro I, Capítulo V, págs.21 y 25).
\item
  Ricardo (2004, pag. 7).
\item
  Ricardo (2004, pag. 19).
\item
  Ricardo (2004, pag. 48).
\item
  Ricardo (2004, págs. 48, 52-53); Herrero (1811, Libro I, Capítulo 5).
\item
  ``La abstinencia, es decir, no consumir, fue sugerida por John Stuart Mill y varios otros escritores, como la `contribución' del capital {[}a la producción{]}'': Heilbroner (1988, pag. 117).
\item
  Robinson1961, pag. 55).
\item
  Grillete (1980). El ensayo se reproduce en Ford, JL: Time, Expectations and Uncertainty in Economics , Edward Elgar, 1990.
\item
  Hobsbawm1997, pag. 144).
\item
  Porter (2000, págs. 138-139 y 149).
\item
  Berlín (2000, pag. 277).
\item
  Molino (2018). Las citas incluidas en el texto son de las págs. 5--9, 19--20, 22--25, 30--37.
\end{enumerate}

sesenta y cinco.
Malthus (1926). Sus postulados son: ``que la alimentación es necesaria para la existencia del hombre'', y ``que la pasión entre los sexos es necesaria y permanecerá casi en el estado actual''. Su inferencia a partir de estos ``postulados'' es que ``el poder de la población es indefinidamente mayor que el poder de la tierra para producir la subsistencia del hombre'', y que, en consecuencia, ``esto implica un fuerte y constante control de la población a partir de la dificultad de subsistencia''(págs. 11, 13-14).

\begin{enumerate}
\def\labelenumi{\arabic{enumi}.}
\setcounter{enumi}{65}
\item
  Matemáticas, astronomía, física, química, biología, sociología o ciencias sociales.
\item
  Spencer1992).
\item
  Fawcett2014, pag. 79).
\item
  Bobbio (1977, pag. 82).
\item
  Andresky1974, págs. 192 y 197). Pareto, positivista convencido, parece tomar distancia de esta asimilación extrema, cuando hace una distinción entre residuos y derivaciones en las ideologías del hombre. Las primeras son expresiones de sentimientos, de instinto, las segundas son expresiones de la necesidad humana de usar la razón. La ideología es una mezcla de impulsos lógicos y no lógicos. Esta es la diferencia básica con los animales, no tienen (no pueden) tener derivaciones, solo instinto. Ver Bobbio, pág. 101.
\item
  Molino (2018, pag. 25).
\item
  págs. 30--31.
\item
  págs. 33--35.
\item
  Molino (2010, pag. 18).
\item
  págs. 138-139.
\item
  Molino (2018, págs. 33-35).
\item
  Heilbroner1988, pag. 12).
\item
  La utilidad marginal ``es notoriamente una invención de economistas burgueses, posmarxistas y antimarxistas'': Sraffa (2017, pag. 3).
\item
  Dobb1937a, págs. 24-25).
\item
  Hayek1955, pag. 203).
\item
  Andresky1974, pag. 17).
\item
  Heilbroner, Milberg (1995, págs. 22-23).
\item
  Heilbroner, Milberg, pág. 23.
\item
  Schumpeter (1954, pag. 188).
\item
  Jevons1965).
\item
  pag. XVI.
\item
  pag. XVI.
\item
  pag. 1.
\item
  pag. VII.
\item
  pag. 23.
\item
  págs. 23 y VI.
\item
  El italiano Maffeo Pantaleoni fue un importante exponente de la escuela neoclásica.
\item
  Naldi2000, pag. 92).
\item
  Jevons1965, págs. 25-26).
\item
  Marshall (1966).
\item
  págs. 636 y 643.
\item
  págs. 70--71.
\item
  pag. 301.
\item
  págs. 4--8.
\item
  Dzionek-Kozlowska (2015).
\item
  Walras1954, pag. 201).
\item
  Walras hace una distinción entre economía pura (como se mencionó anteriormente); economía aplicada (que se ocupa de las implicaciones políticas de la teoría pura); y economía social (que se refiere a la distribución de la riqueza). Ver Tarascio (1967).
\item
  Walras1954, pag. 142). La discusión que sigue se basa en Jaffé (1977).
\item
  Jaffé, pág. 375.
\item
  Maurice Dobb encuentra aquí una inconsistencia lógica: ``{[}E{]} aquí parece haber una contradicción hegeliana en la `competencia perfecta' como concepto, ya que, si la competencia funcionara perfectamente y sin fricciones, nunca sería de interés para un vendedor para reducir su precio, sabiendo como él que todos los competidores lo seguirían inmediatamente y lo privarían de todas las ganancias al hacerlo''(1937b, pag. 203).
\item
  Pareto (1971).
\item
  Libro II, Sezione 592.
\item
  Sobre la ofelimidad, véase Libro II, Sezione 642--653.
\item
  Libro III, Sezione 958.
\item
  Sezione 962.
\item
  Sezione 965. Pareto añade: ``Una vez más necesito recurrir a las matemáticas para explicar esta proposición''. Parece revestir en una fórmula matemática argumentos que no están suficientemente explicados en el lenguaje ordinario.
\item
  Pareto (1919, pag. 370). Véase también Findlay Shirras (1935).
\item
  Pigou2013, pag. 649).
\item
  Einaudi1967, págs. 244-245).
\item
  Schumpeter (1997, pag. 120).
\item
  Schumpeter (1942, págs. 65-66).
\item
  Picketty2014, pag. 367).
\item
  El mismo sentido de futilidad de los intentos de redistribuir la riqueza a través de la intervención del Estado se muestra, un siglo antes de Pareto y sin utilizar una sola fórmula matemática, por Robert Malthus. Su objetivo es demostrar que ``la inmensa suma recaudada en Inglaterra para los pobres {[}las llamadas Leyes de los Pobres{]} no mejora su condición'', es un desperdicio (1926, pag. 71).
\item
  Cristo1990, pag. 39).
\item
  Pareto (1971) Sezione 967. Observa Jaffé, en su artículo ya citado sobre Walras: ``{[}La maximización de la satisfacción social de Walras es{]} esencialmente definitoria \ldots{} es nada más y nada menos que una anticipación de la optimalidad de Pareto, con las mismas virtudes y los mismos defectos \ldots{} Walras's El objetivo, incluso en su `economía pura', era prescriptivo o normativo en lugar de positivo o descriptivo''(p.~379). El mismo objetivo es aplicable a la optimalidad de Pareto.
\item
  Sen (1970).
\item
  Ver Bobbio (1977, Capítulo III, Sección 6).
\item
  Hegel1899, págs.105 y 15).
\item
  Clark (2019, págs.163 y 156).
\item
  Spirito1939).
\item
  Lunghini2001).
\item
  Según Sraffa, citado por Lunghini (2001, pag. 265).
\item
  Lista (1885, Capítulo XXIX).
\item
  De Cecco (1971, págs.20 y 25).
\item
  Lunghini2003, pag. 186).
\item
  Hawes (2014, págs. 12-15).
\item
  Maddison2001).
\item
  Trevelyan1944, pag. 557).
\item
  ``Debo pensar que tal gobierno {[}el gobierno francés{]} bien merecía que se elevaran sus excelencias, se corrigieran sus fallas; y sus capacidades mejoraron hasta convertirse en una constitución británica'': Burke (1790, pag. 195).
\item
  La versión alemana original del libro se publicó en 1844 y se basa en obras escritas por List en las dos décadas anteriores. Ver tribu (1995, pag. 33).
\item
  List se convirtió en ciudadano estadounidense y luego en cónsul estadounidense cuando regresó a su país de origen.
\item
  Para una unificación efectiva del mercado interno, List promovió el desarrollo de la red y la tecnología ferroviarias (locomotoras de vapor). En la lista como ``americano'': Tribe, págs. 32--65.
\item
  List, Capítulo XI.
\item
  GM Trevelyan (1944) correctamente observado: ``El Estado prusiano estaba educando a todo el pueblo prusiano. Los gobernantes paternos de Alemania a principios del siglo XIX educaron a sus súbditos, pero les dieron poca libertad política y ninguna participación en el gobierno. El Estado inglés dio a la gente común una gran libertad política y algo de participación en el gobierno, pero dejó que fueran educados por la caridad religiosa privada''(p.~518).
\item
  List, Capítulo XII.
\item
  Capítulo XIII.
\item
  Capítulo XXI.
\item
  Capítulo XIV.
\item
  Capítulo XXVII.
\item
  Capítulo XXIX.
\item
  Capítulo XXXV.
\item
  Marshall (1966).
\item
  Schumpeter (1954, pag. 100).
\item
  Schmoller (1904, vol.~Yo, p.~178).
\item
  Schmoller (1967, págs. 3-4).
\item
  Schmoller (1904, vol.~II, pág. 1096).
\item
  Grimmer-Solem (2003, pag. 136).
\item
  Schumpeter (1954, pag. 179).
\item
  Grimmer-Solem pág. 33. Más en general, véanse las págs. 19--34.
\item
  Grimmer-Solem, pág. 139.
\item
  Schmoller (1904, vol.~II, págs. 606-661).
\item
  Stolper (1967, págs. 44-46).
\item
  Schmoller (1904, vol.~II, pág. 1017).
\item
  Pierenkemper, Tilly (2004, págs. 8 y 9).
\item
  Pierenkemper, Tilly, págs. 136-141. Según estos autores, la economía alemana en el siglo XIX estaba más orientada al libre mercado de lo que comúnmente se piensa. El punto de vista diferente estaría sesgado por la opinión prevaleciente, pero incorrecta, de los economistas alemanes, sobre el papel omnipresente del Estado en la economía como factor impulsor de la industrialización: una hostilidad hacia el liberalismo del siglo XIX. Ver Zussman (2002); Stolper (1967, págs. 35-37).
\item
  Marx (nd {[}1867{]}, vol.~Yo, Libro Uno). Epílogo de la segunda edición alemana del 24 de enero de 1873, págs. 14-15 ( www.marxists.org ).
\item
  ``Si Marx de hecho hubiera tomado prestados elementos de pensamiento o incluso simplemente su método de especulaciones metafísicas, sería un pobre diablo, no vale la pena tomarlo en serio'': Schumpeter (1954, pag. 119).
\item
  Dobb1937a, pag. 23).
\item
  Lasky1948, pag. 14).
\item
  Dobb1937c, págs. 56-64).
\item
  La identificación del capital con la ``espera'', según los economistas neoclásicos, está bien explicada por Joan Robinson: ``{[}El capital{]} produce una producción extra que hace posible una gestación más larga. Dado que el capital es productivo, el capitalista tiene derecho a su proporción''(1974, pag. 58).
\item
  Marshall (1966, págs. 487-488).
\item
  Lasky1948, pag. 15).
\item
  Streek2017, pag. 230).
\item
  Marx (nd {[}1894{]}), editado por F. Engels, vol.~III, Parte III.
\item
  Porque se puede ajustar el empleo y la remuneración de los trabajadores.
\item
  Marx (1956 {[}1885{]}), editado por F. Engels, vol.~II, parte 1, pág. 22.
\item
  Marx (nd {[}1894{]}, vol.~III, Capítulo 13, págs. 153-154).
\item
  Dobb1937d, págs. 109-110).
\item
  Piero Sraffa escribe en 1947: ``El texto disponible de Marx no es claro (tenemos fragmentos publicados por Engels), y está abierto a diferentes interpretaciones. Mi opinión es que la ley de Marx es metodológica y no histórica, por lo que no es verificable estadísticamente. Por lo que sabemos, parece que en cada sociedad capitalista específica la relación entre la plusvalía y la tasa de ganancia son extraordinariamente estables en el tiempo. Esto no está en contradicción con la ley de Marx si esa `tendencia' se entiende como una abstracción particular, es decir, como resultado de un grupo de fuerzas (acumulación), mientras que otras fuerzas (progreso tecnológico, nuevos inventos y descubrimientos) no operan. El resultado es que esta caída de tendencia obliga a los capitalistas a interminables revoluciones técnicas, para evitar la caída de la tasa de ganancia''. Ver Sraffa (2017, págs. 7-8). Sobre Sraffa y la teoría del valor trabajo, ver este ensayo, Capítulo III.
\item
  Marx (nd {[}1867{]}, vol.~Yo, p.~58).
\item
  Marx (nd {[}1867{]}, vol I, pág. 354); Marx, K., Engels, F .: Manifiesto del Partido Comunista , 1872, en Lasky (1948, pag. 126).
\item
  Lasky1948, págs. 132-135).
\item
  Marx, Engels, Manifiesto , en Lasky, págs. 151-153.
\item
  Marx, Engels, Manifiesto , en Lasky, p.~152.
\item
  Herrero (1811, Libro V, Capítulo I, Parte II, p.~164).
\item
  Lasky1948, págs. 67-71).
\item
  Manifiesto , III.I, Socialismo reaccionario: Socialismo feudal; Socialismo pequeño burgués, págs. 153-161.
\item
  Lasky1948, pag. 51).
\item
  Lasky, págs. 53--54 y 58.
\item
  Manifiesto , pág. 139.
\item
  Manifiesto , III. II Socialismo conservador o burgués (161-166).
\item
  Lasky1948, págs. 54-55).
\item
  Normando2018, pag. 183).
\item
  Heilbroner, Milberg (1995, pag. 25).
\item
  Marx (nd {[}1867{]}, pag. 58).
\end{enumerate}

\hypertarget{part-siglo-xx}{%
\part{Siglo XX}\label{part-siglo-xx}}

\hypertarget{metamorfosis-del-liberalismo-en-el-siglo-xx}{%
\chapter*{Metamorfosis del liberalismo en el siglo XX}\label{metamorfosis-del-liberalismo-en-el-siglo-xx}}
\addcontentsline{toc}{chapter}{Metamorfosis del liberalismo en el siglo XX}

La visión individualista, impulsada por la utilidad, del siglo XIX es objeto de críticas relevantes en el siglo siguiente, bajo la influencia de diferentes circunstancias: la consolidación del nacionalismo alemán; las estructuras industriales en evolución, con la consiguiente concentración del poder económico y el crecimiento de los sindicatos; la Gran Guerra y, luego, la dificultad de volver a las estructuras sociales y los arreglos económicos y monetarios de antes de la guerra; la Depresión económica y la presión por un papel más importante del Estado; y la afirmación del socialismo marxista en la Unión Soviética. Los desarrollos filosóficos ---el idealismo de Croce--- enfatizan los contenidos éticos de la idea liberal, que no se considera necesariamente coincidente con el liberalismo económico. Como consecuencia de estas circunstancias, el liberalismo toma diferentes direcciones: el primero enfatiza los problemas de fallas del mercado, distribución de la riqueza, presencia del Estado en la economía. El segundo tiene un acento libertario y antiestatalista, alejándose sin embargo de la Escuela Neoclásica ``científica''. El tercero inserta elementos poderosos de la economía de mercado y la competencia en la tradición estatista alemana. Pigou, Keynes y Beveridge pueden considerarse economistas liberales que reaccionan a los desequilibrios de una sociedad liberal: Pigou, al insistir en las ``externalidades'' negativas del capitalismo, que deben ser abordadas por el Estado mediante el uso de dispositivos coercitivos para dirigir el interés propio hacia canales sociales; Keynes, al enfrentar la incapacidad del sistema para asegurar el pleno empleo y abordar la desigualdad en la distribución de la riqueza y el ingreso, y al enfatizar la economía como ciencia moral; Beveridge, ampliando el campo del liberalismo a través de una larga lista de servicios que depende del presupuesto del Estado proporcionar, incluso más allá del pleno empleo (el Estado del Bienestar). En otra dirección se mueven los economistas de la Escuela Austriaca y su principal exponente, Hayek. Su postura individualista, sin embargo, está lejos de la clásicalaissez - fairey la actitud científica y positivista de los economistas neoclásicos. Un sistema racional de preferencias, basado en la utilidad y expresado en forma matemática, no es posible, dados los fragmentos dispersos de conocimiento que están disponibles, pero el mercado proporciona la conexión necesaria entre agentes económicos ``desinformados'' a través del sistema de precios. Para operar, la competencia en el mercado requiere la ausencia de una planificación centralizada: solo un conjunto de reglas básicas destinadas a ser instrumentales para la búsqueda de las necesidades individuales. La Escuela de Chicago, con Friedman, reitera con énfasis esta actitud libertaria, enfocándose particularmente en la constitución monetaria, como un instrumento de estabilidad y de mantener el dinero fuera de la discreción de las autoridades. El ordoliberalismo es un giro típico del liberalismo en la Alemania de entreguerras, cuya influencia parece, sin embargo, duradera y viva en nuestros días. El Estado está en el centro mismo del sistema económico y regula y protege la competencia del mercado. El Estado debe actuar para desproletarizar las estructuras sociales del capitalismo, mejorando la libertad y responsabilidad de los trabajadores y no aprisionándolos en un estado de bienestar. El ordoliberalismo representa un fuerte giro hacia la economía normativa, en oposición a la positiva. Como tal, y de manera similar a la Escuela Histórica Alemana del siglo anterior, es inadecuado para ser estudiado como un ``modelo'' formal. Es más bien un esquema prescriptivo de estructura y organización del sistema económico. mejorando la libertad y la responsabilidad de los trabajadores y no aprisionándolos en un estado de bienestar. El ordoliberalismo representa un fuerte giro hacia la economía normativa, en oposición a la positiva. Como tal, y de manera similar a la Escuela Histórica Alemana del siglo anterior, es inadecuado para ser estudiado como un ``modelo'' formal. Es más bien un esquema prescriptivo de estructura y organización del sistema económico. mejorando la libertad y la responsabilidad de los trabajadores y no aprisionándolos en un estado de bienestar. El ordoliberalismo representa un fuerte giro hacia la economía normativa, en oposición a la positiva. Como tal, y de manera similar a la Escuela Histórica Alemana del siglo anterior, es inadecuado para ser estudiado como un ``modelo'' formal. Es más bien un esquema prescriptivo de estructura y organización del sistema económico.

Palabras clave

\begin{itemize}
\tightlist
\item
  Externalidades pigouvianas
\item
  Keynes
\item
  Estado de bienestar
\item
  Escuela austriaca
\item
  Escuela de Chicago
\item
  Ordoliberalismo
\end{itemize}

\hypertarget{causas-del-nuevo-pensamiento-sobre-el-liberalismo}{%
\section*{Causas del nuevo pensamiento sobre el liberalismo}\label{causas-del-nuevo-pensamiento-sobre-el-liberalismo}}
\addcontentsline{toc}{section}{Causas del nuevo pensamiento sobre el liberalismo}

Si distinguimos dos temas abrumadores de cualquier doctrina económica, la producción de riqueza y su distribución, el segundo tema emerge con fuerza en el siglo XX, y el liberalismo sufre amplias y profundas metamorfosis.

El siglo pasado vio ``profesiones del liberalismo cada vez más extendidas'', 1 pero declararse ``liberal'' podría insinuar visiones bastante diversas. La cuestión de la distribución de la riqueza, y el tema conexo de un papel más amplio del Estado en la economía, significó, por un lado, avanzar hacia ideas más cercanas al estatismo y al socialismo, como en el caso del socialismo liberal, la economía social de mercado, incluso el ordoliberalismo 2 . ideas a veces vistas como una ``tercera vía'', como una alternativa entre el liberalismo y el socialismo. Por otro lado, significó un alejamiento del ``cosmopolitismo'' de los pensadores clásicos ---que había sido un hito del liberalismo decimonónico--- hacia una nueva relevancia del interés nacional.

Estas tendencias, sin embargo, no agotaron el amplio campo de las profesiones del liberalismo, porque, al mismo tiempo, un fuerte componente ideológico también estuvo presente en las teorías libertarias que afirmaron el valor ético de reinstaurar la posición central del individuo como agente económico.

Las relaciones entre diferentes corrientes de pensamiento se volvieron borrosas. A menudo dificultaban descubrir la filosofía subyacente de una teoría económica y evaluar hasta qué punto las visiones liberalistas, socialistas y nacionalistas podían converger en la misma persona; Además, la filosofía económica de un escritor podría cambiar y su teoría económica podría verse afectada como resultado. 3

¿Cómo podemos resumir la actitud de una mente liberal a principios del siglo XX, tal como la moldearon las doctrinas económicas clásicas y luego neoclásicas del siglo XIX? ¿Hubo una sabiduría común, o al menos una opinión predominante, del hombre liberal en el cambio de siglo, tal como se plasmó en las convenciones sociales, o incluso religiosas, y en las teorías económicas centradas en el concepto de la utilidad individual del hombre? Keynes ofrece un buen currículum vitaede este consenso, al observar, en 1926: ``Trazo la unidad peculiar de la filosofía política cotidiana del siglo XIX hasta el éxito con el que armonizó escuelas diversificadas y en guerra y unió todas las cosas buenas en una sola mano. Hume y Paley, Burke y Rousseau, Goodwin y Malthus, Cobbet y Huskisson, Bentham y Coleridge, Darwin y el obispo de Oxford, fueron todos, se descubrió, alcanzando prácticamente lo mismo: el individualismo y el laissez-faire \ldots{} la compañía del los economistas estaban allí para demostrar que la menor desviación hacia la impiedad {[}es decir, un desapego de esa sabiduría consolidada{]} implicaba la ruina financiera. Estas razones y esta atmósfera son las explicaciones \ldots{} por qué sentimos un sesgo tan fuerte a favor del laissez-faire y por qué la acción del Estado para regular el valor del dinero, o el curso de la inversión,4

¿Podemos ver esa sabiduría común, tal como la expone Keynes, como una visión realmente liberal? ¿O debería el liberalismo tener un espacio conceptual más amplio (ético, político) y no necesariamente identificarse con el liberalismo económico impulsado por la utilidad que surgió de ese enfoque? Como veremos, Keynes afirmaría explícitamente que la economía es una ciencia moral. 5 Encontró ``repugnante'' la ``mezcla de lenguaje hegeliano y biológico'', 6 rechazando así tanto cualquier visión estatista como la actitud positivista de los economistas neoclásicos. La pregunta surgió del descontento de varios pensadores, y varios factores contribuyeron a reexaminar la relación entre el liberalismo y esa visión individualista, como sigue.

\hypertarget{factores-poluxedticos-econuxf3micos-y-sociales}{%
\subsection*{Factores políticos, económicos y sociales}\label{factores-poluxedticos-econuxf3micos-y-sociales}}
\addcontentsline{toc}{subsection}{Factores políticos, económicos y sociales}

Hasta que Gran Bretaña mantuvo su posición hegemónica, prevaleció una visión global o ``cosmopolita'', parte esencial del análisis clásico y neoclásico. Durante mucho tiempo, el crecimiento del nacionalismo, con el establecimiento y consolidación de estados a menudo poderosos, y el atractivo internacional del socialismo, no fueron suficientes para desafiar esa posición intelectual y política preeminente. La libertad de empresa y la libertad de intercambio en mercados competitivos, y un conjunto de instituciones funcionales a ese sistema económico, fueron el corolario de esta visión.

Pero estaba surgiendo un peso cada vez mayor de Alemania, desafiando la supremacía británica. Alemania se posicionó, al mismo tiempo, como el principal antagonista político de Gran Bretaña y la expresión de filosofías económicas, y políticas económicas, bastante lejos de esa perspectiva liberal imperante en Gran Bretaña. En las últimas décadas del siglo XIX, los intelectuales británicos advirtieron a su propio país que la educación nacional y la disciplina nacional ``en el corazón teutónico de Europa'' estaban creando un nuevo poder que desafiaba celosamente la ``riqueza mal distribuida'' de Gran Bretaña. 7Como se mencionó en el capítulo anterior, según las estimaciones de hoy, en los primeros años del nuevo siglo el PIB per cápita alemán superó al británico. Este tipo de estadísticas no estaba disponible para los contemporáneos, pero pudieron ver el éxito de los enfoques intelectuales y políticos alemanes, su peso creciente en la economía internacional y la confianza de Alemania en una visión centrada en el Estado sobre la perspectiva cosmopolita y enfocada individualmente. La Entente Cordiale entre Gran Bretaña y Francia (1904) puede verse desde esta perspectiva.

Al mismo tiempo, una estructura productiva cambiante de los países industriales, bastante alejada de la existente en la Gran Bretaña de Adam Smith, dificultaba la puesta en práctica de un mercado realmente libre, la condición previa asumida por la economía neoclásica para una competencia sin trabas. . De hecho, esta estructura en evolución estaba formada por grandes complejos industriales y planteaba nuevos problemas que afectaban la concentración del poder económico y la alteración del sistema de precios. De ahí la necesidad de poner barreras a los cárteles y monopolios industriales. La competencia no se puede dar por sentada. Como hemos visto anteriormente (Capítulo 1 Marshall en particular), el tema de la competencia surge en la literatura de los economistas neoclásicos, pero aún con una actitud cautelosa, en la incertidumbre de que la libertad del empresario pueda verse afectada negativamente.

``El sistema en el que {[}el pueblo estadounidense{]} tenía confianza ---escribe Herbert Stein con referencia al capitalismo de su país en la década de 1920--- no era el sistema de libre mercado de competencia atomista, de la Mano Invisible. Era el sistema empresarial, que es otra cosa. Era un sistema, en el que los beneficios fluían del carácter y la sabiduría de empresarios identificables''. 8

En estos nuevos modos de producción los trabajadores tomarían cada vez más conciencia de sus condiciones, no solo en lo que respecta a su remuneración, sino también a sus necesidades de salud y jubilación, y tratarían de alcanzar la igualdad social. Bajo el mismo techo de enormes fábricas, los trabajadores vivían muy cerca y así podían organizarse en sindicatos. Los sindicatos de trabajadores aumentaron sus voces contra ``la capital''. Se reclamaría una distribución más equitativa del producto industrial. La visión socialista iba ganando terreno, tanto en la versión marxista como en otras corrientes de pensamiento no tan extremas. Este fue, por ejemplo, el caso de Gran Bretaña, donde poco antes de la Primera Guerra Mundial, grupos de socialistas numéricamente significativos se reunieron bajo la bandera de la sociedad fabiana. 9

\hypertarget{la-guerra}{%
\subsection*{La guerra}\label{la-guerra}}
\addcontentsline{toc}{subsection}{La guerra}

La Gran Guerra da otro golpe fuerte a la visión económica liberal: vemos intervenciones del Estado en la vida económica para financiar el esfuerzo bélico y organizar a toda la sociedad junto con esquemas funcionales a los propósitos bélicos; se introducen obstáculos al comercio internacional; en todas partes, se registra una expansión anormal de la oferta monetaria y la inflación y, como consecuencia, el patrón oro, el sistema monetario como emblema de la sociedad liberal, se suspende en todos los países involucrados en la guerra, con la excepción de Estados Unidos. . Pero, no por casualidad, ``suspensión'' fue la palabra, porque todos los gobiernos tenían en mente, una vez cerrado el paréntesis de la guerra, restaurar las condiciones económicas y monetarias que antes imperaban, la libertad de cambio, la globalización perdida.

Sin embargo, son vanos los intentos de los ganadores y los perdedores de volver a las condiciones económicas y sociales de antes de la guerra. No nos detendremos en la evolución política de estos años de posguerra. Basta recordar algunos desarrollos relevantes. No solo el Imperio Ruso había desaparecido totalmente del concierto de las grandes potencias, sino que el Estado soviético se proponía como una alternativa radical al Estado liberal y como un sistema de gobierno que buscaba una plena puesta en práctica de esa filosofía socialista que hemos descrito. en el capítulo anterior, poniendo patas arriba los arreglos sociales preexistentes, erigiéndose como una antítesis del orden capitalista liberal y finalmente estableciendo un orden proletario sobre una base mundana.

En otros lugares, se estaban produciendo cambios importantes: la difícil reconversión industrial de la guerra a la paz, la inflación duradera, el malestar social, en parte relacionado con el descontento y la ilusión de los veteranos, y también alimentado por la propaganda socialista y por la observación de lo que estaba sucediendo en Rusia. , las reparaciones de guerra golpean a la derrotada Alemania, el peso de la deuda entre los Aliados sobre las finanzas públicas. Todos estos factores dificultaron enormemente cualquier esfuerzo por volver a las estructuras sociales y los acuerdos económicos y monetarios de antes de la guerra. De hecho, hubo un retorno a las instituciones de antes de la guerra, incluida una reconstrucción del patrón oro, pero esto sucedió sobre la base de tipos de cambio que no reflejaban las condiciones económicas y financieras de los respectivos países, y sin embargo en un contexto de difíciles condiciones sociales que dificultaron cualquier ``juego según las reglas'', en primer lugar, la deflación de precios y salarios necesaria para que los países sean competitivos internacionalmente. Durante la década de 1920, el globalismo de antes de la guerra se apoyó en cimientos frágiles.

Los movimientos internacionales de mercancías y capitales, alentados por la precaria restauración de tipos de cambio fijos, favorecieron a las economías que habían salido de la guerra en mejor forma (Estados Unidos), o habían sido lo suficientemente astutas para volver al patrón oro con un intercambio competitivo. tarifas (Francia). A las economías más desfavorecidas por el retorno de sus monedas a la paridad del oro a tasas poco realistas, a veces alentadas por razones de prestigio político (ver la ``cuota 90'' de Mussolini 10), apegarse a esa paridad significaba un ejercicio inútil y costoso, no solo en términos de producción, sino también en términos de deflación interna y malestar social. Lo que pasó bajo el nombre de ``guerra de dinero'' representó una perturbación mucho más profunda, fue evidencia de la desintegración del orden internacional, atestiguada por el fracaso sustancial de la Sociedad de Naciones. El antiglobalismo, en formas fuertemente nacionalistas y corporativas, y el socialismo, tomaron la delantera.

Todos estos factores crearon un entorno propicio para el éxito de diferentes teorías económicas, apoyadas ambas en la figura abrumadora del Estado: el Estado ético en la raíz del nacionalismo, y el materialismo histórico del socialismo marxista conducente al Estado proletario.

\hypertarget{la-gran-depresiuxf3n}{%
\subsection*{La gran Depresión}\label{la-gran-depresiuxf3n}}
\addcontentsline{toc}{subsection}{La gran Depresión}

El colapso financiero y la consiguiente Gran Depresión fueron una razón más para repensar esquemas anteriores de pensamiento económico. Los economistas neoclásicos, centrados en las teorías walrasianas del equilibrio general, no tenían la clave para poner estos eventos en sus esquemas lógicos, y sus explicaciones, principalmente monetarias, de la crisis sufrieron irrelevancia. El economista estadounidense Irving Fisher escribió en The New York Times, poco antes del colapso, que el mercado de valores había alcanzado lo que parecía una ``meseta permanentemente alta''; y después del accidente, agregó que el deslizamiento fue solo temporal. El desconcierto intelectual y político que acompañó al colapso, el colapso de la producción y el sufrimiento social en varias de las principales economías, se han descrito extensamente en otros lugares y aquí no hay necesidad de dedicar más palabras a eso.

La sensación de crisis que había traído el nuevo siglo, la agitación económica y social relacionada con la Primera Guerra Mundial, ya había hecho que el siglo anterior pareciera una larga fase de tranquilidad, estabilidad económica y social, para entonces definitivamente perdida. En este entorno, la Gran Depresión se notó aún más con una sensación de sorpresa no deseada porque, de hecho, había seguido a casi una década de crecimiento económico y euforia financiera. Fortaleció la búsqueda de nuevas formas de abordar estos nuevos desafíos. El pensamiento liberal necesitaba una reevaluación.

La Depresión reforzó el atractivo de las ideologías muy lejos de la idea liberal. Por un lado, los países nacionalistas encontraron en estos desarrollos una confirmación de la necesidad de un papel importante del Estado en la economía, facilitado políticamente por el mismo autoritarismo de sus gobernantes. Por otro lado, la Unión Soviética y los economistas marxistas de todo el mundo podían contemplar con complacencia un crecimiento aparentemente ininterrumpido de su economía hasta la Segunda Guerra Mundial, ciegos ante los horrores de su régimen (véase el capítulo 3 ).

Basil Blackett, director del Banco de Inglaterra y, con título completo, miembro del establecimiento británico, expresa bien la conciencia de estas nuevas condiciones, quien observó, en 1931: ``La difusión de la técnica de la organización sindical y paralelamente al aumento de la conciencia humanitaria y social sobre los problemas de vivienda, salud, saneamiento y condiciones de trabajo en general, han hecho imposibles o inadmisibles muchos de esos brutales ajustes económicos que nuestros abuelos pudieron considerar como consecuencia de la intervención. de una providencia sabia, que utilizó el interés propio ilustrado y la competitividad humana no regulada como su medio misterioso para realizar maravillas en la causa del progreso moral y material''. 11

\hypertarget{desarrollos-filosuxf3ficos-historicismo-ideal}{%
\subsection*{Desarrollos filosóficos: historicismo ideal}\label{desarrollos-filosuxf3ficos-historicismo-ideal}}
\addcontentsline{toc}{subsection}{Desarrollos filosóficos: historicismo ideal}

En las primeras décadas del siglo surgieron diferentes visiones del liberalismo, hasta entonces centradas en los factores económicos: después de la Ilustración, el historicismo, el marxismo y el positivismo, nuevas líneas del pensamiento liberal miraron la relación entre Economía y Moralidad, como el ``historicismo ideal''. ''Del filósofo italiano Benedetto Croce y sus seguidores, como Robin G. Collingwood en Gran Bretaña.

En cuanto a Economía y moralidad, el problema de la relación entre la Filosofía de la economía y la Ciencia de la economía fue planteado en términos radicales, desplazando a Croce en varias obras, siendo la principal publicada en 1908. 12 Su enfoque de este tema debe ser puesto en el marco de su propio sistema filosófico, que clasifica tanto la economía como la moral dentro de un mismo campo de la ``Filosofía de lo práctico''. 13

Para explicar su sistema y cómo está conectado a la economía, Croce se remonta al origen de la economía política en términos bastante similares a los de Schumpeter (ver arriba, Capítulo 1 ): es en el siglo XVIII cuando los filósofos escoceses, como Hutcheson y Hume, querían ``poner la boca'' ( mettere bocca ) sobre economía y, a su vez, los economistas no querían descuidar las cuestiones relacionadas con la ética. Adam Smith, a la vez filósofo y economista, es la expresión de esta tendencia. Sin embargo, con el tiempo, los economistas introdujeron el concepto de utilidad como una cuestión de ``especulación'' individual, desprovista de contenido ético. El núcleo del desacuerdo, que hemos mencionado anteriormente, estaba en el concepto de ``valor'': un concepto objetivo, inspirado en instancias morales, con el primero; y una subjetiva, inspirada en consideraciones puramente económicas, hasta el extremo hedonistas, siendo esta última: una diferencia entre ``valor como es y valor como, en cierto modo, debe ser''. 14

Con una notable similitud de acentos entre él y los marxistas del siglo XX (cuyas ideas sobre economía política se abordarán en el capítulo 3 ), Croce: (a) ataca la banalización del concepto de utilidad de los economistas neoclásicos, (b) reduce la ciencia económica a sus esquemas abstractos, y (c) cree que esta ciencia económica específica ---abstracta e individualmente utilitaria--- no tiene nada que ver con ninguna idea filosófica de utilidad.

Para ser claro en este punto, debe enfatizarse que Croce identifica la economía con las teorías neoclásicas y descuida totalmente otras direcciones alternativas que fueron tomadas por la disciplina económica, la economía keynesiana, por ejemplo. De hecho, no he encontrado ninguna evidencia de la reacción de Croce al pensamiento keynesiano, en particular a la firme visión de Keynes de que la economía es ``una ciencia moral''. Es una cuestión de especulación que la reacción de Croce hubiera sido muy comprensiva. 15 Esto explica por qué muchos economistas, incluso respetuosos de Croce como filósofo e historiador, se han quejado con pesar de que Croce no entendía economía.

Sólo confinando la economía en los límites de una ciencia utilitaria individual es posible aceptar la idea de Croce de que las acciones económicas y morales deben mantenerse distintas: la primera tiene que ver con la búsqueda egoísta de la utilidad, como lo explican los economistas neoclásicos; y el segundo, con la búsqueda de un nivel superior de ``pensamiento universal'' (``utilidad'', en términos filosóficos).

Lo que significa este ``pensamiento universal'', el propio Croce subraya que no tiene un contenido específico 16 ; con certeza, no se resuelve simplemente en el altruismo, la observancia religiosa u otros conceptos similares, ni se limita a un genérico ``hacer el bien'', en el sentido común de la palabra. Significa, más bien, ir más allá del propósito egoísta y comportarse de conformidad con el interés general de toda la humanidad, como emergiendo de las circunstancias históricas reales del tiempo y lugar bajo las cuales el individuo tiene que actuar (este es el `` historicismo ideal ''de la filosofía de Croce).

Por muy abstracta que pueda ser esta idea, se comprende mejor si se traduce a terminología concreta ---práctica, si queremos---. El enfoque de Croce, en ese momento histórico específico (es decir, en el momento de su escritura, principios del siglo XX), puede verse como ``pedagogía política, un intento de educar a la clase dominante italiana para que esté a la altura de sus deberes''. , una invitación a mirar un contexto europeo más amplio y abandonar los sueños nacionalistas y coloniales entonces generalizados. Al mismo tiempo, Croce, un antimarxista intransigente, busca mantener un equilibrio entre los intereses contrastantes de las diferentes clases sociales, sin dar lugar a los impulsos social revolucionarios. 17

Este es el núcleo del liberalismo en el pensamiento de Croce: los esquemas intelectuales históricamente determinados ---también en el campo de la economía--- no deben confundirse con ideas de validez universal y atemporal. Aquí cesa toda simpatía por los economistas británicos de la Ilustración, de la escuela clásica. No puede aceptar el esquema ricardiano de ideas que tienen validez ``en todos los lugares y en todos los tiempos'' (ver arriba, Capítulo 1 ). 18

Por eso Croce trae un ataque a todas las teorías económicas desarrolladas en el transcurso del siglo anterior:

Sobre la escuela clásica: ``los conceptos empíricos del liberismo (liberismo, como liberalismo económico 19 ) fueron elevados al nivel de leyes naturales. Sin embargo, no tenían un valor absoluto, sino simplemente empíricos, es decir, basados \hspace{0pt}\hspace{0pt}en hechos históricos y contingentes. Los economistas que formularon y apoyaron esas `leyes' defendieron, en nombre de la ciencia, intereses particulares de ciertas clases sociales 20 (esto se parece a Marx)''.

Respecto a la Escuela Histórica Alemana: ``se soñó, especialmente por los economistas alemanes,\ldots{} con una ciencia económica fundada en la ética'', pero la Escuela confunde valores económicos y éticos, y en la práctica favorece los intereses de ciertas naciones. 21

Y sobre los economistas neoclásicos: ``la aplicación de las matemáticas a la ciencia económica se ha hecho similar a la aplicación de las matemáticas a la mecánica; y el homo oeconomicus ha aparecido totalmente similar al `punto material' en la mecánica''\ldots{} El límite de este enfoque está en el hecho de que la economía basada en el concepto de utilidad descuida las distinciones cualitativas, y no puede dar ninguna relevancia a los hechos morales {[}que se entiende en el sentido croceano como se explicó anteriormente{]}. Los dos órdenes de actos son, en el esquema neoclásico de la ``economía pura'', indistinguibles. La economía pura, que trabaja a través del lenguaje matemático, tiene una base sólida, si dejamos de lado el hecho de que no considera ni el más mínimo vestigio del concepto de acción humana. 22Croce añade: ``{[}En conjunto{]} la naturaleza de la disciplina económica como disciplina cuantitativa\ldots{} donde no se puede superar el atomismo de postulados y definiciones, no permite su desarrollo orgánico a partir de un principio superior, que sólo pertenece a la filosofía''. 23

Entre la filosofía de la economía y la ciencia económica no puede haber desacuerdo, porque son conceptos heterogéneos. La filosofía cometería un abuso si invadiera el campo de la economía. Mezclar los dos --- piensa Croce --- da origen a al menos tres errores: (1) ver el cálculo económico como el único capaz de dar al hombre todas las verdades que necesita: ``a los economistas más puros y matemáticos - escribe Croce irónicamente - Me gustaría decir: libérate de las penas de filosofar: calcula, no pienses''; (2) el nacionalismo y el liberalismo económico son hechos históricos, no ``leyes'', y los economistas que sostienen la primera o la segunda no son científicos, sino políticos; (3) homo oeconomicus, constructor de diagramas y calculador de niveles de utilidad y curvas de indiferencia, se cree que es un animal realmente existente, pero el balance de la vida humana no se puede construir como una cuenta de pérdidas / ganancias, medido según su intensidad o duración. Esto genera ``la creencia falaz de que las construcciones matemáticas y el cálculo económico se identifican con la psique real {[}del hombre{]} o el Espíritu''. 24

Croce, al identificar la economía con el pensamiento neoclásico entonces imperante, se ve conducido a una actitud escéptica hacia la disciplina económica en general: observa la debilidad filosófica de los principios adoptados por estos economistas como fundamento de sus teorías, y muestra una actitud de superioridad, cordialmente correspondida por los economistas con los que mantuvo largos y duros debates, en particular con Pareto 25 y Einaudi, a pesar de la actitud deferente y generosa de Einaudi hacia el filósofo (véase también el capítulo 5 ).

Si el acto económico se descarta simplemente como resultado del egoísmo individual, y la disciplina económica se abstiene de considerar sus implicaciones sociales más amplias y profundas, se niega la presencia de cualquier ``visión'' en el establecimiento de los términos del análisis económico. Esto puede explicar el papel relativamente menor que desempeñaron los acontecimientos económicos en las principales obras históricas de Croce.

Esta larga digresión sobre la relación difícil, casi dolorosa entre Croce y la disciplina económica debería ayudar a definir un enfoque de nuestro tema de las ``filosofías económicas'' a lo largo de algunos conceptos:

\begin{enumerate}
\def\labelenumi{\arabic{enumi}.}
\tightlist
\item
  Cualquier teoría económica que aspire a escalar por encima del nivel de la ``economía vulgar'' debe basarse en postulados que en su mayoría son de carácter no económico.
\item
  no hay una visión única, exclusiva y correcta que sustente la teoría económica, y como consecuencia no puede haber una única teoría económica superior. Sólo una visión que responda al avance del espíritu humano en un tiempo y lugar determinados tiene derecho a ser definida como ``liberal'';
\item
  como consecuencia, el liberalismo puede comprender diversas organizaciones sociales y económicas, cuerpos de instituciones, modos de producción, que no son necesariamente válidos ``en todo momento y lugar'' (para usar, nuevamente, la terminología de Ricardo, ver Capítulo 1 ), pero relacionados con las circunstancias históricas específicas en las que surgen y se desarrollan (listas, por así decirlo, para ser reemplazadas por otras cuando llegue el momento).
\item
  la conexión entre el liberalismo ético y el liberalismo económico tiene, por tanto, sólo un carácter histórico. La dificultad surge cuando el liberalismo económico reivindica para sí el valor de la ley suprema de la vida social e insiste en estar cerca del liberalismo ético, que, a su vez, se considera a sí mismo como la ley suprema de la vida social. ``Dos leyes de igual rango y sobre un mismo asunto son demasiadas, una es redundante''. Dado que el liberalismo ético rechaza cualquier regulación autoritaria de la actividad económica como una mortificación de la inventiva humana, avanza en la misma línea del liberalismo económico. Es posible ---escribe Croce--- que el liberalismo apruebe muchos de los preceptos del liberalismo económico, que han traído tantos beneficios a la civilización moderna; pero esta aprobación se da por motivos éticos, no económicos.26 sino más bien si es ``liberal''; no si es cuantitativamente productivo, sino cualitativamente valioso. 27
\end{enumerate}

El verdadero hombre liberal, si seguimos el razonamiento de Croce, tiene en mente un valor ético superior - ``universal'' - que, según diversas circunstancias históricas, puede subsumir cualquier teoría económica que sea la más adecuada para afrontar la situación contingente que tiene. para afrontar y resolver. 28

El enfoque intelectual de Croce está lejos del del economista, pero las cuestiones filosóficas que plantea responden a necesidades que también son relevantes para la disciplina económica. 29

La afirmación de los Estados totalitarios en las décadas posteriores a la Primera Guerra Mundial, llevó a Croce a nuevas reflexiones sobre el liberalismo, su fundamento ético y su relación con la disciplina económica. Dio pruebas de lo que debe ser un medio liberal en las circunstancias políticas y económicas específicas de esos años. En 1925, Croce observó: ``Es con particular insistencia que escuchamos en estos días que la idea del liberalismo ya se ha extinguido, y que el mundo de hoy y de mañana pertenece a la oposición y lucha entre dos tendencias fundamentales, el socialismo y el comunismo en uno. de lado, y reacción o fascismo del otro''. 30Croce no entró en el dominio de la economía, pero reconoció que la superioridad de la idea liberal radicaba precisamente en la ``necesidad de mantener, en la medida de lo posible, un campo de juego libre para las fuerzas espontáneas e inventivas de los individuos y los grupos sociales, porque sólo de estas fuerzas se puede esperar cualquier progreso mental, moral y económico''. Y agregó, para confirmar al liberalismo como una idea ética que está por encima de cualquier arreglo político, económico o institucional: ``A un liberal verdaderamente consciente le suena imposible adherirse a ideales autoritarios y reaccionarios, o comunistas, porque el liberalismo los incluye a todos, dentro de sus límites aceptables''. 31

Vale la pena recalcar nuevamente que, según él, el liberalismo, como idea ética, debe identificarse con la evolución dialéctica de la historia humana misma: dialéctica, porque es capaz de incluir las diversas organizaciones de la economía, la política, el derecho. El liberalismo bien puede aceptar organizaciones de propiedad y producción diferentes y cambiantes. El liberalismo tout court y el liberalismo económico, o ``liberismo'', bien pueden coexistir, pero su conexión es de carácter histórico relativo. El liberalismo económico no puede verse como una regla de vida suprema; por el contrario, se puede hablar de un ``socialismo liberal'', cuando las medidas que la doctrina económica califica como ``socialistas'' son coherentes con la visión liberal, definida como antes.32 {[}``di che lacrime grondi e di che sangue''{]}. 33

Secciones 2 - 5 de este capítulo tratan formas de liberalismo que hacen hincapié en el contenido ético de la idea, y adoptar una postura crítica frente a dejar hacer. Las secciones 6 y 7 se centran en el aspecto explícitamente libertario del liberalismo; Sin embargo, alejándose de la doctrina positivista relacionada con las ciencias naturales de los economistas neoclásicos, estos pensadores subrayan el valor moral del individualismo económico. La sección 8 está dedicada a esa peculiar forma de liberalismo que es el ordoliberalismo ---en su mayoría alemán---.

\hypertarget{el-estatismo-de-rathenau-frente-al-liberismo-de-einaudi}{%
\section*{El estatismo de Rathenau frente al ``liberismo'' de Einaudi}\label{el-estatismo-de-rathenau-frente-al-liberismo-de-einaudi}}
\addcontentsline{toc}{section}{El estatismo de Rathenau frente al ``liberismo'' de Einaudi}

Con la guerra aún en curso, un singular industrial, estadista e intelectual de Alemania, Walther Rathenau, observó que ``de la catástrofe económica mundial más grande de la historia no podemos deshacernos de los arreglos financieros manchados y los viejos dispositivos de expiación como préstamos, derechos y monopolios''. 34Por tanto, no debemos dar un paso atrás en ``interferir en la libertad industrial y los derechos personales, en la colaboración del Estado y la igualdad social, incluso en los trastornos sociales y geográficos''. Rathenau criticó el sistema económico liberal y se basó en la experiencia de la economía de guerra planificada, es decir, en ``el sistema de disciplina económica del Estado''. Señaló críticamente que los liberales querían abolir esa ``disciplina económica'' una vez que terminara la guerra para que ``la libertad económica y la superlibertad pudieran ser restauradas, a veces confiando en las necesidades de la empresa privada para encontrar direcciones para nuestra vida colectiva''. 35En cambio, buscaba la especialización y la coordinación productiva, lo que podría eliminar las duplicaciones y el desperdicio en la producción de la misma mercancía, eliminar gradualmente la competencia desenfrenada, presagio de gastos inútiles ``para atraer clientes''. 36 ``No es cierto - escribió Rathenau - que una ansiedad desesperada por ser competitivos nos hace más fuertes \ldots{} {[}S{]} e que, en el año siguiente, los derrotados intentarán ansiosamente vencer al ganador, sería mejor para ellos alcanzar un acuerdo, en lugar de pelearnos sobre nuestros hombros la lucha por la supremacía en ser capaces e inventivos''. 37También creía en el proteccionismo en el comercio de productos básicos y en las cuotas de importación. Su programa debería haber sido promulgado a través de sindicatos de profesiones y sindicatos industriales, que deben ser corporaciones reconocidas y supervisadas por el Estado con amplios poderes de intervención. 38 ``¡Alguien dirá - agregó Rathenau - que estos son los viejos gremios y las viejas corporaciones de artes y oficios!'' En absoluto, respondió. Los sindicatos son un ``colectivo'' de producción, ``donde todos los miembros están unidos orgánicamente \ldots{} unidos en una unidad viva \ldots{} no una confederación sino un organismo''. 39 Y, sin embargo, ``La nueva economía no será una economía de Estado, sino una economía privada sujeta al juicio de los poderes públicos, una economía privada\ldots{} que necesitará la colaboración del Estado''. 40Lo que tenemos aquí con Rathenau es una visión pro-estatal, proteccionista y corporativista, que es la antítesis del pensamiento liberal. Como se observará fácilmente, su filosofía es similar a la de List (véase el capítulo 1 ) y al naciente corporativismo italiano (véase el capítulo 3 ).

Por tanto, no es casualidad que se pueda encontrar una feroz crítica al libro de Rathenau en la reseña escrita poco después de su publicación por el economista liberal Luigi Einaudi. Refiriéndose al libro de Rathenau, rechazó la idea de que la guerra en curso era ``la hoguera del viejo mundo económico'', y observó que el pensamiento de Rathenau era ``poco claro, vago, indefinido''. ``La guerra actual - escribe Einaudi - no es diferente de tantas otras guerras, salvo la adopción de nuevas técnicas, \ldots{} y sus consecuencias serán similares \ldots{} La cultura de Rathenau no es realmente profunda, necesita creer en una palabra regeneración \ldots{} y él piensa ser el profeta de este nuevo orden económico''. El liberalismo de Einaudi no puede aceptar la creación, bajo la dirección del Estado, de sindicatos profesionales e industriales, o cárteles. ``La voluntad de operar de acuerdo con el propio deber, de actuar con sabiduría,41

Los principales puntos de la visión económica de Einaudi se pueden resumir de la siguiente manera:

\begin{quote}
un sistema económico basado en la libre lucha de los agentes económicos;

el componente moral de esta lucha;

el rol del Estado en permitir que esta lucha se desarrolle con equidad, sin injerencias pero asegurando la igualdad de oportunidades para todos los actores;

una hacienda pública diseñada para hacer del Estado un factor de producción.
\end{quote}

Como veremos, el liberalismo de Einaudi se acerca bastante al de Hayek, en su insistencia en el valor ético de la libertad individual, y al de los ordoliberales, al enfatizar el fuerte papel del Estado.

Podemos considerar por separado los puntos de su filosofía económica.

Respecto al primer punto, Einaudi puede ser visto como perteneciente a la corriente de economistas adheridos a la revolución marginalista neoclásica y, en realidad, no aporta aportes teóricos relevantes a ese esquema lógico, pero su interés no está tanto en el nivel de equilibrio alcanzado por una economía que opera en un régimen de libre competencia, como en la forma en que se alcanza el equilibrio económico. 42 Observa que el equilibrio general walrasiano del sistema económico no puede ser el resultado del funcionamiento espontáneo de los agentes económicos, cuya interacción tiene una explicación mecánica, matemáticamente expresada, sino que se alcanza a través de una lucha interminable de esos agentes económicos, individuos y empresas.

El segundo punto significa que esta lucha tiene un sentido moral, es vista como expresión de libertad. Para dar ejemplos, con referencia a las fuerzas en conflicto de trabajadores y empresarios, Einaudi escribe: ``Un industrial es liberal si cree en su propio espíritu de iniciativa\ldots{} es socialista cuando pide deberes protectores por parte del Estado. Un trabajador es liberal si se une a sus compañeros de trabajo para crear un instrumento común de cooperación o defensa; es socialista si invoca al Estado un privilegio exclusivo para proteger su organización, o pide una ley o sentencia judicial que prohíba las obras a los rompehuelgas \ldots{} Liberal es aquel que cree en la mejora material o moral lograda a través del esfuerzo voluntario, el sacrificio y voluntad de trabajar en armonía con los demás; socialista es el que quiere imponer la mejora a través de la fuerza''.43 Por lo tanto, favorece los sindicatos de libre creación como instrumento de esa lucha, pero rechaza enérgicamente los subsidios o la protección que otorga el Estado a uno u otro lado de la lucha. Este aspecto ético de la libre competencia acerca el pensamiento de Einaudi a la Escuela Clásica de Adam Smith; suWeltanshauungno puedecompararsecon el enfoque ``científico'' que mantiene la ética fuera de la economía.

Tercer punto: Einaudi es consciente del papel central, aunque cuantitativamente limitado, del Estado en la economía. Para ello, el propio Estado debe estar en sintonía con el sistema económico libre: no puede ser una entidad autoritaria. Tenemos aquí una inversión del esquema de Croce: según Croce, un Estado liberal puede ser compatible con sistemas económicos que no necesariamente deben identificarse con el liberalismo económico. Según Einaudi, la idea liberal es seguramente ética, pero una sola idea, porque no se puede hacer una distinción entre liberalismo ético y económico, son exactamente el mismo concepto: este es el principal punto de contraste con Croce.

El papel del Estado no es solo de no injerencia en la lucha, sino también de preservar y mejorar la competencia justa. En cuanto al mercado, la intervención del Estado consistirá en primer lugar en combatir los monopolios privados y transformar los monopolios naturales en servicios públicos. En cuanto a los agentes económicos, el Estado tiene un rol proactivo, que es otro aspecto de la visión del liberalismo como lucha: como liberal, Einaudi no puede aceptar el concepto de igualdad absoluta de los agentes (está claro que está más cerca de La visión moderada de Adam Smith, que a la visión igualitaria de la Ilustración radical francesa); pero piensa que la prioridad de un Estado es asegurar la igualdad desde el principio, es decir, en términos de oportunidades que cualquier persona debe tener. Solo este tipo de igualdad puede permitir que una lucha se libere en términos justos. Significa ``bajar los picos'' mediante la tributación progresiva y ``subir los mínimos'' mediante la legislación social, en lo que respecta al salario mínimo, la limitación de los horarios de trabajo, la prohibición del trabajo infantil, la protección extendida a los trabajadores no sindicalizados, el seguro contra accidentes de trabajo, la discapacidad y pensiones de jubilación. Esta legislación social, lejos de estar en contradicción con el Estado liberal, es la condición previa para estar más cerca de esa hipótesis abstracta de la libre competencia que es el eje principal del pensamiento económico liberal.44

El enfoque de Einaudi parece aquí cercano al pensamiento ordoliberal, madurando exactamente en ese período en Alemania. Los ordoliberales pensaban que, para alcanzar ese esquema abstracto de libre competencia, la competencia de mercado efectiva, lejos de identificarse con el laissez-faire, requiere un conjunto de reglas destinadas a colocar a individuos y empresas al mismo nivel en términos de oportunidades para competir.

El cuarto punto es el énfasis que pone Einaudi en el papel del Estado como entidad económica propiamente dicha, cuya actividad se examina tanto como agente productivo, productor de bienes colectivos, como agente optimizador, orientado a la maximización de ingresos. 45 Su suposición es que las teorías de las finanzas públicas deben utilizar las herramientas analíticas familiares en el estudio del funcionamiento del sector privado. 46Es uno de los principales exponentes de un grupo que creó una ``Escuela Fiscalista Italiana'' sobre finanzas públicas. Su investigación contribuyó al crecimiento de esta disciplina como rama de la economía, lejos de un enfoque puramente contable, las finanzas públicas tomaron desde diferentes caminos. Por un lado, los economistas keynesianos miraron el problema desde una perspectiva diferente, y con diferentes resultados, basándose en una fuerte intervención proactiva del Estado en la configuración de la economía (las finanzas públicas ``funcionales'', bastante alejadas de las de Einaudi `` liberist''vista (véase cap. 5 en este capítulo). por otro lado, también la teoría libertaria de Buchanan de la`elección pública', basado en un papel muy limitado del Estado, se ha visto como estrechamente conectada a la escuela italiana (véase Capítulo 4 ).

\hypertarget{las-dudas-de-pigou}{%
\section*{Las dudas de Pigou}\label{las-dudas-de-pigou}}
\addcontentsline{toc}{section}{Las dudas de Pigou}

La adopción de un concepto ético del liberalismo es, en efecto, necesaria para poner bajo sus alas extendidas las teorías de los economistas que otorgan al Estado un papel central adicional en la configuración del sistema económico. Este rol puede tener diferentes acentos y dimensiones: puede estar relacionado con una alineación de los costos privados y sociales en la producción de bienes y servicios a través de la tributación (como en Pigou), con la gestión macroeconómica (Keynes), con una amplia provisión de servicios a través de gasto público (Beveridge), Esencialmente, el Estado pone remedio a las fallas del mercado, sin su necesaria implicación en la propiedad de los medios de producción, esencia del socialismo. Es preferible la experiencia de la planificación centralizada en lo que sigue siendo una sociedad capitalista, como en la Gran Bretaña de los años treinta, al socialismo real de la Rusia soviética (Pigou); la gestión de la demanda no implica nacionalizaciones (Keynes); la necesidad del socialismo ---como propiedad estatal del capital--- aún no se ha demostrado (Beveridge). Estos tres pensadores son ejemplos notables de la metamorfosis de una idea que sigue siendo, esencialmente, una idea liberal.

Puede haber cierta renuencia a incluir a Arthur C. Pigou en la lista de pensadores liberales del siglo XX. Por un lado, es ajeno a cualquier especulación de la filosofía social y subraya que su investigación es esencialmente práctica: su impulso no es un impulso filosófico, ``el conocimiento por el conocimiento \ldots{} ¿Sobre qué base filosófica generalizaciones de este tipo {[}el leyes de la ciencia económica{]} resto, aquí no nos preocupamos de indagar''. La razón de la investigación del economista es ``el estudio del comportamiento social de los hombres {[}que conduce a{]} resultados prácticos de mejora social'', bajo ciertas circunstancias económicas. Uno está casi tentado a dejar de lado a Pigou cuando se trata de ``filosofías económicas''. Además, ``expone el argumento de su Economía del bienestaren términos de excepciones a la regla de que el laissez-faire asegura la máxima satisfacción; no cuestionó la regla''. 47

Por otro lado, sin embargo, está completamente inmerso en esa corriente de pensamiento que no está satisfecho con apoyarse en los esquemas neoclásicos ortodoxos que prevalecen en su entorno de investigación. Su realismo práctico le hace desconfiar de cualquier esquema matemático abstracto, puro y ``científico''.

Desde esta perspectiva, Pigou ``ayudó {[}ed{]} a los economistas a convertir molestas controversias políticas en problemas técnicos''. Él ---y Keynes--- ``establecieron a los economistas como un conjunto de herramientas para ser utilizado por los responsables de la formulación de políticas y fueron pioneros en el papel de los asesores económicos del gobierno''. 48

En Cambridge, donde ocupa la cátedra de economía política que perteneció a Alfred Marshall ---en una aparente continuidad de hombres y doctrinas--- su enfoque realista lo lleva a mirar un ``bienestar económico'' con ojos diferentes al utilitarismo de esos esquemas. Como Keynes, Pigou se distancia de la economía del laissez-faire. En The Economics of Welfare of 1920 ---su obra magna--- muestra cierto escepticismo en la suposición optimista según la cual ``si solo el gobierno se abstiene de intervenir, automáticamente hará que la tierra, el capital y el trabajo de cualquier país se distribuyan de tal manera que produzcan una producción mayor y, por lo tanto, más bienestar económico que el que podría lograrse con cualquier otro arreglo que no sea el que surge `naturalmente'\,'' 49. Incluso si el propio Smith calificó esta libertad natural, al admitir la acción del Estado a las extensiones limitadas, no llegó a partir de darse cuenta de que, Pigou escribe 50 - ``el funcionamiento del propio interés es generalmente beneficioso, no debido a una coincidencia natural entre el interés propio de todos y para el bien de todos, sino porque las instituciones humanas están dispuestas de modo que obliguen al interés propio a trabajar en direcciones en las que resulte beneficioso''. 51

Los editores de la edición 2013 de Palgrave de The Economics of Welfare 52 señalan que es incorrecto asumir este libro como la inspiración intelectual del Estado de Bienestar británico, tal y como se estableció después de la Segunda Guerra Mundial. 53 En este sentido, no hay duda de que Beveridge debe ser visto como su principal engendrador. Pero esa referencia a que las ``instituciones humanas'' actúan de manera beneficiosa es una apertura a la relevancia que debe darse al interés público, que corrige el interés propio del individuo. ¿Qué hacen (o deberían hacer) estas instituciones?

Pigou relaciona el bienestar económico con el concepto de ``dividendo nacional'', por lo que se refiere al total de ``servicios objetivos, algunos de los cuales se prestan en forma de mercancías, otros en forma directa'', puestos a disposición del público. Es el volumen de la producción neta corriente: la adición neta a los recursos de la comunidad disponibles para el consumo o para la retención del stock de capital, después de tener en cuenta el despilfarro del stock de capital real existente al inicio de cada período. 54

Sin embargo, la prestación de estos servicios se ve alterada por costos (o beneficios) que no se reflejan en su precio. Las curvas de oferta y demanda específicas de cualquier producto o servicio ---en las que se centró la atención de los economistas neoclásicos--- no pueden reflejar esos costos (o beneficios). Aquí está su desapego de su maestro, Marshall, y la contribución duradera de su enfoque ``práctico'': lo que ahora se ve como la extraordinaria actualidad de Pigou es su teoría de los costos y beneficios externos causados \hspace{0pt}\hspace{0pt}por la actividad económica. 55 Estas ``externalidades'' alteran la relación costo / beneficio, visto solo desde una perspectiva privada. 56 El valor total de una mercancía o un servicio debe desglosarse en dos componentes: valor privado y valor social. 57De esta manera, abandonamos el concepto de valor como relacionado con un bien específico y con un solo agente económico, el valor social debe estar relacionado con toda una comunidad. El valor social mide el costo / beneficio generado por la empresa fuera de sí misma {[}el ejemplo típico en el discurso actual es la contaminación de las fábricas o, si así lo preferimos, el cambio climático inducido por la industria. Por otro lado, los nuevos bienes digitales, como motores de búsqueda en Internet, tienen beneficios enormes e inconmensurables para los consumidores, un excedente del consumidor 58{]}. Si la externalidad es un costo, la producción de la empresa es mayor de lo que sería si ese costo externo fuera internalizado; ocurre lo contrario si la externalidad es un beneficio (la producción es menor que en caso contrario). Esta internalización, en ambos casos, alinearía los valores privados y sociales de la producción. Ante la presencia de externalidades, la mejor manera de mitigar las diferencias entre los valores privados y sociales es que el Estado utilice ``dispositivos legales coercitivos para dirigir el interés propio hacia los canales sociales'': incentivar la reducción / aumento de las actividades interesadas, y la forma más obvia es la de impuestos / subsidios. 59 Como veremos (Capítulo 4 ), James Buchanan enfrentará el mismo problema desde una perspectiva individualista diferente: por ejemplo, la elección entre contaminación y crecimiento económico por un lado, y un medio ambiente más limpio por el otro, debe exigirse exclusivamente al consenso individual, no a el estado.

Si pensamos en la abrumadora importancia de los problemas relacionados con el medio ambiente en el mundo actual, la contribución de Pigou no puede subestimarse.

Es importante destacar que incluso si el bienestar económico (que es ``económico'' porque puede medirse con una vara de dinero) no coincide con el bienestar más general, que también se compone de componentes no económicos, es propicio, en un juicio de probabilidad ---Para realzar este último. A pesar de la renuencia de Pigou a entrar en el lado filosófico / institucional, la inferencia que se puede hacer de esta conexión es que las instituciones políticas y económicas coherentes con el bienestar general también deben ser coherentes con el bienestar económico (aunque Pigou escribe que su enfoque es una ciencia positiva de lo que es y tiende a ser, pero no normativo, de lo que debería ser. 60El economista no ``defiende ni se opone a ningún programa político''. Se podría agregar que el punto de vista de Pigou es una evidencia de la renuencia de algunos economistas a verse a sí mismos como ``economistas normativos'' en lugar de ``científicos positivos'').

Por tanto, el velo de la economía es particularmente grueso en la economía de Pigou. Pero su filosofía emerge abiertamente en Socialismo versus capitalismo. 61 ¿Cuál de estos dos sistemas políticos es más propicio para la igualación de los valores sociales y privados de los productos? Comienza con su advertencia habitual: ``No es asunto de un economista académico, ni está dentro de su competencia, defender o contra cualquier programa político. Pero es asunto suyo, y debe ser de su competencia, exponer de forma ordenada las consideraciones dominantes, en la medida en que sean económicas, relevantes para el argumento''. 62El lector no encontrará nada doctrinario en su libro: el materialismo dialéctico marxista o la inevitabilidad del conflicto de clases sociales. Sólo hay cuestiones prácticas: de la distribución de la riqueza y el ingreso, particularmente en presencia de ``una clase que vive de la propiedad, que no sólo no necesita trabajar, sino que de hecho no hace nada \ldots{} el espectáculo de esta clase es repulsivo para las personas del público espíritu'' 63 ; y de la asignación eficiente de recursos entre diferentes sectores de producción. ¿Es necesario el socialismo para reducir las desigualdades e ineficiencias?

Puso el tema en el contexto de la política británica. Sobre la distribución, señala que si se introdujera el socialismo mediante la confiscación de los medios de producción, el Estado podría asegurar una gran parte de los ingresos que ahora fluyen en gran parte a los ricos, y retenerlos o redistribuirlos entre los pobres, y se reduciría la desigualdad. Pero si el Partido Laborista británico decidiera comprar esos medios a un valor justo, el socialismo no tendría ningún efecto; simplemente, los accionistas se convertirían en rentistas. 64 La tributación, principalmente a través del impuesto sobre la herencia y el impuesto sobre la renta altamente progresivo, sería entonces el instrumento de redistribución; sin embargo, se vería obstaculizado por el temor a dañar la acumulación de capital. Solo un socialismo confiscatorio sería el único remedio efectivo para reparar los ingresos y la riqueza.

Sobre la asignación eficiente de recursos, Pigou utiliza su concepto de externalidades: señala que, bajo el capitalismo, las externalidades de costos, por las cuales los empleadores arrojan parte de sus costos a los de afuera, les permiten producir más de lo que producirían si estos costos fueran internalizados: el costo no es una carga para el empleador. Pero, en la práctica, la evaluación de este costo y de la consiguiente tributación es difícil y una autoridad central de planificación no tendría una ventaja comparativa en este sentido.

En conclusión ---dice Pigou--- es preferible un concepto vago de socialismo al capitalismo; pero de nuevo de manera pragmática, si tomamos como socialismo el sistema de la Rusia actual, y como capitalismo el sistema británico, donde el capitalismo coexiste con la planificación central socialista, esta última es preferible, sin embargo -agrega- con un uso extensivo de impuestos y una nacionalización de grandes sectores de la industria británica (el Banco de Inglaterra incluyó, por así decirlo, en 1946).

\hypertarget{el-liberalismo-de-keynes}{%
\section*{El liberalismo de Keynes}\label{el-liberalismo-de-keynes}}
\addcontentsline{toc}{section}{El liberalismo de Keynes}

El librito Economic Philosophy , de Joan Robinson, 65 es un excelente resumen de las premisas ideológicas del liberalismo económico en el siglo XIX, preparando así el escenario para las nuevas ``reglas del juego'' previstas por la ``revolución keynesiana''. ``La Teoría General sacó a la luz el problema de la elección y el juicio que los economistas neoclásicos habían logrado sofocar. La ideología para acabar con todas las ideologías se vino abajo. La economía se convirtió una vez más en Economía Política'' 66 \ldots`` Keynes devolvió el problema moral a la economía al destruir la reconciliación neoclásica del egoísmo privado y el servicio público ''. 67

De hecho, está claro que, para Keynes, el aspecto ético es fundamental para la disciplina económica, y esto es suficiente para mantenerla separada de las ciencias naturales. En una carta a Roy Harrod, señala que ``En química, física y otras ciencias naturales, el objeto del experimento es completar los valores reales de las diversas cantidades y factores que aparecen en una ecuación o una fórmula; y el trabajo cuando esté hecho es de una vez por todas. En economía este no es el caso y convertir un modelo en una fórmula cuantitativa es destruir su utilidad como instrumento de pensamiento \ldots{} la pseudo-analogía con las ciencias físicas lleva directamente en contra del hábito de la mente que es más importante para un economista. propio de adquirir \ldots{} se trata de la introspección y de los valores \ldots{} de los motivos, las expectativas, la incertidumbre psicológica''.68

Es interesante observar que las ``primeras creencias'' de Keynes, remontándose a sus años universitarios, ya muestran un ``escape de la tradición benthamita \ldots{} fue este escape de Bentham, unido al insuperable individualismo de nuestra filosofía, lo que ha servido para proteger a todos nosotros desde la reducción final ad absurdum del benthamismo conocido como marxismo'' 69 ): un desapego paulatino del pensamiento económico dominante que le llevaría a preguntarse, mucho antes de su Teoría General , si él mismo podría calificarse de`` liberal". 70

En un artículo de 1933 poco citado, 71Keynes escribe: ``Fui educado, como la mayoría de los ingleses, para respetar el libre comercio no solo como una doctrina económica que una persona racional e instruida no podría dudar, sino casi como parte de la ley moral \ldots{} Sin embargo, la orientación de mi mente es cambiado {[}y{]} se modifican mis antecedentes de teoría económica''. Entonces, escribe, el proteccionismo del siglo XIX no debe verse como una mancha en la eficiencia y el buen sentido (se parece a List). Muchos países (Rusia, Italia, Alemania y, en parte, Estados Unidos y Gran Bretaña) se están embarcando en una variedad de experimentos político-económicos que van más allá de la maximización de beneficios como guía para las decisiones de inversión. Se revaloriza la autosuficiencia nacional y, con ella, protección arancelaria para levantar el desempleo y los controles de capital a fin de evitar que una tasa de interés demasiado alta para atraer inversiones extranjeras vaya en detrimento de las políticas internas. Los viejos liberales pensaban que servirían no sólo a la supervivencia de los más aptos, sino también a ``la gran causa de la paz, la gloriosa fertilidad de la mente libre contra las fuerzas del privilegio, el monopolio y la obsolescencia''. Keynes mira más bien con cierta simpatía los nuevos experimentos económicos, el nacionalismo económico, la autosuficiencia, aunque ve tres peligros: la estupidez del doctrinario (``Mussolini quizás está adquiriendo muelas del juicio''), la prisa, la intolerancia. Existe un peligro claro, o casi seguro, de que estas economías no puedan conectarse con la idea liberal, que es la dificultad para aceptar sus políticas sin una conversión al autoritarismo.

Con respecto a la competencia, la visión crítica de Keynes ya se había expresado en El fin del laissez-faire : competencia como ``un estado de cosas en el que la distribución ideal de los recursos productivos puede lograrse a través de individuos que actúan de manera independiente mediante el método de prueba y error en tales condiciones. una forma en que los individuos que se mueven en la dirección correcta destruirán mediante la competencia a los que se mueven en la dirección equivocada \ldots{} Es un método para llevar a los más exitosos a la cima mediante una lucha despiadada por la supervivencia, que selecciona a los más eficientes por la quiebra de los menos eficientes'' 72 (Esto suena a Rathenau, o Schumpeter. 73 )

Esas nuevas ideas no sólo echaron raíces en países que se estaban moviendo hacia regímenes abiertamente autoritarios, sino que también ---como señaló Keynes en su artículo--- tuvieron una influencia en países de tradición liberal bien establecida. Sin embargo, en lo que respecta al comercio, el proteccionismo de los Estados Unidos no es nuevo. El país había crecido muy rápido entre el final de la Guerra Civil y 1917 con la ayuda de aranceles estrictos. El atractivo del aislacionismo, después de la Primera Guerra Mundial, fue fuerte. La introducción del arancel Smoot Hawley en 1930 erigió importantes barreras a las importaciones, a pesar de un fuerte superávit comercial. Con respecto al dinero, una postura monetaria muy estricta de la Reserva Federal alentó enormes entradas de capital, sin embargo, esterilizado por el banco central con una política de ``dinero administrado'' que contrastaba con la ``regla del juego'' que el patrón oro requería para los países con grandes superávits. El abandono de su paridad oro por la libra esterlina en 1931, y luego por el dólar estadounidense en 1933, marcó el predominio de los intereses nacionalistas y el punto crítico del colapso del patrón oro. El Reino Unido adoptó la Preferencia Imperial en 1932.

Frente a los trastornos económicos masivos y al desempleo, y a los fuertes ataques a la filosofía económica liberal, las reacciones de la disciplina económica fueron dobles: por un lado, la causa principal se vio en una demanda insuficiente de bienes y servicios que el sistema económico es capaz de producir, y por lo tanto en una ``falla de mercado''; en el lado opuesto, se atribuyó a las externalidades y la rigidez de los precios un impedimento para que un mercado, por lo demás eficiente, funcionara correctamente. La primera causalidad es la sustancia de la ``revolución'' macroeconómica keynesiana; este último se queda atrás de una serie de explicaciones que, más allá de las teorías neoclásicas, exploran nuevas vías de un estricto liberalismo económico.

Keynes no tiene nada que objetar a la teoría clásica, dentro de sus propios límites: ``Considerada como la teoría de la empresa individual y de la distribución del producto de una determinada cantidad de recursos, la teoría clásica ha hecho una contribución al pensamiento económico que no puede ser impugnada''. 74 Pero Keynes quiere ir más allá de esos límites, y su Teoría general (1936) es importante, ante todo, por el salto cualitativo que da con respecto a las teorías económicas entonces imperantes, en particular a la teoría neoclásica.

Hemos mencionado anteriormente los dos enfoques neoclásicos principales (walrasiano y marshalliano) para la definición de un equilibrio económico, el primero centrado en el sistema económico general y el segundo en muchos mercados individuales. Ambos muestran que la producción total resulta de la interacción de agentes individuales, movidos por sus utilidades y costos marginales. Keynes pasa de un análisis centrado individualmente al comportamiento agregado de grupos de agentes económicos (consumidores, inversores, rentistas\ldots), cuya demanda no está relacionada con funciones de utilidad individuales. ``Es la sustitución de la determinación de precios como tarea esencial de la economía, por la tarea previamente inexistente de determinar el nivel de demanda agregada''. 75El enfoque agregado de Keynes reemplaza al enfoque sumativo para definir el nuevo objeto de la economía. Se pierde la economía ``científica'', ``pura'' (basada en la utilidad marginal, medida matemáticamente). Se recordará que, según Ysidro Edgeworth, el economista neoclásico, ``el principio utilitario de que la política debe dirigirse al mayor bien para el mayor número requiere la suma de la felicidad de individuos separados''. 76

La teoría macroeconómica es la rama de la disciplina económica que analiza todo el sistema económico, como una entidad diferente de la suma total de sus componentes individuales. La invención de la macroeconomía requiere insertar la sociedad y su estructura en el estudio de la economía: una convicción compartida tanto por Marx como por Keynes. 77

La nuestra no es una historia del pensamiento económico, nuestro interés se centra en qué filosofía social descansa el pensamiento de Keynes. Sobre este tema, Keynes no fue explícito, y esto puede explicar por qué todavía se le considera, según una opinión, como un colectivista reacio, o como un socialista liberal según otros, o incluso de otras formas. 78 Es de su análisis como economista que Keynes deriva elementos de su filosofía social, y esto ocurre sólo en el capítulo final de su Teoría general. 79 Por tanto, es necesario dedicar algunas palabras a las principales características de su teoría. 80

Los principales problemas de la sociedad en la que vivimos son, según Keynes, la incapacidad de asegurar el pleno empleo y la desigualdad en la distribución de la riqueza y el ingreso. 81Por tanto, Keynes incide en los dos temas fundamentales de la disciplina económica en el siglo XX: la producción y distribución de una determinada producción, con un nuevo énfasis en el segundo. Con respecto al empleo, Keynes escribe que depende de los ingresos que el empresario espera recibir de la producción, que busca maximizar con respecto a sus costos. Por lo tanto, aumenta la producción y el empleo solo en la medida en que el aumento de los costos esté más que compensado por una mayor demanda de los consumidores. Los economistas clásicos piensan que la demanda del consumidor es necesariamente igual al ingreso recibido por los factores de producción empleados por el empresario, de modo que la producción y el empleo pueden expandirse pari passuhasta llegar a una situación de pleno empleo. Pero ---observa Keynes--- no se consume toda la producción (la propensión a consumir es menor que ``uno''). Un crecimiento en el empleo puede ocurrir solo cuando el monto de las inversiones puede absorber el exceso de lo que se produce sobre lo que la comunidad decide consumir.

La demanda de inversiones por parte del empresario no se suma a la demanda de los consumidores de forma que se asegure automáticamente el pleno empleo. La demanda de inversión depende: (1) del rendimiento esperado de la inversión; (2) Sobre su costo, es decir, sobre la tasa de interés. Cuanto menor sea el rendimiento esperado y cuanto mayor sea la tasa de interés, menor será la demanda de inversiones. Esto bien puede estabilizarse en un nivel inferior al que es coherente con el pleno empleo. La demanda agregada, de consumo e inversión, asociada al pleno empleo es, por tanto, ``un caso especial, sólo realizado cuando la propensión a consumir y el incentivo a invertir mantienen una relación particular entre sí'', ya sea por accidente o por diseño. Es deber del Estado actuar de manera que esto suceda por diseño, si ese ``accidente'' no ocurre.

La otra cara del problema es la desigualdad social, ``la paradoja de la pobreza en medio de la abundancia''. 82Está relacionada con la diferente propensión a consumir entre las diferentes clases sociales: esta propensión disminuye con el aumento de los ingresos y la riqueza. Los ricos gastan relativamente menos en bienes de consumo. Particularmente en una sociedad que ya es rica, que ya tiene un gran stock de riqueza disponible, esa propensión es relativamente débil, y la debilidad de la demanda del consumidor es un incentivo para reducir la demanda de inversiones: la demanda agregada (para consumo e inversión) es, en consecuencia, bajo, y seguirá una reducción de la producción y el empleo. En resumen, el crecimiento de la inversión y del empleo no depende de una baja propensión a consumir (``lejos de depender de la abstinencia de los ricos''), es decir, de un mayor ahorro, como explican los economistas clásicos, se ve obstaculizado por eso. Hasta que se alcance el pleno empleo,

En el capítulo final de su Teoría general, Keynes pasa a sus ``notas de filosofía social''. La primera inferencia de su teoría está relacionada con la desigualdad. Esto se ha abordado a través del esquema de tributación directa, que sin embargo ha encontrado severos límites a la evasión tributaria y sobre todo en la opinión --- errónea, como se vio anteriormente --- de que una alta tributación desalienta la acumulación de ahorros que se considera necesaria para promover las inversiones. ``El crecimiento de la riqueza, lejos de depender de la abstinencia de los ricos, como se supone comúnmente, es más probable que se vea obstaculizado por ella''. 83

Su nota de filosofía social es: ``Creo que hay una justificación social y psicológica para desigualdades significativas de ingresos y riqueza, pero no para disparidades tan grandes como las que existen hoy. Hay actividades humanas valiosas que requieren el motivo de hacer dinero y el entorno de la propiedad privada de la riqueza para su plena función \ldots{} Es mejor que un hombre ejerza una tiranía sobre su saldo bancario que sobre sus conciudadanos \ldots{} Pero no es necesario para el estímulo de estas actividades \ldots{} que el juego debería jugarse con apuestas tan altas como en la actualidad. Las apuestas mucho más bajas servirán igualmente bien, tan pronto como los jugadores se acostumbren a ellas''. 84

La segunda inferencia importante, que también tiene consecuencias sobre la distribución de la riqueza, se refiere a la tasa de interés. No es correcto pensar que una tasa alta conduce a mayores inversiones a través de mayores ahorros. Dado que el interés es un factor de costo para el empresario, se realizarán cantidades adicionales de inversión hasta que la eficiencia marginal del capital, definida como el rendimiento esperado de capital adicional, sea igual a la tasa de interés del mercado. 85De ello se deduce que esta tasa debe reducirse hasta el punto en que la rentabilidad esperada implique un nivel de inversión coherente con el pleno empleo. En ese momento, el rendimiento esperado será tan bajo que cubrirá poco más que el desperdicio y la obsolescencia de los instrumentos de capital, y ``algún margen para cubrir el riesgo y el ejercicio de la habilidad y el juicio''. La consecuencia social de las bajas tasas de interés está relacionada con sus efectos sobre los inversores. Los ricos que disfrutaron de altos rendimientos de sus ahorros se verán privados de dicha renta, es decir, de una renta que no se obtiene con inversiones de riesgo o con su trabajo: ``La eutanasia del rentista''. ``Veo el aspecto rentista del capitalismo como una fase de transición que desaparecerá cuando haya hecho su trabajo''. 86

En la teoría económica de Keynes, los determinantes del sistema económico no son el ahorro y la inversión (como sugería la teoría clásica), sino la propensión a consumir, la eficiencia marginal del capital y la tasa de interés. 87Su filosofía económica implica la intervención del gobierno en campos hasta entonces abandonados principalmente a la iniciativa privada, y una ampliación de las funciones tradicionales del Estado, en particular para aumentar la demanda agregada (por influencia de la propensión a consumir e invertir) y a bajar la tasa de interés. , para compatibilizarlos con el objetivo del pleno empleo. Por otro lado, el objetivo de Keynes es salvaguardar las ventajas del individualismo, de la eficiencia económica basada en la descentralización de decisiones y la interacción de los intereses privados, siempre que estas ventajas estén contenidas dentro de los límites antes mencionados. De hecho, cuando gracias a esos ``controles centrales'' la economía se acerca al pleno empleo, la teoría clásica se vuelve aplicable. Este individualismo libertario sigue siendo importante:88 Y, en lo que respecta a las inversiones, es importante que la intervención del gobierno aborde el volumen de las inversiones, no su dirección (que es decisión de la iniciativa privada: ``hay, por supuesto, errores de previsión; pero estos no serían evitado centralizando las decisiones''. 89 )

Incluso recientemente, Keynes ha sido etiquetado como ``planificador''. 90 Considerando su filosofía como se describe aquí, esta etiqueta es incorrecta. Una demanda agregada con pleno empleo no implica ni nacionalizaciones, ni orientaciones de producción hacia bienes y servicios específicos, ni fijación de precios fuera de esquemas de mercado. En la Teoría GeneralSe resuelven las propias perplejidades expresadas en su artículo de 1933: la expansión del papel del Estado no llega al colectivismo ni a la planificación; y defiende las elecciones individuales en nombre de la eficiencia y rechaza el autoritarismo en nombre de la libertad. ¿Estamos lejos, quizás, de la frase de Croce, antes mencionada, de que ``Para un liberal verdaderamente consciente le suena imposible adherirse a ideales autoritarios y reaccionarios, o comunistas, porque el liberalismo los incluye a todos, dentro de sus límites aceptables''?

Un aspecto del pensamiento keynesiano que merece especial atención está relacionado con las relaciones internacionales. Hoy es recordado como uno de los promotores de la cooperación internacional, organizada bajo la bandera del Sistema de Bretton Woods. Pero es necesario mencionar una actitud nacionalista, como reacción al colapso del orden internacional del patrón oro, actitud que enfatiza la capacidad reactiva de su propio país ante la Gran Depresión e incluso lo acerca a ideas mercantilistas. Hay un aspecto, no enfatizado, en la visión de Keynes: su acento en el carácter doméstico de sus propuestas, que es complementario a su rechazo del patrón oro. No es casualidad que, en el capítulo 23 de la Teoría generaltitulado Notas sobre mercantilismo, recuerda, en analogía a lo que Friedrich List había escrito un siglo antes, cuatro puntos: (1) ``El pensamiento mercantilista nunca supuso que hubiera una tendencia autoajustable por la cual la tasa de interés se establecería al nivel apropiado, más bien pensando que una tasa de interés indebidamente alta era el principal obstáculo para el crecimiento de la riqueza''; (2) ``Los mercantilistas eran conscientes de la falacia de la baratura y del peligro de que la competencia excesiva pueda volver los términos de intercambio en contra de un país'', de ahí la protección del mercado interno; (3) ``El deseo del individuo de aumentar su riqueza personal absteniéndose del consumo ha sido generalmente más fuerte que el incentivo al empresario de aumentar la riqueza nacional empleando mano de obra en la construcción de activos duraderos'', de ahí la necesidad de bajar la tasa de interés; (4) los mercantilistas eran conscientes de que el proteccionismo podría llevar a la guerra, pero ---escribe Keynes--- ``su realismo es mucho preferible al pensamiento confuso de los defensores contemporáneos de un patrón oro fijado internacionalmente y del laissez-faire en los préstamos internacionales, que creen que es precisamente estas políticas que mejor promoverán la paz''.91

Keynes, en resumen, no duda en criticar las ``teorías defectuosas'' de la City de Londres, cuando piensan que para mantener la rígida paridad de las divisas ---es decir, para mantener el equilibrio de las cuentas exteriores--- la ``tasa bancaria'', si es necesario, debe ajustarse al alza, hasta niveles totalmente incompatibles con el pleno empleo. 92 Los partidarios del patrón oro no se inmutaron. ``Pero el señor Keynes ni siquiera insistiría en una estabilización de las divisas. Si tuviera que elegir entre una fluctuación en el nivel de precios y una fluctuación en las divisas, optaría por la última''. 93

Sin embargo, la convivencia del liberalismo internacional y el interés nacional no había sido un problema hasta que Gran Bretaña fue el director de la orquesta internacional, pero las cosas cambiaron cuando la batuta pasó a manos estadounidenses. Sólo durante la Segunda Guerra Mundial Keynes avanzó hacia la idea de una unión de compensación multilateral que superaría las dificultades y complicaciones de un gran número de acuerdos bilaterales (con Gran Bretaña en el centro de su Imperio); tendría una institución supranacional como cámara de compensación central; se basaría en una nueva unidad monetaria internacional, el ``bancor'', que la cámara de compensación emitirá a los países miembros contra el pago de cuotas en oro y monedas nacionales. Este esquema chocaría en Bretton Woods con las políticas centradas en el dólar del estadounidense Dexter White,94 Como sabemos, no se creó una unidad monetaria internacional, el dólar ocupó el lugar como la principal moneda de reserva y el sistema de Bretton Woods permaneció inherentemente inestable. La posición de superávit, y luego deudor, del país hegemónico, Estados Unidos, tuvo como consecuencia escasez y luego exceso de liquidez internacional y transmitió presiones deflacionarias y luego inflacionarias sobre el resto del mundo, hasta su finalización. fallecimiento en 1971.

Estas consideraciones muestran cuán débil es la frontera entre una visión basada en un liberalismo global y otra basada en el interés nacional, y cómo un liberal puede encontrar difícil conciliar las dos visiones en diversas circunstancias. En nuestro caso, la conciencia, madurada después de la Primera Guerra, de que el papel hegemónico de Gran Bretaña había llegado inexorablemente a su fin, empujó a Keynes a elaborar un sistema de cooperación internacional que pudiera contrarrestar la hegemonía estadounidense entrante.

\hypertarget{estado-de-bienestar-de-beveridge}{%
\section*{Estado de bienestar de Beveridge}\label{estado-de-bienestar-de-beveridge}}
\addcontentsline{toc}{section}{Estado de bienestar de Beveridge}

Aún más que en Keynes, es en las obras de Beveridge donde se persigue una visión más amplia del papel del Estado. Es evidente de inmediato que la atención de Beveridge va más allá de los límites de la economía y toca las características esenciales de un Estado liberal. El objetivo del pleno empleo debe ir acompañado de la condición de que se conserven todas las libertades esenciales, y solo un gobierno democrático puede mantener y defender lo que él considera libertades esenciales de los ciudadanos: libertad de escritura, estudio y enseñanza; libertad de reunión con fines políticos y de otro tipo; libertad de elección de ocupación; y libertad en la gestión de los ingresos personales. Es deber del Estado implementarlos. Por lo tanto, esta larga lista significa excluir cualquier solución totalitaria de pleno empleo en una sociedad completamente planificada y reglamentada por un ``dictador inamovible'': la necesidad del socialismo aún no se ha demostrado, dice Beveridge. Pero, de manera significativa, no incluye, en estas libertades esenciales, la de la propiedad privada de los medios de producción: si la experiencia mostrara que la abolición de la propiedad privada es necesaria para el pleno empleo, esta abolición tendría que emprenderse.95 Un Estado liberal no estaría necesariamente en conflicto con una estructura productiva de propiedad pública (podemos recordar en este punto el liberalismo ético de Croce y la coincidencia no necesaria del liberalismo económico y ético).

Para hacer efectivas esas libertades, no una simple declaración de intenciones, el Estado debe brindar no solo servicios públicos bien establecidos como defensa o policía o justicia, sino también otros servicios como educación gratuita, seguridad social, servicio nacional de salud, infraestructuras. 96

Las finanzas públicas, según Beveridge, tienen una variedad de subfunciones: generar una distorsión de los recursos para la satisfacción de los deseos públicos (un objetivo reconocido también por los economistas clásicos, pero ampliado por Beveridge al ampliar la lista de servicios que le corresponde a la Estado a proporcionar); corregir la distribución de la renta y la riqueza; utilizar el instrumento fiscal para la estabilización del ingreso y el crecimiento: el aspecto distributivo parece tener prioridad sobre el crecimiento (esto es el llamado ``financiamiento funcional'', ver más abajo). Incluso un gasto público improductivo, que sin embargo genera empleo, es preferible a una mano de obra obligada a la ociosidad. Pertenece a Beveridge la frase, a veces atribuida a Keynes, de que ``es mejor emplear personas para cavar agujeros y rellenarlos de nuevo, que no emplearlos en absoluto''.97

Hasta ahora ---observa Beveridge--- dos principios fundamentales han regido el presupuesto del Estado: mantener el gasto estatal en el mínimo necesario para satisfacer necesidades ineludibles y equilibrar los ingresos y los desembolsos de cada año. Para el logro de esos objetivos más amplios de asegurar la satisfacción de los deseos públicos en un marco de pleno empleo, Beveridge piensa que el presupuesto público debe interpretarse determinando ``no sólo los ingresos y gastos del gobierno, sino los ingresos y egresos estimados de la nación como entero''. 98Beveridge es keynesiano, porque según ambos el Estado es el último responsable del nivel de los desembolsos totales (la ``demanda efectiva'' keynesiana) ---privada y pública--- consistente con una situación de pleno empleo. Beveridge reconoce que se necesitan presupuestos públicos más grandes, porque el gasto total, público y privado, para el consumo o la inversión: la renta nacional debe ser igual a la cantidad necesaria para asegurar que la mano de obra esté plenamente empleada. 99 Dentro de este marco macroeconómico, establece tres reglas de las finanzas nacionales, en orden de prioridad: los desembolsos deben dirigirse al pleno empleo; deben estar dirigidos por prioridades sociales, como se indicó anteriormente; sujeto a la primera y segunda prioridad, es mejor proporcionar medios para los desembolsos mediante impuestos que mediante préstamos. 100Incluso admitiendo déficits públicos cuando es necesario, Beveridge se muestra reacio a alentarlos, porque el endeudamiento público significa aumentar los ingresos y la riqueza de los rentistas: personas que tienen reclamos contra la comunidad sin contribuir con su propio trabajo. 101 Aquí, Beveridge parece olvidar que una cantidad cada vez mayor de deuda pública estaba en manos de la clase trabajadora, que está lejos de ser definida como ``rentistas''.

En resumen, las ideas de Beveridge y las ``finanzas funcionales'' cuyos principios desarrollaría Abba Lerner 102 tienen implicaciones más amplias que el gasto deficitario de Keynes, porque tienen un impacto profundo en la estructura y funcionamiento del sistema económico. Las políticas económicas de la posguerra se sienten probablemente más influidas por Beveridge que por Keynes. Y Beveridge, más que Keynes, podría ser visto como el principal objetivo de las críticas de los economistas radicalmente libertarios, y aún más de los ordoliberalistas alemanes (véanse las Secciones 6 , 7 y 8 de este capítulo).

Lerner representa un paso más hacia la ampliación del sector público. Afirmó, al comienzo de su Economía del control , 103ese control significa la aplicación deliberada de cualquier política que sirva mejor a la intención social, sin prejuzgar la cuestión entre la propiedad colectiva y la administración o alguna forma de empresa privada. Tres cuestiones están ante la economía de control para resolver: el empleo, el monopolio y la distribución de la renta. Hay dogmas tanto de la derecha (el gobierno que no interfiere con los negocios con fines de lucro) como de la izquierda, que establecerían el 100\% del colectivismo y prohibirían cualquier empresa privada con fines de lucro como inmoral; pero una economía de control puede cosechar los beneficios de la economía capitalista y la economía colectivista, y la economía del bienestar resultante reconciliará el liberalismo y el socialismo. 104

El fuerte internacionalismo de Beveridge es visible en una cortés controversia entre él y Keynes que se remonta al período de la Gran Depresión, cuando ambos participaron, en 1931, en un ciclo de conferencias sobre La crisis económica mundial y la vía de escape. 105 Keynes había invocado la adopción, por parte del Reino Unido, de un arancel protector, porque -dijo- ``es un previo necesario para la recuperación mundial que este país recupere su libertad de acción y su poder de iniciativa internacional''. 106La respuesta de Beveridge acusó a Keynes de nacionalismo, basado en la pretensión británica de actuar como líder. ``Bueno - dijo Beveridge - me inclino a estar de acuerdo en que nosotros {[}los británicos{]} somos {[}los líderes{]} \ldots{} Pero si la justificación para fortalecernos es que somos mejores internacionalistas, para fortalecernos, \ldots{} imitando el nacionalismo tonto de los demás, destruye esa justificación''. 107

Las metamorfosis del liberalismo en el siglo XX muestran que una de sus principales características, el ``cosmopolitismo'' (globalismo), puede perderse o, en cierta medida, debilitarse. Las influencias nacionalistas o socialistas surgen según diferentes circunstancias. Con Beveridge, lo que parece prevalecer es una idea del liberalismo como concepto ético ante cualquier implicación de eficiencia económica. Esta idea supone una intervención activa del Estado para alcanzar objetivos como el pleno empleo y la prestación de una serie ampliada de servicios públicos. Sin estos objetivos, la libertad sería un concepto vacío, no podría concretarse.

\hypertarget{la-escuela-austriaca-y-el-liberalismo-de-hayek}{%
\section*{La escuela austriaca y el liberalismo de Hayek}\label{la-escuela-austriaca-y-el-liberalismo-de-hayek}}
\addcontentsline{toc}{section}{La escuela austriaca y el liberalismo de Hayek}

``No hay nada en los principios básicos del liberalismo que lo convierta en un credo estacionario ---escribe Hayek---, no hay reglas estrictas fijadas de una vez por todas. El principio fundamental de que en el ordenamiento de nuestros asuntos debemos aprovechar al máximo las fuerzas espontáneas de la sociedad y recurrir lo menos posible a la coacción, tiene una infinita variedad de aplicaciones''. 108

Esta frase es evidencia de lo que hemos subrayado al comienzo de este capítulo: las metamorfosis del liberalismo en el siglo XX. De hecho, la Primera Guerra Mundial, la crisis de Wall Street, la Gran Depresión y los desarrollos sociales y políticos relacionados, incluso las nuevas tendencias en el pensamiento filosófico que separaron los conceptos de liberalismo ético y económico, fueron razones para un gran replanteamiento del bien. -visiones económicas establecidas. Pero no todas las reflexiones iban en la misma dirección: desde luego, no todos en la dirección descrita en los apartados anteriores (en particular, 3 - 5) de este capítulo. A diferencia de los autores que acabamos de mencionar (Keynes, Beveridge o incluso Pigou), otros pensadores liberales reaccionaron a las crisis políticas y al colapso de las principales economías con igual escepticismo hacia ambas ideas de una amplia participación del Estado en la conducción de la actividad económica ( que fue criticado como un socialismo progresivo antilibertario), y las ideologías totalitarias, poniendo en pie de igualdad el nacionalismo fascista y el socialismo marxista. El mejor epítome de esta visión está representado por lo que generalmente se identifica como la Escuela Austriaca y la Escuela Americana de Chicago. Los defensores de ambos estuvieron bien representados en la Sociedad Mont Pèlerin, una organización creada en 1947, para elaborar principios destinados a crear y preservar una sociedad libre. El ordoliberalismo alemán merece una mención distinta, por sus peculiaridades.

No hemos mencionado en el primer capítulo al economista político que a menudo se incluye, con Jevons y Walras, entre los tres principales representantes de la teoría de la utilidad marginal, el austriaco Carl Menger. 109 La razón es que Menger ocupa una posición peculiar, siendo al mismo tiempo un exponente importante de la escuela neoclásica y el engendrador de una corriente de pensamiento bastante diferente, la Escuela Austriaca de Economía. Como intentaremos aclarar, aparece más como un hijo rebelde del historicismo alemán que como una expresión de la nueva ciencia positiva en el campo de la economía. 110

Es el momento de mencionarlo ahora, como el iniciador de esa Escuela Austriaca, que se desarrolló en el siglo XX, teniendo a Hayek como probablemente su mayor exponente, y dio un renovado prestigio moral al liberalismo económico. En esta Escuela se incluye un grupo de economistas ``liberales'' que comprende, además de Hayek, autores como O. Morgenstern, F. Machlup, G. von Haberler, P. Rosenstein Rodan, L. von Mises: economistas de origen austriaco o centroeuropeo , quienes, debido a sus ideas, se trasladaron principalmente a Estados Unidos o Gran Bretaña.

Menger, con Jevons y Walras, reaccionó contra la Escuela Clásica (mayoritariamente británica) de Smith o Ricardo. 111 Como sabemos, la Escuela Clásica apoyó la visión de una teoría objetiva del valor trabajo y no dio relevancia a la teoría subjetiva de la utilidad individual. ``Una de las razones por las que las doctrinas clásicas nunca se habían establecido firmemente en Alemania era que los economistas alemanes siempre habían sido conscientes de ciertas contradicciones inherentes a cualquier teoría del valor del costo o del trabajo'' 112(la teoría ricardiana). Menger se centró en cambio en el individuo y su utilidad, contribuyendo así a construir su propia versión del marginalismo. El camino de Menger hacia el marginalismo fue, sin embargo, diferente al de los economistas neoclásicos, y no puede explicarse sin considerar su origen cultural: él, licenciado en Derecho en Viena y Cracovia, tuvo que liberarse en primer lugar de los esquemas de la Historia. Escuela de Economía, entonces imperante en Alemania.

Es bien conocido el duro debate de Menger con el director de la Escuela Histórica, Gustav Schmoller. Schmoller criticó a Menger porque, en su opinión, Menger no hizo más que repetir la ``ficción errónea y obsoleta'' de los economistas del libre comercio británicos: un método deductivo basado en la asunción de proposiciones elementales sobre un hombre abstracto y medio. 113Menger reaccionó a Schmoller: el error de la Escuela Histórica fue que consideraba la economía nacional, que es un complejo de economías individuales individuales, como, en sí misma, un abstracto, ``un todo'': una gran economía en la que la nación debe representar en sí mismo como sujeto productor y consumidor. En cambio, el desarrollo de un sistema económico debería verse, según Menger, como el resultado involuntario del comportamiento de varios individuos que operan en su propio interés.

Habiendo descrito a Menger como un antagonista tanto de la Escuela Clásica (en su mayoría británica) como de la Escuela Histórica Alemana, queda por ver por qué aparece como una figura peculiar también entre los economistas neoclásicos. Como ellos, confiaba en la utilidad individual como una medida subjetiva de valor: exactamente lo que estaban haciendo los economistas neoclásicos en otros lugares. Pero al mismo tiempo, Menger adoptó una posición bien caracterizada.

Se ha observado que, a diferencia de Jevons y Walras, Menger favoreció un enfoque humanista 114que no les pertenece. No nos detendremos en su teoría que, al igual que otros escritores neoclásicos, se centra en una idea subjetiva del valor, en el análisis marginal, en el individualismo metodológico; pero su enfoque también se centra en la elección humana intencionada, en el acto de preferencia como juicio y en la relación entre medios y fines. Jevons y Walras rechazaron la relación medios-fines, en lugar de favorecer la técnica de modelar relaciones complejas como sistemas de ecuaciones simultáneas en las que ninguna variable es causa de otra variable. Menger vio en cambio, a través de la relación causal de medios y fines, la estructura temporal de la producción: el elemento del tiempo había sido descuidado por los economistas neoclásicos,115 )

La Escuela Austriaca de Economía se desarrolló en el siglo XX, consolidando su propia postura en el campo cada vez más diversificado del liberalismo y oponiéndose a diferentes puntos de vista liberales, como keynesianos. Esta Escuela agregó vitalidad a la idea liberal a través de un credo radicalmente libertario, donde un tipo específico de ética (de una manera similar a la visión de Einaudi, como se describió anteriormente) sigue siendo un componente básico.

Friedrich von Hayek puede considerarse el principal exponente de este tipo de liberalismo del siglo XX. En oposición a los economistas neoclásicos y a su padre natural, el positivismo, se distancia de su postura ``científica'' al hacer una revalorización explícita de la ``ideología'', que él ve ``simplemente {[}como{]} el análisis de las ideas y acciones humanas''. No es casualidad que haya escrito una nueva y hermosa Introducción a los Principios de Economía de Menger .

Hayek hace una fuerte contraposición de las ciencias sociales y naturales, justo lo contrario de los seguidores de la filosofía positivista. Los primeros no tratan de las relaciones entre las cosas, sino de las relaciones entre el hombre y las cosas, o las relaciones entre el hombre y el hombre. Su objetivo es explicar los resultados no deseados o no deseados de las acciones de muchos hombres. Estas acciones -tal como las estudian las ciencias sociales en sentido estricto (solían ser descritas -escribe- como las ciencias morales) - son, de manera consciente o reflexiva, llevadas a cabo por el individuo cuando tiene que elegir entre varios cursos abiertos. a él. El enfoque de las ciencias naturales es objetivo, el de las ciencias sociales y morales es subjetivo y los ``hechos'' de las ciencias naturales se convierten ---dentro de las ciencias sociales--- en ``opiniones, ideas, conceptos''. Como hemos visto en el capítulo 1 Hayek, en su actitud polémica hacia el positivismo, llega a asimilar a Comte a Hegel, en su postura anti-libertaria, incluso colocando a Comte por debajo del Hegel oscurantista. 116 Este dúo poco acogedor, que se basa en leyes simples para explicar la realidad, lo completa Marx, que se basa en ambos: Comte: leyes naturales; Hegel: principios metafísicos; Marx: interpretación materialista. 117

Dentro del marco de su actitud en gran parte anti-positivista, hay al menos dos elementos que diferencian la visión de Hayek de la de los economistas neoclásicos: (a) un enfoque teórico diferente del concepto de racionalidad en economía; yb) el papel del Estado en un contexto social, político y económico en evolución.

Sobre el primer punto, relativo a la racionalidad, Hayek se distanció de los marginalistas neoclásicos y negó que un orden económico racional resulte del supuesto de que poseemos toda la información relevante, podemos formular un sistema racional de preferencias y conocer los medios disponibles. para ponerlos en práctica. Si este fuera el caso, la solución óptima para la mejor utilización de los recursos disponibles (trabajo, capital, tierra) sería, de hecho, solo una deducción lógica, mejor expresada en forma matemática. Sin embargo, estas condiciones para un orden racional no existen en realidad, ya que sólo hemos ``fragmentados fragmentos de conocimiento incompleto y frecuentemente contradictorio'', ``el conocimiento de las circunstancias particulares de tiempo y lugar''. 118¿Cómo utilizar este ``conocimiento que no se le da a nadie en su totalidad''? Según Hayek, el mercado libre es el arreglo económico más eficiente, no porque cada agente económico tenga un programa completo y racional de cualquier comportamiento alternativo posible (la incertidumbre es inevitable), sino porque los conocimientos parciales dispersos están conectados por el sistema de precios. . El mercado proporciona la conexión necesaria entre los agentes económicos y los precios son una información objetiva sobre los recursos disponibles. En el mercado, se crea un pedido espontáneo.

El rechazo de una autoridad de planificación es el sequitur necesario : al ser las preferencias individuales poco claras e incognoscibles, la autoridad no puede utilizar ninguna información sobre ellas para gobernar la economía en una determinada dirección. Un plan central está destinado al fracaso porque ningún vehículo puede transmitir a la autoridad lo que es, per se , incierto y cambiante.

La formación del precio correcto requiere un sistema basado en la competencia. Y esto, a su vez, requiere un papel para el Estado: este es el segundo elemento de diferencia con los economistas neoclásicos. Al papel del Estado ---como garante de la competencia más que como planificador central--- Hayek dedica gran parte de The Road to Serfdom. 119No rechaza la idea de planificación si se refiere al agente económico único, es decir, si significa emplear la previsión y el pensamiento sistemático en la planificación de los asuntos comunes del hombre, sino la tarea del ``poseedor del poder coercitivo'' (el Estado). se limita a crear condiciones que permitan al individuo planificar con éxito. El pensador liberal está a favor de hacer el mejor uso posible de las fuerzas de la competencia como medio para mejorar la planificación individual, es decir, para coordinar los esfuerzos humanos.

La competencia, subraya, no es seguir un laissez-faire dogmático: Hayek reconoce que la estructura de la sociedad ha cambiado drásticamente con respecto al siglo XIX. El laissez-faire debe verse en un contexto histórico y como una regla empírica, basada como está en una aceptación pasiva de las instituciones como son, como una ``regla cruda'' en la que se expresaron los principios de la política económica del siglo XIX. . Pero esos principios fueron ``sólo el comienzo'', y seguir confiando en esa regla práctica ha hecho mucho daño a la causa liberal. Hay espacio para una mejora gradual del marco institucional de una sociedad libre. El problema ---añade Hayek--- es que este avance ha sido lento: ahora hay un campo amplio e incuestionable de actividad estatal.

``El funcionamiento de una competencia no solo requiere una adecuada organización de ciertas instituciones como los mercados de dinero y canales de información, algunos de los cuales nunca pueden ser proporcionados por empresas privadas, sino que depende, sobre todo, de la existencia de un sistema legal adecuado \ldots{} diseñado tanto para preservar la competencia como para que funcione de la manera más beneficiosa posible. De ninguna manera es suficiente que la ley reconozca el principio de propiedad privada y libertad de contratación; mucho depende de la definición precisa del derecho de propiedad aplicado a diferentes cosas''. 120 La competencia definida en esta línea implica que no se puede permitir ningún tipo de acuerdo de organización sindicalista o ``corporativa'' de la industria: un estado de cosas, un monopolio, correctamente opuesto tanto por liberales como por socialistas.121

El Camino de servidumbre de Hayek fue escrito durante la Segunda Guerra Mundial, bajo el impulso de los desarrollos políticos y militares, incluso si es el resultado de trabajos anteriores. 122 Evidentemente se ve afectado por ese clima específico, por la enorme presión proveniente de los estados totalitarios; es una mezcla de reflexiones teóricas e históricas e incitaciones a volver al ``camino abandonado'' del liberalismo de Adam Smith, Hume, Locke, y al individualismo de Cicerón, Erasmo, Montaigne, apoyándose en las fuerzas espontáneas de la sociedad, contra cualquier forma de coerción. 123Esta ideología estrictamente libertaria y de libre mercado es una guía para Hayek para reflexionar sobre los temas de ``planificación y democracia'' y ``planificación y Estado de derecho''. Y es igualmente duro con las economías ``occidentales'' y totalitarias. Analicemos estos dos temas.

Sobre planificación y democracia, la afirmación fundamental de Hayek es que ``la democracia es esencialmente un medio, un dispositivo utilitario para salvaguardar la paz interna y la libertad individual'': un dispositivo instrumental, ya que no reconoce a la sociedad y el bien público como la pieza central de la democracia. El objetivo social colectivista, el bienestar general, se realiza a través de un plan único, mediante el cual cada necesidad individual recibe un cierto rango, en una escala de valores definida por el planificador, que es una especie de código ético completo. Un código ético común lo suficientemente amplio como para incluir un plan económico unitario es una inversión de la tendencia a avanzar hacia una ampliación de la esfera de la libertad individual. Este código no solo sería inconveniente desde el punto de vista de la libertad individual,124

Hayek sigue un enfoque benthamita de la utilidad cuando escribe que la filosofía del individualismo no asume que el hombre es egoísta o egoísta, sino simplemente que tiene su propia escala de valores, que no debería estar sujeta a ningún dictado de otros. 125Esta visión está comprensiblemente en curso de colisión con el sistema colectivista, definido como ``la organización deliberada de los trabajos de la sociedad para un objetivo social definido''. Los tipos de colectivismo, ya sean fascismo o comunismo, pueden diferenciarse por la naturaleza de tal objetivo, pero cualquier colectivismo es diferente del liberalismo y el individualismo en su negativa a reconocer una esfera de autonomía donde los fines individuales son supremos. Desde 1931 {[}la Depresión{]} ---lamenta --- Estados Unidos y Gran Bretaña --- ``Occidente'' --- se han movido lentamente hacia el socialismo (esta es una crítica severa de los economistas liberales mencionados anteriormente), aunque sabemos que socialismo significa, en el fin, esclavitud. 126

Hayek fija entonces un concepto fundamental de Estado liberal: el Estado de derecho, en contraposición al gobierno arbitrario. En virtud del Estado de derecho, ``el gobierno se limita a fijar reglas que determinen las condiciones en las que se pueden utilizar los recursos disponibles, dejando a las personas la decisión de los fines con los que se utilizarán''. 127 Las reglas se establecen de antemano, como reglas formales destinadas a ser meramente instrumentales en la búsqueda de los diversos fines individuales de las personas. En consecuencia, la discrecionalidad dejada al Poder Ejecutivo del gobierno que ejerce poderes coercitivos se reduce tanto como sea posible.

En el extremo opuesto está el gobierno arbitrario, por el cual el gobierno dirige el uso de los medios de producción a sus fines particulares, a través de reglas sustantivas y siempre provisionales. El gobierno atiende las necesidades reales de las personas y elige entre ellas, según las circunstancias a medida que surgen. Como ejemplo de gobierno arbitrario, Hayek toma el caso de la Alemania nazi 128 : Los nazis siempre han sido intolerantes con una justicia meramente formal, con el Estado de derecho, es decir, con una ley que no tiene una visión de lo bien que deberían estar determinadas personas. ser y querer una socialización de la ley, atacando la independencia de los jueces e invocando la Freirechtsshule. 129 En la Alemania nazi, el Estado Gerechte (el Estado justo), defendido por el jurista Carl Schmitt,130 es el sustituto del Estado de derecho.

Es cierto, señala Hayek, que el Estado de derecho tolera las desigualdades económicas. Pero, en una especie de compensación entre desigualdad y pérdida de libertad individual, el péndulo de Hayek se inclina hacia el primero. En su apasionada defensa de la libertad individual y la aversión al bienestar colectivista, llega a reconocer que los valores económicos son menos importantes para nosotros que muchas cosas precisamente porque en materia económica somos libres de decidir lo que consideramos marginal \ldots{} la planificación implicaría la dirección de casi la totalidad de nuestra vida. 131 Esta notación de Hayek da la esencia del Estado liberal, que debe dar libertad a los ciudadanos para elegir, no solo {[}y no tanto, se podría agregar{]} en sus elecciones económicas, sino más importantemente en cualquier aspecto de la vida privada y pública, cuando hay que tomar una decisión.

Mientras que un punto de vista croceano podría inducir a pensar que un liberalismo ético puede coexistir con estructuras económicas no liberales (ver arriba), Hayek está fuertemente en contra de la posibilidad de que la democracia pueda sobrevivir si los asuntos económicos fueran dirigidos por una autoridad superior. En este sentido, se puede decir con seguridad que, tanto con Hayek como con Luigi Einaudi, no se puede distinguir el liberalismo ético, económico y político, son solo una idea: una posición que los pone en la corriente del pensamiento que se remonta a el liberalismo del siglo XIX, y toma la forma de un libertarismo que abre una amplia brecha entre ellos y el liberalismo de Keynes o Beveridge.

Hayek hace una observación aguda para motivar el surgimiento del totalitarismo de extrema derecha, de la actual amenaza fascista-nazi. Señala que la teoría socialista, marxista o no, está dominada por la idea de una división de la sociedad en dos clases con intereses comunes pero mutuamente opuestos: capitalistas y trabajadores industriales. Esta contraposición binaria la hacen los socialistas en el supuesto de que la clase media desaparecería. Pero, según Hayek, los socialistas han hecho caso omiso del surgimiento de una clase media nueva y muy grande (compuesta por empleados, trabajadores administrativos, comerciantes, pequeños funcionarios). Contrariamente a las expectativas de los socialistas, esta nueva clase media, la pequeña burguesía, sobrevivió y ganó fuerza, a pesar de que su posición económica se estaba deteriorando frente a los trabajadores industriales. Ambas clases, obreros industriales y pequeña burguesía, estaban asociadas por un odio compartido al sistema capitalista, y en este sentido, según Hayek, ambas eran socialistas, pero los ideales revolucionarios no tenían atractivo para esta clase media, su idea de la justicia era diferente a la del viejo socialismo. El fascismo y el nacionalsocialismo ---el resultado político del odio de la clase pequeñoburguesa al capitalismo liberal--- son una especie de socialismo de clase media. El conflicto entre estos movimientos políticos, por un lado, y el viejo socialismo, por el otro, debe verse como un conflicto de facciones socialistas rivales. su idea de justicia era diferente a la del viejo socialismo. El fascismo y el nacionalsocialismo ---el resultado político del odio de la clase pequeñoburguesa al capitalismo liberal--- son una especie de socialismo de clase media. El conflicto entre estos movimientos políticos, por un lado, y el viejo socialismo, por el otro, debe verse como un conflicto de facciones socialistas rivales. su idea de justicia era diferente a la del viejo socialismo. El fascismo y el nacionalsocialismo ---el resultado político del odio de la clase pequeñoburguesa al capitalismo liberal--- son una especie de socialismo de clase media. El conflicto entre estos movimientos políticos, por un lado, y el viejo socialismo, por el otro, debe verse como un conflicto de facciones socialistas rivales.132

Esta visión de Hayek, por singular que parezca, puede compararse con los comentarios de Marx sobre la pequeña burguesía: ambos escritores comprenden plenamente la relevancia social y económica de esta clase. Marx lo vio como reaccionario, adverso al socialismo y estaba convencido de que finalmente desaparecería (ver Capítulo 1 ); Hayek, por el contrario, ve a esta clase como fortaleciéndose y ganando poder, y como una nueva facción rival del socialismo mismo. Sin duda, Hayek tenía razón al subrayar la importancia no transitoria de la pequeña burguesía y su ideología totalitaria no liberal, pero su asimilación de esta ideología al socialismo puede ser discutible, si por socialismo pretendemos un derrocamiento revolucionario de la clase capitalista.

\hypertarget{la-escuela-de-chicago}{%
\section*{La escuela de Chicago}\label{la-escuela-de-chicago}}
\addcontentsline{toc}{section}{La escuela de Chicago}

La ``vieja'' escuela de Chicago ha sido opacada durante mucho tiempo por la ``nueva escuela'', la defendida por Milton Friedman y George Stigler, quienes vieron la vieja escuela como ``intervencionista'', siendo su teoría caracterizada por un conjunto firme de reglas, con que el mercado y sus participantes deben cumplir. Según el Antiguo, la función apropiada del Estado no es prestar servicios ni realizar algunas actividades económicas, sino dar un marco de reglas dentro del cual las actividades económicas de los agentes económicos públicos y privados puedan desarrollarse libremente. Uno de sus principales exponentes, Henry Simons, afirma explícitamente que el suyo es ``un esquema coherente de ética práctica, una filosofía político-económica o, si se quiere, una posición ideológica bien definida\ldots{} libertaria o, en el inglés-continental sentido, liberal''.133 Simons también presta atención al aspecto distributivo del sistema económico, que sin embargo se resuelve en lo que él llama ---al igual que Walras--- justicia conmutativa 134 : ``cada uno recibirá según contribuya (o contribuya) a la producción organizada, cooperativa y conjunta , o en lenguaje técnico económico según la productividad de su propiedad, capital o capacidad (incluida la capacidad personal)''. 135

Simons da por sentada su oposición al comunismo y al fascismo, pero los verdaderos enemigos de sus ideas son ``nuestros reformadores liberales e intelectuales políticamente ambiciosos \ldots{} los defensores ingenuos de la economía dirigida o la planificación nacional'' (no es difícil identificar a estos reformadores como el Nuevo Distribuidores). 136 ``Es una responsabilidad obvia del Estado \ldots{} mantener el tipo de marco legal e institucional dentro del cual la competencia puede funcionar efectivamente como una agencia de control \ldots{} el llamado fracaso del capitalismo (del sistema de libre empresa, de competencia) puede razonablemente ser interpretado principalmente como un fracaso del Estado político en el cumplimiento de sus responsabilidades mínimas bajo el capitalismo''. 137La responsabilidad de las fluctuaciones económicas pertenece al Estado que desestabiliza el sistema económico al expandir y contraer la cantidad de dinero en circulación (el lado monetario de la teoría de Simons se considerará más adelante).

Con los ``defensores ingenuos de la economía gestionada o la planificación nacional \ldots{} debemos estar de acuerdo en un punto vital, a saber, que ahora existe una necesidad imperiosa de un programa de legislación económica sólido y positivo''. 138 Sin embargo, este programa va en contra de las políticas de los New Dealers: la legislación debe establecer que el gobierno tiene pocas funciones: mantener el orden interno, hacer la guerra, promover el libre comercio frente a cualquier forma de mercantilismo. A las conocidas libertades rooseveltianas ---la libertad de expresión, de culto, de la miseria, del miedo--- sólo hay que añadir una: la libertad de empresa, mediante la minimización de las responsabilidades del Estado. 139

``Los objetivos próximos de una política económica tradicionalmente liberal, en las condiciones modernas, pueden definirse en términos de los problemas: primero, del dinero; segundo, de monopolio y regulación; y, tercero, de la desigualdad''. 140

Acerca del dinero, la creciente atención de Simons, y muchos otros, es el signo obvio de la dificultad de reemplazar el antiguo patrón oro con nuevos arreglos monetarios, dadas las implicaciones sociales y políticas de un ``dinero sólido''. La administración del dinero implica reglas de juego estables bien definidas por la ley, destinadas a controlar el dinero en cantidad y valor, mientras que cualquier discreción en la administración del dinero debe rechazarse. Estas reglas deberían ser una especie de mandato extraconstitucional o cuasi constitucional. También requerirían una reforma del sistema bancario, que debería separar la función monetaria (que es de carácter público) de la ``movilización de fondos para fines de inversión'', es decir, del crédito (que es un negocio privado). 141

Las reglas monetarias también deben afectar la política fiscal, porque es a través de ella que se mejoran; en otras palabras, la política fiscal no debería desestabilizar el dinero. Como consecuencia, Simons tiene una visión crítica de la deuda pública. Cuando se emite como sustituto del dinero, puede surgir inflación si se daña la confianza de los inversores. El dinero no se gestiona fácilmente junto con la deuda (este tipo de constitución monetaria también lo concibe Irving Fisher, quien, como economista puro, no se detiene en la visión filosófica que es la raíz del pensamiento de Simons). 142

Con referencia al segundo objetivo, la regulación del mercado, el mayor enemigo de una democracia liberal es el monopolio, ya sea en forma de grandes corporaciones, cárteles industriales, agencias de control de precios, o también sindicatos. De hecho, el mejor criterio de eficiencia económica es el sistema de precios. Los precios deben determinarse libremente en cualquier mercado (incluido el mercado laboral): el resultado de compras competitivas por personas libres de utilizar el poder adquisitivo como les plazca. Por lo tanto, Simons es crítico con los sindicatos, porque a través de la negociación colectiva, intentan elevar el nivel de los salarios por encima del nivel determinado por la competencia, reduciendo así las oportunidades de empleo.

La mitigación de las desigualdades, este es su tercer punto, es incompatible con la eficiencia económica. Es cierto que, como consecuencia de un sistema de precios de libre mercado (incluidos los salarios), los individuos se encuentran en circunstancias de ingresos muy diferentes, pero los ``problemas de ineficiencia y desigualdad \ldots{} son, dentro de límites bastante amplios, distintos e independientes'' . 143 La desigualdad es un problema aparte que debe resolverse mediante impuestos progresivos.

Como se mencionó anteriormente, Simons es muy crítico con ese tipo de liberales que ven un papel activo del Estado en la economía, a quienes llama ``colectivistas''. Al revisar sus libros, ataca tanto al keynesiano Alvin Hansen como, aún más, a William Beveridge. Refiriéndose a su Pleno empleo en una sociedad libre , Simons escribe que ``está escrito por un liberal nominal, radical-reaccionario en sus propuestas sustantivas, libertario en su retórica'', y se queja de que el libro ``puede pronosticar o determinar en gran medida el curso de Política británica de posguerra''. Es filosofía política más que economía. Recordando las libertades fundamentales de Beveridge (ver arriba, Sección 3), Simons los compara con la política económica nazi e irónicamente los define como ``Una cruzada contra la miseria, la enfermedad, la miseria y la ignorancia, lo cual es bueno si te gusta el esquema alemán de antes de la guerra como forma de vida nacional''; una planificación colectivista donde no hay nada que imponga la competencia. El trabajo de Beveridge no es más que un ``esquema hiperkeynesiano de economía estrictamente reglamentada y nacionalismo económico extremo''. 144

Desde la perspectiva fiscal y monetaria, la crítica de Simons se dirige a la discrecionalidad otorgada al banco central: todo el poder ---el de emisión y de endeudamiento--- debe concentrarse en el Tesoro. El Tesoro, a su vez, debe asegurar la estabilidad monetaria bajo reglas definidas y coherentes, con una mínima intervención en los mercados.

Este punto de vista radical y libertario no podría ser aceptado acríticamente ni siquiera por un economista liberal como Lionel Robbins quien, no muchos años después, escribirá: ``La conveniencia de las reglas en lugar de las autoridades, para usar el contraste tan vívidamente planteado por Henry Simons, es absolutamente central para la posición libertaria principal \ldots{} {[}pero{]} creo que es una deficiencia del caso libertario \ldots{} que incluso cuando repudia explícitamente la superficialidad del laisse-faire extremo, tiende a sugerir una concepción de gobierno que está demasiado limitada a la ejecución de leyes conocidas, con exclusión de funciones de iniciativa y discreción que no pueden quedar fuera del cuadro sin distorsión''. Y cita ``el ámbito de las finanzas'': ``Seguramente sería imprudente\ldots{} asumir que no puede surgir ninguna situación que no pueda ser tratada por mecanismos puramente automáticos''.145

¿Qué ha agregado la ``Nueva Escuela'' de Chicago a la filosofía de Simons? No mucho, aparte de un mayor acento en el ``liberismo''. Según su principal exponente, Milton Friedman, ``el gobierno es fundamental tanto como foro para determinar las `reglas del juego' como árbitro para interpretar y hacer cumplir las reglas decididas''. 146Si comparamos las ideas de Simons y Friedman, podemos encontrar similitudes y diferencias. Sobre el problema de la distribución de la riqueza, Friedman está de acuerdo con Simons en el principio de que todo el mundo debería recibir una remuneración proporcional a su contribución al proceso de producción, salvo cualquier forma de igualitarismo. Sin embargo, Friedman se opone a los impuestos con tasas progresivas como instrumento de redistribución de la riqueza (``Me resulta difícil, como liberal, ver alguna justificación para la imposición gradual únicamente para redistribuir la renta''), prefiriendo un impuesto de tasa plana. Las elevadas tasas impositivas nominales parecen ``un caso claro de utilizar la coacción para tomar de unos con el fin de dar a otros y así entrar en conflicto frontalmente con la libertad individual''. 147

En cuanto a la competencia, la función de árbitro del Estado significa que debe evitar que se confunda competencia con libertad de colusión: no se puede dejar fuera del mercado a ningún competidor que no sea vendiendo un mejor producto al mismo precio o el mismo producto a un precio más económico. (mientras que en la tradición ``continental'', observa, la libertad de empresa significa que las empresas son libres de fijar precios, no de competir en el mismo mercado o de adoptar prácticas para mantener fuera a los competidores potenciales). Pero cuando el monopolio tiene que ser el resultado final como la solución técnicamente más eficiente y hay tres posibles alternativas disponibles: monopolio público, monopolio privado, regulación pública del mercado de un bien o servicio específico, Friedman, aunque reacia, prefiere el monopolio privado.148 (que preferiría un mercado regulado públicamente). La motivación de Friedman es que la tecnología en rápida evolución podría permitir pasar del monopolio privado a una situación de competencia, mientras que el monopolio público, una vez establecido, sería más difícil de desmantelar. 149

La constitución monetaria de Simons toma con Friedman un carácter más definido. Él piensa que el patrón oro, nunca promulgado completamente en su automatismo implícito y, de hecho, sujeto a la discreción de los gobiernos, no es más adecuado para las condiciones actuales, pero aún más inadecuado es otorgar responsabilidades monetarias y amplios poderes discrecionales a un grupo. de tecnócratas, reunidos en un banco central independiente: un arreglo que solo ha traído inestabilidad, medida por las fluctuaciones en el stock de dinero, los precios o la producción. Un liberal tiene miedo de tal concentración de poder. Reglas en lugar de discreción:150 Estas reglas no deberían abordar un nivel óptimo de precios, lo que dejaría demasiadas maniobras discrecionales a las autoridades, sino el stock de dinero, que debería crecer en un porcentaje estable, cuantificado por Friedman en alrededor del 3-5\% anual. 151

Y aquí está la motivación subyacente de su desapego de Simons: ``Muchos liberales anteriores \ldots{} escribiendo en un momento en que el gobierno era pequeño para el estándar actual, estaban dispuestos a que el gobierno emprendiera actividades que los liberales de hoy no aceptarían ahora que el gobierno se ha vuelto tan desbordado'' . 152

\hypertarget{ordoliberalismo-o-liberalismo-autoritario}{%
\section*{Ordoliberalismo o liberalismo autoritario}\label{ordoliberalismo-o-liberalismo-autoritario}}
\addcontentsline{toc}{section}{Ordoliberalismo o liberalismo autoritario}

El papel del Estado en el esquema ordoliberal es suficiente para distinguir a los ordoliberales alemanes de Hayek y los ``austriacos'' en general. A primera vista, todos ellos podrían verse como muy similares a los ordoliberales, y ciertamente existía un vínculo de simpatía entre ellos, ligado como estaban por su confianza compartida en la eficiencia de los mercados, en el sistema de precios como indicador de valor de los productos y factores de producción, en la libertad de entrada al mercado y en el énfasis en las opciones individuales. Pero, como veremos, hay diferencias. Estas diferencias enfatizan las peculiaridades de los ordoliberales no solo en relación con los austriacos, sino también con la nueva Escuela de Chicago, reacia a reconocer el papel central del Estado en la economía.

En definitiva, la visión de los ordoliberales se basa en el reconocimiento de la fuerza del Estado en el centro mismo del sistema económico. Su enfoque es coherente con el enfoque intelectual específicamente alemán que intentamos resumir con cierta extensión en el Capítulo 1. (Secs. 6-10), y es un reconocimiento explícito del amplio papel que definitivamente estaba asumiendo el Estado en el gobierno de las economías nacionales en el siglo XX. Es oportuno agregar que, de manera similar a la Escuela Histórica Alemana, el enfoque ordoliberal no puede ser estudiado como una especie de ``modelo'', en el sentido que un economista podría atribuir a esta palabra; es más bien un esquema esencialmente prescriptivo sobre la estructura y organización del sistema económico, dentro de un marco político bien definido. Este enfoque requiere una conexión estricta entre economía y derecho: se enfatiza con especial énfasis que el marco legal de un sistema económico debe construirse cuidadosamente, siendo esencial para su correcto funcionamiento.

Se puede plantear un interrogante acerca de si el ordoliberalismo debe incluirse dentro de una definición amplia de liberalismo o si debe estar conectado a un nacionalismo autoritario. Lo que hay que señalar al comienzo de esta sección es que el ordoliberalismo no puede verse como un monolito: algunos ordoliberales centran su atención en un Estado omnipotente (``El Leviatán estaba y tenía que estar allí'' 153 ), otros en la eficiencia de los mercados libres. Un corolario del primer rasgo es el poder abrumador del Estado para dictar leyes en el interés superior de toda la nación, como lo prevén los gobernantes, y el papel político mínimo atribuido a los individuos, como entidades no politizadas que simplemente tienen que buscar su bien. -ser sin suplicar el apoyo del Estado.

Para entender el ordoliberalismo, acabamos de hacer una referencia al marco intelectual alemán, y aquí viene a la mente el nombre de Carl Schmitt. Schmitt fue un jurista destacado y, si bien no es nuestra intención detenernos en sus pensamientos sobre la disciplina del derecho y el papel del Estado, un hecho que no puede pasarse por alto es que fue un ideólogo y experto constitucional del Estado nazi. Compartió con los primeros ordoliberales el disgusto por la democracia liberal, poniendo más bien énfasis en una visión peculiar y aparentemente contradictoria de un Estado liberal-autoritario, que protege al mercado de las solicitudes de redistribución de la riqueza.

Algunas palabras sobre Schmitt son necesarias, porque su forma de razonar está muy cerca de los primeros pensadores ordoliberales. 154Distingue el ``Estado total'', encarnado por la República de Weimar, y el Estado ``autoritario''. Aquí, ``total'' significa una democracia pluralista, donde la representación parlamentaria refleja una homogeneidad de gobernantes y gobernados, y ya no se centra en los intereses burgueses; mientras que el gobierno es responsable de las emociones y pasiones masivas. La República de Weimar es vista críticamente como el emblema de la crisis de una sociedad de masas desenfrenada en rebelión, que de hecho había ``suplantado al Estado liberal''. Demasiado Estado democrático había significado una sobrecarga política y económica excesiva, y de esta manera una contradicción efectiva de los principios liberales. Una línea de separación entre Estado y sociedad resultaría útil para ambos: por un lado, significaría una capacidad efectiva del Estado para gobernar, y por otro, un libre ejercicio de la fuerza de trabajo por parte de todas las personas económicamente activas, donde todos serían tratados por igual en virtud del Estado de derecho. La tarea del Estado es liberar la economía despolitizando las relaciones socioeconómicas. El Estado no debe ser el objetivo de los trabajadores que buscan el bienestar; Los proletarios rebeldes deben transformarse en ejercitantes independientes y dispuestos de su propia fuerza de trabajo: este es ``el heroísmo de la pobreza, el sacrificio y la disciplina''. En resumen, una economía libre no es un orden natural, sino el resultado de una práctica gubernamental. El Estado es la categoría predominante de economía política. Los proletarios rebeldes deben transformarse en ejercitantes independientes y dispuestos de su propia fuerza de trabajo: este es ``el heroísmo de la pobreza, el sacrificio y la disciplina''. En resumen, una economía libre no es un orden natural, sino el resultado de una práctica gubernamental. El Estado es la categoría predominante de economía política. Los proletarios rebeldes deben transformarse en ejercitantes independientes y dispuestos de su propia fuerza de trabajo: este es ``el heroísmo de la pobreza, el sacrificio y la disciplina''. En resumen, una economía libre no es un orden natural, sino el resultado de una práctica gubernamental. El Estado es la categoría predominante de economía política.

Por tanto, era necesario, según Schmitt, restaurar el Estado como una institución independiente de toma de decisiones autorizada (independiente de la sociedad de masas). En épocas anormales, como las vividas bajo Weimar, no sería posible una democracia real.

En un discurso de noviembre de 1932, en las etapas finales de la República de Weimar, titulado Estado fuerte y economía sana, Schmitt sostiene que en el siglo XX todos los estados tienden a volverse ``totales'', pero Weimar era un ``estado total cuantitativo'', incapaz de resistir presiones provenientes de partidos y partidos contrarios; el Estado total que se necesita es de tipo ``cualitativo'', con control sobre el ejército y la burocracia, mientras que otras áreas se dejan a la autogestión y a la economía libre: un nuevo orden, potencialmente corporativista y autoritario, para preservar la propiedad, pero al mismo tiempo, y de manera un tanto contradictoria, un orden que defendería los derechos liberales tradicionales del hombre frente al Estado, así como defendería al Estado frente a la amenaza de la democracia liberal: una democracia despolitizada, con un marginado parlamento.155

Al comienzo mismo del Tercer Reich, escribe que la nueva Ley de Habilitación de 1933, que marca el comienzo del Reich de Hitler, aunque se presenta formalmente como un cambio a la anterior constitución de Weimar, débil y ``neutral'', representa un cambio radical: la ley ha ha sido decidido por el parlamento sólo en obediencia a la voluntad del pueblo expresada en las elecciones políticas que se acaban de celebrar; en realidad es un referéndum popular, un plebiscito, que reconoce a Hitler como líder político del pueblo alemán. Se abolieron la libertad de propaganda, opinión, conciencia y actividad, y la neutralidad ideológica de la constitución de Weimar. Esa constitución ``de ninguna manera fue capaz de reconocer ni siquiera a un enemigo mortal del pueblo alemán para abolir al Partido Comunista, enemigo del Estado y del pueblo''. 4 de este ensayo.

En cuanto al Estado, Schmitt afirma una unidad fundamental del cuerpo político, que es el resultado de tres estructuras: el Estado mismo, el Movimiento y el Pueblo. No están en posiciones paralelas: el Movimiento es la estructura dinámica y está encarnado en el partido, el partido único de los trabajadores nacionalsocialistas alemanes; sostiene a los otros dos, los penetra y los conduce. El Estado es el miembro estático de esta estructura triádica: la organización del mando, la administración y la justicia. El Pueblo es la vida social y económica de la nación, que crece al amparo y a la sombra de las decisiones políticas.

Todo el mundo liberal ha caído. Como en la Italia fascista, las calles sin salida de la democracia liberal están abandonadas. La constitución democrática liberal típica ---observa Schmitt--- se basaba en la oposición de dos entidades, ya fueran Estado / sociedad; o Estado / individuo; o poder del Estado / libertad individual; o política / dominio privado. Esta separación tiene por objeto separar el poder del Estado de la sociedad, de modo que esta última pueda controlar al primero, como defensa de la sociedad contra el poder del Estado. La estructura triádica alemana supera esta división. En Alemania, hasta mediados del siglo XIX, el Estado no reconoció esa división y tuvo un papel abrumador, hegeliano. Pero incluso después de eso, con el advenimiento del liberalismo y el positivismo, el Estado siguió siendo en Alemania un Estado administrativo, con su clase de funcionarios públicos que no era un instrumento, sino una fuerza independiente del propio Estado. Por tanto, la visión hegeliana se mantuvo, y escritores como Adolf Wagner y Gustav Schmoller {[}el principal exponente de la Escuela Histórica de Economía Alemana, véase el capítulo 1 {]} mantuvo vivo ese gran concepto alemán: la conciencia de que la clase culta e incorruptible de los funcionarios públicos alemanes la hace superior a la sociedad burguesa.

Hay similitudes entre las líneas de pensamiento de Schmitt y la doctrina económica ordoliberal: la visión de Weimar como un fracaso que resulta de la intervención del Estado en la vida económica, de una sobreexpansión social y fiscal, luego seguida de una política deflacionaria y, en general, cambios económicos parciales; actitud crítica hacia la democracia de masas y preferencia por un gobierno autoritario. Schmitt y los primeros ordoliberales también mantienen la forma muy característica de ver a la clase trabajadora como formada por individuos individuales con ``fuerza de trabajo libre'', capaces de ejercer sus capacidades sin referencia a intereses comunes y solidaridad de clase, y por lo tanto de ver la sociedad como un conjunto. de relaciones socioeconómicas despolitizadas. Por otro lado, y de manera muy clara, los ordoliberales reaccionan contra el dirigismo,156

Según los ordoliberales, la ``destrucción creativa'' de los mercados, adoptada por la economía austriaca, no es suficiente para asegurar el dinamismo necesario para la economía (por ejemplo, la existencia de grandes monopolios se debe, piensa von Mises, a la interferencia del gobierno en la libertad de los mercados; por el contrario, para los ordoliberales, corresponde al gobierno resistir los monopolios creados por los mercados sin trabas). Los ordoliberales piensan que solo la autoridad del Estado puede ejercer la fuerza necesaria para crear un sistema económico eficiente (para los austriacos, este objetivo solo puede ser alcanzado por el sector privado, a través del interés propio de sus componentes). La libertad económica descansa sobre un Estado fuerte, que garantiza un equilibrio social, definido como una situación en la que el individuo está protegido contra la hegemonía de los mercados descontrolados.

El ordoliberalismo nació en el contexto de la Gran Depresión, la crisis de la República de Weimar y la dictadura nazi. Más allá de las diferencias que caracterizan a sus partidarios individuales, generalmente se considera como una forma alternativa neoliberal de economía política al liberalismo del laissez-faire y al colectivismo, en sus diversas formas. Se ha escrito que ``el dicho de que la economía libre depende del estado fuerte es clave para su postura teórica''. 157 Pero, al mismo tiempo, esta relevancia atribuida al Estado está lejos de las ideas de Keynes o Beveridge: nuevamente, otra evidencia de las metamorfosis del liberalismo del siglo XX.

Los puntos centrales de la filosofía ordoliberal se pueden identificar de la siguiente manera:

\begin{quote}
la competencia de mercado, lejos de ser espontánea, está definida y amparada por las regulaciones del Estado: un Estado fuerte, muy diferente del Estado limitado y débil del pensamiento liberal clásico;

sólo en un mercado basado en la competencia, regulado y protegido por el Estado, el emprendedor puede operar, con su reconocida vitalidad y energía y liderazgo innovador: no entrará como tal en el debate político sobre las estructuras y transacciones del mercado.
\end{quote}

El Estado debe actuar de manera que desproletarice las estructuras sociales del capitalismo. Esta desproletarización apunta a vaciar las estructuras sociales marxistas de cualquier contenido. ``La solución a la condición proletaria consiste en el esfuerzo constantemente renovado por eliminar al proletariado mediante una política social acorde con el mercado que, en lugar de aprisionar a los trabajadores en el estado de bienestar, facilita su libertad y responsabilidad, haciendo a cada uno como un propietario. empresario''. 158 Como se dijo anteriormente, el trabajador se integra socialmente en el proceso de producción. Mantiene firmes valores sociales y éticos, arraigados en la tradición, la familia y la comunidad. La imagen de un trabajador desproletarizado y seguro de sí mismo acerca el esquema de los ordoliberales al de Einaudi.

Todo esto representa un fuerte vuelco de la economía positiva, a favor de la economía prescriptiva: el problema no es desarrollar un modelo analítico para explicar el mundo real, sino cambiar el mundo real haciéndolo coherente con el modelo ordoliberal. Este modelo ve la autoridad política como el instrumento necesario para establecer un sistema económico libre.

En 1936 se publicó en Alemania un Manifiesto Ordoliberal , firmado por Franz Böhm, Walter Euken y Hans Grossman Dörth. 159 Según una orientación profundamente arraigada de la tradición intelectual alemana (véase el capítulo 1 ), los autores abordan las disciplinas tanto del derecho como de la economía pero, separándose de la Escuela Histórica, piensan que estas dos disciplinas fueron efectivamente desatendidas durante los siglos XIX y XX. Esta negligencia se debió de hecho a la prevalencia del historicismo: por un lado, había observado correctamente, en contra de las ideas de la Ilustración, que en realidad no existe ningún sistema natural de derecho y economía (como lo subraya completa y correctamente List, en el campo de economía, y Savigny, en el campo del derecho). Pero, apoyarse en el historicismo había expuesto esas disciplinas al riesgo de extinción, por su relativismo y fatalismo.

El relativismo está implícito en la afirmación de que el derecho debe ser desarrollado por las ``fuerzas internas silenciosas'' de la sociedad, no a discreción del legislador: la sustancia del derecho surge de desarrollos históricos y leyes hechas por juristas, que también gobiernan las relaciones económicas. De esta manera, gracias a este enfoque, ``el capitalismo siempre ha encontrado formas y medios para triunfar de lege, praeter legem et contra legem''. De hecho, si los engendradores del derecho son esas ``fuerzas silenciosas internas'', si el derecho es generado por la sociedad misma y no por la voluntad del legislador, el derecho mismo no puede reaccionar contra esas fuerzas (por ejemplo, esta concepción de una especie de El poder creativo de los jueces no impidió la creación de grandes carteles industriales, es decir, la negación de un mercado eficiente y competitivo).

Los ordoliberales ven los trabajos de economistas de la Escuela Histórica de Economía Alemana como ejercicios de relativismo, también desde un punto de vista metodológico. Fracasaron por su empirismo, que consiste en observar y recopilar una enorme cantidad de hechos sin notar sus interdependencias. La realidad económica, de hecho, no puede entenderse si se la considera como una masa de eventos no correlacionados. Su relativismo les impidió utilizar la economía política clásica tal como se construyó durante el siglo XIX. Ellos ``no sabían cómo utilizar el aparato de pensamiento abstracto de la economía política'' y les era ``imposible llegar a una comprensión de las interdependencias dentro del sistema económico''. 160 Terminaron en una confianza en el progreso, de sabor hegeliano, justificada por la fe, no sustentada en un esquema lógico.

El fatalismo ---observan los autores del Manifiesto de Ordo--- es común a escritores tan diversos como Marx y Spengler, el primero determinista, el segundo escéptico: según ambos, las cosas no se pueden cambiar, solo se pueden observar. El conjunto de la vida social, política e intelectual no puede verse como otra cosa que una ``superestructura'', superpuesta a una estructura que evoluciona según sus propias leyes, las ``fuerzas silenciosas internas'' mencionadas anteriormente.

El programa de los ordoliberales tiene como objetivo la construcción y organización de un sistema económico basado en una serie de principios 161 :

\begin{quote}
Adoptar una actitud científica en el estudio del derecho y la economía.

yendo más allá del relativismo del historicismo

al mismo tiempo, reconociendo la importancia de la evidencia histórica

comprender la constitución (estructura) de la economía para decidir cómo se debe reorganizar la actividad económica a través de la legislación. En este sentido, el problema de comprender y diseñar los instrumentos legales necesarios para una constitución económica sólo puede resolverse si el legislador se vale de los hallazgos de la investigación económica.
\end{quote}

El pensamiento ordoliberal parte, por tanto, de una crítica tanto del sistema ``natural'' en el que se basa la economía clásica como del determinismo historicista, pero acaba por injertar la economía clásica en el árbol del historicismo alemán. Esto quiere decir que toma de ambos lo que se considera convincente en sus aportaciones.

Las opiniones expresadas en los aportes de los ordoliberales no son unánimes: algunos insisten en el papel central del Estado, otros en la eficiencia de un mercado libre y competitivo. Estas diferentes visiones nos devuelven a las dos raíces del pensamiento económico que hemos tratado de describir, siguiendo los pasos de Schumpeter, en el capítulo 1. : uno basado en el individuo, alimentado por la filosofía de la Ilustración, y el otro centrado en el Estado, tal como lo definieron históricamente y lo describieron los filósofos alemanes. Los ordoliberales se basan en ambas visiones, pero prevalece la segunda visión. Probablemente esto se deba al hecho de que el renacimiento del liberalismo en Alemania tiene lugar en los años que preceden y siguen a la Segunda Guerra Mundial, y en Alemania las tradiciones de la disciplina de la economía habían ido en una dirección opuesta a la del liberalismo. de la doctrina económica clásica. No por casualidad, Gustav Schmoller había ignorado esta doctrina como ``economía de los vendedores ambulantes''. 162 Aunque el ordoliberalismo era crítico con la Escuela Histórica, como hemos visto, los ordoliberales estaban condicionados por ese mismo historicismo, y el papel del Estado era mucho mayor que en la doctrina clásica.

La tradición historicista y estatista es claramente visible en los escritos de Euken. Probablemente en ninguna parte más que en un artículo de 1948163---Donde se pregunta qué tipo de sistema económico debería construirse en la derrotada Alemania. Su enfoque es típicamente microeconómico. Euken observa que las reglas que rigen los sistemas económicos de los países industrializados deberían obtener idealmente los mismos resultados que los obtenidos en una pequeña economía de subsistencia cerrada. Cualquiera que gobierne esta pequeña economía debe tener un conocimiento detallado de lo que está sucediendo en toda la economía y ser capaz de evaluar su utilidad. También debería poder dar instrucciones sobre la forma más eficiente de utilizar todos los factores de producción. Todas las interdependencias de la actividad económica en esa pequeña economía son percibidas directamente por el gobernante, que puede tomar correctamente las decisiones adecuadas. 164

De manera similar, existen interdependencias en un gran país industrializado, pero no se detectan fácilmente de manera completa y actualizada. Entonces, lo que se necesita es ``alguna medida de escasez'', que indicaría qué bienes escasean y cómo se deben combinar los factores de producción para producir lo que se necesita. Este ``indicador de escasez'' viene dado, para las empresas y los hogares, por el sistema de precios, y sobre la base del precio los agentes económicos pueden calcular lo que se va a producir y la combinación más eficiente de los factores de producción, de una manera que sea similar a una máquina de calcular. Hasta ahora, Euken parece seguir la doctrina clásica (como lo ha vuelto a exponer, por ejemplo, Hayek). Pero, dado que los precios se determinan de manera diferente según las diferentes condiciones de los mercados en cuestión, no pueden ser un indicador de escasez en toda la economía. Las interconexiones entre los diferentes elementos de la economía hacen necesario ver cada acto de política económica en el contexto de todo el proceso económico. ``Un sistema económico tiene que controlar adecuadamente todo el proceso económico de forma razonable y para ello es necesario que sus componentes individuales se complementen entre sí''.165

Sin embargo, este indicador de escasez a nivel sistémico debe diseñarse como un instrumento para hacer el sistema más eficiente, no para alcanzar el pleno empleo. De hecho, el objetivo del pleno empleo puede conducir a un empleo de mano de obra en sectores improductivos, mientras que en otros sectores pueden surgir cuellos de botella como consecuencia de la escasez de factores de producción disponibles. Las inversiones deben elegirse bajo el control del Estado en la proporción correcta, en todos los sectores, industrias o incluso regiones o empresas específicas. El control de todo el proceso de producción y, a través de él, la consecución del pleno empleo, es una política sabia; no así una política que busque a priori un pleno empleo que enmascare o ignore el tema de la eficiencia. El Estado ordoliberal es un planificador, no con fines sociales sino para evitar el mal funcionamiento del mercado.166

En este esquema, la política fiscal está al margen: no hay espacio para ella, es decir, para fines keynesianos o de ``finanzas funcionales''. En un orden liberal perfecto, no hay necesidad de políticas de estabilización porque la actividad económica es estable por definición. 167 Por lo tanto, Euken no considera el uso de la política fiscal para fines de gestión de la demanda tanto en recesiones cíclicas como en el caso de problemas estructurales: incluso en este último caso, solo se necesitan reformas estructurales, operando del lado de la oferta de la economía, que es haciendo que la oferta sea más eficiente, no gestionando la demanda. 168

Si miramos la relación entre el sistema fiscal y el sistema económico, hay una diferencia notable entre la taxonomía de Beveridge, que implica un papel activo y funcional otorgado al primero, y la taxonomía aséptica adelantada, unos años después, por un ordoliberal. escritor, Kurt Schmidt, 169 donde no se prevé un vínculo entre la política presupuestaria pública y el empleo y, más aún, no se menciona el uso de la deuda pública para promoverla.

Los ordoliberales muestran igual renuencia a apoyar un papel activo de la política monetaria. En este sentido, los primeros ordoliberales expresaron, no por casualidad, una clara preferencia por el patrón oro. Friedrich Lutz, en un artículo que se remonta a 1935, 170escrito durante los últimos suspiros del antiguo régimen monetario, sigue destacando las ventajas de un sistema de tipos de cambio fijos, de contención del stock de dinero dentro de los límites de la reserva de oro, de una distribución internacional de las reservas de oro en relación con la evolución de la balanza de pagos, y de un mecanismo automático basado en unas pocas reglas de juego precisas, donde no se deja absolutamente nada a la gestión de los bancos centrales. El buen funcionamiento de este sistema implica la exclusión de las políticas monetarias independientes, ligadas a las condiciones específicas del ciclo económico, y del proteccionismo. Lutz reconoce sin embargo la profunda crisis del patrón oro: la pérdida de oro de un determinado país con un déficit en las cuentas exteriores es seguida por una caída, difícil de aceptar social y políticamente, en salarios y precios y al final por desempleo, para recuperar competitividad. Por eso ---observa--- este régimen ha sido recientemente abandonado por Reino Unido y Estados Unidos (y sería abandonado al año siguiente por Italia y Francia). El éxito decisivo de las ideas nacionalistas (estamos, vale la pena repetirlo, en 1935) lleva a Lutz a preguntarse si se podría introducir un sistema monetario ``controlado a nivel nacional'', como vía de escape del patrón oro. Contempla una serie de medidas técnicas para alcanzar ese objetivo, El éxito decisivo de las ideas nacionalistas (estamos, vale la pena repetirlo, en 1935) lleva a Lutz a preguntarse si se podría introducir un sistema monetario ``controlado a nivel nacional'', como vía de escape del patrón oro. Contempla una serie de medidas técnicas para alcanzar ese objetivo, El éxito decisivo de las ideas nacionalistas (estamos, vale la pena repetirlo, en 1935) lleva a Lutz a preguntarse si se podría introducir un sistema monetario ``controlado a nivel nacional'', como vía de escape del patrón oro. Contempla una serie de medidas técnicas para alcanzar ese objetivo,171 pero agrega que, como paso preliminar, debe resolverse el problema más general de la alternativa entre una economía libre y una economía planificada. 172Este tema no resuelto puede explicar por qué el pensamiento ordoliberal sobre política monetaria nunca fue más allá de un ``monetarismo'' genérico, es decir, prestar atención a mantener bajo control la base monetaria mediante reglas que excluyen la discreción, políticamente maniobrable, de los bancos centrales, para obtener dinero estable (ver Friedman, arriba). Esta posición va acompañada de una actitud escéptica sobre el uso de la política monetaria para sostener o frenar el crecimiento económico (un uso de la política monetaria que podría denominarse keynesiana) y, en cambio, de la afirmación de la neutralidad del dinero, cuya gestión activa pone en riesgo generando inestabilidad. En resumen, lo importante para los pensadores liberalistas no es la política monetaria, sino una constitución monetaria.173 Desconfían particularmente de la expansión de la oferta monetaria a través del crédito bancario, desconfianza que incluso llevó a Euken a adherirse a la Escuela de Chicago, de la que los ordoliberales estaban, en otros aspectos, principalmente el papel del Estado, bastante distantes.

Alfred Müller-Armack está más orientado hacia el neoliberalismo de la escuela austriaca y, por tanto, menos centrado en la centralidad del Estado. 174 En primer lugar, en todo caso confirma que el concepto de libre competencia tiene un papel central, no identificable con el laissez-faire porque requiere de una serie de garantías institucionales, encaminadas a prevenir restricciones comerciales y controlar monopolios, oligopolios y cárteles, para el beneficio de los consumidores. De esta manera, el sistema económico puede funcionar correctamente y realizar al mismo tiempo una función social. La competencia es un requisito previo de la libertad económica que no se puede encontrar ni generar en la esfera del sector privado.

Müller-Armack también se interesa por el Estado del Bienestar, pero en una perspectiva muy alejada de la de Beveridge. Las salvaguardias proporcionadas por el Estado de Bienestar deben mantenerse, pero cualquier intervención pública debe ser acorde con el mercado. Esto significa una mezcla de libertad de mercado y equilibrio social: una ``economía social de mercado'', una expresión familiarizada por las políticas del gobierno alemán, particularmente cuando Ludwig Erhard era ministro de Economía (1949-1963). Al respecto, se reconoce que es deber del Estado intervenir en la redistribución de la producción -a través del sistema estatal de pensiones, seguros sociales y diversos subsidios a las clases menos acomodadas- pero el gasto social no debe sobrepasar el umbral en que obstaculizaría el funcionamiento del mercado competitivo y la producción de ingresos. Respecto a la fiscalidad, Cumplir con el mismo principio significaría que tasas impositivas demasiado altas, diseñadas para financiar el gasto social, dañarían la producción de ingresos. En un mercado que funcione correctamente, escribe, la creación de nueva riqueza sería suficiente y capaz de tolerar una redistribución considerable de la riqueza sin un aumento excesivo de las tasas impositivas. Una vez más, no se le da ningún papel al financiamiento del déficit público con el propósito de redistribuirlo.

Wilhelm Röpke desarrolla más este tema: la ``libertad de la necesidad'' de Roosevelt es ``un concepto negativo'', porque significa que los necesitados tienen que depender de los demás, es decir, de tomar de los demás, para su propio sustento. Esto también significa que el Estado debe usar su poder de coerción para obligar a otros a sostener a los necesitados. Pero si la fiscalidad alcanza niveles excesivos, los recursos disponibles se agotan, en perjuicio de todos. En una sociedad libre, el mismo objetivo debe perseguirse principalmente de forma voluntaria, a través del ahorro, los seguros y las contribuciones voluntarias. Sólo así se superará la ``forma de existencia proletaria''. 175En la misma línea, Müller-Armack piensa que mantener los tipos de interés artificialmente bajos para facilitar el crédito a los deudores desfavorecidos no es coherente con el mercado, así como congelar las rentas en todo el mercado inmobiliario sin tener en cuenta el grado en que los arrendatarios pueden pagar las rentas, mientras que un sistema de rentas subvencionadas solo para los pobres sería compatible con el mercado. 176

El énfasis puesto por los ordoliberales en los temas políticos en un sentido amplio ha suscitado la pregunta: ¿qué nuevas ideas han aportado a la teoría económica? Se ha señalado que sus temas ---como la estabilidad monetaria, el mercado y su regulación, la competencia y la libertad comercial--- eran conceptos ya familiares para la economía, y que ``la verdadera razón {[}de su éxito{]} reside en el corazón mismo de la filosofía de la economía social de mercado\ldots{} sus teóricos produjeron pocas sugerencias concretas para la prevención del crecimiento excesivo del Estado''. 177Este es un comentario que solo puede aceptarse parcialmente. Observamos que, por un lado, no encontraremos en su trabajo, por ejemplo, ningún análisis profundo sobre las características de los diferentes tipos de mercados, ni ningún cálculo del efecto multiplicador de un determinado gasto, mientras que, por otro, Por otro lado, la economía dominante, que es analíticamente fuerte, a menudo no está interesada en el aspecto institucional, como un factor exógeno que debe darse por sentado. Los fenómenos que observan los ordoliberales son los mismos que los analizados por los economistas convencionales, pero desde un punto de vista diferente: no les interesan las regularidades legales de comportamiento que demarcan la economía como un campo de análisis social, ni en la construcción de ``modelos''. Como se mencionó anteriormente, un ``modelo ordoliberal'' simplemente no existe. Miran el lado prescriptivo, armados con instrumentos generalmente ignorados por los economistas, como la historia y el derecho, y están más interesados \hspace{0pt}\hspace{0pt}en una constitución económica que en una macrogestión activa de la economía (se ignora la macroeconomía). Si sus recetas fueron (son) propicias para el bienestar económico, es un tema de debate. Pero lo cierto es que ``la prevención del crecimiento del Estado'' encuentra en sus teorías condiciones y límites, más estructurados y argumentados que en otros análisis de economistas con diferente bagaje. Y el resultado fue una estrategia nacional del lado de la oferta que utilizó recursos culturales e institucionales tradicionales para asumir un papel principal a nivel europeo y global (como muestran claramente incluso los desarrollos actuales en Europa; pero este es un tema que se tratará más adelante en este ensayo). como la historia y el derecho, y están más interesados \hspace{0pt}\hspace{0pt}en una constitución económica que en una macrogestión activa de la economía (se ignora la macroeconomía). Si sus recetas fueron (son) propicias para el bienestar económico, es un tema de debate. Pero lo cierto es que ``la prevención del crecimiento del Estado'' encuentra en sus teorías condiciones y límites, más estructurados y argumentados que en otros análisis de economistas con diferente bagaje. Y el resultado fue una estrategia nacional del lado de la oferta que utilizó recursos culturales e institucionales tradicionales para asumir un papel principal a nivel europeo y global (como muestran claramente incluso los desarrollos actuales en Europa; pero este es un tema que se tratará más adelante en este ensayo). como la historia y el derecho, y están más interesados \hspace{0pt}\hspace{0pt}en una constitución económica que en una macrogestión activa de la economía (se ignora la macroeconomía). Si sus recetas fueron (son) propicias para el bienestar económico, es un tema de debate. Pero lo cierto es que ``la prevención del crecimiento del Estado'' encuentra en sus teorías condiciones y límites, más estructurados y argumentados que en otros análisis de economistas con diferente bagaje. Y el resultado fue una estrategia nacional del lado de la oferta que utilizó recursos culturales e institucionales tradicionales para asumir un papel principal a nivel europeo y global (como muestran claramente incluso los desarrollos actuales en Europa; pero este es un tema que se tratará más adelante en este ensayo). y están más interesados \hspace{0pt}\hspace{0pt}en una constitución económica que en una macrogestión activa de la economía (se ignora la macroeconomía). Si sus recetas fueron (son) propicias para el bienestar económico, es un tema de debate. Pero lo cierto es que ``la prevención del crecimiento del Estado'' encuentra en sus teorías condiciones y límites, más estructurados y argumentados que en otros análisis de economistas con diferente bagaje. Y el resultado fue una estrategia nacional del lado de la oferta que utilizó recursos culturales e institucionales tradicionales para asumir un papel principal a nivel europeo y global (como muestran claramente incluso los desarrollos actuales en Europa; pero este es un tema que se tratará más adelante en este ensayo). y están más interesados \hspace{0pt}\hspace{0pt}en una constitución económica que en una macrogestión activa de la economía (se ignora la macroeconomía). Si sus recetas fueron (son) propicias para el bienestar económico, es un tema de debate. Pero lo cierto es que ``la prevención del crecimiento del Estado'' encuentra en sus teorías condiciones y límites, más estructurados y argumentados que en otros análisis de economistas con diferente bagaje. Y el resultado fue una estrategia nacional del lado de la oferta que utilizó recursos culturales e institucionales tradicionales para asumir un papel principal a nivel europeo y global (como muestran claramente incluso los desarrollos actuales en Europa; pero este es un tema que se tratará más adelante en este ensayo). Pero lo cierto es que ``la prevención del crecimiento del Estado'' encuentra en sus teorías condiciones y límites, más estructurados y argumentados que en otros análisis de economistas con diferente bagaje. Y el resultado fue una estrategia nacional del lado de la oferta que utilizó recursos culturales e institucionales tradicionales para asumir un papel principal a nivel europeo y global (como muestran claramente incluso los desarrollos actuales en Europa; pero este es un tema que se tratará más adelante en este ensayo). Pero lo cierto es que ``la prevención del crecimiento del Estado'' encuentra en sus teorías condiciones y límites, más estructurados y argumentados que en otros análisis de economistas con diferente bagaje. Y el resultado fue una estrategia nacional del lado de la oferta que utilizó recursos culturales e institucionales tradicionales para asumir un papel principal a nivel europeo y global (como muestran claramente incluso los desarrollos actuales en Europa; pero este es un tema que se tratará más adelante en este ensayo).

Notas
1.
Baffigi2009, pag. 8).

\begin{enumerate}
\def\labelenumi{\arabic{enumi}.}
\setcounter{enumi}{1}
\item
  Si nos centramos principalmente en su origen.
\item
  ``Cuando los hechos cambian, cambio de opinión'', dijo Keynes, según se informa.
\item
  Keynes (1926, págs. 14-15).
\item
  Joan Robinson afirmaría entonces que con Keynes ``la economía volvió a convertirse en economía política''.
\item
  Skidelsky1992, pag. 224).
\item
  Trevelyan1944, pag. 557).
\item
  Stein1990, pag. 6).
\item
  Esta sociedad, que existe desde finales del siglo XIX, tomó su nombre del romano Quintus Fabius Maximus, el cunctator , que es el general de movimiento lento pero tenaz. Sería un símbolo del inexorable, aunque gradual, movimiento hacia el socialismo, en contraposición a la estrategia diversa, de cambios revolucionarios repentinos.
\item
  90 liras por libra británica, muy por encima de la tasa de mercado imperante que había llegado a 125 liras por libra.
\item
  Blackett1932, pag. 96).
\item
  Croce1973).
\item
  A diferencia de la ``Filosofía del Espíritu'', en la que no nos detendremos.
\item
  Croce1973, pag. 286). Joan Robinson definió el valor como ``una idea metafísica''; con la identificación del valor con el producto del trabajo (Ricardo, y luego Marx), la idea metafísica, dice Robinson, se convierte en una ``hipótesis'' (1974, págs. 29-30).
\item
  Al menos en un caso tenemos evidencia de una relación entre los dos: Keynes encargó a Croce un artículo para su serie de informes sobre Reconstrucción en Europa, para el Manchester Guardian, en 1922: La visión de la población de un filósofo. Ver Kelly (2019).
\end{enumerate}

dieciséis.
Croce1973, pag. 301).

\begin{enumerate}
\def\labelenumi{\arabic{enumi}.}
\setcounter{enumi}{16}
\item
  Bodei (2003).
\item
  Schumpeter, en una línea similar, cuestionó si la competencia perfecta es una construcción teórica o una realidad histórica (1947, pag. 107).
\item
  En este texto, me atengo, en su caso, a la traducción literal de la palabra italiana como neologismo en inglés.
\item
  Croce1973, pag. 264). Sin embargo, podríamos seguir el enfoque de Croce y ver la doctrina económica de Adam Smith como ``ética'', porque, en las circunstancias históricas de la Escocia de Smith (ese momento y lugar), estaba perfectamente en sintonía con la búsqueda de un sistema económico que definitivamente libraría a ese país. de las prácticas comerciales feudales anteriores.
\item
  págs. 259 y 264.
\item
  págs. 255, 256, 257.
\item
  pag. 260.
\item
  págs. 262-266.
\item
  ``Los economistas que emplean métodos cuantitativos, embrujados por la evidencia de sus procedimientos y no conscientes de que la suya es una evidencia nula, en lugar de limitarse a la construcción de sus esquemas muy útiles, aumentan la confusión filosofando de manera extravagante: como nosotros puede verlo en uno de los economistas más astutos y eruditos de nuestro tiempo''Pareto (p.~287).
\item
  ``Liberista'', adjetivo de ``Liberismo'', ``liberalismo económico''.
\item
  Croce2015, págs. 298-302) {[}publicado originalmente como ensayo único, 1927{]}.
\item
  pag. 303.
\item
  Montesano2003).
\item
  Luego hablará explícitamente de ``nacionalismo'' Croce (1955, pag. 283).
\item
  págs. 284-285 y 288.
\item
  Croce1941, pag. 163).
\item
  El verso es del poema ``I sepolcri'' de Ugo Foscolo, donde escribe que Maquiavelo mostró las terribles consecuencias del poder absoluto y desenfrenado del Príncipe. Lo mismo hizo Marx ---escribe Croce--- al mostrar las consecuencias de la ganancia capitalista.
\item
  Rathenau1919, pag. 62). Rathenau, copropietario y presidente de AEG, había sido director de la oficina alemana de Material de Guerra, en el Ministerio de Guerra durante la Primera Guerra Mundial. Luego sería nombrado secretario de Relaciones Exteriores de la República de Weimar, hasta su asesinato en 1922.
\item
  pag. 20.
\item
  pag. 54.
\item
  pag. 85.
\item
  pag. 64.
\item
  págs. 68-69.
\item
  págs. 62-87.
\item
  Einaudi1918, págs.450-456).
\item
  Einaudi y col.~(2006, págs.5 y 7).
\item
  Einaudi1972, pag. 6).
\item
  Einaudi, Lezioni (1964, págs.66-81). Véase también Baffigi, págs. 25-37.
\item
  Instituto de Nuevo Pensamiento Económico. La tradición italiana , www.hetwebsite.net/net/schools/italian.htm .
\item
  Einaudi y col.~(2006, pag. 17).
\item
  Robinson1974, pag. 72).
\item
  El economista (2016).
\item
  Pigou2013, pag. 127).
\item
  Citando a Edwin Cannan.
\item
  pag. 128.
\item
  Aslanbeigui, Oakes: Introducción: Reclamar a un maestro olvidado (Pigou, La economía del bienestar ).
\item
  pag. XIV.
\item
  Vea la Parte II. Sobre el dividendo nacional, Pigou sigue a Marshall y critica a Irving Fisher. Sobre este tema, Keynes cita a Pigou y lo explica en términos más claros (1936, pag. 38).
\item
  Vea la Parte II, Capítulos 1 y 2.
\item
  págs. 127-130.
\item
  págs. 134-135.
\item
  Hartford2018). Hartford agrega: ``el economista William Nordhaus ha estimado que durante la segunda mitad del siglo XX, las empresas innovadoras generalmente lograron capturar como ganancias solo el 3.7\% del valor social que crearon; el otro 96,3\% se destinó a otros, mayoritariamente consumidores. Por ejemplo, la penicilina salva la vida por unos centavos''. Si este discurso puede ser válido para las redes sociales, es discutible (ver Capítulo 4 ).
\item
  Pigou2013, pag. 192).
\item
  pag. 5.
\item
  Macmillan, 1949 {[}1937{]}.
\item
  pag. V.
\item
  págs. 15-16. Sobre la distribución, le indigna que el 1\% de las personas mayores de 25 años posea el 60\% del capital total (p.~13).
\item
  Esto recuerda, por ejemplo, la nacionalización de la energía eléctrica en Italia en la década de 1960.
\end{enumerate}

sesenta y cinco.
Robinson1974).

\begin{enumerate}
\def\labelenumi{\arabic{enumi}.}
\setcounter{enumi}{65}
\item
  pag. 73.
\item
  pag. 80.
\item
  JMK a R. Harrod, 16 de julio de 1938 (1973, págs.299-300).
\item
  Keynes (1949, págs.96 y 98).
\item
  Keynes (1925).
\item
  Keynes (1933) (Hitler acababa de tomar el poder).
\item
  Keynes (1926, págs.28-29).
\item
  Sobre Schumpeter, consulte el Capítulo 3.
\item
  Keynes (1964, págs.339-40).
\item
  Heibroner, Milberg (1995, pag. 31).
\item
  Según Joan Robinson (1974, pag. sesenta y cinco).
\item
  Skidelsky2018, pag. 386).
\item
  Jones (2013).
\item
  Notas finales sobre la filosofía social hacia la que podría conducir la teoría general. Es el capítulo 24 de la Teoría general del empleo, el interés y el dinero.
\item
  Capítulo 3: El principio de la demanda efectiva.
\item
  pag. 372.
\item
  pag. 30.
\item
  pag. 373.
\item
  pag. 374.
\item
  pág 136
\item
  pag. 376.
\item
  págs. 183-184.
\item
  pag. 381.
\item
  pag. 379.
\item
  Hoerber2017, Capítulo 7).
\item
  pag. 348.
\item
  pag. 339.
\item
  Hawtrey (1931, pag. 102).
\item
  Steil (2013).
\item
  Beveridge (1944, págs. 22-23).
\item
  Ver Beveridge (1942).
\item
  Beveridge (1944, pag. 147).
\item
  pag. 135 . Pleno empleo se publicó en 1944, teniendo muy en cuenta la estructura y metodología de un presupuesto de guerra.
\item
  Consulte la Parte IV y el Apéndice C de Pleno empleo, por Nicholas Kaldor.
\item
  Beveridge (1944) Parte IV, secc. 2
\item
  Beveridge (1944, pag. 148). Se tiene en cuenta la ``eutanasia del rentista'', mencionada por Keynes.
\item
  Lerner1944); (1951).
\item
  Macmillan, 1946.
\item
  Lerner1946, págs. 1-4).
\item
  Conferencia de Halley Stewart de 1931.
\item
  pag. 80.
\item
  págs. 168-169.
\item
  Hayek2008, pag. 71).
\item
  En realidad nació en Galicia, ahora Polonia, pero entonces (1840) formaba parte del Imperio Austriaco.
\item
  Una de sus principales obras es (1976).
\item
  Por esta razón, la acusación de Schmoller contra él como economista clásico fue mal dirigida.
\item
  Hayek, FA: Introducción a los principios económicos de Carl Menger , p.~13.
\item
  El trabajo de Menger fue descartado por Schmoller como ``meramente austríaco'' (!). Yagi1997).
\item
  Klein, P: Prólogo a los principios de Menger , p.~7.
\item
  Esto recuerda la frase atribuida a Joan Robinson: ``No estoy entrenado matemáticamente, luego tengo que pensar''.
\item
  Hayek1955, pag. 203).
\item
  págs. 189-196.
\item
  Hayek1945). Véase también lo citado anteriormente The Counter - Revolution in Science, 1955.
\item
  Hayek1944).
\item
  pag. 38.
\item
  pag. 89.
\item
  Consulte la Introducción a la edición de Routledge de 2008, editada por Bruce Caldwell.
\item
  pag. 13.
\item
  pag. sesenta y cinco.
\item
  Véase Bentham, más arriba (Capítulo 1 ).
\item
  pag. 13.
\item
  pag. 73.
\item
  Su libro está escrito en medio de la Guerra Mundial.
\item
  Esta escuela jurídica otorga a las decisiones del poder judicial un predominio sobre el derecho formal, con el riesgo de poner en peligro el principio de certeza del derecho. La posibilidad de decisiones altamente voluntaristas y arbitrarias es potencialmente explotada por dictaduras.
\item
  Sobre Schmitt, véase la Secta. 8 (ordoliberalismo).
\item
  pag. 91.
\item
  págs. 115-116.
\item
  Simons1945, pag. 1). Este escrito, como otros citados aquí, se incluye en Economic Policy for a Free Society , University of Chicago Press, 1948, publicado póstumamente después de la prematura muerte de Simons.
\item
  Un término usado por Walras en oposición a justicia distributiva (ver Capítulo 1 ). Este no es el único significado que se le da al término. Adam Smith se refiere a la justicia conmutativa como ``hacer voluntariamente todo lo que podamos con decoro ser forzados a hacer'' ( Teoría de los sentimientos morales , p.~334).
\item
  Simons1945, pag. 5).
\item
  Simons1934, pag. 41).
\item
  pag. 43.
\item
  pag. 41.
\item
  págs. 41-42.
\item
  Simons1945, págs.15-16); (1936, pag. 79).
\item
  Los requisitos (1936, pag. 79).
\item
  Fisher (1935).
\item
  Simons1934, págs.46-47).
\item
  Simons1945).
\item
  Robbins1963, págs.48-49).
\item
  Friedman (mil novecientos ochenta y dos, pag. 15).
\item
  pag. 174.
\item
  Consulte la siguiente sección de este capítulo.
\item
  págs. 26-28.
\item
  pag. 51.
\item
  págs. 53-54.
\item
  pag. 32.
\item
  Streek2015).
\item
  Ver sobre este punto Bonefeld (2016).
\item
  Caldwell2005, págs. 365-366). Caldwell se basa en un trabajo anterior de R.Cristi.
\item
  Pavo real y Willgerodt (1989, pag. 3).
\item
  Bonefeld (2012) ( \url{https://eprints.whiterose.ac.uk} ).
\item
  pag. 12.
\item
  El Manifiesto de Ordo de 1936 (Nuestra tarea), en Peacock, Willgerodt (Peacock, A. 1989).
\item
  pag. 21.
\item
  págs. 22-25.
\item
  Barry (1989, pag. 106).
\item
  Euken (1948).
\item
  Se ha observado que ``el ordoliberalismo es en el fondo un modelo microeconómico que desautoriza la política macroeconómica porque trata a los países, o incluso a toda una zona monetaria, como si fueran hogares individuales. Tiene sentido que las personas ahorren cuando están endeudadas, como lo hace el proverbial ama de casa suaba en Alemania. Pero si todas las personas recortan el gasto al mismo tiempo, el resultado puede ser un déficit en la demanda que anula los beneficios de las reformas microeconómicas. De vez en cuando es mejor romper las reglas que todos sufrir una miseria que respeta la ley''( The Economist {[}2015{]}).
\item
  Euken (1948, págs.28-29).
\item
  Bonefeld (2012, pag. 5).
\item
  Zettelmeier (2017, pag. 158).
\item
  Euken (1951).
\item
  Schmidt (1956).
\item
  Lutz1935).
\item
  Operaciones de mercado abierto por parte del banco central, adopción de un patrón de cambio de oro para ``ahorrar'' el uso de oro, intervenciones de tipo de cambio por parte del banco central, creación de un banco central internacional (págs. 238-240).
\item
  pag. 241.
\item
  Un ordoliberal como Wilhelm Röpke pidió un retorno al patrón oro (1951).
\item
  Müller-Armack (1956).
\item
  Röpke (1957).
\item
  Müller-Armack (1956, págs.82-86).
\item
  Barry (1989) pag. 121. Streek (2017).
\end{enumerate}

\hypertarget{part-entender-el-liberalismo}{%
\part{Entender el liberalismo}\label{part-entender-el-liberalismo}}

\hypertarget{enemigos-del-liberalismo}{%
\chapter*{Enemigos del liberalismo}\label{enemigos-del-liberalismo}}
\addcontentsline{toc}{chapter}{Enemigos del liberalismo}

Lejos de las alas anchas, el liberalismo, el nacionalismo estatista y el socialismo de Marx ocupan una posición fuerte en la economía política del siglo XX, con una fuerte influencia en las grandes dictaduras europeas. El corporativismo italiano representa una corriente de pensamiento original, en parte construida sobre la Escuela Histórica Alemana del siglo anterior, en parte funcional a los intereses proteccionistas de Italia, en parte basada en el concepto de Estado ético, donde se representan los intereses en conflicto de todas las clases sociales. en las ``corporaciones'' y reconciliado en el interés superior de la nación. Esto implica una política dirigista y la creación de un conjunto de instituciones cuasi gubernamentales: características que, en un contexto político diferente, volvemos a encontrar en la Italia posterior a la Segunda Guerra Mundial. El marxismo sigue siendo una ideología estática, en comparación con el dinamismo del liberalismo. Esto se debe a que el materialismo histórico es una interpretación de la realidad económica que no admite desviaciones y, posiblemente, al desempeño económico relativamente mejor de la Unión Soviética durante el largo período de Depresión que aflige a los países capitalistas en la década de 1930. La inflexibilidad doctrinaria hizo que los economistas marxistas fueran incapaces de deducir las inferencias apropiadas de los cambios que ocurrían en la estructura de la economía, en los modos de producción. En particular, la competencia ---que Marx había visto como la forma predominante de mercado del capitalismo--- había sido reemplazada por estructuras monopólicas donde la destrucción creativa del capitalismo era una fuente continua de fuerza (Schumpeter). La supervivencia de las ``leyes económicas'' en una economía socialista, negada por los marxistas puros, fue en sí misma un objeto de controversia. En el final, Los economistas socialistas veían su disciplina como una ciencia neutral de la gestión económica, reducida a una especie de ingeniero social y búsqueda de la eficiencia. El ``socialismo por defecto'' es una fórmula que aglutina a dos pensadores bastante diversos y no marxistas, pero con un fuerte sentido histórico, que conduce a ambos a un pronóstico básicamente erróneo: la caída del capitalismo y el advenimiento del socialismo. Schumpeter, criticando a Weber que había dicho que viviremos con el capitalismo ``hasta que se queme la última tonelada de carbón fosilizado'', cree que el capitalismo morirá de una especie de agotamiento, no por una revolución sino como consecuencia del aburrimiento de la clase burguesa y la burocratización de industrias gigantes, donde los administradores reemplazarán al empresario en fuga (a diferencia del propietario de la empresa). La visión cristiana de Polanyi es crítica con el liberalismo económico. La sociedad en su conjunto, a diferencia de cualquier clase social, corre el riesgo de autodestrucción por las fuerzas de la economía de libre mercado, donde desempeñan un papel fundamental.haute finance . En una nueva sociedad socialista, el trabajo, la tierra y el dinero se liberarán de las limitaciones del mercado libre. Esta sociedad se apoyará en las tradiciones cristianas, como atestigua el Antiguo Testamento, las enseñanzas de Jesús, el socialismo utópico de Robert Owen.

Palabras clave

\begin{itemize}
\tightlist
\item
  Corporativismo
\item
  Economía marxista
\item
  Schumpeter
\item
  Polanyi
\end{itemize}

\hypertarget{nacionalismo-y-corporativismo}{%
\section*{Nacionalismo y corporativismo}\label{nacionalismo-y-corporativismo}}
\addcontentsline{toc}{section}{Nacionalismo y corporativismo}

El siglo XX vio al liberalismo desafiado por el nacionalismo y el socialismo. Esto fue particularmente relevante en el campo de las doctrinas económicas. De hecho, las metamorfosis del liberalismo en el siglo XX también se debieron a su influencia: los pensadores políticos y económicos liberales fueron a menudo seducidos por estas doctrinas, a veces por el encanto de un Estado omnipresente, otras veces por el impulso igualitario de una sociedad socialista.

A principios del siglo XX, y en particular después de la Primera Guerra Mundial, el nacionalismo significó por un lado la culminación de ese proceso de independencia de varios estados europeos, que había caracterizado el siglo anterior. Por otro lado, el nacionalismo perdió su ímpetu liberal, y en estados más grandes y ya bien establecidos dio un fuerte giro hacia el autoritarismo, incluso apoyando regímenes abiertamente dictatoriales; en estos estados el nacionalismo, como doctrina económica, no hizo más que reafirmar la idea que ya había planteado la Escuela Histórica Alemana de Economía: un sistema económico centrado en un ``Estado ético''. Esta mezcla de autoritarismo y nacionalismo económico encontró una expresión muy lograda en Italia: como veremos, el sistema corporativista italiano fue un hijastro del nacionalismo económico dentro de un marco fascista.

Después de la Segunda Guerra Mundial, con la derrota de las principales potencias nacionalistas y autoritarias, el nacionalismo siguió un camino descendente. A veces se convirtió en un modelo político para áreas subdesarrolladas del mundo que querían deshacerse de su pasado colonial. Sin embargo, el nacionalismo ha vuelto recientemente con fuerza, como reacción a la globalización extendida que acompaña a la expansión del liberalismo económico y a la Gran Recesión que siguió a la crisis financiera de los últimos años. Centrado generalmente en los intereses imperantes de la nación y potencialmente inclinado hacia el mercantilismo o proteccionismo, muestra una línea de continuidad con el antiguo nacionalismo, pero sus características aún no están claras, como veremos más adelante en el capítulo 4 .

Como se considera el socialismo, tuvo una influencia notable en el pensamiento económico de muchos autores de origen liberal, como hemos visto en el capítulo anterior; pero, por otro lado, los economistas marxistas, los intérpretes ortodoxos de la doctrina, permanecieron esencialmente apegados al verbo del maestro, en una especie de inflexibilidad ideológica. Los economistas marxistas Baran y Sweezy escribieron explícitamente sobre un ``estancamiento de las ciencias sociales marxistas''. 1 Como Marx, dedicaron sus estudios más a una crítica amplia y destructiva de los regímenes capitalistas de libre mercado, que a ajustar su pensamiento a un entorno social y económico en profunda evolución, a los inevitables cambios incluso en sociedades de orientación socialista. El marxismo sufrió las deficiencias y el colapso final del Estado, que era la principal encarnación del socialismo.

Consideraremos el nacionalismo autoritario de la primera mitad del siglo XX en Sectas. 3.1 y 3.2 , y socialismo marxista en las sectas. 3.3 - 3.6 . La sección 3.7 está dedicada principalmente a dos pensadores no marxistas, Schumpeter y Polanyi, que hicieron el pronóstico fundamentalmente erróneo de un advenimiento del socialismo: una especie de socialismo que no es perseguido realmente por una revolución ``necesaria'', sino que sigue el agotamiento ``necesario''. del libre mercado, sociedad liberal: una especie de socialismo por defecto. Terminaron siendo más (erróneamente) deterministas que el propio Marx.

En cuanto al pensamiento económico, Italia es quizás el país que en el cambio de siglo estuvo intelectualmente a la vanguardia en la reanudación de las viejas ideas de la Escuela Histórica Alemana, que estaba, a principios del siglo XX, en un camino de decadencia. Sin embargo, sus teorías dieron fuerza a los nacionalistas italianos, dispuestos a afirmar en la arena internacional la posición de su país, que emergía de la reciente lucha por su propia independencia: una posición que Alemania había conquistado en años no muy lejanos. Esos pensadores italianos lucharon contra la visión positivista, individualista y utilitaria de los economistas neoclásicos, y trasladaron el foco de sus reflexiones del individuo al Estado ético que todo lo absorbe, pasando, por así decirlo, de Comte y Marshall a Hegel y Schmoller. desde la visión de la economía como una ciencia a estudiar como ciencia natural, hasta una investigación inductiva, históricamente arraigada, de las condiciones económicas italianas específicas. Como se acaba de mencionar, Italia estaba de hecho en una posición similar a la Alemania de mediados del siglo XIX: ``La menor de las grandes potencias'',2 pero ambicioso por ganar esa posición preeminente que, en la retórica de la época, se había perdido durante tanto tiempo, después de las glorias del imperio romano. Y de nuevo de manera similar a Alemania, elSistema Nacional de Economía Políticay el proteccionismo económico deFriedrich Listganaron terreno dentro de los círculos académicos italianos.

En este contexto, si tenemos en cuenta a los grandes economistas italianos de tradición liberal, no se pudo evitar un acalorado debate entre proteccionistas y librecambistas. Originalmente, se centró en el alto arancel que se había introducido en Italia en 1887, y había sido seguido por un fuerte aumento de los derechos de importación sobre el trigo y el azúcar. Esta protección ha tenido consecuencias desiguales en diferentes sectores de la economía italiana. Sin embargo, durante el largo mandato de Giolitti al frente del gobierno italiano, la relevancia de esas medidas había ido disminuyendo: la mayoría de ellas eran impuestos especiales (impuestos por unidad) y, por lo tanto, en una larga fase de aumento del nivel de precios, su el efecto se había reducido; Además, la fuerza relativa de la lira en el mercado de divisas contribuyó a que su carga fuera menos pesada. 3Si bien esta evolución tendió a dar ventaja a los argumentos de los librecambistas, en 1913 el ministro de Agricultura, Industria y Comercio, Francesco S. Nitti, nombró una Comisión Real para investigar todo el asunto del régimen de derechos y los tratados comerciales. .

No nos detendremos en este debate, animado ---particularmente en el frente del libre comercio--- por algunas de las mejores mentes de la disciplina económica italiana --- como Einaudi, De Viti de Marco, Luzzatto, Borgatta, Ricci, todos economistas liberales. Con los ojos de hoy, podemos establecer ese debate en una discusión entre los economistas de la corriente principal y aquellos que se inclinaban por la Escuela Histórica, a quienes los primeros incluso se mostraron reacios a calificar como economistas. Se puede considerar que el primer grupo lleva adelante argumentos analíticos, mientras que el segundo se inclina hacia un enfoque pragmático, que refleja intereses sectoriales concretos y posiciones oficiales. 4

La falta de argumentos analíticos fue, sin embargo, contrarrestada desde el lado proteccionista y nacionalista al enfatizar cuestiones que estaban cobrando nueva fuerza, en particular las ideas que habían caracterizado al historicismo económico alemán: la centralidad atribuida al Estado en el gobierno de la economía, el rechazo de los principios liberales de la doctrina clásica y neoclásica. En el aspecto político, se tuvo muy en cuenta la protección otorgada a la industria alemana bajo el emperador Wilhelm II. Los nacionalistas italianos dieron especial énfasis a los aspectos autoritarios de una política proteccionista, con un giro de pensamiento que luego los llevaría al campo fascista. Una figura destacada en este sentido es Alfredo Rocco, quien, ya en 1914, había sentado las bases de lo que luego se llamaría ``doctrina corporativista''.

En un texto firmado conjuntamente por Alfredo Rocco y Filippo Carli, se destacan dos principios: (1) los métodos de producción deben ajustarse a la producción en masa. Esto implica, según ellos, un rechazo a la libre competencia: ``el régimen de competencia es esencialmente un régimen de crisis'', y la lucha por captar clientes solo significa destrucción recíproca. Lo que se necesita es ``solidaridad'', ``asociacionismo'', que se promulgue a través de los sindicatos industriales (cárteles), que de hecho son el resultado del ``malestar general de una competencia desenfrenada y de la sobreproducción'': grandes conglomerados industriales, integrados tanto horizontal como verticalmente 5; (2) La producción nacional debe ser defendida, no solo confirmando, sino también reforzando la protección de la industria, para colmar el retraso actual con respecto a otras economías más avanzadas: una defensa defendida en particular por ciertos sectores industriales, como acero, construcción naval, azúcar. Los autores critican la teoría ricardiana de las ventajas comparativas, porque obliga a los países pobres a mantener indefinidamente su modelo de producción, sin ninguna posibilidad de crecimiento económico que vendría de la diversificación de su economía (un argumento claramente tomado de los escritos de List). 6

List, el teórico del proteccionismo, es visto por Rocco y Carli, precisamente por su batalla contra la Escuela Clásica, como el fundador de la ciencia económica alemana. 7La principal razón por la que el historicismo económico alemán merece ser elogiado, según estos autores, es su enfoque rigurosamente empírico-inductivo, basado como está en el estudio de la economía nacional de un pueblo específico en un momento específico de su historia. Contra el ``cosmopolitismo'', la ``sociedad mundana'' de la Escuela Clásica (el ``globalismo'', en el lenguaje actual), los economistas alemanes habían elaborado propuestas a la medida de la Alemania de su época, encaminadas a promover la intervención del Estado en el ámbito económico y económico. campos sociales. Desde esta perspectiva, Rocco y Carli condenaron tanto el liberalismo como el socialismo, que compartían la idea de ``desintegración'' 8 de la comunidad nacional, cuyo propósito común los historicistas alemanes y los dos italianos, por el contrario, mantenían en la más alta consideración. 9Rocco y Carli escriben que ``los individuos {[}deben verse{]} ya no como un fin, sino como simples instrumentos y órganos de la sociedad nacional'' 10 (es como leer a Hegel). Por tanto, rechazan la economía basada en el individuo, en el utilitarismo benthamita, en el materialismo, en el internacionalismo. El principal objeto del análisis del economista debe ser ``el estudio de las condiciones de la economía nacional italiana''. Rocco y Carli sostienen que las ``causas de {[}su{]} inferioridad'', 11 que limitan su capacidad productiva y, en términos más concretos, hacen necesaria la extensión de la protección arancelaria de 1887, dependen tanto de factores ``naturales'' como ``transitorios''. Los primeros están relacionados con el territorio nacional, en gran parte estériles, 12con escasez de materia prima y no apto para comunicaciones fáciles; mientras que los segundos dependen de la falta de capital y espíritu empresarial, y de la escasez de capacidades técnicas y de gestión.

Este llamado a una fuerte protección aduanera respondió a los intereses no solo de la industria pesada, sino también del estamento militar. No por casualidad, como corolario de su estudio del desarrollo italiano, los dos autores invocaron fuertemente una guerra colonial, con el fin de obtener las materias primas necesarias para las necesidades energéticas nacionales, un nuevo espacio donde destinar la mano de obra italiana en exceso. y abrir nuevos mercados a los productos nacionales. En conclusión, parece que el nacionalismo económico se basa en el proteccionismo, la concentración monopolística de la producción, la expansión colonial. 13

Pero el significado del corporativismo va más allá de la conducción de una política económica nacionalista y proteccionista, y no puede captarse sin su raíz filosófica: el Actualismo del filósofo Giovanni Gentile. Arremetió contra la visión del Estado que sólo realiza actividades auxiliares en beneficio del individuo y, desde una perspectiva hegeliana y antiliberal, veía al individuo indisolublemente conectado con el Estado, muy lejos del hombre ``abstracto'' de la sociedad. Ilustración y filosofía liberal. ``Una economía corporativista reconocería el carácter social de la producción, con la iniciativa individual regida por las necesidades sociales y los fines sociales'' 14 , en el interés superior de la nación.

A diferencia de la opinión predominante que vincula el corporativismo con la tradición hegeliana y estatista, el historiador estadounidense James Gregor encontró en este enfoque un vínculo inesperado con la ``Voluntad General'' de Rousseau y con el papel del Estado en el campo de la educación ciudadana, de modo que las personas pueden expresarse con una voz colectiva (ver arriba, Capítulo 1 ). Este sistema educativo estatista haría que los individuos fueran lo más uniformes posible en sus valores y aspiraciones, de modo que se pudiera lograr una armonía de la voluntad general y tomar decisiones compartidas. La acción del gobierno simplemente ejecutaría esta voluntad general, en una especie de ``democracia totalitaria''. 15 La visión rousseauniana de que la voluntad del pueblo no puede delegarse en el parlamento es fundamental para la visión corporativista del Estado y resurge en el populismo actual (capítulo 4 ).

La idea corporativista también es central en el pensamiento de Ugo Spirito, filósofo e ideólogo del régimen fascista: el mundo real, que ``históricamente está evolucionando,\ldots{} ha sido alejado por el economista de las circunstancias siempre cambiantes''. El agente económico ha sido visto como ``una especie botánica naturalista: homo oeconomicus , natural y científicamente analizable''. ``Contra la economía liberal queremos enfrentar a la economía corporativa. El primero dice que el hombre es todo lo que importa, y que la sociedad, o el Estado, es sólo una garantía para el individuo; el segundo establece que el individuo debe identificarse con el Estado, y que estudiar al individuo significa estudiar al Estado como organismo''. dieciséisComo sello a estas palabras, la entrada de la Enciclopedia Treccani titulada ``Fascismo'', firmada por Mussolini pero probablemente escrita por Giovanni Gentile, dice lo siguiente: ``El que dice liberalismo significa individuo, el que dice fascismo significa Estado''. El resultado inevitable es que la distinción artificial entre lo que se concibe como ``público'' y lo que se concibe como ``privado'' en la economía nacional va a desaparecer.

Para tener un corporativismo realista fascista real, Spirito concibió un sistema de empresa que reemplazaría a la sociedad anónima liberal: una empresa donde desaparecería la distinción entre propiedad y gestión responsable y entre empresarios y trabajadores, en una comunidad que desaparecería. descansar en el interés colectivo, el esfuerzo colectivo, las recompensas colectivas. 17 Si esta posición puede interpretarse solo como una tapadera de los intereses industriales privados, es decir, como una forma de callar a la clase trabajadora, o como un paso hacia una economía socializada, si no socialista, es un tema sobre el que volveremos en la Secta. 3.2. Sin duda, había industriales que temían una evolución bolchevique guiada por los izquierdistas del régimen. Otros se quejaban ---entre ellos--- de que la ausencia de disturbios y huelgas obreras era una ventaja, pero tener que obedecer las directivas del régimen en sus proyectos industriales era frustrante. 18

Coherente con esta filosofía económica abiertamente antiliberal es la idea de un modelo específico de relación capital-trabajo dentro del Estado corporativo. Hemos visto anteriormente (Capítulos 1 y 2 ) cómo ---para un economista clásico--- el salario de equilibrio es el que corresponde a ``algo más'' que la manutención del trabajador; cómo ---para un economista marxista, que se basa en fundamentos ricardianos--- hay una identificación del salario con el valor total de lo que se produce (y el reclamo de un trabajador de la reapropiación de lo que se resta por la ganancia capitalista); cómo -para un liberal del siglo XX- el salario tiene que estar alineado con la competitividad de la empresa, o -según otros- ajustado, con la intervención del Estado, con la condición de inferioridad del trabajador frente al capitalista (ya que el poder de negociación de las partes sociales opuestas es diferente).

Por su parte, Rocco y Carli, cuyo objetivo es la ``elevación de la clase obrera'', piensan que el salario de equilibrio se puede alcanzar a través de la ``corporación'' (``Nuestro viejo corporativismo'' -escriben- que ha sido abrumado por la jus- individualismo naturalista (ley natural), y por la revolución francesa). Los agentes de la economía nacional son solo dos: el Estado y el individuo, y este último, empleador o trabajador, tiene un mandato social, lo que significa que no opera en su exclusivo interés. 19 En consecuencia, los conflictos sociales, que expresan intereses específicos, deben superarse y conciliarse en interés de la nación. La lucha de clases existe, pero el sindicalismo marxista es ``antinacional y antiestatal''.

Para prohibir la autodefensa de clase (podríamos decir: un sindicato independiente de trabajadores), es necesario establecer un sistema que lo haga imposible. Este sistema está conformado por dos instituciones: (1) Sindicatos que agrupan a empleadores y trabajadores: sólo este tipo de sindicatos, legalmente reconocidos y sometidos al control del Estado, pueden representar legalmente a todos los trabajadores y empleadores de una determinada industria / sector; (2) Contratos colectivos de trabajo: son estipulados por estos sindicatos y vinculan a todas las personas que pertenecen a ese sector, miembros y no miembros por igual: de hecho, el contrato colectivo encarna la solidaridad de todos los factores de producción y concilia intereses particulares opuestos. ,20 ). El contrato colectivo sería una solución interclasista a los conflictos laborales.

Estos conceptos están consignados en la Carta Laboral de 1927. 21 La nación es una unidad ética, política y económica; es un cuerpo que está por encima de los individuos, singularmente tomados o agrupados, que son los componentes de la nación. Las corporaciones son órganos del Estado que constituyen la organización unitaria de las fuerzas productivas, cuyos intereses representan. 22Estructuradas a lo largo de varias ramas productivas, las corporaciones son el lugar donde los grupos de interés, que en los sindicatos permanecen en líneas paralelas, pueden encontrarse y resolver sus diferencias. Las corporaciones tienen poderes tanto consultivos como normativos, por lo que pueden dictar reglas obligatorias sobre las relaciones laborales y la coordinación de la producción. La iniciativa privada es la más eficiente para atender los intereses nacionales, mientras que la intervención del Estado se da solo cuando falta la iniciativa privada o entran en juego los intereses políticos superiores del Estado. Los sindicatos patronales están obligados a estimular el aumento o elevar la calidad de la producción, disminuyendo los costos. Mientras que, en un régimen de libre competencia, el salario tiende al nivel de los costos de producción, en el régimen corporativo el trabajador puede obtener más pero no por encima del umbral más allá del cual otros resultarían perjudicados.23 En opinión de Filippo Carli, el ``salario corporativo'' incorporaría componentes éticos e históricos, sea lo que sea que esto signifique.

Lo que surge del sistema corporativista es lo opuesto a una economía de libre mercado: un monopolio bilateral controlado por el Estado, en lo que respecta al mercado laboral; y un régimen oligopólico, con cárteles y consorcios, para la mayoría de los demás mercados.

Según estudios recientes, a través del sistema corporativista, la distribución del ingreso se hizo menos desigual; durante la Depresión de la década de 1930, los salarios reales estaban protegidos, aunque en una situación de disminución del empleo y la jornada laboral. ``Se puede suponer razonablemente que, sin la protección de los contratos colectivos, las cosas podrían haber sido mucho peor para los trabajadores''. 24

\hypertarget{diferentes-interpretaciones-del-corporativismo}{%
\section*{Diferentes interpretaciones del corporativismo}\label{diferentes-interpretaciones-del-corporativismo}}
\addcontentsline{toc}{section}{Diferentes interpretaciones del corporativismo}

Como bien sabemos, el corporativismo no sobrevivió a la experiencia histórica de la Italia fascista. Sin embargo, fue otro giro, uno autoritario, el que dio el nacionalismo en el siglo pasado. Su interpretación puede resultar interesante para comprobar cómo algunas ideas de su organización económica y social siguen afectando a las filosofías económicas nacionalistas o social-liberales de nuestro tiempo. Su interpretación también es útil para verificar si existió una ``política económica corporativa fascista'', como sugieren algunos estudios. 25 Mencionaré aquí el punto de vista izquierdista y marxista; la visión dirigista y la visión tecnocrática. Más adelante, revisaré brevemente lo que quedó del corporativismo en la Italia posguerra y posfascista.

Según la primera interpretación, las grandes empresas (industria pesada y finanzas) apoyaron activamente al movimiento fascista y fueron el principal beneficiario del régimen, en un do ut des social contract, al mantener los salarios reales no por encima del nivel de subsistencia, 26y por medidas proteccionistas. El corporativismo fue un instrumento para proteger los intereses industriales. La pequeña burguesía, atrapada en el medio entre la gran burguesía y el proletariado, fue víctima de la Gran Guerra, la inflación, la crisis del capitalismo y, por lo tanto, se frustró y se empobreció. Estas personas, sin embargo, no podían ir a la izquierda, su objetivo no era la ``lucha de clases''; por el contrario, no querían perder su estatus social, aunque en su mayoría aparente; odiaban el desorden social y tenían un fuerte concepto de nación: ``empobrecidos pero no proletarizados''. 27 De nuevo en un do ut des bargain, el fascismo les dio una lira fuerte (hasta que duró) que benefició a las clases medias como rentistas, 28y un Estado de Bienestar ampliado: el fascismo como alianza entre la gran y la pequeña burguesía. 29 Su proclamada postura anticapitalista era simplemente demagogia. Y, por supuesto, el Estado ético era una ficción, si no una mera estafa, sobre los hombros de la clase trabajadora.

Michal Kalecki dio una explicación marxista del apoyo de las grandes empresas a las dictaduras fascistas y nazis. 30Escribe que, en un sistema capitalista de libre mercado, la oposición de la clase capitalista a la gran intervención estatal a través de políticas de gasto público se basa en tres razones: la aversión a la interferencia del gobierno en el problema del empleo; la aversión a las orientaciones del gasto público, que podría extenderse para incluir nacionalizaciones y provocar un desplazamiento de las inversiones privadas; y la aversión a los subsidios al consumo, que van en contra del principio moral de la ética capitalista de ``te ganarás el pan con sudor''. Además y sobre todo, en una situación de pleno empleo así creada, la posición social del capitalista se vería socavada por la conciencia de la clase obrera, con huelgas y malestar social y tensiones políticas. Las dictaduras fascistas o nazis eliminan estas objeciones capitalistas poniendo la maquinaria del Estado bajo el firme control de las grandes empresas: la disciplina en las fábricas y la estabilidad política se mantienen con el ``nuevo orden, que va desde la supresión de los sindicatos hasta la campo de concentración. La presión política reemplaza la presión económica del paro''.31

En cuanto al dirigismo, Ugo Spirito señaló que la economía política del régimen, liberal en su primera fase, luego Estado-socialista en la segunda (cuando los rescates de los grandes bancos fueron decretados por el Estado), se movió en la dirección del ``corporativismo integral''. '', Concepto que va`` mucho más allá del liberalismo y el socialismo '', a través de un alejamiento crítico de la idea liberal del individuo libre, llegando a la identificación de individuo y Estado. El corporativismo resolvía la contraposición del capital y el trabajo mediante su unificación en el interés superior del Estado. 32El hecho de que cualquier distinción entre propietarios, empresarios y trabajadores se derritiera en un sistema de intereses y recompensas colectivas, explica los temores, antes mencionados, de importantes líderes industriales de que el corporativismo pueda evolucionar hacia una especie de bolchevismo. Aunque, de hecho, las corporaciones fascistas aparecieron más cerca de los intereses de los empresarios, el resultado fue un foro de ``concertación entre productores'' 33 , una especie de burocratización de la economía.

La visión tecnocrática es un derivado de la interpretación dirigista, en el sentido de que también se basa en una fuerte participación del Estado en la economía. Enfrentado a problemas urgentes y sistémicos, Mussolini se limitó a hablar de labios para afuera al corporativismo y, además, dejó de lado al partido fascista, desconfiando de la competencia y temiendo la ``rivalidad de los caciques despreocupados de su partido''. 34 Trasladar el foco del proceso de toma de decisiones a la persona del propio ``Il Duce'', tuvo como consecuencia el surgimiento de un grupo de tecnócratas, a menudo personalidades destacadas, vinculados sólo en parte al partido fascista. La perspectiva tecnocrática tiene su expresión institucional en la creación de más de 300 enti pubblici (organismos cuasi estatales), según un nuevo modelo de organización de una administración pública seccional. 35

Estos dos últimos puntos de vista, tomados en conjunto, ven el corporativismo como una ``tercera vía'' entre el liberalismo económico y el socialismo. Pueden reconectarse a una línea de pensamiento italiana específica, de ``economía civil'', que se remonta al pensamiento de la Ilustración (Genovesi, Filangieri) hasta Giuseppe Toniolo y FS Nitti. Siguen un paradigma de bienestar común, cooperación interclasista, distribución de ingresos más equitativa, gran intervención estatal. El homo corporativus respondería a los objetivos de reequilibrar los intereses de las diferentes clases sociales y perseguir una ``economía mixta''.

En cuanto a la supervivencia de las ideas corporativas en la república italiana de la posguerra, es posible decir que el énfasis en el trabajo como pieza central del sistema económico todavía es bien visible: la constitución italiana, en su primer artículo, establece que ``Italia es una República fundada en el trabajo''; Además, el sistema de contratos colectivos sigue vigente. Sobre la intervención del Estado en la economía, hay que tener en cuenta que esas entidades cuasi estatales e instituciones financieras encarnaban, más que la idea de un Estado corporativo, un mecanismo de desarrollo económico apoyado en una élite tecnocrática apolítica. Incluso si este método no fue inmune a la intromisión, las prácticas colusorias y las presiones políticas sobre los organismos ``independientes'', el marco institucional era sólido y, de hecho, duradero, sobrevivió a la caída del fascismo y, nuevamente, caracterizó varias décadas de la recién nacida República Italiana y su economía. Lo que definitivamente terminó con la caída del fascismo fue la actitud proteccionista. La Italia de la posguerra abrió sus fronteras a la competencia internacional: desde este punto de vista, el corporativismo estaba definitivamente muerto.

Cabría preguntarse si en la Alemania nazi ocurrió una experiencia similar al corporativismo italiano. La respuesta probablemente sea negativa, y la experiencia corporativista siguió siendo un producto típico italiano. En el mismo período de tiempo, el ordoliberalismo representó en Alemania el verdadero desarrollo intelectual nuevo en el campo de la filosofía económica. Y si existía un vínculo entre el profundamente antiliberal Carl Schmitt y el ordoliberalismo (véase el capítulo 2 ), no se puede encontrar tal vínculo entre él y el fascismo italiano. ``El fascismo, en pos de objetivos antidemocráticos, prefiere apoyarse en el apoyo ideológico del Actualismo de Giovanni Gentile y en el Estado ético, mucho más comedido y tranquilizador que el vertiginoso y extremista pensamiento schmittiano''. 36

\hypertarget{filosofuxeda-econuxf3mica-marxista-despuuxe9s-de-marx-sin-cambios}{%
\section*{Filosofía económica marxista después de Marx: sin cambios}\label{filosofuxeda-econuxf3mica-marxista-despuuxe9s-de-marx-sin-cambios}}
\addcontentsline{toc}{section}{Filosofía económica marxista después de Marx: sin cambios}

Entre las filosofías y teorías económicas por un lado, y las instituciones y políticas económicas por el otro, ha existido constantemente una relación bidireccional. En los capítulos anteriores hemos visto cómo el liberalismo influyó profundamente en la política y la economía entre los siglos XIX y XX, siendo a su vez afectado por los dramáticos acontecimientos de la Primera Guerra Mundial y la Gran Depresión, y por los logros del socialismo y el nacionalismo en el primeras décadas del XX.

En cuanto a la doctrina marxista: ¿tuvo, en el transcurso del siglo pasado, algún tipo de evolución, particularmente a la luz de las tendencias de la economía soviética, es decir, del país que quiso incrustar la idea misma de sus engendradores? Parece que durante mucho tiempo los economistas marxistas no dedicaron muchos esfuerzos a desarrollar la visión marxista más allá de lo que surge de los trabajos de Marx y Engels, y a ajustar esta visión a las circunstancias cambiantes del mundo real, tanto del libre mercado como del socialismo. economías.

El socialismo ---en su versión marxista--- siguió siendo en el siglo XX una ideología estática, en comparación con el dinamismo del liberalismo, agitado en diferentes direcciones por un continuo replanteamiento intelectual de visiones anteriores o en competencia. Se pueden mencionar algunos factores al respecto.

La ausencia de una actitud autocrítica similar se debe, al menos en parte, a que el marxismo es una doctrina en el sentido estricto de la palabra: el materialismo histórico es una interpretación de la realidad social y económica que no admite desviaciones, no a mencionar perspectivas alternativas. Al leer a los economistas marxistas, incluso a los más abiertos a una visión benévola, aunque crítica, del pensamiento liberal (de los economistas de la escuela clásica, en particular), no se puede evitar la impresión de que, para ellos, la doctrina de Marx tiene un contenido realmente sagrado. es un ``credo'', casi un acto de fe: las desviaciones son actos para ser estigmatizados.

Durante la década de 1930, cuando las principales economías capitalistas seguían estancadas tras la Gran Depresión, mientras la Unión Soviética disfrutaba de una fase de crecimiento relativamente benigna e ininterrumpida, Sidney y Beatrice Webb elogiaron al comunismo como una nueva civilización: ``Esta transformación fundamental de El orden social ---la sustitución de una producción planificada por el consumo comunitario, en lugar de la obtención de beneficios capitalista de la llamada''civilización occidental" - me parece {[}sic{]} un cambio tan vital para mejor, tan propicio para el progreso de la humanidad a un nivel superior de riqueza y felicidad, virtud y sabiduría, como para constituir una nueva civilización ''. Y el comunismo no les parecía tan alejado de los valores cristianos, porque las nuevas instituciones ``no eran contrarias a la filosofía viva de la religión cristiana''. 37Los líderes políticos de las democracias capitalistas ---añadieron--- consideraban esa filosofía como la piedra angular de la sociedad, pero de hecho estaba muy lejos del impulso fundamental de una sociedad lucrativa. 38

Los Webb a veces son vistos como ``idiotas útiles de Stalin'', pero esto es lo que un filósofo liberal como Bertrand Russell escribió en 1920: ``Las ideas fundamentales del comunismo no son de ninguna manera impracticables y, si se hicieran realidad, se sumarían inconmensurablemente al bienestar''. ser de la humanidad ''. 39

Se puede dar otra explicación de la inflexibilidad de la doctrina marxista incluso en términos de una perspectiva marxista. Da una importancia esencial a la ``estructura'' de la sociedad, en contraposición a su ``superestructura'' (ver Capítulo 1 ): se podría argumentar que el pensamiento marxista fue ---ha sido--- incapaz de adaptarse a los cambios ocurridos a lo largo del tiempo en la estructura misma de la sociedad capitalista, es decir, en sus modos de producción. Estaban cambiando en muchos aspectos: en la tecnología en profunda evolución, en las relaciones de capitalistas, gerentes y trabajadores, en estructuras de mercado, en la importancia relativa de diferentes sectores de la economía y clases sociales. Con referencia específica a la naturaleza de la empresa capitalista, la pieza central del sistema capitalista, Schumpeter argumentó que ``el proceso de cambio industrial no fue entendido correctamente por Marx \ldots{} con él, el mecanismo {[}del cambio industrial{]} se resuelve en mera mecánica de masa de capital.40

Otro aspecto de las estructuras capitalistas no considerado por Marx fue señalado más tarde por dos economistas marxistas, Baran y Sweezy. Observaron que Marx había centrado su atención en una estructura del capitalismo que ya estaba desactualizada en el momento de escribir este artículo: ``Marx trató a los monopolios como restos del pasado feudal y mercantilista'', y vio la competencia como la forma predominante de mercado en el siglo XIX. Gran Bretaña del siglo. Pero ``ellos {[}los monopolios{]} se estaban convirtiendo en una característica permanente del sistema; y también sus sucesores {[}de Marx{]} no lograron explicar o, a veces, ni siquiera reconocer su existencia''. 41Estas empresas a gran escala, solo por su participación significativa en la producción de una industria, como empresas monopolísticas u oligopólicas, generan una estructura ascendente de precios y un control total de los volúmenes de producción e inversión, de una manera que es ``nada menos que devastadora al capitalismo como orden social racional'', 42 y en contradicción con la estructura de precios que emerge de un régimen de competencia perfecta inexistente. El capitalismo monopolista genera así una tendencia al excedente, definido como ``la diferencia entre lo que la sociedad produce y el costo de producirlo'' 43-levantar. Sin embargo, según Baran y Sweezy, la estructura del capitalismo no cuenta con un mecanismo adecuado de absorción de excedentes, carencia que es estadísticamente visible en las cifras de desempleo y subutilización de los recursos disponibles. Esto, a su vez, es causa de estancamiento, solo contrastado, hasta ahora, por innovaciones técnicas y políticas militares e imperialistas que hacen época. 44

El tema de las estructuras monopolísticas del capitalismo es una constante del pensamiento marxista, pero fue Schumpeter (del que Sweezy había sido asistente de investigación en Harvard), quien en su Capitalismo, socialismo y democracia 45 negó la relevancia de la competencia para que floreciera el capitalismo, y más bien atribuido a la ``destrucción creativa'', promulgada a través de empresas a gran escala, su fuerza continua (véase más adelante, sección 3.7 ).

Más adelante desarrollaremos este tema de los ``modos de producción'' en constante evolución, pero por el momento basta con observar que, a pesar de esta evolución, un marxista podría poner en evidencia que una contraposición esencial y continua de intereses entre el propietario y el el trabajador todavía está presente en una sociedad capitalista. El peso cambiante de los empresarios y propietarios dentro de la empresa, el desplazamiento tecnológico de la mano de obra por maquinaria, la ampliación del sector de servicios, la capacidad de los trabajadores para iniciar negocios propios, ``la conversión de trabajadores inseguros en consumidores confiados'' no serían suficientes deshacerse de una lucha esencial de clases sociales. 46

Esta importancia de la evolución de la ``estructura'' no fue apreciada por Marx. Además, si la crítica de Marx a la producción capitalista tenía sus defectos, la forma en que la estructura productiva operaría concretamente en una sociedad socialista escapaba a su atención. Un economista marxista observó que ``los fundadores del socialismo científico, Marx y Engels, dedicaron todos sus esfuerzos al análisis de la economía capitalista. Hicieron solo unas pocas observaciones muy generalizadas sobre la economía socialista. Por principio, se negaron a profundizar en el problema, por miedo a resultar más utópicos que científicos''. 47

Esta visión escatológica y determinista fue quizás una de las razones por las que el proceso de transición del capitalismo al socialismo, que fue abrupto en Rusia, no solo en términos de tiempo, fue descuidado como campo de investigación. La fase inicial, el ``comunismo de guerra'' bolchevique, fue la introducción más dura, y lejos de ser gradual, de una sociedad socialista, seguida de una reversión igualmente abrupta a la política opuesta, la Nueva Política Económica-NEP, a su vez abolida después de pocos años. . Además, los acontecimientos revolucionarios que tuvieron lugar en 1917 habían ocurrido en un país cuya estructura social y económica estaba lejos de la que Marx imaginaba como un entorno maduro para un resultado revolucionario exitoso. 48La revolución socialista se logró, en pocos meses, antes de lo esperado por Marx, y en un país aún lejos de la fase industrial del capitalismo avanzado. Según él, un requisito previo para el levantamiento socialista era una estructura industrial suficientemente desarrollada y un proletariado motivado y consciente de sí mismo: condiciones muy alejadas de las que prevalecían en la Rusia zarista. Esta circunstancia complicó especialmente los problemas organizativos de la transición al socialismo.

La cuestión principal para los economistas marxistas, en el territorio realmente inexplorado de alternativas políticas y económicas que se abría ante ellos, era si, y en qué medida, las ``leyes'' económicas de una sociedad capitalista serían válidas en un sistema socialista. Sus reflexiones estaban constreñidas por el marco intelectual establecido por Marx, pero no podían permanecer ajenas a la situación real de la economía soviética en el momento de su redacción. Recién en la segunda mitad del siglo XX surge un replanteamiento de la doctrina marxista. ``Al igual que en el siglo XVIII, un cambio radical en la organización socioeconómica está generando un cambio radical en la naturaleza y función de la economía'', escribió un economista marxista, Ronald Meek, en 1964. 49 Este replanteamiento se consideró necesario para tener en cuenta la creciente debilidad estructural de la economía soviética.

La doctrina socialista marxista permaneció desconectada de la conducción concreta de la política económica soviética. Como escribió Edward Carr: ``El cumplimiento de las promesas escatológicas del marxismo se retrasó, como la Segunda Venida, mucho más allá de las expectativas originales de los fieles''. 50En particular, la política evolucionó en líneas que tenían más que ver con la evolución de las circunstancias políticas y económicas. De hecho, la política soviética se ha comparado con la conducción de una economía de guerra de un país capitalista. Esencialmente, esta evolución de la política siguió fuerzas endógenas destinadas a construir un poder nacional durante muchas décadas a través de una economía dirigida planificada, basada en una intervención generalizada del Estado, la propiedad pública de los medios de producción, el monopolio del comercio exterior: ``El socialismo en un solo país'' , según la teoría de Stalin y Buckharin, más que un diseño coherente, de una revolución permanente y universal, como la concibió Marx. 51

Esta dicotomía entre una ideología intransigente y la evolución concreta de las políticas soviéticas a lo largo de los años está bien ilustrada por un observador comprensivo, Rudolf Schlesinger, quien escribe en 1947: ``Puede ser irrazonable esperar que los líderes de la URSS declaren abiertamente que ha habido Ha sido un cambio bastante natural, no solo de la política, sino incluso de las ideas dominantes desde los días de 1917. De modo que el público en el extranjero se enfrenta a dos afirmaciones contradictorias: la de los críticos (principalmente de Trotskyte) de que el régimen soviético ha abandonado sus objetivos originales y concepciones {[}y{]} ha degenerado, y la del propio régimen que no sólo sostiene que ha sido fiel a esas concepciones originales, sino que para probar su caso incluso intenta interpretar su pasado a la luz del presente''. 52

En la misma línea, dentro de este marco de economía dirigida, ``ha habido una serie de economías soviéticas a lo largo de los años, cada una significativamente diferente de las demás''. La organización de la economía cambió en consecuencia, mientras que ``la ideología comunista {[}fue{]} interpretada y reinterpretada para justificar estos cambios organizativos''. 53

En resumen, el gobierno soviético trajo producción en masa y ejércitos en masa, aumentó la educación, creó mejores oportunidades para las mujeres, pero también, en una especie de cuenta de pérdidas y ganancias, provocó fallas en los campos de una mejor productividad y un mayor bienestar y, lo que es más importante, pérdidas profundas en términos de hambrunas, asesinatos en masa, uso de trabajo forzoso.

En cuanto al crecimiento real del PIB, si comparamos, con todas las incertidumbres de este tipo de comparaciones, los niveles del PIB en 1913, 54poco antes de la guerra y la Revolución, y en 1991, al final de la Unión Soviética, el PIB se situó en 1991 en poco más de 8 veces el nivel de 1913. En el mismo período, la economía de Estados Unidos creció más de 11 veces. El PIB estadounidense en 1913 era 2,2 veces mayor que el ruso; en 1991, era 3,1 veces mayor. Desde este punto de vista, el intento soviético de alcanzar y vencer al país archienemigo estuvo lejos de ser exitoso. Pero si limitamos nuestra comparación al período 1913-1939 (año del estallido de la Segunda Guerra Mundial), la conclusión es diferente, porque la divergencia en tamaño de las dos economías se redujo, con respecto a 1913, aunque ligeramente, a 2 solo veces. Este estrechamiento de la divergencia puede explicarse, al menos en parte, por el colapso de la economía estadounidense durante la década de 1930: -29\% entre pico y valle,55

Las principales fases de la evolución de las políticas económicas soviéticas en los 74 años de su poder, desde la revolución de 1917 hasta su colapso en 1991, se pueden resumir de la siguiente manera. 56

\hypertarget{los-bolcheviques-en-el-poder}{%
\subsection*{Los bolcheviques en el poder}\label{los-bolcheviques-en-el-poder}}
\addcontentsline{toc}{subsection}{Los bolcheviques en el poder}

Después de sus promesas hechas durante la revolución, Lenin, en 1917, arrasó con las propiedades de los terratenientes, dando legalidad a la apropiación espontánea de tierras campesinas y sancionando así la expansión de la propiedad privada, en un movimiento no marxista. Pero, en presencia de una grave hambruna y de una fuerte reacción antisoviética (los Ejércitos Blancos y la guerra civil que siguió), cualquier intento de colaborar con lo que quedaba de la clase capitalista llegó a su fin, y el sistema de ``comunismo de guerra''Se introdujo poniendo a todos y todo al servicio de la lucha del Estado por la supervivencia militar y económica: se nacionalizaron todas las fábricas y el comercio se convirtió en monopolio estatal; el funcionamiento del mercado basado en el uso del dinero se vio perturbado en gran medida. Un sistema de trueque, salarios pagados en especie, requisa forzosa caracterizó la fase del Comunismo de Guerra.

\hypertarget{la-nep}{%
\subsection*{La NEP}\label{la-nep}}
\addcontentsline{toc}{subsection}{La NEP}

En 1921, el comunismo de guerra fue descartado por Lenin quien, en un cambio total por supuesto, introdujo la Nueva Política Económica-NEP, que aprobó y alentó el interés propio como un incentivo para la actividad económica, buscando restaurar formas de libre mercado. ``Sin capitalistas, pero con la legislación laboral más progresista del mundo, las fábricas estatales funcionaban mejor que en manos privadas''. Esta política dio resultados notables, aunque sólo en 1928 el PIB de la URSS alcanzó su nivel de preguerra. 57

\hypertarget{la-era-de-stalin-y-luego-jruschov-una-gran-acumulaciuxf3n-de-capital}{%
\subsection*{La era de Stalin, y luego Jruschov: una gran acumulación de capital}\label{la-era-de-stalin-y-luego-jruschov-una-gran-acumulaciuxf3n-de-capital}}
\addcontentsline{toc}{subsection}{La era de Stalin, y luego Jruschov: una gran acumulación de capital}

Tras la muerte de Lenin en 1924, Trotsky y el ala izquierda del partido se opusieron enérgicamente a la NEP, instando a que se volviera a la planificación económica plena y a la rápida industrialización inducida por el Estado. La elección fue por una alta tasa de acumulación y, en consecuencia, por una fuerte contención del consumo. Los campesinos, la gran mayoría del país, soportarían el peso de esta industrialización mediante una compresión de sus salarios reales. Las políticas de derecha, que ponían el acento en el sector agrícola, fueron derrotadas, y bajo Stalin, que había triunfado contra Trotsky en la lucha por la sucesión de Lenin, el recién mencionado proceso de fuerte industrialización se reanudó con venganza, también en vista de lo que se consideraba un conflicto internacional inminente e inevitable. La inversión en bienes instrumentales para la industria fue una prioridad,

Stalin definió la planificación como ``no previsiones, sino instrucciones''. El órgano central de planificación fue la Gosplan (Comisión Estatal de Planificación), a cargo de la administración del sistema de precios y la definición de metas de producción física. Los gerentes de las empresas estatales eran responsables ante el Estado, no ante los clientes, siendo incentivados por los volúmenes de producción, no por los costos. El potencial de producción se definió por el stock de capital fijo y la disponibilidad de capital circulante y mano de obra, no por la minimización de costos. Los precios fueron fijados por la autoridad de planificación en relación con los salarios, de tal manera que se lograra la plena utilización de las plantas (por lo tanto, el problema keynesiano de la ``demanda efectiva'' se consideró inexistente). Los objetivos de volumen y la fijación de precios por parte de los planificadores llevaron a la exageración del desempeño de las empresas o la escasez y las colas de clientes.

La propiedad de la tierra se colectivizó ampliamente desde la década de 1930 (cooperativas como koljós, granjas estatales como sovjós). También con un impulso de la tecnología occidental, la producción industrial aumentó sustancialmente durante la década de 1930. Como se mencionó anteriormente, la Unión Soviética escapó de lo peor de la Gran Depresión que asoló a la mayoría de los países occidentales. 58 Al estallar la guerra, en 1939, el PIB soviético se situó en un nivel alrededor de un 85,5\% más alto que en 1928, un logro notable.

Después de las penurias de la Segunda Guerra Mundial, se reanudó la rápida expansión de la economía, favorecida por muchos años de tranquilidad interna y paz internacional. La Unión Soviética, como otros países de libre mercado, se recuperó de la guerra con relativa rapidez, a alturas de desarrollo económico nunca antes vistas en la historia de ese país.

El gobierno se basó, como antes de la guerra, en el crecimiento de la producción industrial pesada más rápido que la producción de bienes de consumo; se siguió haciendo hincapié en la defensa y las inversiones de capital.

Si bien, hasta la muerte de Stalin en 1953, el aumento de la producción se debió principalmente a un aumento en la cantidad de recursos dedicados a la producción, de 1953 a 1961 hubo un aumento notable en la productividad. ``Un tema importante en la planificación soviética y la exhortación pública {[}se convirtió en{]} la necesidad de aumentar la productividad de la mano de obra y otros recursos'', 59 mediante la mejora de la tecnología y el uso generalizado de planes de pago de incentivos para los trabajadores.

Desde 1955 hasta su desaparición en 1964, Jruschov dio un nuevo impulso a la economía, tratando de alcanzar un mejor equilibrio entre la producción de bienes industriales y de consumo. Sin embargo, dado el énfasis en la inversión y los bienes de inversión, el porcentaje de bienes de consumo continuó disminuyendo (aún en la década de 1980, la capacidad militar de la Unión Soviética, construida sobre un fuerte complejo industrial orientado a las armas, seguía siendo desproporcionada con respecto a su tamaño económico). . Jruschov también incentivó al sector agrícola, afirmando que la lucha contra el capitalismo occidental no debía ganarse mediante la guerra, sino mediante una carrera de productividad que daría a los rusos un nivel de vida más alto, mientras que gran parte del sistema estalinista de coerción de los trabajadores soviéticos. fue desmantelado.

\hypertarget{acumulaciuxf3n-intensiva-de-capital.-desacelerar}{%
\subsection*{Acumulación intensiva de capital. Desacelerar}\label{acumulaciuxf3n-intensiva-de-capital.-desacelerar}}
\addcontentsline{toc}{subsection}{Acumulación intensiva de capital. Desacelerar}

Desde finales de la década de 1960, la economía sufrió un agotamiento de las reservas de trabajadores agrícolas y de recursos naturales. La política económica se desplazó hacia una acumulación intensiva de capital, encaminada a minimizar los costos y aumentar la eficiencia. Este esfuerzo fue sustancialmente infructuoso, debido a los retrasos tecnológicos y la militarización de la economía que llevaron a una especie de confiscación de la innovación por parte de la industria militar. Al mismo tiempo, se produjo un deterioro de la disciplina obrera, alentado por un proceso de desestalinización, un sistema político más abierto y la presión salarial proveniente de una situación de pleno empleo. Abandonando los esquemas anteriores de una economía controlada aislada, el comercio exterior ---durante un tiempo un sector relativamente menor de la economía--- aumentó, con las exportaciones impulsadas por el petróleo y el gas, y las armas.

\hypertarget{perestroika}{%
\subsection*{Perestroika}\label{perestroika}}
\addcontentsline{toc}{subsection}{Perestroika}

Gorbachov, secretario general del partido desde 1985, tenía dos objetivos: reactivar la economía y elevar el nivel de vida de la población. Comenzó por suavizar la planificación central, por la descentralización de las decisiones y la participación de los trabajadores en la gestión empresarial. Algunas leyes deben citarse como particularmente relevantes: la ley sobre la actividad laboral individual de 1986, que inició un sector privado efectivo de la economía; la ley de empresa estatal de 1987, que otorgó autonomía a las empresas estatales, haciendo los planes centrales indicativos y no obligatorios. El sistema de precios dejó de estar controlado por el Gosplan, moviendo la economía hacia un mecanismo de mercado; al mismo tiempo, se buscó una integración más profunda con el comercio mundial. La disminución de la disciplina en el sector laboral se combatió introduciendo la autogestión de los trabajadores, pero eso no fue suficiente: las huelgas continuaron y la inflación creció. La ley de Cooperativas de 1988 favoreció el desarrollo de una clase directiva que realmente se comportó según los mecanismos de una economía capitalista, y la extensión de su actividad de la economía real a la intermediación financiera marcó un paso más hacia una economía de mercado. Con una ley de 1988, también se descentralizó el comercio exterior.

Una clase de administradores prósperos de empresas sustancialmente privatizadas (los ``oligarcas'') ``desempeñaron un papel central en el colapso de la URSS al financiar la coalición procapitalista y profundizar los desequilibrios económicos''. Los enormes desequilibrios en la estructura salarial y en las cuentas exteriores, que agotaron las reservas de oro y divisas de la Unión Soviética, fueron el golpe final para la Unión, que llegó a su fin el 25 de diciembre de 1991. Después del colapso, esa clase de gerentes habría aprovechó en gran medida las terapias de choque de la Rusia postsoviética en la década de 1990.

\hypertarget{ajustar-e-interpretar-a-marx}{%
\section*{Ajustar e interpretar a Marx}\label{ajustar-e-interpretar-a-marx}}
\addcontentsline{toc}{section}{Ajustar e interpretar a Marx}

El problema del ``cálculo económico'', es decir, asegurar el equilibrio entre la disponibilidad y el uso de cualquier bien a través de un mecanismo de mercado, no fue examinado adecuadamente por los planificadores soviéticos. Esta atención inadecuada era coherente con un sistema económico que, en perspectiva, haría del mecanismo de precios un instrumento obsoleto. Sin embargo, cabe señalar que los planificadores eran conscientes de este problema. El propio Stalin hizo dos comentarios relevantes en un pequeño libro, en 1952 60 :

\begin{quote}
\begin{itemize}
\item
  la nacionalización total no sería posible en la agricultura, donde ``las granjas colectivas deberían {[}más bien{]} colocarse sobre la base técnica moderna de la producción a gran escala, no expropiarlas, sino al contrario suministrarlas con tractores de primera y otras máquinas''. Esto implicaría la coexistencia de industrias estatales en las ciudades y granjas cooperativas y colectivas en el campo. Entre estos dos sectores, ``el intercambio por compra y venta {[}es decir, un mecanismo de precio de mercado{]} debe preservarse durante un período determinado, siendo la forma de vínculo económico con la ciudad la única aceptable para los campesinos. Y el comercio soviético, las granjas estatales, cooperativas y colectivas, debe desarrollarse al máximo, y los capitalistas de todo tipo y descripción {[}es decir, propietarios privados de los medios de producción{]} deben ser desterrados de la actividad comercial''.61
\item
  ``Por supuesto, cuando en lugar de dos sectores productivos básicos, el sector estatal y el sector agrícola colectivo, haya un solo sector productivo que lo abarque todo, el derecho a disponer de todos los bienes de consumo producidos en el país, la circulación de mercancías, con su `economía monetaria', desaparecerá, como elemento innecesario de la economía nacional''. 62
\end{itemize}
\end{quote}

En dos sectores el mecanismo de precios de mercado seguiría funcionando, al menos por un tiempo y aunque dentro de una economía planificada: el mercado de bienes de consumo (donde los consumidores tenían libertad para elegir bienes en los que gastar sus ingresos), y las relaciones de cambio entre ciudades y campo (donde operaba el intercambio entre industrias estatales y granjas cooperativas). Habría sido, según Stalin, un error si la apresurada abolición de los precios de mercado contribuyese a agravar los problemas de sobreproducción y escasez antes mencionados.

El rígido marco de la doctrina marxista permaneció indiscutido hasta que aparecieron algunas grietas graves en la organización y conducción de la economía soviética.

Puede resultar interesante comparar tres enfoques sucesivos del marxismo de Maurice Dobb, Oskar Lange y Ronald Meek. El enfoque de Dobb que consideramos aquí está relacionado con la situación de la década de 1930, cuando el avance de la economía soviética continuó, relativamente sin ser perturbado por la larga Gran Depresión del mundo capitalista. El proceso de acumulación de capital, favorecido por una gran disponibilidad de recursos y mano de obra, es el trasfondo de la teorización de Dobb. Quiere destacar algunos puntos: el hiato entre los sistemas capitalista y socialista, que es imposible de cumplir con un ajuste del primero, y menos en sentido contrario; la contraposición de la coordinación ex ante de la actividad económica bajo el socialismo, y la única coordinación ex post inadecuada del sistema de mercado; y el hecho de que la cuestión de la distribución de la riqueza se elimina de raíz en un sistema socialista, al resolver el antagonismo de las clases sociales en competencia. El segundo, de Lange, de finales de la década de 1950, puede leerse frente a las próximas dificultades económicas que surgieron bajo el liderazgo de Jruschov. Lange se ocupa de los continuos conflictos sociales a pesar de la abolición de las clases sociales, la relevancia, incluso en una sociedad socialista, del concepto de ``valor'' económico, a evaluar en términos monetarios, y la necesidad de descentralizar las decisiones económicas. El tercero, de Meek, de los años sesenta, es una especie de invitación a repensar en términos similares los problemas económicos que enfrentan los países socialistas y liberales ``occidentales'': al mismo tiempo, una negación implícita del hiato mencionado por Dobb en el 1930,

Escribe Dobb, en 1937: ``El crecimiento de la economía soviética en los últimos años, además de su capacidad para mantener una tasa de expansión constante `boom' durante una década, los esfuerzos de construcción a gran escala que ha logrado y su sustitución de un estado de escasez de excedentes en el mercado laboral, no sólo han despertado el interés, el estudio y la polémica, sino que han proporcionado una base de comparación concreta que antes faltaba''. 63

La coordinación directa de las partes constitutivas del sistema nunca puede lograrse en una sociedad capitalista, ``debido a los derechos de propiedad atomísticos sobre los que {[}este{]} sistema descansa''. 64En el campo de las inversiones, en un sistema capitalista el acto de invertir está guiado por las expectativas de ganancia, y estas se ven afectadas, además de la demanda esperada del producto y la innovación técnica futura, por factores que el emprendedor es mayoritariamente ignorante: rival actos de inversión, actos de inversión complementarios a los propios, monto de ahorros e inversiones en todo el sistema, acumulación futura de capital. Estos dos últimos son de gran importancia y los menos comprendidos, pero en un sistema socialista se convierten en un problema de reparto del trabajo entre varios tipos de producción: la decisión relativa la toma una sola autoridad, para evitar la inconsistencia provocada por la independencia de decisiones separadas. . Presumiblemente,

En una economía capitalista, las ``leyes'' tienen la forma de afirmar que, dadas ciertas condiciones de naturaleza y técnica y ciertas preferencias de los consumidores, los productores se comportarán siguiendo ciertas relaciones de valor. En una economía socialista, su guía será comportarse de acuerdo con un propósito determinado. Esto debe verse como un postulado, aunque no arbitrariamente determinado sino condicionado por el nuevo tipo de organización social y seleccionado en función de la situación concreta. Dado ese postulado (propósito), la ley económica ya no será la ley ricardiana / marxista del valor del trabajo.

En una sociedad de clases, la acumulación de capital está sujeta a un límite que la retarda, el límite es la resistencia a acercarse a una condición de pleno empleo en el mercado de trabajo, porque el aumento salarial encoge la plusvalía del capitalista. Una sociedad socialista supera el problema de la distribución del excedente: ya no es necesaria ninguna investigación ética sobre esta distribución, gracias a la abolición del beneficio capitalista. En una sociedad socialista, la ganancia deja de ser una categoría de ingresos (el único ingreso es el ingreso salarial) y un incentivo económico, y el único incentivo presumiblemente será aumentar los salarios en mayor medida: el límite lo establecen únicamente los poderes y consideraciones productivas existentes. de futuros equipos productivos. ``Esto es ---concluye Dobb--- marchar en la mejor tradición de Economía Política''. sesenta y cinco

Esta visión intransigente da paso a una perspectiva más matizada en las reflexiones del economista polaco Oskar Lange, a finales de la década de 1950. Después de la muerte de Stalin y el acceso al poder de Jruschov, escribiendo en 1959 muestra la dificultad que enfrenta la economía soviética para ir más allá de las rigideces anteriores de una economía dirigida y adaptarse a las necesidades emergentes de una sociedad en evolución. .

Los conceptos más relevantes que surgen de la reflexión de Lange parecen ser los siguientes: la persistencia de conflictos sociales en una economía socialista entre diferentes ``estratos'' sociales más que entre ``clases''; la perdurable relevancia de la ``ley'' ricardiana / marxiana del valor del trabajo y de los valores expresados \hspace{0pt}\hspace{0pt}en términos monetarios; la necesidad de fortalecer el autogobierno y los incentivos a la producción en las empresas, para lograr eficiencia y evitar la burocracia.

Por supuesto, esto puede ser cuestionable. La sustitución del término ``estrato'' por ``clase'' significa que no hay dueños de la estructura productiva en una sociedad socialista. Pero la falta de incentivos a las ganancias significó agregar más burocracia que eficiencia (como demuestra ampliamente la evolución de la economía soviética antes mencionada).

El discurso de Lange se realiza en el marco de la doctrina marxista y su lenguaje típico, y teniendo en cuenta las observaciones de Stalin mencionadas anteriormente en esta sección. 66 Observa que es incompatible con la teoría marxista creer que el advenimiento del socialismo ha resuelto todos los problemas de la nueva organización social y económica, como si todas las contradicciones sociales en la vida humana desaparecieran automáticamente en una sociedad socialista. El advenimiento del socialismo no es la ``realización del Reino de Dios''. 67 Lange añade que cualquier generalización teórica debe tener en cuenta no solo la experiencia acumulada madurada en décadas anteriores en Rusia, sino también las vicisitudes más recientes de otros países socialistas europeos y China.

Siguiendo la doctrina del materialismo histórico, Lange analiza las contradicciones marxistas (contradicciones entendidas como una incompatibilidad creciente en el tiempo entre magnitudes económicas e instituciones) que son la fuerza motriz del desarrollo social. Son dos:

\begin{quote}
\begin{itemize}
\item
  el primero surge entre el desarrollo de las fuerzas productivas ---trabajo empleado en un mecanismo de producción, combinado con recursos según el estado del conocimiento técnico--- y la lucha de clases que caracteriza las relaciones de producción entre obrero y capitalista;
\item
  el segundo es la contradicción entre los modos de producción relacionados con cada tipo de organización social (esclavitud, feudalismo, capitalismo, socialismo, comunismo) y la superestructura de la economía, como sistema jurídico y político levantado sobre un modo de producción específico, con su correspondientes formas de vida social, política e intelectual. 68
\end{itemize}
\end{quote}

La primera contradicción en una sociedad capitalista aparece en las relaciones de producción, tomando la forma de una lucha de clases de intereses opuestos de trabajadores y capitalistas; en una sociedad socialista, en ausencia de capitalistas, desaparece como tal. Pero el problema reaparece con la segunda contradicción, en cuanto al ajuste de la superestructura a los nuevos modos y relaciones de producción. En este proceso de ajuste, los intereses de los diferentes grupos sociales existentes en la superestructura - que, como se mencionó anteriormente, Lange llama ``estratos sociales'' para diversificarlos de las ``clases sociales'' - pueden chocar, por ejemplo con referencia a métodos de economía. gestión u organización política. Derrotando los intereses creados en la superestructura, aún persistentes a pesar de la abolición de las relaciones capitalistas de producción,

En cuanto a las leyes económicas, Lange ---confirmando, nuevamente, la visión de Stalin--- desconoce la opinión de muchos marxistas (Rosa Luxemburg, Nikolaj Bukharin entre ellos) que piensan que la economía política, como ciencia del capitalismo, no necesita sobrevivir en una sociedad socialista. Al tomar distancia de Dobb, Lange escribe que las leyes económicas todavía existen, con la única diferencia de que deben expresar el funcionamiento de una sociedad diferente y libre, la socialista. La ley económica que organiza el sistema económico depende de las relaciones de producción existentes. Bajo el capitalismo, esas relaciones se basan en la contradicción entre clases sociales, y la propiedad privada de los medios de producción existe para el beneficio del propietario; el modo de producción socialista vence que la lucha y la propiedad existen para la satisfacción de las necesidades humanas.

Además, en una sociedad socialista las leyes del valor y la circulación monetaria continúan operando para determinar el precio de las mercancías que se intercambian entre diferentes propietarios. Hay diferentes formas de propiedad socialista, no solo una, que es la propiedad nacional. Una sola propiedad nacional (un sector estatal que incluyera a todos) podría haber sido posible si la economía rusa hubiera sido una economía capitalista bien desarrollada; pero la necesidad de mantener un sector no estatal de la economía surge del hecho de que la afirmación del socialismo ocurrió en una sociedad capitalista no completamente desarrollada, donde aún sobrevivían formas de producción no capitalistas (como la producción de pequeñas mercancías a través de cooperativas). Estas condiciones históricas particulares hacen oportuno no pasar directamente a la propiedad nacional. Además,

Pero, ¿cómo se intercambian las mercancías dentro de los sectores nacionalizados, es decir, sin un cambio de propiedad? Las transferencias dentro de los sectores nacionalizados se tasan por imputación. Es ``un proceso contable reflejado hacia atrás a los medios de producción que se utilizan para producirlos'' 69 {[}los bienes que realmente se intercambian{]}. Mientras funcione esta ley, una junta de planificación socialista podría usar el precio resultante para ingresar los precios apropiados para los bienes de capital producidos e intercambiados en los sectores nacionalizados. Usando esta ``imputación'', los planificadores socialistas podrían construir un sistema de ecuaciones de costos e ingresos, y resolver el sistema para la cantidad eficiente de cualquier bien intercambiado en los sectores nacionalizados, para ser producido de tal manera que se minimicen los costos.

Luego, están las leyes que afectan la superestructura de la gestión de una economía socialista (ver más arriba la referencia de Lange a los ``estratos''). Bajo el socialismo, el único modo de interacción social es la planificación, dirigida por el propósito del bienestar social. La empresa socialista debe actuar como fideicomisaria del interés social general, y debe ser un organismo autónomo. Por tanto, la planificación requiere mantener incentivos y oportunidades para racionalizar el uso de los medios de producción. La asignación de bienes a través de la planificación debe conciliarse con el respeto de la ley del valor, para evitar que la interacción de la asignación administrativa y la resistencia burocrática de algunos estratos impida a los trabajadores influir en el uso de los medios de producción; Se necesita una participación democrática efectiva, evitando sin embargo que, por ejemplo, los objetivos del plan se fijan demasiado bajos, con el fin de obtener primas de producción más fácilmente; o que se cumplan solo levemente para que no se recauden demasiado el año siguiente.

Escribiendo en 1964, Robert Meek reflexiona sobre la historia de la economía y observa, en una especie de tono positivista, que toda la historia económica está hecha de un intento de liberar la economía de los juicios de valor, es decir, pasar de la economía normativa a la economía positiva. 70 : a partir de Adam Smith, los economistas querían asumir una actitud pasiva hacia la distinción entre lo que es la vida económica y lo que debería ser. El sistema económico apareció como una máquina gigantesca, cuyas interrelaciones objetivas podrían describirse como ``leyes''. La nueva ciencia quería cortar los valores morales, los juicios de valor. Las opiniones morales y políticas fueron vistas como excrecencias en esa máquina.

Sin embargo, agrega Meek, los juicios de valor regresaron, de hecho, al análisis económico: no por mera fragilidad humana, sino por razones más profundas: la máquina estudiada por los economistas es muy compleja y no puede haber una sola explicación válida para cualquier parte de ella. Por tanto, se necesitan explicaciones alternativas, y consideraciones ideológicas, morales y políticas son la motivación de la elección de una de ellas. Sin ellos, los resultados de nuestro análisis quedarían de una validez limitada: ``un conocimiento incómodo''.

Pero, en los últimos 10/20 años, continúa Meek, están surgiendo dos tendencias para eliminar los prejuicios ideológicos:

\begin{quote}
\begin{itemize}
\item
  La evolución de ciertas técnicas matemáticas, diseñadas para ayudar en la solución, por parte de administradores públicos y privados, de problemas difíciles de elección económica;
\item
  La afirmación de la economía del bienestar, como un conjunto de reglas por las cuales diversas situaciones económicas, disponibles para una sociedad, pueden ordenarse y compararse en cuanto a su conveniencia.
\end{itemize}
\end{quote}

Ambas situaciones, que son aún más frecuentes, implican elecciones económicas en las que no se puede esperar que funcione el mecanismo de precios, como foco de la investigación económica.

Como consecuencia de estas tendencias, ``el colapso y destrucción de esa compleja máquina es evidente'', señala Meek, tanto en los países comunistas como en Occidente, donde las técnicas de planificación y las grandes y complejas empresas públicas tienden a prevalecer, creando una convergencia de políticas económicas. El resultado es que la economía se está transformando, independientemente del tipo de sistema económico, en una ciencia de la gestión económica, de la ingeniería social, de la eficiencia de la ingeniería. La economía llegó a ser considerada y enseñada como una asignatura de resolución de problemas, un poco como la ingeniería. Surgieron tendencias similares en la academia occidental. Por ejemplo, George Shackle argumentó que, a partir del análisis de insumo-producto de Wassily Leontief a principios de la década de 1930, un esquema de ``planificación general indicativa o`` la Matriz de Contabilidad Social''podría construirse como`` una compleja red productiva de industrias que se abastecen y recurren unas a otras, cuando la `lista de bienes' final \ldots{} podría calcularse \ldots{} de un solo golpe (aunque ese `golpe' consistió en la solución de un gran sistema de ecuaciones ''). En estos esquemas de coherencia, que abarcan toda la economía, la economía tiene la mayor esperanza de justificarse ante una herramienta de la mente humana capaz de igualar, aunque no de imitar, los logros de las ciencias naturales ''.71 La historia económica era irrelevante, ya que por definición estaba desactualizada.

¿Cuál es el papel de los juicios de valor en esta situación? Según Meek y Shackle, parece muy pequeño, simplemente debido a la ``eficiencia de los ingenieros'' que prevalece. Pero, nuevamente, los criterios de eficiencia, comunes a las economías socialista y capitalista, pueden ser valorados de manera diferente en diversos sistemas económicos, ¿y cuáles deberían adoptarse? El papel de los prejuicios ideológicos permanece, concluye Meek.

Sin considerar lo inconcluso de la solución, las observaciones de Meek son relevantes porque, por un lado, apuntan a una transformación de la ciencia económica en una especie de gestión económica (aparentemente) no ideológica, operando en gran parte a través de técnicas matemáticas. Se trata de un anticipo de las tendencias de la disciplina económica que surgirían en las últimas décadas del siglo; por otro lado, están revelando las dificultades que los esquemas puramente marxistas estaban encontrando en la interpretación de las estructuras socioeconómicas en evolución, incluso dentro del sistema soviético.

La introducción de las computadoras para la planificación económica refuerza un enfoque no ideológico de los problemas económicos, donde se difuminan las fronteras entre las economías de libre mercado y las socialistas. El mercado ya no se ve como un dispositivo transitorio hacia el socialismo pleno, 72mientras que la planificación económica se pone de moda incluso en las economías de mercado, generalmente entre los años cuarenta y sesenta de la posguerra. Lange enfatiza la importancia de las técnicas matemáticas impulsadas por computadora, que pueden cerrar la brecha ideológica entre los economistas liberales y marxistas. ``La programación matemática asistida por computadoras electrónicas se convierte en el instrumento fundamental de la planificación económica a largo plazo, así como de la resolución de problemas económicos dinámicos de alcance más limitado. Aquí, la computadora electrónica no reemplaza al mercado. Cumple una función que el mercado nunca pudo realizar''. 73

La desaceleración de la economía soviética hacia fines de la década de 1960 y la aceptación de formas de economía de mercado y, por otro lado, una creciente actitud crítica hacia el capitalismo en las economías liberales occidentales, llevan a los economistas de ambos lados a comparar sus respectivas experiencias y a intentar un intento ---que al final resultaría estéril--- de avanzar hacia una planificación extensiva en las economías capitalistas y una mayor autonomía en las decisiones empresariales en las socialistas.

Este acercamiento está bien expuesto por Joan Robinson. ``La historia ha visto dos métodos de llevar a cabo la acumulación necesaria para instalar tecnología científica. El primero, que ha estado en funcionamiento durante casi dos siglos, se basa en la codicia individual; el segundo, que funciona desde hace menos de medio siglo, se basa en la planificación socialista''. 74Los frutos de la acumulación están ahora disponibles ---escribe--- pero en cada sistema las instituciones y los hábitos mentales están poniendo obstáculos en el camino hacia su disfrute racional. En los países capitalistas, el igualitarismo, que se ha establecido gracias al proceso democrático, es derrotado por los arreglos legales que favorecen la propiedad y por la aceptación de la estructura de clases, necesaria para fomentar la acumulación. Ahora, la propiedad privada se ha vuelto ``ociosa'': tanto los accionistas como los rentistas se dedican al lucrativo negocio de canjear valores entre ellos sin dar una contribución efectiva al proceso productivo. El desarrollo económico no está limitado por la falta de ahorro privado (aquí hay un eco de la economía keynesiana): la industria extrae de sí misma los recursos necesarios a través de fondos de amortización y ganancias retenidas. Pero no se puede confiar en las grandes corporaciones independientes para asegurar el pleno empleo continuo y un patrón constante de desarrollo. La independencia de la industria privada impide que la economía cree órganos de control, en interés general. Es necesario un programa decidido democráticamente, una ``planificación nacional'', para superar el sesgo sistemático en el patrón de producción de bienes y servicios que pueden venderse por partes, a fin de proporcionar un margen de beneficio y dirigir la producción hacia bienes de uso colectivo. consumo, que debe financiarse mediante impuestos.75

Con referencia a los países socialistas, el problema ---escribe Robinson--- es el contrario, incluso con el mismo objetivo de disfrutar de los frutos de la acumulación: mover la producción del sector de la industria pesada y cuidar los intereses de los consumidores. Esto es impedido por un sistema de mando desde arriba que priva al gerente individual de autoridad e iniciativa, haciendo que la planificación sea rígida y torpe, y obstaculiza el progreso futuro. Los cambios para estimular la industria ligera y la agricultura deben realizarse de manera centralizada, pero en detalle debe darse más espacio a las empresas individuales, superando ``el horror exagerado del riesgo''. Ajustar el sistema de precios desde la contabilidad de costos y los objetivos de producción en términos físicos hasta la demanda del mercado evitaría la acumulación de bienes no vendibles en los sótanos de las tiendas. 76

De hecho, gran parte de los debates que siguieron en la década de 1970, no solo dentro de la academia soviética, sino también a través de sus discusiones con economistas de países no socialistas, refleja por un lado la intención de los economistas socialistas de hacer su sistema más eficiente en un micronivel, aunque en el lecho procusteano de la doctrina marxista, por otro lado el malestar de los economistas ``occidentales'', ante las dificultades del estancamiento del producto y la inflación, y abandonando el ``consenso keynesiano'', tras el largo período de crecimiento y la estabilidad que siguió a la Segunda Guerra Mundial. Tomó cierto tiempo darse cuenta de que en realidad había terminado.

Las observaciones finales de una conferencia internacional dedicada a estos temas, por el profesor Michael Kaser de Oxford, enfatizaron que ``La conclusión más clara a la que ha llegado esta conferencia es que el mecanismo del mercado no ha resuelto los problemas sociales y las externalidades, aunque admitió una divergencia bastante amplia de opinión sobre si pueden corregirse mediante la adaptación de precios o mediante la planificación. En el otro extremo, todos insistieron ---concluye Kaser--- en el valor y la aplicación real de la microplanificación''. 77

\hypertarget{cruxedtica-de-dobb-a-la-economuxeda-de-libre-mercado.-la-reconciliaciuxf3n-de-sraffa-entre-la-economuxeda-cluxe1sica-y-el-marxismo}{%
\section*{Crítica de Dobb a la economía de libre mercado. La reconciliación de Sraffa entre la economía clásica y el marxismo}\label{cruxedtica-de-dobb-a-la-economuxeda-de-libre-mercado.-la-reconciliaciuxf3n-de-sraffa-entre-la-economuxeda-cluxe1sica-y-el-marxismo}}
\addcontentsline{toc}{section}{Crítica de Dobb a la economía de libre mercado. La reconciliación de Sraffa entre la economía clásica y el marxismo}

En términos de la filosofía económica, la hostilidad de los economistas marxistas se dirige más a los teóricos de la utilidad marginal de la economía neoclásica que a la escuela clásica. Según los marxistas, la teoría del valor del trabajo de la escuela clásica no parece tan anticuada como pretendían los economistas de la utilidad marginal. Los economistas marxistas piensan que la teoría del valor trabajo, incapaz de explicar el funcionamiento de un sistema económico socialista, sigue siendo un instrumento útil para comprender la sociedad capitalista y reconocen esa teoría como el punto de partida del análisis marxista.

Maurice Dobb analiza la economía política del capitalismo. Sabemos ---observa--- que una teoría científica debe basarse en una abstracción específica, que debe ser adecuada al interés del investigador. Pero una abstracción general debe contener a su vez una dosis suficiente de realismo, con el riesgo de que lo que la abstracción gana en amplitud, lo pierde en profundidad, de modo que los corolarios deducibles de la abstracción serán de significado limitado: aunque se presenten como `` leyes ''del mundo real, estos corolarios están vacíos de contenido real. Es mejor mantener un pie en el suelo que perderse en la ``precisión de la formulación algebraica''. 78

Esa dosis de realismo significa tener en cuenta ``las relaciones productivas y las instituciones de propiedad y de clase de las que son expresión''; Los economistas neoclásicos no hicieron eso y llegaron a generalizaciones - ``leyes'' - consideradas válidas para cualquier tipo de economía de cambio. En una sociedad de clases, las ideas que se derivaban de esa sociedad tendían a asumir un ``carácter fetichista'' (en palabras de Marx): pueden haber jugado un papel positivo de ilustración como armas de crítica contra ideas e instituciones de una época anterior, pero más tarde se han vuelto reaccionarios y oscurantistas, y la representación de la realidad resultó velada. 79

La realidad ---agrega Dobb--- es que las llamadas ``leyes'' de la economía no deben basarse en el aspecto subjetivo de las interrelaciones económicas, como los deseos y elecciones individuales, como hacen los economistas neoclásicos de utilidad marginal, sino más bien. relaciones fundamentales y modos de producción. Las relaciones de clase fueron completamente olvidadas por la doctrina económica cuando surgió el nuevo capitalismo industrial en el transcurso del siglo XIX, dando una nueva conciencia al proletariado industrial. La economía política de Marx es el análisis de esta realidad social y económica, y el pensamiento de Marx traza una línea divisoria entre la Escuela Clásica de Ricardo y los teóricos de la utilidad marginal. En ese momento, la economía se convirtió, como escribe Marx, en una disciplina ``vulgar''. 80

La utilidad, la escuela neoclásica se retira en puro formalismo, impotente para emitir juicios sustanciales sobre problemas que son propios de un cierto tipo de sociedad. La economía se convierte en una especie de ``álgebra de la elección humana'', una ``cáscara vacía''. La visión de la sociedad de Dobb no es en términos de individuos sino de clases sociales, y esto es válido también en la esfera política y económica.

En el ámbito político, el Estado, según la teoría tradicional de la política, es la expresión de la ``voluntad general'' 81.resultante de la voluntad autónoma de individuos libres e iguales. Asimismo, en el ámbito económico, ``la mayoría de los escritos económicos se refieren a la regla del consumidor {[}la libre elección del consumidor{]} porque existe como mercado''. Pero esta es una ``imagen idílica'' (que esconde la estructura de clases de la sociedad), como la pinta la Prensa capitalista en el campo de la política y la industria publicitaria en el campo de la economía. En el primero, el carácter atomista del cuerpo social (el individuo guiado por la utilidad de los economistas neoclásicos) es la antesala de las dictaduras: escribiendo en 1937 sobre la Alemania nazi, la opinión de Dobb es que pensar de manera diferente ``es tan ingenuamente como ver Herr Hitler y su Estado totalitario como producto de una voluntad popular porque celebró un plebiscito''. En el segundo campo, Las valoraciones autónomas del mercado bajo el capitalismo son ilusorias y representan, en sí mismas, un grado muy alto de autoritarismo. Las opciones de los consumidores, aparentemente libres, son en realidad la expresión de la diferencia de estatus económico y social y de la ``dependencia de los sin dueño con respecto al dueño''.82

Una armonía esencial de intereses entre clases niega la existencia de una plusvalía marxista y de la explotación del trabajador, y se convierte en ``un simple caso de petitio principii''. 83

Con los economistas marxistas del siglo XX, los modos de producción y la estructura de clases de la sociedad siguen siendo el supuesto de partida para la teorización económica; y el enfoque fundamental de la producción y distribución de la riqueza ---la teoría del valor del trabajo--- no se modifica sustancialmente.

La visión de la estructura de la sociedad en diferentes clases y la teoría del valor del trabajo permite un acercamiento del marxismo a la escuela clásica. Como se mencionó, tanto los economistas clásicos como los marxistas ven la estructura de la sociedad dividida en clases sociales. Pero los primeros piensan que estas clases, en sus respectivos roles, contribuyen a crear una condición de bienestar óptimo de otra manera no alcanzable, y por lo tanto ven esta estructura social, incluso como resultado de una evolución histórica, como válida ``en todos los tiempos y lugares''. Estos últimos ven esta estructura, solo porque históricamente determinada, como posicionada específicamente en términos de tiempo y lugar: la estructura social y económica del capitalismo, que evoluciona hacia una nueva sociedad socialista una vez que el proletariado derroca a la clase capitalista.

Dentro de esta estructura de clases de la sociedad, tanto Ricardo como Marx reconocen que el valor de una mercancía se basa en el trabajo necesario para producirla. Esto es inmediatamente evidente en una economía de subsistencia primitiva, donde el producto total es justo lo que se necesita para mantener, año tras año, el nivel de producción tal como está, y donde el trabajador toma el control de todo el proceso de producción y obtiene el total. producto de su trabajo. En esta sociedad, los precios de equilibrio relativo de las mercancías tenderían a ser iguales a las cantidades relativas de trabajo necesarias para producirlas (esta es la teoría clásica, o ``ley'', del valor del trabajo). Marx está de acuerdo con este análisis: estos son los que Keynes llamó ``los fundamentos ricardianos del marxismo''. 84

Cuando una clase capitalista entra en escena y se obtiene un excedente de subsistencia, el producto neto debe distribuirse entre los participantes en el proceso de producción, en particular entre el trabajador y el capitalista. La ley clásica del valor podría continuar operando si todo el producto neto fuera al trabajador: los precios continuarían siendo determinados como se indicó anteriormente y no surgiría ninguna ganancia: una suposición imposible en una economía capitalista. Por otro lado, el producto neto no se puede acumular completamente para el capitalista: en una sociedad de no esclavos, una sociedad capitalista, el trabajador vende su fuerza de trabajo al capitalista a un precio (salario), y el salario no puede ser cero.

En este caso, los precios difieren de la cantidad de trabajo empleado en la producción. Adam Smith dio una respuesta formal, no resolutiva: este producto neto se dividiría entre los factores de producción de acuerdo con su respectiva contribución al proceso de producción, resultando en su ``precio natural'' (Capítulo 1 ); pero, ¿cómo se puede evaluar este precio natural? ¿Y cómo se determinarían en consecuencia los precios de las materias primas? ¿Cuál es la relación que vincula precios, salarios y ganancias? En particular, ¿se divide el producto neto entre los factores de producción de manera casual o responde a una determinada ``ley''? ¿Hay consideraciones éticas a considerar? ¿Hay alguna clase que se lleve más de lo que se merece en relación a su aporte productivo? Estos puntos habían quedado indefinidos dentro de la Escuela Clásica. Joan Robinson escribió: ``necesitamos conocer los precios para valorar el excedente que se va a dividir. Este fue el problema que desconcertó a Ricardo''. 85

Marx había resuelto el problema de manera radical. Introdujo el concepto de ``plusvalía'', como la cantidad de trabajo que se apropia el capitalista. En palabras de Marx, ``el trabajo excedente de la fuerza de trabajo es el trabajo barato del capital y, por lo tanto, forma un supervalor para el capitalista, un valor que no le cuesta ningún rendimiento equivalente'' (véase también el capítulo 1 ). Pero, incluso con Marx, la relación entre el supervalor, los salarios y los medios de producción y, en consecuencia, el precio de la mercancía, sigue sin resolverse.

Los economistas marxistas han abordado el trabajo del economista italiano Piero Sraffa, de la Universidad de Cambridge, como evidencia de una continuidad entre el pensamiento clásico y marxista: una forma de reconciliar la economía de la Escuela Clásica y la doctrina marxista, y la adecuada adaptación del esquema de Marx a un modelo moderno. sociedad capitalista.

Su obra principal fue publicada en 1960: fruto de largos años de reflexiones concentradas en un libro bastante compacto, donde las matemáticas son la forma de expresión predominante. 86 Algunos economistas de la corriente principal vieron el libro de Sraffa como una interpretación ricardiana de la sociedad: existe un vínculo entre su dirección editorial de una edición crítica de las obras de Ricardo 87 y sus propios intereses de investigación. Otros criticaron el esquema de Sraffa como abstractamente lógico pero no respondía a una experiencia verificable.

Lejos de discutir el controvertido ``modelo'' elaborado por Sraffa (vale la pena repetirlo, esto no es una historia del pensamiento económico), queremos aquí resaltar, detrás del velo de su razonamiento, las ``notas de la filosofía social'', para usar la de Keynes. terminología --- que se puede inferir de su trabajo. Pero hacerlo requiere algunos indicios de su línea de pensamiento.

Su propósito teórico es llenar el vacío que Ricardo y Marx habían dejado sin explicar: encontrar la conexión lógica que vincula salarios, ganancias y precios, o, dicho de otra manera, resolver el problema de la determinación de precios y, con ello, el problema de la distribución del ingreso entre salarios y ganancias, de una manera diferente a la de los economistas neoclásicos. 88

Sus supuestos 89 son:

\begin{quote}
que los productos de ciertas industrias deben constituir los insumos de otros: lo que es insumo en una industria es el producto de otra. Cada sector de la economía no puede funcionar si no es conectándose con otros sectores;

que ---en línea con la Escuela Clásica--- el valor es independiente de la utilidad individual, y por tanto del concepto de ``demanda'': \hspace{0pt}\hspace{0pt}no sólo el trabajo de Sraffa ni siquiera entra en los argumentos marginalistas de la teoría neoclásica 90 ; pero también descuida la relevancia macroeconómica de la demanda agregada en el pensamiento keynesiano 91 ;

que ``su'' sistema económico no depende de cambios en la escala de producción o en la proporción de factores de producción: de esta manera, no se involucra, nuevamente, en cuestiones relacionadas con cambios en el ``producto marginal'';

que el rendimiento es constante: la tasa de ganancia, definida como la relación entre la ganancia y los medios de producción (es decir, la inversión del propietario), debe ser la misma en cualquier industria: se distribuye por todo el sistema económico en proporción a los medios de producción empleados en el proceso productivo.
\end{quote}

Sraffa analiza la elaboración de Marx de la teoría del valor trabajo. Sin embargo, hace una adaptación al esquema de Marx. Marx había escrito que el ``capital variable'' 92 es la contribución del trabajador al proceso de producción, y que se divide en dos partes: el ``capital de trabajo'', pagado al trabajador como salario, y la ``plusvalía'', que es el resto, expropiado por el capitalista. Sraffa no distingue entre capital-trabajo y plusvalía (rechazando implícitamente la visión marxista de que la ganancia es una expropiación de lo que se le debe al trabajador), sino que utiliza el concepto de ``producto neto'', que incluye a ambos.

En una sociedad capitalista, este producto neto se divide entre el trabajador como salario y el capitalista como ganancia. A la luz de los supuestos mencionados anteriormente, el punto crítico a examinar es, según Sraffa, la diferente proporción en la que se emplean mano de obra y medios de producción (insumos, como se acaba de decir) en cada industria, porque la ganancia surge como resultado . La ganancia no es una expropiación del trabajador, como en Marx, sino un valor residual que se puede determinar cuando conocemos: (a) el salario, que se ve como resultado de las luchas sociales; y (b) la proporción trabajo / medios de producción.

Si esta proporción es diferente en diferentes industrias, para tener la misma tasa de ganancia, dado un cierto nivel de salarios, la ganancia debe ser mayor donde los medios de producción están en mayor proporción en relación con el trabajo.

Pero esta no es la conclusión, según Sraffa, porque cualquier mercancía se produce utilizando medios de producción que son, a su vez, mercancías producidas a través del trabajo y los medios de producción combinados en diferentes proporciones. Por lo tanto, ``los movimientos relativos de los precios de dos productos cualesquiera \ldots{} llegan a depender \ldots{} no sólo de la proporción de trabajo a los medios de producción por los que se producen respectivamente, sino también de las proporciones en las que esos medios se han producido, y también sobre las proporciones en las que se han producido los medios de producción de esos medios de producción, etc.'' 93

El problema que ``desconcertó a Ricardo'' ---las desviaciones de los precios del valor del trabajo--- puede resolverse, por lo tanto, observando que la tasa de ganancias sobre la economía en su conjunto se determina tan pronto como conocemos la razón del producto neto (salarios y ganancias ) a los medios de producción y la proporción del producto neto que se destina a los salarios. O, en otros términos, cuando se da la proporción del producto neto que se destina a los salarios, la tasa media de ganancia depende del nivel de la relación entre el producto neto y los medios de producción.

No profundizaremos más en el razonamiento de Sraffa. Parece privar a la teoría de Marx del componente ideológico y construir un modelo de funcionamiento de una sociedad capitalista que responde a la economía como una ``ciencia''. Al igual que Walras o Pareto, no ve ninguna forma de expresar su teoría más que en una secuencia de ecuaciones: un hábito, o una necesidad, por así decirlo, que se generalizaría cada vez más en la economía. Como se mencionó, su teoría encontró voces críticas de sus colegas en Cambridge, cuya crítica se basa, más que en una consistencia abstracta, en la verificabilidad empírica. 94

Pero esta sería una lectura parcial de su modelo. Sraffa tenía un trasfondo liberal culturalmente sólido; Sin embargo, estaba insatisfecho con la forma en que el capitalismo funcionaba efectivamente en países donde las ideas liberales se interpretaban como una protección pura de intereses privados creados, y simpatizaba cada vez más con las ideas socialistas, estando particularmente cerca de la posición de los comunistas (la de Gramsci, en particular) 95 y de los economistas marxistas. en Gran Bretaña (como Dobb). 96

Su componente ideológico es bien visible cuando, de manera marxista, piensa que la ganancia del capitalista, la recompensa del capital como factor específico de producción, es el resultado de la interacción -o lucha- entre el propietario de los medios de producción y el trabajador que le presta su fuerza. En la práctica, su visión encaja bien en una actitud generalmente crítica hacia el funcionamiento real de una sociedad capitalista dentro del marco institucional de un sistema político democrático. Esta visión daría un sustento teórico a los movimientos políticos de izquierda y a los trabajadores altamente sindicalizados, en las crecientes tensiones entre el capital y el trabajo que caracterizaron a fines de los años sesenta y setenta. 97

\hypertarget{cruxedtica-liberal-del-marxismo}{%
\section*{Crítica liberal del marxismo}\label{cruxedtica-liberal-del-marxismo}}
\addcontentsline{toc}{section}{Crítica liberal del marxismo}

Las cifras de capitalista, terrateniente y trabajador, y las categorías correspondientes de ganancia, renta, salario, están bien firmes en el trabajo de Adam Smith, y esta ``distinción de rangos'' ---para citar sus palabras en The Theory of Moral Sentiments--- es la base de la ``paz y orden de la sociedad''. 98 La misma relevancia de las clases sociales, pero en sentido contrario, es decir, para mostrar la explotación del trabajador por parte del propietario de los medios de producción, se mantiene en la doctrina marxista.

Con la afirmación del pensamiento neoclásico individualista y sin clases guiado por la utilidad del cambio de siglo, la orientación centrada en la clase tanto de la escuela clásica como del marxismo desaparece. Y durante el siglo XX, el liberalismo de cualquier matiz no se ocupa de las cuestiones básicas de la organización económica en términos de diferentes clases sociales. Sigue siendo ciego a las clases, tal vez bajo la influencia de los valores políticos democráticos que subyacen al liberalismo del siglo XX. 99 Por tanto, la posición adoptada por los economistas liberales sobre el marxismo debe evaluarse teniendo en cuenta que la clase social no forma parte explícita de su vocabulario. Esto no significa que el liberalismo del siglo XX asuma una sociedad sin clases, solo significa que su razonamiento no se basa en esa distinción.

Otro punto a tener en cuenta es que cada una de las diferentes corrientes de pensamiento (metamorfosis) del liberalismo en el siglo XX tiene su propia actitud hacia el marxismo: tenemos por un lado a economistas que enfatizan el tema de la distribución de la riqueza, y por otro a economistas que confiar en la maximización del producto, en la producción de riqueza y en la libertad de elección del individuo: la posición de Keynes o Beveridge frente al marxismo no puede ser la misma que la de Hayek o von Mises o Friedman.

Como se mencionó anteriormente, Keynes afirmó con bastante firmeza que su obra principal significaría la destrucción de los ``fundamentos ricardianos del marxismo'', pero su actitud hacia el marxismo era más benévola que la de los economistas neoclásicos y libertarios. En palabras de Schumpeter, ``no existía un abismo entre Marx y Keynes como el que había entre Marx, Marshall y Wicksell''. 100En la Teoría General, solo hay menciones pasajeras de Marx. Por otra parte, su crítica no se basa en una irracionalidad esencial del sistema económico socialista, en su imposibilidad lógica de alcanzar una situación de equilibrio en el sistema económico, sino en la incapacidad de los socialistas marxistas de su tiempo para comprender la ``estructura'' en evolución. del capitalismo, es decir, que el capitalismo como se describe en El Capital ha cambiado mucho: el capitalismo actual es apenas un recuerdo del antiguo. Stalin ---escribe Keynes--- ``mira hacia atrás a lo que era el capitalismo, no hacia adelante a lo que se está convirtiendo. 101Ese es el destino de quienes dogmatizan en el ámbito social y económico donde la evolución avanza a un ritmo vertiginoso de una forma de sociedad a otra \ldots{} por una razón u otra, el Tiempo y la Sociedad Anónima y la Función Pública han traído silenciosamente la clase asalariada en el poder. Todavía no es un proletariado. Pero un Salariat, sin duda. Y marca una gran diferencia''. Sobre el comunismo, Keynes escribe irónicamente que ``se nos ofrece como un medio para mejorar la situación económica, es un insulto a nuestra inteligencia. Pero ofrecido como medio para empeorar la situación económica, ese es su atractivo sutil, casi irresistible''. 102

Beveridge se inclina más hacia el socialismo, cuyas propuestas, tal como figuran en Pleno empleo (capítulo 2 ), ``no son ni el socialismo ni una alternativa al socialismo:\ldots{} Un control consciente del sistema económico al más alto nivel, un nuevo tipo de presupuesto que requiere la mano de obra como su dato ---demanda directa sostenida adecuada de los productos de la industria --- organización del mercado de trabajo --- éstos son necesarios en cualquier sociedad moderna''. 103

Un libertario radical como von Mises lanza un ataque muy diferente al socialismo de Marx, y esto se puede entender mejor si tenemos en cuenta lo que es el liberalismo, según él. ``El liberalismo nunca ha pretendido ser más que una filosofía de la vida terrena\ldots{} Nunca ha pretendido agotar el Último o Mayor Secreto del Hombre. El pensamiento antiliberal lo promete todo''. 104 De hecho, solo dos visiones se oponen a la organización social y económica de la sociedad. Por un lado, está el liberalismo y la economía de mercado; por el otro, la siguiente lista de organizaciones sociales: Estado de Bienestar, Socialismo, regímenes nazi y fascista, New Deal, incluso ---en una edición posterior de su obra--- la Argentina de Perón, todos bajo la bandera abrazadora del ``socialismo pleno'', sus rivalidades a pesar de. 105Y, más tarde, ``Nuestra propia civilización descansa en el hecho de que los hombres siempre han logrado vencer los ataques de los redistribuidores'', donde la redistribución es ``la consigna de los socialistas''. 106 A pesar de su significado aparentemente diminuto del concepto de liberalismo (una especie de materia ``terrenal''), su concepto es tal que sólo el liberalismo da contenido real a la idea de democracia: ``la democracia sin liberalismo es una forma hueca''. 107

En cuanto al marxismo, von Mises, confirmando un juicio generalizado que es común tanto a los economistas liberales como a los marxistas, subraya que, a propósito, Marx no dedicó atención a la organización de una economía socialista. Según von Mises, ``el propósito de la prohibición de estudiar el funcionamiento de una comunidad socialista \ldots{} realmente tenía la intención de evitar que la debilidad de las doctrinas marxistas saliera realmente a la luz en la discusión sobre la creación de una sociedad socialista practicable''. 108

Pero von Mises es uno de los pocos economistas que brindó con claridad la explicación de la imposibilidad de establecer una contabilidad económica adecuada bajo el socialismo. Observó que en cualquier economía las transacciones se realizan en términos de un medio general de intercambio y no en términos de valores de uso subjetivos (es decir, en términos de juicios de valor sobre la utilidad de un determinado bien). Lo que quiere lograr una economía socialista es la sustitución de cálculos en especie por cálculos en términos de dinero. Esto es una ilusión y la producción racional de bienes se vuelve imposible. En la producción de un determinado bien de consumo, se debe crear una cadena de suministro de bienes intermedios, a través de todos los establecimientos involucrados en el proceso de producción. El ``mando de una autoridad suprema regiría el negocio del suministro'', pero la administración económica no tendría un sentido real de dirección al tomar las decisiones sobre cuánto producir y de qué manera combinar la producción de esos bienes intermedios. En la larga cadena de producción de un bien de consumo a través de una serie de fábricas interconectadas que producen los bienes intermedios, ``no hay forma de determinar si una determinada pieza de trabajo es realmente necesaria, si no se desperdicia mano de obra y material para completarla''. ¿Cuál de los procesos alternativos de producción es más satisfactorio? Se puede comparar la cantidad producida, pero no el gasto incurrido en su producción, y debe ser el gasto más pequeño. Lo que se necesita es un cálculo del valor en términos de dinero, no un cálculo técnico del valor de uso.109

No por casualidad, cuando a finales de la década de 1950, frente a las crecientes dificultades de un buen funcionamiento de la economía soviética, Oskar Lange se volvió hacia el tema de la contabilidad económica en una economía socialista, reconoció, citando entre otros a Ludwig von Mises, que el socialismo no pudo evitar un metro de cálculo que iba más allá de las medidas físicas. Como hemos visto anteriormente (Art. 3.5 ), tuvo que recurrir al método complicado e ineficaz de valorar la transferencia de bienes en los sectores nacionalizados ``por imputación''.

\hypertarget{socialismo-por-defecto-religiuxf3n-schumpeter-y-polanyi}{%
\section*{Socialismo por defecto: religión, Schumpeter y Polanyi}\label{socialismo-por-defecto-religiuxf3n-schumpeter-y-polanyi}}
\addcontentsline{toc}{section}{Socialismo por defecto: religión, Schumpeter y Polanyi}

Hemos visto, al comienzo de este ensayo, cómo Schumpeter explicó el origen de la economía política. Reconectó esta disciplina a dos raíces: los estudios filosóficos que se centraban en el hombre como entidad social, cuya actividad debía estudiarse a partir de la observación empírica y explicarse mediante una relación causa-efecto; y las opiniones de personas cuyo interés, rico en experiencia empresarial, se centra principalmente en asuntos prácticos y cotidianos relacionados con su actividad económica.

Más o menos en los mismos años de las reflexiones de Schumpeter, otros pensadores prestaron cada vez más atención a esta segunda raíz, es decir, al funcionamiento real del sistema capitalista, a la experiencia concreta de las personas orientadas a los negocios que ponen en práctica el sistema. Esto sucedía cuando la afirmación y el crecimiento del capitalismo, particularmente en las economías occidentales, estaba teniendo enormes consecuencias en la producción y distribución de la producción, en las relaciones sociales, en las estructuras políticas y en la actividad normativa. En Alemania, Max Weber, estudiando la estructura capitalista de la sociedad, intentó dar una respuesta a la siguiente pregunta: ¿qué tipo de condiciones psicológicas hicieron el desarrollo de la civilización capitalista moderna? 110

La respuesta, según Weber, debería encontrarse en la revolución religiosa protestante del siglo XVI que generó los movimientos que dieron origen al capitalismo tal como lo vemos. Esas condiciones fueron, en efecto, el resultado de una nueva actitud que las personas religiosas mostraron en el desempeño de sus deberes diarios, en particular su actividad empresarial. Los primeros capitalistas adoptaron un código de conducta económica y un sistema de relaciones, desafiando esquemas de ética social muy diferentes, antiguos y consolidados, y convenciones y leyes preexistentes que también contaban con el apoyo de la Iglesia y los Estados. El ascetismo protestante interpretó los negocios y el trabajo como un ``llamado'' divino, que debe observarse a través de la capacidad e iniciativa personal. Este llamado no fue visto como una condición en la que el individuo nació, como tal para ser aceptado pasivamente, sino como una ruta elegida por el individuo, a seguir con responsabilidad. Era ``lo más característico de la ética social de la cultura capitalista y es, en cierto sentido, la base fundamental de la misma''. Llamar era una obligación que debía cumplirse ``no importa \ldots{} si aparece en la superficie como una utilización de los poderes personales {[}del individuo{]}, o sólo de su posesión material''.111 La búsqueda de la riqueza era ``no solo una ventaja, sino un deber''. Nuevos estándares morales canonizaron como virtudes económicas lo que antes se condenaba como vicios. 112

``El summum bonum de esta ética, la obtención de más y más dinero, combinado con la estricta evitación de todo disfrute espontáneo de la vida, está sobre todo desprovisto de cualquier mezcla eudemonista, por no decir edonista''. 113 Esto hizo que el capitalismo protestante moderno específico fuera radicalmente diferente, no solo del capitalismo de otras épocas y lugares (desde China, a la India, al mundo clásico, a la Edad Media), sino también del utilitarismo crudo, según el cual la honestidad, la puntualidad, la laboriosidad, la frugalidad son sólo un excedente innecesario.

¿Cómo esta actitud religiosa hacia la actividad económica llegó a cambiar la estructura económica preexistente? Apoyándose en la iniciativa individual perseguida religiosamente, los calvinistas holandeses se opusieron a cualquier forma de capitalismo monopolista y políticamente privilegiado que representara la base de un fundamento ético social cristiano \ldots{} Los puritanos, igualmente, con el mismo espíritu ``repudiaron todas las conexiones con países capitalistas a gran escala \ldots{} Como una clase éticamente sospechosa, y se enorgullecían de su propia moralidad empresarial superior de clase media''. 114Eran, específicamente, enemigos apasionados del capitalismo privilegiado de estado, exaltando los impulsos individualistas de comportamiento racional, contribuyendo a industrias nacidas fuera de la asistencia de los poderes públicos establecidos: el contraste de dos formas de capitalismo era paralelo a los contrastes de carácter religioso. 115

Este impulso original, sin embargo, fue entonces completamente secularizado, se convirtió en el celo del capitalismo moderno, asumiendo una forma hedonista, desprendida del impulso religioso original. De manera fáustica ---escribe Weber citando a Goethe--- la adquisición de riquezas materiales se convierte en el objetivo principal de la vida, mientras que la ``vocación'' divina se pierde totalmente. ``Los puritanos querían trabajar en una vocación; nos vemos obligados a hacerlo. Porque cuando el ascetismo se trasladó de las células monásticas a la vida cotidiana ---continúa Weber con acentos casi marxistas 116-, y comenzó a dominar la moral mundana, hizo su parte en la construcción del tremendo cosmos del orden económico moderno. Este orden está ahora ligado a las condiciones técnicas y económicas de la producción de máquinas que hoy condicionan la vida de todos los individuos que nacen en este mecanismo'', quizás\ldots`` hasta que se queme la última tonelada de carbón fosilizado ''. 117

``En el campo de su mayor desarrollo, en Estados Unidos, la búsqueda de la riqueza, despojada de su significado religioso y ético, tiende a asociarse a pasiones puramente mundanas''. 118

Schumpeter, que incluye a Weber en la Escuela Histórica de Economía de Alemania (véase el capítulo 1 ), no prestó más que una escasa atención a su tesis. En una nota a pie de página de su Historia del análisis económico , explica que el error metodológico de Weber consiste en la adopción del ``método de los tipos ideales'': Weber pone al hombre feudal ``ideal'' contra el hombre capitalista ``ideal'', presenta el nuevo espíritu del capitalista: una actitud diferente hacia la vida y sus valores , nacido de la Reforma Protestante --- como transición del primero al último Tipo. Schumpeter escribe que se trata de un ``problema espurio'', que debe descartarse: este tipo de transición ideal no tiene contrapartida en la esfera de los hechos históricos. ``Él {[}Weber{]} se propuso encontrar una explicación para un proceso que la atención suficiente a los detalles históricos hace que se explique por sí mismo''. 119

Como se mencionó anteriormente, Weber había concluido que el capitalismo duraría ``hasta que se queme la última tonelada de carbón fosilizado'', es decir, indefinidamente. Joseph Schumpeter, en Capitalism, Socialism, and Democracy 120 , después de haber observado que cualquier intento de pronóstico social, si se basa en hechos y argumentos, es científico en sus resultados finales, concluyó que el capitalismo no podría sobrevivir.

La idea central de Schumpeter, que el capitalismo morirá por su propio éxito, se presenta de hecho no como una posición ideológica, preanalítica, sino más bien como una visión científica, basada en hechos y argumentos que apoyan ciertas inferencias (ciertamente, escribe, difíciles de probar). como un teorema de Euclides). No es un marxista ideológico, pero su conclusión es la misma que la de Marx. Fue víctima del determinismo, el término aborrecido de todo economista, que sin embargo es más propenso a seguirlo.

El hecho de que un agudo analista social y un gran economista produjera esta predicción y que, al menos hasta ahora, la experiencia parezca estar en conformidad con la predicción de Weber, plantea dudas sobre el carácter científico del pronóstico de Schumpeter o sobre la economía como ciencia, o al menos conduce a estrechar el alcance de la ciencia propiamente dicha ---es decir, de conclusiones sustentadas experimental y lógicamente--- dentro de la disciplina económica 121 . La experiencia parece respaldar las ideas históricas de la escuela alemana.

Comienza escribiendo que la historia del capitalismo es una de producción incremental, sin afectar sustancialmente la distribución de la riqueza. El éxito del capitalismo, en términos de bienestar, no se debe a una gama más amplia de bienes y servicios producidos (Luis XIV habría permanecido perfectamente feliz incluso sin la invención de la bombilla eléctrica, pudiendo gastar en cantidades ilimitadas de velas y tener a su disposición a todos los sirvientes que los atienden 122 ), pero hacer fácilmente disponibles bienes y servicios a precios baratos a estratos cada vez más grandes de la población.

Schumpeter luego pregunta si la estructura capitalista de la sociedad fue favorable para su desempeño exitoso. Sobre esto, comienza por observar que la sociedad burguesa se moldea sobre una base económica: el éxito se identifica con el éxito económico. El prototipo del hombre de éxito es el empresario, cuya actividad, según los economistas clásicos británicos, incluso realizada por interés propio, está orientada al interés de todos. Sin embargo, quedaba sin explicar un abismo entre el interés propio y el interés de todos, y se desarrollaron dos corrientes de pensamiento para llenar ese abismo. El enfoque neoclásico, basado en la competencia perfecta y la maximización del producto, teorizó un estado de equilibrio ``en el que todos los productos están al máximo y todos los factores se emplean plenamente''. 123Sin embargo, una segunda vertiente criticó este punto de vista: la competencia perfecta es la excepción; prevalecen la competencia monopolística y el oligopolio. Además, bajo las condiciones previstas por los economistas neoclásicos, ese equilibrio sería generalmente inconsistente con la producción máxima y el pleno empleo (curiosamente, en este punto Schumpeter no hace ninguna referencia a Keynes).

El hecho relevante es que, incluso en aquellas condiciones que hacen de ``una edad de oro de la competencia enteramente imaginaria'', en ese tipo de estructura, es decir, en un entorno de grandes empresas en condición cuasimonopolística, la tasa de crecimiento de la producción continuó. sin cesar: ``Una sospecha impactante de que las grandes empresas tienen que ver con un mayor nivel de vida''. Esto prueba que el capitalismo nunca es un estado de equilibrio estacionario, sino un proceso evolutivo, un ``proceso de destrucción creativa'' 124 , al contrario de un estado de calma permanente. El impulso del capitalismo proviene de nuevos bienes de consumo, nuevos métodos de producción, nuevos mercados, nuevas formas de organización industrial, no de una condición de competencia perfecta en el mercado, reputada como buena en todo momento.

¿Por qué esta estructura, basada en grandes empresas y en una competencia lejos de ser perfecta, ha tenido tanto éxito? Schumpeter analiza la historia económica y política y menciona cinco ``circunstancias excepcionales'' que favorecen el crecimiento y mejores niveles de vida: una acción gubernamental benevolente que, después de la fase del capitalismo sin restricciones (después de alrededor de 1870), levantó nuevas trabas como sistemas de seguridad social ---Pero no mucho para dañar la tendencia anterior; nuevos descubrimientos de oro, que, en un régimen de patrón oro, permitieron condiciones y políticas monetarias adaptativas; Aumento de población; nuevos descubrimientos de fuentes de materias primas, como carbón, petróleo; y la fuerza de las nuevas tecnologías.

Sin embargo, estas condiciones favorables no conducen, según Schumpeter, a mantener un pronóstico favorable para su futuro. ``El desempeño capitalista {[}real{]} no es\ldots{} relevante para el pronóstico\ldots{} Por lo tanto, no voy a argumentar, sobre la base de ese desempeño, que es probable que el intermezzo capitalista se prolongue. De hecho, voy a sacar la inferencia exactamente opuesta''. 125 ¿Por qué?

En consonancia con el estado evolutivo del capitalismo, han estado operando dos factores, adversos al capitalismo. La primera, en la que tanto Marx como Keynes estarían de acuerdo, se denomina ``Teoría de la desaparición de las oportunidades de inversión'', 126 según la cual, y en términos generales, los factores excepcionales mencionados anteriormente dejarían de operar gradualmente (Schumpeter no confía en ``Pump priming'' a través del gasto público en inversión, incluso en déficit).

Luego, y más importante, está la ``Evaporación de la sustancia de la propiedad'', 127 que tiene a su vez dos componentes: el lado industrial y el lado del consumidor.

Sobre el primer factor -oportunidades de inversión- Schumpeter piensa en un estado de saciedad de deseos y perfección tecnológica absoluta, con una consecuente mecanización del progreso que afectaría al emprendimiento y a la sociedad capitalista: un estado que está lejos de nosotros, pero cuya perspectiva ya es observable. . El capitalismo, como proceso evolutivo, no pudo sobrevivir. Aquí está el acento en una distinción de roles que está bastante borrosa en la visión de Marx: el empresario y la burguesía.

La función del emprendedor ``no consiste esencialmente ni en inventar nada ni en crear las condiciones que explota la empresa \ldots{} consiste en hacer las cosas'' 128 ; es una función fáctica ---el emprendedor como emprendedor--- que pierde importancia cuando su trabajo se convierte en una especie de rutina. El progreso económico tiende a despersonalizarse y automatizarse, ``el trabajo de las oficinas y los comités tiende a reemplazar la acción individual''. 129 Se trata de ``la obsolescencia de la función empresarial''. 130 El empresario no es, per se , una clase social, pero la clase burguesa lo absorbe a él y a su familia y conexiones.

¿Y la burguesía? Schumpeter lo ve, más allá de ``los recintos de consideraciones puramente económicas'', como ``el componente cultural de la economía capitalista \ldots{} su superestructura socio-psicológica'', en términos marxistas. 131

El primero es una parte pequeña pero esencial del segundo; pero, al mismo tiempo, la burguesía depende del empresario. ``Entre, está el grueso {[}de{]} industriales, comerciantes, financieros y banqueros: se encuentran en la etapa intermedia entre el emprendimiento empresarial y la mera administración de un dominio heredado''. 132 Las industrias gigantes están cada vez más formadas por estos administradores, mientras que el empresario tiende a desaparecer. El empresario que desaparece arrastra a su clase social hacia su propio declive y muerte.

En cuanto al segundo factor del declive capitalista, la evaporación de la sustancia de la propiedad, significa, en primer lugar, la ``Evaporación de la propiedad industrial''. 133 El empresario moderno, emprendedor o administrador gerente, ``racional y poco heroico'', adquiere paulatinamente la psicología del asalariado que trabaja en una organización cada vez más burocrática: su voluntad de luchar ya no es la voluntad del destructor creativo; la corporación moderna socializa la mente burguesa y eventualmente matará las raíces del empresario. Pero la burguesía no puede salvarse a sí misma tomando el liderazgo del gobierno: nunca se ha acostumbrado a gobernar: ``el libro de contabilidad y el cálculo de costos absorben y limitan''. 134 El proceso capitalista, después de haber destruido el marco institucional de la sociedad feudal, se destruye al final a sí mismo.

Conectada a la primera ``evaporación'' está la ``Evaporación de la propiedad de los consumidores'' 135 : la desintegración de la familia, pari passu con el progreso del capitalismo, hace que las comodidades del hogar burgués sean menos evidentes que sus cargas; la hospitalidad, en lugar de la recepción en casa, se ``traslada cada vez más al restaurante o club''. 136 El trabajo y el ahorro para la esposa y los hijos se desvanecen de la visión moral del empresario; este componente hedonista es un factor negativo para la eficiencia capitalista. La familia solía ser la fuente principal del afán de lucro. Esto conduce a ``un tipo diferente de homo oeconomicus\ldots{} Que se preocupa por cosas diferentes y actúa de diferentes maneras;\ldots{} desde el punto de vista de su utilitarismo individualista, el comportamiento de ese viejo tipo sería de hecho completamente irracional''. 137 El horizonte temporal del empresario se reduce a su esperanza de vida.

La desaparición gradual de los ``valores'' capitalistas lleva a una actitud crítica hacia el capitalismo mismo. La burguesía encuentra que ``su actitud {[}crítica{]} no se limita a las credenciales de reyes y papas, sino que ataca la propiedad privada y todo el esquema de valores burgueses''. 138La vida se saca de la idea de propiedad. La propiedad se desmaterializa, se disfuncionaliza y se ausenta. Hay un poder impulsor extraracional y el capitalismo se enfrenta a un juicio donde los jueces ya han expresado una sentencia de muerte. Sin embargo, los agravios y los ataques no bastarían para generar una hostilidad activa hacia el orden social. Debe haber grupos que organicen el resentimiento, lo alimenten y le den voz. El ataque final al capitalismo vendrá de los ``intelectuales'', a quienes la propia clase burguesa nutre y defiende, otorgándoles el papel de expresar su propio descontento y frustración. Tienen un interés personal en el malestar social, incluso sin tener responsabilidades concretas en la conducción de los asuntos públicos. No son profesionales, no tienen la responsabilidad directa del conocimiento de primera mano para los asuntos prácticos, generalmente no son políticos, pero tienen el rol de asesores políticos. Invocan la libertad, una libertad que puede desagradar a la clase burguesa, pero ``la libertad que la burguesía desaprueba no puede ser aplastada sin aplastar también la libertad que ella {[}la misma burguesía{]} aprueba''.139 Schumpeter concluye: ``No hay tanta diferencia como podría pensarse entre decir que la decadencia del capitalismo se debe a su éxito y decir que se debe a su fracaso''. 140

La ``visión'' de Schumpeter ---porque de ninguna otra manera que como visión se pueda definir su pensamiento--- parece correcta y profética cuando se ocupa de un declive bien arraigado de los valores burgueses en las décadas posteriores a su libro (en particular, lo que él etiquetas de ``evaporación de la propiedad de los consumidores''), pero parece haber pasado por alto totalmente el posterior resurgimiento de los ``valores'' capitalistas, de la ``destrucción creativa'' que estamos presenciando, en nuevas formas. Si este desarrollo puede verse de una manera marxista, como un signo de las ``estructuras'' en evolución del capitalismo o como una poderosa influencia de las ``superestructuras'' en evolución, de las filosofías económicas del neoliberalismo, lo veremos en el siguiente capítulo.

La actitud crítica de Karl Polanyi hacia el liberalismo económico está arraigada en una visión cristiana, y su análisis político y económico del orden liberal lo lleva a vislumbrar el advenimiento de un socialismo espiritual, cuyos primeros signos ---observa Polanyi--- ya se pueden detectar en un una serie de iniciativas desconectadas, algunas que se remontan al siglo XIX: una especie de socialismo muy alejado de la doctrina del socialismo marxista.

Su obra principal, La gran transformación. Los orígenes políticos y económicos de nuestro tiempo , 141 es más o menos contemporáneo de Schumpeter, Capitalism, Socialism and Democracy. Ambos se publicaron a principios de la década de 1940, cuando el impulso de luchar contra las dictaduras se mezcló con un replanteamiento radical de los puntos de vista económicos y políticos establecidos, con un futuro particularmente incierto por delante (el fascismo es el objetivo de comentarios duros en ambos libros, pero particularmente en Polanyi). Ambos reservan escasa o nula atención a Keynes, cuya obra maestra había sido publicada unos años antes. El liberalismo está claramente bajo presión en ambos libros. Ambos se caracterizan por una profunda comprensión histórica de la evolución del sistema capitalista y ven su insostenibilidad final. Ambos consideran una especie de socialismo como el resultado de un proceso gradual pero inevitable, aunque ambos no ven a Marx como una clave decisiva para explicar los problemas de su propio tiempo.

El concepto de ``clase'', muy presente en Schumpeter, está ausente en Polanyi, que se apoya en la ``sociedad''. La sociedad corre el riesgo de autodestruirse por las fuerzas de la economía de libre mercado. Este último autor tiene una especie de inspiración moral que está totalmente ausente en el primero: los juicios de valor del autor están, con Polanyi, expresados \hspace{0pt}\hspace{0pt}sin reservas y de manera amplia, y contrastan con el enfoque ``científico'' de Schumpeter: Polanyi nunca vería el aburrimiento de la clase burguesa y el agotamiento del espíritu animal del empresario como factor principal de la desaparición del capitalismo. Ataca el núcleo de la sociedad liberal, visto como una expresión, o un derivado, del mercado autorregulado y antitético de la sustancia de la democracia. Critica la opinión ampliamente compartida de que nuestra sociedad comenzó aproximadamente con la publicación de La riqueza de las naciones de Adam Smith y que las culturas anteriores son irrelevantes para comprender los problemas de nuestra época. Su análisis del siglo XIX se lleva a cabo, por tanto, no porque la civilización parta de allí, sino porque los problemas actuales (siglo XX) no pueden entenderse sin mirar un rasgo típico de ese siglo: el sistema de mercado autorregulado. Polanyi ve el mercado autorregulado como la ``matriz'' de todo el sistema liberal y el Estado liberal como su creación. Desde entonces, la economía clásica ``acechó a la ciencia del hombre, y la reintegración de la sociedad en el mundo humano se convirtió en el objetivo perseguido persistentemente de la evolución del pensamiento social''.142 (conviene recordar aquí los ``fundamentos ricardianos del marxismo'', como escribió Keynes).

Es bien conocido su esquema de explicación de la estructura política y económica imperante en el siglo XIX, que aseguró los ``Cien Años de Paz''. Se caracterizó por cuatro arreglos institucionales: por un lado, el sistema de equilibrio de poder y el Estado liberal, basado en instituciones políticas y nacionales; por otro, el régimen monetario del patrón oro y el mercado autorregulado, que son, a la vez, instituciones económicas e internacionales. 143

Entonces, las altas finanzas desempeñaron un papel fundamental, encarnado por la familia Rothschild. No estaban sujetos a ningún gobierno, simbolizaban el principio abstracto del internacionalismo, respondían a las necesidades de los Estados de la época teniendo en cualquier Estado agentes que contaban con la confianza de gobiernos e inversionistas. 144 Incluso al no estar diseñada como un instrumento de paz, la influencia de las altas finanzas, ejercida sobre los gobiernos nacionales, fue en sí misma un factor alentador de la paz: una guerra general no podría ser funcional para el buen funcionamiento del sistema monetario internacional, el patrón oro, sobre el cual florecerían el comercio y el crédito internacionales.

El problema esencial con este sistema económico (Polanyi parece pensar en el sistema político liberal como una ``superestructura'' del económico, casi de una manera marxista) fue el error de juicio del liberalismo económico sobre las necesidades sociales, visto solo desde el punto de vista económico. El comportamiento del hombre no es únicamente económico. Este error de juicio no fue por casualidad: una vez establecido, el sistema de mercado puede funcionar correctamente solo en ausencia de interferencia externa de ningún tipo. Este sistema está ``controlado, regulado y dirigido únicamente por los mercados; el orden en la producción y distribución de bienes está encomendado a esta autoridad autorreguladora'' 145(No es necesario enfatizar que esta visión del homo oeconomicus ya había sido sometida a tensiones incluso por pensadores liberales, quienes, según diferentes corrientes de pensamiento, señalaron, señalarán y señalarán, a motivaciones no económicas, la irracionalidad en las elecciones u otros factores que influyen en sus acciones).

Para funcionar correctamente, el sistema liberal necesita tres principios: que la mano de obra debe encontrar su precio en el mercado; que la creación de dinero debe estar sujeta a un sistema automático (patrón oro); que las mercancías se comercialicen internacionalmente sin preferencias ni obstáculos. La Revolución Industrial, los contratos laborales gratuitos estipulados sin la protección de ningún trabajador, la abolición de los deberes protectores son términos que definen la sumisión de la sociedad al mercado autorregulado. El hombre, y significativamente la naturaleza, estaban sujetos a las leyes de la oferta y la demanda, tratados como mercancías, como bienes producidos para la venta. La subordinación de los deseos sociales a las leyes del mercado no significa que la separación de las dos esferas haya existido en todo tipo de sociedad en todo momento. ``Normalmente, el orden económico es meramente una función del social, en el que está contenido. Ni bajo condiciones tribales, ni feudales ni mercantiles había\ldots{} un sistema económico separado en la sociedad. La sociedad del siglo XIX \ldots{} fue, en efecto, una salida singular''.146

Polanyi se entrega a una visión benévola de las estructuras sociales pasadas (una especie de laudator temporis acti), donde el trueque en lugar del intercambio de mercado, el mercantilismo en lugar del libre comercio parecía responder más directamente a las necesidades sociales y la igualdad: ``el mayor número de pobres es \ldots{} encontrarse en aquellas {[}naciones{]} que son las más fértiles y las más civilizadas''. 147 La economía, en palabras de Polanyi, debe estar insertada en instituciones no económicas: esto significa que los actos de producción y distribución deben realizarse como una descarga de las obligaciones sociales. 148

La subordinación de los deseos sociales a los sistemas de libre mercado no podría durar sin una autodestrucción de las estructuras sociales. Incluso el libre mercado y el libre comercio y la competencia requerían, como consecuencia, la intervención externa para ser viable. El marxismo mismo era, según Polanyi, un ``mito liberal'': en el marxismo, en realidad, la perspectiva económica liberal encontró un apoyo poderoso. Rechaza la opinión de que la intervención pública es el resultado de una ``conspiración colectivista''; La intervención pública y las consiguientes restricciones al laissez-faire comenzaron de manera espontánea, como una autoprotección realista de la sociedad, en países de una configuración política e ideológica ampliamente disímil.

En el siglo XIX, la expansión de la economía de mercado comenzó a ser contrarrestada por una reacción contra la dislocación del mercado ``que atacaba el tejido de la sociedad''. Robert Owen, a quien Polanyi ve como una figura imponente que presagia una sociedad entrante y diferente, dio una verdadera visión cristiana al decir que la economía de mercado, si se deja evolucionar de acuerdo con sus propias leyes, crearía un mal grande y permanente. Como reacción al liberalismo surgió la necesidad de protección social: legislación protectora, asociaciones restrictivas.

Por tanto, el liberalismo del siglo XIX no fue destruido por la Primera Guerra Mundial, ni por una revolución del proletariado, ni por el fascismo de las clases media y pequeña; no por la tendencia marxista a la caída de la tasa de ganancia, ni por el subconsumo o la sobreproducción, es decir, por una demanda keynesiana insuficiente. El liberalismo fue destruido por las tensiones y tensiones creadas por el conflicto entre el mercado y los requisitos elementales de una vida social organizada, por la reacción de la sociedad para no ser aniquilada por el mercado autorregulador.

Definición de socialismo de Polanyi: ``El socialismo es esencialmente la tendencia inherente a una organización industrial a trascender el mercado autorregulador subordinándolo conscientemente a una sociedad democrática''.

Este tipo de nueva sociedad socialista puede desarrollarse en diferentes líneas, con un factor unificador: el trabajo, la tierra y el dinero (los tres principios de la sociedad) serán liberados de las limitaciones del mercado autorregulado: la naturaleza de la propiedad sufrirá una profunda cambiar.

Con acentos religiosos, Polany concluye escribiendo que tres hechos representan la ``conciencia del hombre occidental'': el conocimiento de la muerte, de la libertad, de la sociedad. El primero es revelado por el Antiguo Testamento, el segundo por las enseñanzas de Jesús, el tercero proviene de nuestra propia vida en una sociedad industrial: a este último no se le atribuye un gran nombre, sino el de Robert Owen. El fabricante británico convertido en reformador social fue el más cercano a una comprensión completa de lo que significa vivir en nuestra sociedad. ``Reconoció que la libertad que obtuvimos a través de las enseñanzas de Jesús era inaplicable a una sociedad industrial compleja. Su socialismo fue la defensa del derecho del hombre a la libertad en una sociedad así. Había comenzado la era poscristiana de la civilización occidental, en la que el evangelio ya no era suficiente. Y, sin embargo, siguió siendo la base de nuestra civilización''.149

La ideología de Polanyi termina por vislumbrar una especie de socialismo matizado, no claramente definido, donde el componente religioso prevalece y deja indeterminados ---por no merecer una atención especial--- los temas de maximización y distribución de la producción, en beneficio de ``una relación distintivamente humana de personas que en Europa occidental siempre estuvieron asociadas con las tradiciones cristianas''. 150

Su visión bastante singular puede, por un lado, verse como una anticipación de la organización política y económica del Estado socialdemócrata, una especie de prefiguración del Estado de bienestar del mundo posterior a la Segunda Guerra Mundial; por otro, como sociedad utópica que la experiencia de dos siglos ha anulado como alternativa realista. Una crítica al trabajo de Polanyi es, de hecho, que este papel sociopolítico (no económico) que desempeñan los actores de la sociedad no explica por qué se dedican a actividades de producción y distribución, esa es la motivación de esas actividades. 151

\hypertarget{notas-1}{%
\section*{Notas}\label{notas-1}}
\addcontentsline{toc}{section}{Notas}

\begin{enumerate}
\def\labelenumi{\arabic{enumi}.}
\item
  Baran y Sweezy (1966, pag. 3).
\item
  Para citar el título de un libro de RJB Bosworth.
\item
  Fenoaltea (2011, pag. 136).
\item
  Fenoaltea (2011, págs. 136-137).
\item
  Rocco y Carli (1914, págs. 29-32). Las ideas son casi textualmente las mismas que las expresó Rathenau (véase el capítulo II).
\item
  Rocco y Carli, págs. 24-25.
\item
  Rocco y Carli, p.~27. Para conocer la influencia de List en Rocco, véase Gregor (2005, págs. 43-48).
\item
  Rocco y Carli, pág. 6.
\item
  D'Alfonso (2004, págs. 124-127).
\item
  Rocco y Carli, pág. 5.
\item
  Rocco y Carli, págs. 49--51.
\item
  ``La tierra baldía, para ser explotada, es una fantasía alegre de nuestros liberales y socialistas''.
\item
  D'Alfonso, págs.131 y 136.
\item
  Gregor, pág. 117.
\item
  Gregor, pág. 119.
\end{enumerate}

dieciséis.
Spirito1939, pag. 99).

\begin{enumerate}
\def\labelenumi{\arabic{enumi}.}
\setcounter{enumi}{16}
\item
  Gregor, págs. 131-133.
\item
  Conti (1986, pag. 431).
\item
  Rocco y Carli, págs. 56--57.
\item
  Informe ministerial a la Cámara de Diputados sobre el proyecto de ley 1926/563, relacionado con la disciplina de los contratos colectivos (en 1939 había alrededor de 8500).
\item
  Como se informó en De Felice (1968, págs. 542-547).
\item
  Sin embargo, las corporaciones fueron creadas solo por una ley de 1934.
\item
  Esta es la interpretación que da Papi (1958, vol.~Yo, p.~457).
\item
  Negri Zamagni (2019).
\item
  Toniolo1980).
\item
  Ciocca2007, pag. 203).
\item
  Guerin1939, pag. 28).
\item
  Ciocca, pág. 223.
\item
  Sylos Labini (1975).
\item
  Kalecki (1943).
\item
  pag. 425.
\item
  Spirito1933, págs.97 y 101) y Gregor (2005, capítulo seis).
\item
  Ciocca, pág. 215.
\item
  Paxton2004, pag. 122).
\item
  Merlini1995, pag. 48).
\item
  Galli2010, pag. 12).
\item
  Veremos esta inspiración cristiana también en Polanyi (Sec. 3.7 de este capítulo).
\item
  Webb (1944, págs. XXXVII y L).
\item
  Russell (1920, pag. 90).
\item
  Schumpeter (1947, pag. 32). Esa distinción de roles se enfatiza en el libro como una razón del colapso del capitalismo (ver Sección 3.7 ).
\item
  Baran y Sweezy (1966, págs.3 y 4). El libro tiene un epígrafe: La verdad es el todo (Hegel). Casi para enfatizar, si es necesario, el origen filosófico hegeliano de la doctrina marxista, y su lenguaje a veces ``oscuro'' (Schumpeter).
\item
  Baran y Sweezy, pág. 56.
\item
  pag. 9.
\item
  Capítulo 8.
\item
  Schumpeter (1947).
\item
  Streeck (2016, págs. 2-3).
\item
  Lange1959, pag. 1).
\item
  Sin embargo, Lenin estaba convencido de que lo que Marx había dicho en su ensayo sobre los aspectos constitucionales y políticos de la Comuna de París de 1871 se aplicaba con igual verdad al Soviet ruso de 1917. Véase Webb (1944, pag. 9).
\item
  Manso1964, pag. 95).
\item
  Carr (1958, vol.~1, págs. 21-22).
\item
  ``El proceso de degeneración del ideal puro tomó formas específicamente rusas en un contexto ruso \ldots{} este proceso, sutil y no declarado, estaba muy avanzado cuando Stalin propuso por primera vez la doctrina híbrida del `socialismo en un solo país'\,'' (Carr, ibid.) .
\item
  Schlesinger1947, pag. 10).
\item
  Schwartz1968, pag. 2).
\item
  Tomamos 1913 porque el nivel del PIB no está disponible para 1917, el año de la Revolución.
\item
  Maddison2003, Tablas 2b y 3b).
\item
  Ver Harrison (2017) y Schlesinger (1947).
\item
  Maddison, Tabla 3b.
\item
  Sin embargo, Ucrania se vio afectada por el hambre y la represión soviética de los intelectuales disidentes.
\item
  Schwartz1968, pag. 26).
\item
  Stalin1972).
\item
  pag. 13.
\item
  pag. 15.
\item
  Dobb1937a, pag. 270).
\item
  Dobb1937a, pag. 271).
\end{enumerate}

sesenta y cinco.
Dobb1937a, pag. 338).

\begin{enumerate}
\def\labelenumi{\arabic{enumi}.}
\setcounter{enumi}{65}
\item
  Lange1959).
\item
  pag. 2.
\item
  Marx (nd {[}1867{]}, vol.~I, Parte I, págs.50, 58, 94, 251).
\item
  Lange, pág. 9.
\item
  Manso1964).
\item
  Grillete (1963, pag. 194).
\item
  ``El mercado está encarnado institucionalmente en la actual economía socialista''. Ver Lange (1967, pag. 160).
\item
  Lange, pág. 161.
\item
  Robinson1967, pag. 176).
\item
  págs. 176-178.
\item
  págs. 178-181.
\item
  Kaser (1971, pag. 254). La experiencia de la planificación nacional en algunas de las principales economías europeas fue negativa.
\item
  Dobb1937b, p.131) .El tema de las ``leyes'' económicas, como caracterización de la economía como ``ciencia'', ha sido objeto de debate recurrente, siempre que existen dificultades para ajustarlas a la realidad económica. Cuando esto sucede, lo que era una ``ley'' se degrada a ``regularidad''. Ejemplos notables de leyes económicas defectuosas son la ``distribución invariable del ingreso'' (Pareto, capítulo I) y la ``curva de Phillips'' (que mencionaremos en el capítulo 4). 30 años después de la escritura de Dobb, Axel Leijonhufvud, observó que ``la distinción nítida \ldots{} es uno de los dispositivos que utilizan los economistas para efectuar una separación clara de la economía de las otras disciplinas de las ciencias sociales y para poner los problemas de esta última en el ceteris paribus basurero''. Agrega que esta es una distinción perniciosa si consideramos su artificialidad, y que es sólo gracias a la ``abstracción drástica'' de la economía pura de la totalidad de otras ciencias sociales que los economistas han podido ir ``muy por delante'' de esas ciencias sociales en la construcción teórica. Solo esta abstracción y los paradigmas compartidos han permitido que la economía no hierva a fuego lento en discusiones y conflictos interminables (1968, págs. 233-234).
\item
  Dobb1937b, pag. 132).
\item
  Marx (sin fecha {[}1867{]}), pág. 57
\item
  La referencia implícita de Dodd es a la Voluntad General de Rousseau, ver capítulo I. Es cuestionable que el pensamiento de Rousseau pueda ser visto como la ``teoría tradicional de la política y del Estado''.
\item
  Dobb1937b, págs.177-178.)
\item
  Dobb, pág. 182.
\item
  Lo que Keynes quería ``derribar''. Carta a GB Shaw, 1 de enero de 1935 (1973, págs. 492-493).
\item
  Robinson1972, pag. 200).
\item
  Sraffa1960).
\item
  Sraffa1951, 1952, 1953, 1954, 1955: vol.~I -- X y 1973: vol.~XI -- índices).
\item
  Roncaglia2009, pag. 453).
\item
  Aparte del primero, que impregna todo el libro, estos supuestos se encuentran en el Prefacio del libro.
\item
  Sraffa escribe que la identificación del concepto de valor con la utilidad marginal es ``notoriamente una invención de economistas burgueses, posmarxistas y antimarxistas'' (2017, pag. 3).
\item
  Roy Harrod escribió una crítica favorable al libro de Sraffa, solo quejándose de que en su texto no hay ninguna referencia a la ``demanda'', según la tradición clásica ricardiana (1961, pag. 783).
\item
  A diferencia del ``capital constante'', la maquinaria.
\item
  Sraffa1960, pag. 17). Véase también Meek (1961).
\item
  Sraffa ofrece, por ejemplo, la tasa uniforme de ganancia sobre la industria de una manera que es lógicamente rigurosa, pero históricamente sorda. Ver Napoleoni (1963, pag. 201).
\item
  Ver Naldi (2000).
\item
  Sraffa2017).
\item
  Ver Napoleoni (1963, págs. 194-201); más recientemente, Mazzucato (2018, pag. 70).
\item
  Teoría de los sentimientos morales , pág. 331.
\item
  Heibroner y Milberg (1995, pág.118).
\item
  Schumpeter (1947, pag. 112).
\item
  Como podemos ver, la incapacidad de captar las estructuras en evolución del capitalismo fue ---es--- una crítica recurrente del pensamiento marxista.
\item
  La intervención de Keynes en The New Statement and Nation (1934, págs. 34-35).
\item
  Beveridge (1944, pag. 206).
\item
  Von Mises (1951, pag. 48).
\item
  pag. 13.
\item
  pag. 51.
\item
  pag. 76.
\item
  pag. 29.
\item
  pag. 120.
\item
  Weber1930). Véase el prefacio de Tawney, RH, pág. 1b. Tawney es el autor de Religion and the Rise of Capitalism , John Murray, 1926.
\item
  Weber, pág. 54.
\item
  Tawney, Prefacio , pág. 2.
\item
  Weber, pág. 52.
\item
  Weber, pág. 179.
\item
  Weber, págs. 302-303.
\item
  Weber estaba ``consternado'' por el capitalismo moderno (Gregory {[}2012, pag. 241{]}).
\item
  Weber, pág. 181.
\item
  pag. 182.
\item
  Schumpeter (1954, pag. 80).
\item
  Schumpeter (1947). Ver en particular la Parte II, ``¿Puede sobrevivir el capitalismo?''
\item
  No nos ocuparemos aquí de las Partes III y V de su libro, donde Schumpeter es crítico del socialismo mismo, que corre el riesgo de dañar las conquistas del capitalismo y las libertades de las democracias liberales.
\item
  pag. 67.
\item
  pag. 78.
\item
  págs. 82--83.
\item
  pag. 130.
\item
  Capítulo XIV.
\item
  pag. 156.
\item
  pag. 132.
\item
  pag. 133.
\item
  pag. 131.
\item
  pag. 121.
\item
  pag. 134.
\item
  pag. 158.
\item
  pag. 137.
\item
  pag. 158.
\item
  pag. 159.
\item
  pag. 160.
\item
  pag. 143.
\item
  pag. 150.
\item
  pag. 162.
\item
  Polanyi1957).
\item
  Polanyi, pág. 126.
\item
  Polanyi, pág. 3.
\item
  ``La extraterritorialidad metafísica de una dinastía de banqueros judíos domiciliados en las capitales de Europa'', p.~10.
\item
  pag. 68.
\item
  pag. 71.
\item
  pag. 103.
\item
  Heilbroner1988, págs. 17-18).
\item
  pag. 258.
\item
  pag. 234.
\item
  Heilbroner, pág. 18.
\end{enumerate}

\hypertarget{part-entender-el-neoliberalismo}{%
\part{Entender el neoliberalismo}\label{part-entender-el-neoliberalismo}}

\hypertarget{neoliberalismo}{%
\chapter*{Neoliberalismo}\label{neoliberalismo}}
\addcontentsline{toc}{chapter}{Neoliberalismo}

El Estado de Bienestar que surge de la Segunda Guerra Mundial marca el predominio de la economía keynesiana y las ideas de Beveridge. Sin embargo, el consenso keynesiano en economía adolece de algunos desarrollos económicos y sociales y de un ataque al marco analítico de Keynes. En un entorno en el que las preocupaciones inflacionarias ocupan el lugar del desempleo masivo como tema central para los políticos y los economistas, una contrarrevolución monetarista, una atención renovada por el comportamiento individual y una postura no intervencionista caracterizan la larga fase del neoliberalismo. Encuentra apoyo intelectual en la filosofía económica de James Buchanan y, en el campo de la economía, en los esquemas teóricos de las ``expectativas racionales'' y las hipótesis del ``mercado eficiente''. La confianza en este libertario, El enfoque individualista se ve seriamente afectado por el colapso financiero de principios del siglo XXI y la Gran Recesión. Esta larga crisis proporciona el terreno para el dramático aumento del populismo, cuyas ideologías y posturas económicas ---alimentadas por el dramático impacto de las redes sociales--- son inciertas, más allá de un nacionalismo genérico y un llamado a la intervención estatal.

Palabras clave

\begin{itemize}
\tightlist
\item
  Neoliberalismo
\item
  Monetarismo
\item
  Expectativas racionales e hipótesis de mercado eficientes
\item
  Populismo
\end{itemize}

\hypertarget{la-situaciuxf3n-cluxe1sica-de-keynes}{%
\section*{La ``situación clásica'' de Keynes}\label{la-situaciuxf3n-cluxe1sica-de-keynes}}
\addcontentsline{toc}{section}{La ``situación clásica'' de Keynes}

¿Qué tipo de liberalismo estaba emergiendo de la Segunda Guerra Mundial? Ni las ideas individualistas, radicalmente libertarias presentadas por Hayek, ni las teorías de la Escuela de Chicago, basadas en reglas que limitarían la discreción de la autoridad en la gestión monetaria. Fue más bien una visión keynesiana, que fue incluso más allá del campo de la economía, para apoyar un enfoque que favoreciera un papel importante del Estado en la vida social del país. El Estado de Bienestar se estaba estableciendo, de hecho apoyándose más en el diseño de Beveridge que en la macroeconomía de Keynes.

Esta amplia configuración tomó diferentes formas, según la peculiar situación en la que los países habían salido de la guerra y sus experiencias previas. Por ejemplo, en Gran Bretaña, el gobierno laborista de los años inmediatos de la posguerra tomó medidas para implementar el Estado de Bienestar según las líneas trazadas por su creador, en un clima político inclinado hacia puntos de vista benignos de los desarrollos socialistas en otros lugares. La economía británica se gestionó en gran medida mediante controles administrativos. Si bien esta política pudiera explicarse como la adopción de dispositivos necesarios pero temporales en las difíciles circunstancias de la posguerra, en la profesión económica algunos pensaron que los controles deberían ser una característica permanente de la gestión macroeconómica.

En Italia, después de la caída del fascismo y la derrota en la guerra, la reconstrucción económica pudo llevarse a cabo en un mercado libre; pero al mismo tiempo, la presencia del Estado cubría gran parte de la economía y se siguieron aplicando políticas dirigistas, continuación de las prácticas establecidas en la preguerra (ver Capítulo 3 ). En la profesión económica, la investigación y la enseñanza continuaron basándose en esquemas neoclásicos, a pesar de que los mayores todavía estaban influenciados por el corporativismo y solo los más jóvenes fueron seducidos intelectualmente por Keynes. En Francia, los economistas se dedicaron a la planificación indicativa. Los economistas y políticos alemanes se basaron en gran medida en esa forma peculiar de liberalismo económico que es el ordoliberalismo (Capítulo 2 ).

Es importante notar que en algunos países importantes, grandes sectores del público y de las élites intelectuales se inclinaban hacia formas extremas de socialismo, todavía atraídos por las experiencias de la Unión Soviética. El lanzamiento del Sputnik, en 1957, fue considerado en Occidente como evidencia de los avances tecnológicos y la superioridad científica (en particular vis - a - vis los Estados Unidos), y poderoso era el señuelo de una sociedad aparentemente más igualitaria y bien educados, mientras que sólo lentamente, las deficiencias económicas y las espantosas consecuencias sociales del régimen soviético (véase el capítulo 3 ) se hicieron evidentes para los observadores externos.

Aún fuerte como una forma de lucha política en varias democracias liberales (los partidos comunistas todavía tomaban una gran parte del electorado durante la década de 1970), el atractivo ideológico del socialismo se volvió, sin embargo, más tenue. De hecho, el socialismo marxista desapareció gradualmente como una ideología política y económica alternativa, mientras que los logros de una sociedad liberal libre, en términos de innovaciones tecnológicas, crecimiento económico espectacular, niveles de vida en aumento y un gobierno limitado, fueron evidencia de que la Weltanschauung liberal estaba prevaleciendo. sobre el nacionalismo rígido o el marxismo.

En la disciplina económica, los estudios marxistas prosperaron durante un tiempo, también gracias a importantes y brillantes economistas, a los que hemos mencionado en el capítulo 3 , pero, paralelamente al declive económico de la Unión Soviética, los economistas marxistas tendieron a encontrar una vía de escape. , abandonando las premisas ideológicas marxistas, y viendo la economía como una especie de ciencia neutral de la gestión económica, de la ingeniería social, que al final aproximaría los esquemas de las economías socialistas avanzadas a los de las economías de libre mercado. El triunfo del liberalismo significó que la disciplina económica se identificara cada vez más como el estudio del capitalismo. 1 Esta economía dominante tendió a desalojar formas alternativas de pensamiento económico. Fue reforzado por los desarrollos políticos y económicos de las últimas décadas del siglo XX. La posibilidad de un colapso estructural del orden capitalista parecía remota, mientras que, como ha escrito un pensador de izquierda, la producción y acumulación ilimitadas de capital productivo confirmaba que el mercado podía asegurar la conversión del ``vicio privado de la codicia material en un vicio público''. beneficio '', y los trabajadores inseguros del sistema de libre mercado se convirtieron en consumidores confiados y prósperos,`` incluso frente a la incertidumbre fundamental de los mercados laborales y el empleo ''. 2

En el lado opuesto del debate, la ideología nacionalista, desacreditada por la derrota de los principales países que habían perseguido su versión más autoritaria, sobrevivió (a veces mezclada con un enfoque marxista: tal vez una evidencia de su origen estatista común) como un medio político y económico. ideología, pero en general se veía desde una perspectiva limitada: como una posible fuerza impulsora de los países subdesarrollados, a menudo de nueva creación, para acelerar su desarrollo económico. Los economistas de izquierda dedicaron muchas reflexiones a las economías de los países subdesarrollados. Esta perspectiva nacionalista y de inspiración social se caracterizó por una preferencia ampliamente compartida por la empresa pública o estatal sobre la privada; por políticas destinadas a sustituir la producción nacional por importaciones, y por la negativa o reticencia a admitir inversiones de capital extranjero, a menos que pudiera ser controlado completamente por el gobierno nacional. La visión liberal, según la cual la confianza en un precio de mercado competitivo sería preferible al control y la propiedad del gobierno; que la eficiencia económica estaría mejor servida por el principio de la ventaja comparativa ricardiana; y que el capital interno en los países subdesarrollados era escaso y, por lo tanto, un cuello de botella para el crecimiento, fue generalmente descartado con argumentos que, al final, encontrarían su principal apoyo en la protección de Friedrich List.Sistema de Economía Política , si no en el Capital de Marx . 3 De hecho, el nacionalismo se caracterizaría por políticas, a menudo mal concebidas, principalmente relacionadas con la redistribución de los ingresos en lugar de aumentarlos, y por un énfasis en lo que se denomina ingresos ``psíquicos'', en forma de orgullo por la nación, o incluso peor, en un grupo étnico, incluso en detrimento de los ingresos materiales. 4 Algunas de estas características las encontraremos en el resurgimiento del nacionalismo en nuestro tiempo, en países económicamente avanzados, como se verá en la secc. 4.6 .

La perspectiva liberal generalizada, reforzada por cambios políticos bien conocidos y radicales que ocurrieron en algunos países importantes, particularmente en los Estados Unidos y el Reino Unido, recibió una bendición no oficial pero muy bien publicitada de un ensayo demasiado famoso de Francis Fukuyama. 5 Esto afirmaba que el ``agotamiento total de las alternativas sistemáticas viables'' 6 era evidencia del triunfo del liberalismo occidental. Lo que es algo intrigante en los escritos de Fukuyama es que enmarca su visión en una perspectiva hegeliana mal concebida, 7 es decir, en las obras de un autor que hizo que todo el curso de la historia dependiera de la idea del Estado (Capítulo 1). ), esa misma idea que, como liberal, Fukuyama mira con sospecha, por decir lo menos.

Curiosamente, sin embargo, mientras Fukuyama ve las ideas del fascismo, centradas en un Estado fuerte que forja nuevas personas sobre la base de la exclusividad nacional, como una promesa de conflictos interminables y derrotas desastrosas, no se reserva una opinión negativa similar para el nacionalismo, según se : ``no está claro que el nacionalismo represente una contradicción irreconciliable en el corazón del liberalismo \ldots{} Sólo los nacionalismos sistemáticos de este último tipo {[}como el nacionalsocialismo, el fascismo{]} pueden calificar como una ideología formal en el nivel del liberalismo o el comunismo \ldots{} movimientos nacionalistas no tienen un programa político más allá del deseo negativo de independencia de algunos otros grupos de personas y no ofrecen nada como una agenda integral para la organización socioeconómica'' 8: esto suena plausible, a la luz del resurgimiento actual del nacionalismo (en la Sección 4.6 de este capítulo se presenta una interpretación más sombría del ensayo de Fukuyama).

En cuanto al liberalismo, sin embargo, debe hacerse una distinción entre diferentes puntos de vista dentro de ese amplio lapso de tiempo que abarca alrededor de 60 años de historia política y económica, entre la Segunda Guerra Mundial y el estallido de la crisis financiera a principios de nuestro siglo: bajo el amplias alas del liberalismo económico, se han producido grandes cambios ideológicos. Esta distinción, quizás aún poco clara cuando Fukuyama escribió su ensayo (1989), es importante, porque estos cambios abrieron el camino a desarrollos muy difíciles ---en términos de crecimiento económico, desigualdad social y estabilidad financiera--- a principios del siglo XXI. Estos desarrollos, a su vez, han planteado nuevas preguntas, aún sin respuesta.

Como se mencionó al comienzo de este capítulo, en las décadas inmediatas de la posguerra, dentro de la visión liberal, se consideró que el enfoque de Keynes prevalecía y marcaba un punto de partida de otras corrientes de pensamiento generalmente más libertarias. El pensamiento de Keynes, como hemos observado en el capítulo 2 , se consideraba a veces inclinado hacia el socialismo, una especie de liberalismo social que incluía una economía de mercado gestionada, el objetivo del pleno empleo y una sociedad del bienestar; pero mantuvo las principales características individualistas de una sociedad capitalista libre: de hecho, Keynes pensó que su teoría general señalaría el camino hacia la supervivencia misma del capitalismo, después de la Gran Depresión. A pesar de las interminables revoluciones y contrarrevoluciones que han agitado la disciplina económica después de Keynes, el pensamiento keynesiano, en su sentido más amplio, a veces se representa como el último punto de referencia, o modelo estándar, que por un tiempo mantuvo un amplio consenso no solo dentro de la profesión económica, pero también en cualquier sociedad inspirada por los valores democráticos liberales.

El consenso keynesiano representó el tipo de acuerdo que Schumpeter llama ``situación clásica''. 9 No fue el primero, y seguro que no será el último caso de una situación clásica. Schumpeter no da una definición formal de esta expresión, aunque es recurrente en su Historia del análisis económico., pero su clara referencia es a la consecución de un acuerdo sustancial y ampliamente compartido en el pensamiento económico (en este sentido, podríamos hablar de un consenso ``clásico'' con referencia a cualquier disciplina): un acuerdo que se alcanza, de manera más o menos amplia , después de un período de lucha y controversia en la profesión económica y, quizás más ampliamente, en la forma en que la sociedad ve los asuntos económicos. Schumpeter escribe que una situación clásica típica se caracteriza por el hecho de que las ``obras principales'' que le dan su carácter, exhiben ``una gran extensión de terreno común'' y sugieren ``un sentimiento de reposo''. Este consenso bien establecido crea, ``en el observador superficial, una impresión de finalidad, la finalidad de un templo griego que extiende sus líneas perfectas contra un cielo despejado''. 10 Lo que en este ensayo preferimos nombrar una visión, una ideología -como el punto de referencia explícito o implícito del economista que elabora su propio esquema teórico- se convierte en una ``situación clásica'' cuando se alcanza un consenso suficientemente extendido no solo sobre su construcción teórica, pero sobre todo sobre sus premisas ideológicas.

La complacencia con la que se puede mirar una situación clásica no puede llevarnos a pensar que las ideas y construcciones teóricas que están en la base del ``templo'' sean realmente ``finales'', desvinculadas de situaciones específicas definidas en términos de tiempo y lugar (esto es probablemente el factor principal que separa una situación clásica en las ciencias morales del mismo tipo de situación en las ciencias físicas).

Antes de Keynes, este tipo de situaciones clásicas surgieron, por ejemplo, con la Escuela Clásica de finales del siglo XVIII y principios del XIX, y luego con la teorización sumamente abstracta en torno a la utilidad marginal, aparentemente indiferente a las consideraciones sociales y políticas, de los pensadores neoclásicos de finales del siglo XIX y posteriores: un período en el que se soñaba con los equilibrios más perfectos de la mecánica social. 11

De hecho, en una perspectiva más amplia del cambio histórico, toda situación clásica, en el campo de la economía, tiene en su interior las semillas de su propia evolución hacia algo diferente (signos de decadencia, rupturas a la vista, escribe Schumpeter): una nueva situación clásica que altera radicalmente los objetivos y métodos de investigación previos. 12

Lo que interesa ahora es ver:

A. qué ``signos de decadencia'' o ``rupturas a la vista'' fueron visibles, para observadores posteriores, en la Teoría General de Keynes ;
B. qué nuevas actitudes filosóficas, y revoluciones o contrarrevoluciones en la disciplina de la economía, surgieron de ellas;
C. si una nueva ``situación clásica'' resultó de estos movimientos, que caracterizaron las últimas décadas del siglo XX.

Cada uno de estos tres puntos se desarrolla en las secciones siguientes. En términos muy generales, vale la pena recalcar nuevamente que el liberalismo ocupó todo el territorio de la teorización política, institucional y económica, pero emergió una postura libertaria más aguda, apoyada cada vez más en el egoísmo individual. Esta visión fue más amplia que el campo de la filosofía y el pensamiento económicos, extendiéndose al cuerpo político y a la vida social.

\hypertarget{desapariciuxf3n-del-consenso-keynesiano}{%
\section*{Desaparición del consenso keynesiano}\label{desapariciuxf3n-del-consenso-keynesiano}}
\addcontentsline{toc}{section}{Desaparición del consenso keynesiano}

En cuanto a las ``rupturas en el horizonte'' que se detectaron en la obra de Keynes, la crítica a Keynes se deriva en parte de algunos desarrollos de carácter económico y social, ocurridos en el período posterior a la Segunda Guerra Mundial, y en parte de un atentado sobre el análisis de Keynes, al principio dirigido específicamente contra el papel de la inflación en la Teoría General de Keynes , y luego contra los ``escombros'' de ``ese notable evento intelectual llamado Revolución Keynesiana''. 13

Sobre el primer punto, se deben mencionar algunos avances en las finanzas públicas, las relaciones laborales y el activismo gubernamental:

\begin{quote}
\begin{itemize}
\item
  un papel más amplio de la economía del bienestar, según lo previsto por Beveridge, y la promulgación de una financiación pública ``funcional'' (véase el capítulo 2 ): generalmente iban acompañadas de una fuerte expansión de los presupuestos gubernamentales y del gasto deficitario. Estos desarrollos tuvieron un impacto profundo en la economía: no realmente keynesiano, porque Keynes preconizaba el gasto deficitario en una situación de subempleo de recursos, de demanda efectiva insuficiente, como la que prevalecía en los años treinta. Keynes estaba convencido de que un equilibrio con subutilización de recursos podría ser más que una posibilidad accidental, pero argumentó que cuando hay pleno empleo ``en la medida de lo posible, la teoría clásica vuelve a cobrar vida desde este punto en adelante''. 14 En el período posterior a la Segunda Guerra Mundial, por ejemplo, en Gran Bretaña, la creencia de que el auge inmediato de la posguerra podría ser temporal y que la economía recaería en un alto desempleo persistió durante mucho tiempo, dando lugar a gastar más dinero público; además, el desempleo de los recursos siguió siendo notablemente elevado en otros países y la distribución de la riqueza fue desequilibrada. Pero el impulso a la economía del bienestar a veces superó la sostenibilidad financiera, con un alto componente de subsidio e incentivos perversos. La influencia de la política de partidos no fue, a este respecto, despreciable;
\item
  El hábito de ``concertación'' en la contratación de salarios y otras condiciones de trabajo, entre empleadores y empleados, a menudo negociado por el gobierno: este poder de negociación más fuerte del trabajo contribuyó a la creación de un entorno de trabajo, tanto en términos de niveles salariales como de tasas de empleo. lo cual, como se acaba de observar, estaba bastante lejos de las duraderas condiciones estresantes del mercado laboral del período anterior a la guerra. Estas relaciones laborales, que podrían recordar el corporativismo italiano de la era fascista, en una especie de connivencia del capital con el trabajo, se hicieron frecuentes, sobre todo en Italia (bajo la dirección de gobiernos de los partidos democristiano / socialistas) pero también en otros lugares (en Gran Bretaña). esta política ya era una característica de la administración de Churchill a principios de la década de 1950,15 );
\item
  un creciente activismo de los gobiernos, particularmente en Europa, que no debe confundirse con las políticas macroeconómicas del estado de bienestar, mencionado anteriormente: este activismo tomó diferentes formas, por ejemplo, a través de la ``economía social de mercado'', inspirada en el credo ordoliberal, como en Alemania (Capítulo 2 ), o la ``economía mixta'' en Italia, donde las instituciones cuasi gubernamentales creadas bajo el fascismo continuaron sobreviviendo y prosperando en el período de posguerra. Este activismo gubernamental incorporó en realidad las tendencias dirigistas y tecnocráticas ya bien visibles en el régimen fascista y fue, de hecho, una de las fuerzas impulsoras de la recuperación económica de la posguerra (Capítulo 3). ). Este activismo, que sólo vagamente podría estar asociado con el intervencionismo keynesiano, pareció a muchos observadores libertarios una violación obvia del núcleo mismo del liberalismo, donde los adjetivos ``social'' o ``mixto'' coexistirían incómodamente con el sustantivo ``mercado''.
\end{itemize}
\end{quote}

En cuanto al segundo punto, el ataque teórico a Keynes, desde un punto de vista analítico, la inflación no había sido un punto central en la Teoría General de Keynes . La inflación rara vez se menciona en ese trabajo. La mención más relevante de la inflación es probablemente cuando señala una aparente asimetría entre los efectos de la deflación y la inflación sobre la demanda efectiva: ``Mientras que una deflación de la demanda efectiva por debajo del nivel requerido para el pleno empleo disminuirá el empleo y precios, una inflación de la misma {[}de la demanda efectiva{]} por encima de este nivel simplemente afectará a los precios''. 16 Pensaba en las deprimidas condiciones comerciales de la década de 1930 y en un régimen monetario lo suficientemente estable como para ignorar la inflación como una fuente específica de inestabilidad.

Después de la guerra, el componente doméstico de la inflación reflejó un régimen monetario de dinero ``fiduciario'' manipulador y discrecional, 17 afectado por los acontecimientos políticos y sociales mencionados anteriormente. Además, el repentino salto en los precios de la energía a principios de la década de 1970, evidencia de condiciones geopolíticas cambiantes, tuvo un efecto significativo sobre el componente externo de la inflación, un componente igualmente irrelevante en el contexto de la Teoría General .

De hecho, las presiones inflacionarias en la década de 1970 estuvieron acompañadas de una desaceleración del producto y el desempleo: una combinación ---la ``estanflación'' - que sería difícil de explicar en términos keynesianos.

Sin entrar en detalles sobre las doctrinas de los economistas keynesianos de la posguerra, parecía que esta falta de atención a la inflación podría ser cubierta por el modelo dado a finales de la década de 1950 por el economista AW Phillips. Vio una tendencia a largo plazo de la tasa salarial (y, por inferencia, del nivel de precios) a estar relacionada, en una relación estable e inversa, con la tasa de desempleo. La ``curva de Phillips'' no estaba en la Teoría General de Keynes , y fue introducida ``no sin oposición por algunos keynesianos''. 18Y parecía contradecir la situación de principios de la década de 1970, cuando, como se acaba de mencionar, la estanflación significaba alta inflación y alto desempleo, en un contexto de caída o estancamiento del producto. Esa correlación, incluso si aparentemente está respaldada por evidencia estadística histórica, no podría demostrarse en el contexto actual. Fue desacreditado académicamente, como un ``tipo de ley'' o ``regularidad'' que no podía resistir la evidencia empírica. 19 Puede haber argumentos para apoyar la tesis de Phillips pero, más allá de los debates académicos, lo más relevante fue el cambiante clima político e intelectual de esos años.

Un análisis cáustico pero agudo del economista canadiense Harry Johnson ---un análisis que, si alguna vez fuera necesario implementarlo, sería suficiente para dar evidencia de la mezcla de ideología, teoría y circunstancias sociales y económicas cambiantes--- muestra cómo ``signos de decadencia''y`` rupturas a la vista ''en la`` situación clásica ''keynesiana fueron explotadas por una visión y construcción teórica diferente. 20

Según Johnson, los factores que explicaron el éxito de la Teoría General pueden atribuirse, por un lado, a la existencia de un importante problema social y económico ---el desempleo y la depresión--- que la ortodoxia anterior (la economía neoclásica) había tenido. no ha podido resolver, evidenciando en cambio una confusión general y una evidente irrelevancia; por el otro, a su relevancia social superior y distinción intelectual (apelando a la iconoclasia juvenil de las generaciones más jóvenes de economistas), aunque Keynes incorporó de manera novedosa y confusa algunos elementos válidos de la teoría tradicional. 21

Johnson observa que la revolución de Keynes se convirtió en una ortodoxia establecida, una ``situación clásica'', podemos decir, principalmente a través del trabajo de sus seguidores. Ellos (lo que significa la profesión en general) elaboraron el análisis de Keynes, desarrollado en un conjunto dado de circunstancias históricas, en ``un conjunto atemporal y sin espacio de principios universales 22 \ldots{} y así establecieron el keynesianismo como una ortodoxia {[}en sí{]} lista para el contraataque''. 23

No es de extrañar que en un entorno en el que las preocupaciones inflacionarias reemplazaran al desempleo masivo como tema central para los responsables de la formulación de políticas y los economistas, 24 no sería inesperado un resurgimiento del interés por el dinero. La crítica de la Teoría General estimuló inicialmente una renovada atención a las teorías alternativas ya establecidas y, luego, a nuevas investigaciones basadas en el comportamiento del individuo. Ambos desarrollos dan evidencia adicional de que la teorización económica nunca es realmente ``final'' y permanece inevitablemente conectada a condiciones sociales e intelectuales específicas que pueden prevalecer en ciertos momentos y lugares. Sin embargo, la ``contrarrevolución monetarista'' y las teorías posteriores (que en general se denominan Nueva Economía Clásica: véase la sección 4.4) .), no eran solo una cuestión de renovado interés científico en el comportamiento de los agregados monetarios y en las elecciones racionales de los individuos; también fueron evidencia de un cambio de la ideología económica predominante contra la ortodoxia keynesiana.

La opinión ampliamente aceptada es que este poderoso cambio intelectual puede etiquetarse con el título de neoliberalismo. Se ha proporcionado una definición amplia y aceptable de este cambio: ``El neoliberalismo es, en primera instancia, una teoría de las prácticas económicas políticas que propone que el bienestar humano se puede promover mejor liberando las libertades y habilidades empresariales individuales dentro de un marco institucional caracterizado por fuertes derechos de propiedad privada, mercados libres y libre comercio. El papel del Estado es crear y preservar un marco institucional adecuado a tales prácticas''. 25 Por lo tanto, el neoliberalismo puede ser visto como un intento-cierto, no es la primera en la historia política y económica a revertir el avance del Estado en nuestra vida diaria.

En el aspecto económico, una especificación importante, si no particularmente nueva, es que el neoliberalismo no solo se caracteriza por una postura no intervencionista, sino que se basa en la presunción de que a cada agente solo le importa su utilidad y no le importa la utilidad de otros. Esto se expresa como una afirmación ``positiva'', una afirmación ``científica''. Frente a esta afirmación, el sentimiento de ``simpatía'', al que me refiero también como ``confianza'', de Adam Smith, que es la base de un sistema de libre mercado que funcione bien, se convierte en una ilusión.

El neoliberalismo tiene varias implicaciones, a veces subsumidas bajo la expresión ``fundamentalismo de mercado'': \hspace{0pt}\hspace{0pt}en las relaciones laborales entre empleador y empleado, se pone en el mismo nivel su respectiva fuerza contractual; en cuanto a la organización del mercado, el neoliberalismo refuerza un concepto darwiniano de predominio de los más aptos y más eficientes, hasta el extremo de anular el mismo concepto de mercado y la emergencia paradójica de posiciones de renta; las relaciones comerciales internacionales se rigen por un globalismo que niega formas de protección nacional o regional bajo cualquier circunstancia; Las políticas fiscales y monetarias tienen que ser coherentes con (o restringidas por) reglas que implican su neutralidad sustancial con respecto al buen funcionamiento de los mercados sin obstáculos.

Refiriéndose a los Estados Unidos, Paul Samuelson, en 1997, caracterizó su economía como la ``Economía implacable'' y su trabajo como una ``Fuerza laboral acobardada''. La primera característica, marcada por una retirada de un Estado de bienestar ilimitado, contaba la ``misma historia'', aplicada a la América posterior a Reagan, o al ``caso extremo'' de la británica Margaret Thatcher, oa la mayor parte de Europa occidental, Escandinavia, o Australia. La segunda característica, el mercado laboral acobardado, se refirió a una fuerza laboral intimidada como una indicación de cuán ``se han vuelto los receptores de ingresos inciertos'', pero también a que ``los empleadores también corren en la misma carrera. La competencia despiadada, que demanda constantemente, `¿Qué has hecho por mí últimamente?', Es lo que nos pone a todos en una especie de ansiedad visceral''. 26

Las siguientes dos secciones están dedicadas al lado político y al lado económico del neoliberalismo.

\hypertarget{actitudes-filosuxf3ficas-constitucionalismo-econuxf3mico}{%
\section*{Actitudes filosóficas: constitucionalismo económico}\label{actitudes-filosuxf3ficas-constitucionalismo-econuxf3mico}}
\addcontentsline{toc}{section}{Actitudes filosóficas: constitucionalismo económico}

El primer aspecto a considerar, para explicar el alejamiento de la situación clásica keynesiana, es el poderoso y penetrante cambio intelectual que, en la segunda mitad del siglo XX, afectó la teorización política sobre el gobierno y la economía. Fue un largo hilo de pensamiento que se desarrolló en continuidad con Hayek y los ``austriacos'', es decir, ese grupo de economistas, muchos de origen o ascendencia centroeuropea, 27 que afirmaron la necesidad de preservar ---o volver a--- Estructuras sociales y económicas centradas en el individuo que opera en una economía de libre mercado, mientras se mantiene una actitud escéptica hacia la intervención del Estado, vista como un camino hacia regímenes autoritarios, paternalistas o incluso liberticidas (ver Capítulo 2 ). Este cambio dio a los sistemas económicos capitalistas una marcada postura libertaria.

Para poner al neoliberalismo en su contexto intelectual, vale la pena recordar dos corrientes de pensamiento. Ambos se desarrollaron en la segunda parte del siglo XX y es posible verlos como esquemas en competencia. A costa de una simplificación excesiva, se basan, respectivamente, en la ``simpatía'' y el ``egoísmo''. La diferencia básica es que ``los libertarios restauraron los derechos al individuo, pero no las obligaciones''. 28 ``En las versiones más extremas, el dinero {[}se convirtió{]} en la medida del bienestar, y la justicia {[}no era{]} más que eficacia''. 29 A pesar de ciertas aparentes similitudes, estos dos enfoques vieron la organización de la sociedad de formas muy diferentes, y es inmediatamente claro que el segundo punto de vista ha prevalecido.

Nos referimos aquí al contractualismo de John Rawls y al contractualismo de James Buchanan. Ambas pueden verse como teorías políticas de la legitimidad del gobierno y como teorías morales sobre el origen o el contenido legítimo de las normas morales. La legitimidad de la autoridad política se basa en el consentimiento de los gobernados. Este consentimiento consiste en un mutuo acuerdo que legitima las normas morales. 30 En las filosofías de Rawls y Buchanan, un contrato social, éticamente basado, es de hecho el vínculo básico entre los miembros de una sociedad, y es la justificación para su convivencia. De modo que ambos pertenecen al liberalismo, ampliamente definido.

Sin embargo, si entramos en el campo de la economía propiamente dicha, la relevancia efectiva del pensamiento de Buchanan ha sido ciertamente mayor. Hay dos razones para eso:

\begin{quote}
\begin{itemize}
\item
  No podríamos encontrar en Rawls una conexión con enfoques teóricos específicos en economía, su pensamiento está más en sintonía con la organización política del Estado que con su sistema económico, mientras que en Buchanan hay un sequitur evidente del pensamiento político a las consecuencias económicas. . 31 Y su pensamiento político y económico está más en sintonía con las circunstancias imperantes en su tiempo.
\item
  Pero, lo que es más importante, el modelo de sociedad liberal que podría sugerir Rawls estaba sufriendo la misma sensación de crisis y desaparición que, en el campo económico, caracterizaba a la economía keynesiana.
\end{itemize}
\end{quote}

En ambos casos, estamos lejos de cualquier filosofía orientada al Estado, mientras que el hombre en su individualidad es el centro de atención del filósofo. Si volvemos a las raíces filosóficas de las teorías del contrato social que son el trasfondo de ambas, nos vienen a la mente dos nombres: Immanuel Kant y Thomas Hobbes. Kant está detrás de las reflexiones de Rawls, Hobbes detrás de las de Buchanan.

Kant tiene una visión fundamentalmente benevolente del hombre actuando en sociedad, cuya ética se expresa en tres proposiciones (los ``imperativos categóricos'') que especifican su deber: (1) actuar racionalmente: cualquiera que sea el fin particular de su acción, debe querer su fin particular sólo si puede subsumirse bajo un orden universal, es decir, un orden en el que todas las acciones racionales posibles pueden converger; (2) como corolario: actuar de una manera que trate a la humanidad ---de su propia persona y de cualquier otra persona--- siempre como un fin y nunca como un medio; (3) actúa para que tu voluntad se convierta en regla universal. Por ``regla universal'', Kant quiere decir que la ley deriva de la misma razón que vive en todo hombre, que por tanto está obligado a observarla. 32

Este marco moral está presente en el pensamiento de Rawls, y es necesario comprender la base del contrato social, tal como lo concibe Rawls en su A Theory of Justice . 33 Él ve este contrato no en un sentido histórico, ni surgir de un supuesto estado primitivo de la naturaleza, sino surgir, con ``un cierto nivel de abstracción'', de una ``situación puramente hipotética''. Es una posición original de igualdad, donde los principios de la justicia se eligen detrás de un velo de ignorancia, y son el resultado de un acuerdo o negociación entre los miembros de la sociedad. Aquí está la conexión con Kant, porque Rawls ve a los individuos como personas morales, como ``seres racionales con sus propios fines'', pero también ``capaces de un sentido de justicia''. 34Rawls ve a cada persona como racional y desinteresada con respecto a todos los demás miembros de la sociedad, pero esto no significa que esa persona sea egoísta, solo significa que no se interesa por los intereses de los demás. Afirma dos principios: ``igualdad en la asignación de derechos y deberes básicos; Las desigualdades sociales y económicas, por ejemplo, la desigualdad de riqueza y autoridad, son solo si dan como resultado beneficios compensatorios para todos, y en particular para los miembros menos avanzados de la sociedad \ldots{} Puede ser conveniente, pero no es justo, que algunos deban tener menos para que otros prosperen''. 35 Estamos bastante lejos de la optimalidad de Pareto.

Por lo tanto, la racionalidad no debe considerarse ``en el sentido estricto, estándar en la teoría económica'', escribe, es decir, como tomar los medios más efectivos para los fines de uno; la racionalidad, según Rawls, está bastante lejos del concepto de utilidad, como se entiende generalmente. Dados los ``fuertes y duraderos impulsos benevolentes'' del hombre, el principio de utilidad tal como se pretende generalmente es ``incompatible con la concepción de la cooperación social entre iguales para el beneficio mutuo''. 36

El contrato social suena diferente, que acaba siendo mucho más influyente en la disciplina de la economía de su época, en el pensamiento de Buchanan. En su forma más simple y clara, un predominio de la posición individualista es expresada por Buchanan y Tullock, quienes, al inicio de su obra principal, El cálculo del consentimiento , rechazan explícitamente tanto la teoría orgánica del Estado, para la cual el Estado ha ``Una existencia, un patrón de valores y una motivación independiente de los seres humanos individuales que reclaman ser miembros'': una visión, dicen, ``opuesta a la tradición filosófica occidental'' 37; y la visión marxista, que encarna la explotación de una clase dominante gobernada, ya sea de propietarios de factores de producción o de aristocracias. Entonces, habiendo descartado tanto el nacionalismo como las teorías inspiradas en el socialismo, ``nos quedamos con un concepto de colectividad puramente individualista''. 38 Su análisis se realiza en términos de ``individualismo metodológico'', que encarna un compromiso filosófico y un juicio ético, 39 y rechazan cualquier tipo de enfoque de clase o grupo de la economía.

La economía constitucional de Buchanan (y otros) representa un poderoso intento de construir sobre el pensamiento liberal de Hayek y los austriacos, y de construir un esquema intelectual de reconciliación de los intereses individuales con los intereses de los demás individuos, que son miembros de la misma política. comunidad. Esta reconciliación no se basa en un sentido de confianza mutua, lo que podría llevarnos de regreso a Adam Smith. Es una reconciliación en la que el Estado asume un papel, que es, sin embargo, sólo el de garante. El papel del Estado es mínimo, en términos de funciones a desempeñar, pero también fundamental para permitir la plena realización de las potencialidades de las personas. La teoría de la economía constitucional de Buchanan tiene un fuerte acento institucional, como trasfondo de sus puntos de vista libertarios, y de hecho se basa en dos pilares:

Aunque los comentaristas de Buchanan han insistido a menudo en la naturaleza espontánea de un sistema de economía de mercado basado en los derechos de propiedad privada, la libertad de contrato y la estabilidad monetaria, 40 el aspecto más intrigante de la teoría de Buchanan es que su firme postura de economía de libre mercado debe reconciliarse con las limitaciones que puedan surgir como resultado del proceso político. Esta construcción teórica requiere de la estrecha colaboración de economistas, juristas y politólogos, 41 y tiene como objetivo definir lo que debe ser el Estado (en una afirmación explícita de la economía normativa versus la positiva). Los dos pilares deben examinarse en breve.

En cuanto al pilar libertario, el autor identifica la racionalidad con la utilidad subjetiva que persigue el individuo como entidad separada, una racionalidad que no debe confundirse con alguna investigación de la verdad absoluta, y una utilidad ni necesariamente de naturaleza hedonista e interesada: el hombre puede ser egoísta, altruista o una combinación de dos. La elección racional no significa que el individuo, necesariamente, haga elecciones de acuerdo con su interés económico; significa que ``el individuo autónomo es\ldots{} presume que es capaz de elegir cualquier alternativa de manera suficientemente ordenada'', en una escala de preferencias que no incluye la clasificación entre buenos y malos. 42En una especie de ponderación de las cosas elegidas entre sí, simplemente, ``el precepto central de la racionalidad establece únicamente que un individuo elige más en lugar de menos de bienes y menos en lugar de más de malos''. 43 (Sin embargo, la frontera entre lo bueno y lo malo es inexplicable).

Todo esto puede parecer superficialmente similar al pensamiento de Rawls, ambos enfocados en el individuo; pero la diferencia esencial es que, mientras Rawls, a la manera kantiana, ve una ley moral universal que se deriva de la racionalidad común de todo hombre, Buchanan ve al individuo como el único juez de su propio comportamiento moral.

Igualmente subjetivo es, en Buchanan, el concepto de costo, en relación con la utilidad racional. ``El costo es lo que el tomador de decisiones sacrifica o renuncia cuando toma una decisión. Consiste en su propia valoración del disfrute o la utilidad que anticipa tener que renunciar como resultado de la selección entre cursos de acción alternativos \ldots{} En una teoría de la elección, el costo debe contarse en una dimensión de utilidad'' 44 : un concepto notablemente similar a el ``costo de oportunidad'' de los economistas neoclásicos.

Buchanan examina el problema de la elección con referencia al individuo individual: la elección privada, ya los individuos que interactúan con otros en un grupo organizado: la elección colectiva. En ambos casos, el individuo que toma una decisión, a un costo (definido como arriba), busca maximizar su utilidad. En el caso de una elección privada, se realiza una transacción de intercambio en el mercado; mientras que en el caso de la elección colectiva, se expresa como el ejercicio del poder de voto político.

En el proceso de elección privada, el individuo hace elecciones dentro de ciertas limitaciones que se le presentan, es decir, que están determinadas exógenamente; mientras que, en el proceso de elección colectiva, la elección que hace el individuo no está dentro, sino dentro de las limitaciones: estas limitaciones están determinadas por los mismos individuos que participan en el proceso de votación. Son aceptadas por el individuo a cambio de beneficios que se anticipan frente a las mismas limitaciones impuestas a los demás miembros de su misma colectividad.

Buchanan y Tullock examinan el aspecto ético de la elección tanto privada como colectiva. En el primero, es decir, en un intercambio privado, se alcanza la máxima utilidad en un mercado con competencia perfecta; mientras que en este último, es decir, en un proceso político, se obtiene la máxima utilidad cuando se alcanza la unanimidad. (Su análisis tiene como objetivo establecer los límites de la ética cuando el intercambio privado ocurre en un mercado no competitivo, o terceros se ven afectados negativamente; y, en un proceso político, cuando las decisiones se toman por mayoría, imponer costos a los disidentes. votantes.)

En cuanto a la elección colectiva, Buchanan contrasta la ``economía ortodoxa'' {[}su referencia es principalmente a la teoría neoclásica{]} y su ``economía constitucional''. La economía ortodoxa ---escribe--- ve las limitaciones sobre las que el votante tiene que expresar su preferencia, no como resultado de su propia elección, sino de elecciones impuestas por el gobierno: el gobierno impone estas limitaciones como bienes públicos. Los votantes se convierten en cautivos intelectuales de ``filósofos políticos idealistas, que adoptan variantes de la mentalidad platónica o helena''. 45El surgimiento de la teoría macroeconómica acaba de reforzar esta actitud, centrando la atención en los macroagregados (como el PIB, el empleo, el nivel de precios\ldots), y eligiendo para ellos niveles objetivo considerados objetivamente buenos. Se ha considerado que los gobiernos pueden hacer esta elección, siguiendo el consejo de economistas o filósofos sociales, en una búsqueda idealista del bien único.

Pero, añade Buchanan, de la economía ortodoxa al socialismo el paso no es tan difícil de escalar. Citando al filósofo John Dewey, Buchanan y Tullock escriben: ``un factor significativo en el apoyo popular al socialismo \ldots{} ha sido la fe subyacente de que el cambio de una actividad del ámbito de lo privado al de la elección social implica el reemplazo del motivo de la ganancia privada por la del bien social \ldots{} En la esfera política, la búsqueda de la ganancia privada por parte del participante individual ha sido casi universalmente condenada como `mala' por los filósofos morales de cualquier tipo''. 46

La economía política constitucional toma una dirección diferente: las reglas y limitaciones institucionales no pueden delegarse. Es necesario utilizar el paradigma del intercambio individual, en contraposición a la búsqueda idealista del bien único. La elección colectiva se convierte en nada más que el comportamiento participativo de los miembros individuales. La selección de reglas o instituciones está sujeta a evaluaciones deliberativas y elecciones explícitas por parte de los miembros de la colectividad. 47 La propia definición de bienes públicos debe basarse en un enfoque voluntario. La Economía del Bienestar de Pigou, que ve al gobierno como la entidad desinteresada que corrige las fallas del mercado en nombre de un bien público (Capítulo 2 ), se invierte por completo.

Podríamos preguntarnos, al margen de la elección colectiva de Buchanan frente a las ``externalidades'' de Pigou, cómo se debe enfrentar el tema del cambio climático: solo adoptando medidas aprobadas por la elección colectiva de los individuos, o por el Estado que utilizaría ``dispositivos legales coercitivos para dirigir el interés propio a los canales sociales''?

Para pasar del pilar libertario al pilar estatal de su construcción, el sustento filosófico de Buchanan es la ``filosofía política contractualista'' de Spinoza, Locke, pero sobre todo de Hobbes, que justifica la coacción del Estado solo con el acuerdo de los sujetos sujetos. lo. Hobbes escribe: ``Autorizo \hspace{0pt}\hspace{0pt}y renuncio a mi derecho de gobernarme a mí mismo, a este Hombre, o Asamblea de hombres {[}el Leviatán{]}, con la condición de que tú {[}los otros miembros de la colectividad{]} renuncies a tu derecho sobre él y autorice todas sus acciones de la misma manera''. 48El esquema de Buchanan es similar, pero tanto el origen de la devolución de la coerción al Estado como el propósito de la coerción son diferentes. Hobbes ve la necesidad de la devolución en los instintos básicamente malos de los hombres, Buchanan parece confiar en que el hombre sea bondadoso. Además, el propósito de la coacción es, para Buchanan, deshacerse del poder absoluto del soberano, lo opuesto al Leviatán de Hobbes, por el bien de las elecciones colectivas individuales: ``{[}La{]} tradición intelectual inventó el individuo autónomo despojándose de el capullo comunitario''. 49 Los contractuales anteriores insistían en el aspecto de la coerción, dice Buchanan, porque no tenían idea de la eficacia del orden de mercado. Fue Adam Smith quien, más tarde, confiando en la eficacia del mercado, vio el papel correcto para un Estado mínimo protector. 50

En este marco intelectual, ¿cuál es el papel del economista político? 51Buchanan comienza rindiendo homenaje a la revolución positivista ya la economía ``positiva'', en contraposición a la ``normativa'', suscribiendo así el concepto de que el economista ``positivo'' mira lo que es, no lo que debería ser. En consecuencia, el lugar del economista en las cuestiones de política no puede ser otro que indirecto. Pero, de hecho, pasa a la economía normativa. Se adhiere a la regla de Pareto de la ``optimización'': la optimización en una sociedad significa que cualquier cambio posible desde una determinada posición da como resultado que algunos individuos empeoren; o si lo preferimos, cualquier cambio es óptimo solo si todos están mejor, o si alguien está mejor y nadie está peor. Este es un caso típico de economía positiva, pero también es, escribe Buchanan, una proposición ética, un juicio de valor.

¿Cómo se puede reconciliar esta proposición, la optimalidad como declaración de valor, con la economía positiva de Pareto? La reconciliación se realiza quitando el contenido del juicio de valor del economista, y dejando ese contenido a la elección individual, que es, por definición, ética.

Observa que persiste una ambigüedad en el contenido que se debe dar a los ``más acomodados'' y los ``más desfavorecidos''. La contribución de Buchanan a este tema es que la eficiencia, una posición de ``mejor situación'', es la posición que se elige voluntariamente. Mientras que la economía del bienestar generalmente ha asumido que el economista-observador es omnisciente y, por lo tanto, capaz de leer las preferencias individuales, la economía constitucional coloca al economista en una presunción de ignorancia. Su criterio de eficacia no puede ser otro que presuntivo. Es de suponer que su eficacia conservará las características paretianas. ``La economía política es, pues, positivista en un sentido diferente de la economía positiva concebida de manera más estrecha''. 52El economista político presenta un posible cambio, pero solo si se llega a un consenso, de lo contrario no resultará un beneficio sino un daño. El comportamiento observable de los individuos como tomadores de decisiones colectivos es la única prueba de bienestar. ``El comportamiento político de los individuos, no el desempeño o los resultados del mercado, proporciona los criterios para probar las hipótesis de la economía política''. 53

No existen valores sociales aparte de los valores individuales. El economista político formula hipótesis sobre los valores de los individuos, manteniéndose éticamente neutral. Los valores se pondrán a prueba en la acción colectiva de todas las personas involucradas en la decisión. Las consecuencias de este enfoque son trascendentales, y el campo de las finanzas públicas permite una implementación fructífera del enfoque de la economía constitucional: el ejemplo típico que se da es la coerción estatal en materia de impuestos y gasto público, que encuentra su legitimidad económica sólo en el proceso de elección colectiva. 54Sin embargo, esto puede llevar a conclusiones inconsistentes con la optimalidad de Pareto y con cualquier visión ideológica de libre mercado, aunque consistente con una democracia liberal. ``{[}I{]} n comportamiento individual puede ser totalmente coherente con una reducción en los ingresos o la riqueza personal medidos''. ``Una política que combine impuestos progresivos sobre la renta y gasto público en los servicios sociales puede obtener un apoyo unánime aunque el proceso implique una reducción de los ingresos reales medidos de los ricos''. 55

De esta manera tenemos una paradoja: en nombre del individualismo, la economía constitucional puede llegar a conclusiones bastante diferentes a las sugeridas por los economistas positivos. La eficiencia acaba siendo lo que surge del consentimiento individual, del proceso de acción colectiva.

Buchanan advierte a los economistas contra el abandono de la neutralidad: su postura de libre mercado, que implica ganancias mutuas del comercio, puede presentarse solo como una recomendación, porque solo las preferencias individuales, tal como se expresan en la acción colectiva, tienen una relevancia decisiva.

Por lo tanto, podemos tener reglas, ya que han surgido de acciones colectivas, que no son funcionales para una economía de libre mercado. Este es el caso examinado por Richard Posner, un teórico del derecho de la elección pública, con respecto a la constitución estadounidense (pero el mismo análisis podría aplicarse a otros). La alineación de las reglas y el interés privado es difícil de lograr, particularmente en el caso de reglas cuya formulación está especialmente protegida, porque son centrales para la vida, no solo la vida económica, de una colectividad. Posner ha considerado la relación entre la constitución y el crecimiento económico, y su alineación con la lógica económica implícita de un mercado libre.

El punto importante aquí es que, dentro de su contexto democrático más amplio, la constitución bien podría no ser completamente consistente con la visión económica libertaria de los economistas constitucionales. ``Como cualquier forma de constitucionalismo agresivo\ldots, el enfoque económico libertario {[}a través de la interpretación de la constitución o su enmienda{]} disminuye el papel de la democracia, potencialmente dramáticamente'', escribe Posner. 56Según él, ``para captar la naturaleza y el alcance de la tensión entre el laissez-faire y la teoría política o jurídica democrática es necesario distinguir entre dos concepciones políticas fundamentales que a veces se confunden: gobierno limitado y gobierno democrático. Los defensores del gobierno limitado quieren que el gobierno sea relativamente impotente y, en parte por esta razón, no están muy interesados \hspace{0pt}\hspace{0pt}en cómo se elige a las personas que dirigen el gobierno; su interés es preservar una gran esfera para la acción privada libre de interferencias gubernamentales. Los defensores del gobierno democrático quieren asegurarse de que el gobierno esté de alguna manera en manos del pueblo y confían en que, si se coloca allí, se puede confiar en que promoverá el bienestar general.57

La tensión entre un gobierno limitado, inclinado hacia el individualismo, el libre intercambio y el comercio por un lado, y el gobierno democrático, por el otro, inclinado hacia el bienestar general sobre la base de la acción colectiva, muestra hasta qué punto una ``situación clásica'' de consenso generalizado en materia institucional. las estructuras orientadas al neoliberalismo encuentran dificultades para emerger (más tarde, después de la crisis financiera de 2008, la confianza de Posner en los mercados libres se vio notablemente afectada). 58

\hypertarget{nueva-economuxeda-cluxe1sica}{%
\section*{Nueva economía clásica}\label{nueva-economuxeda-cluxe1sica}}
\addcontentsline{toc}{section}{Nueva economía clásica}

Pasando a las contrarrevoluciones económicas que reaccionaron al consenso keynesiano, la misma postura libertaria marca una evolución de la disciplina de la economía, en primer lugar implicando una reevaluación de la doctrina monetarista, y en segundo lugar poniendo un nuevo énfasis en la lógica del comportamiento económico individual en economías de libre mercado, a través de modelos altamente sofisticados. Como se mencionó anteriormente, el primer enfoque atacó el análisis insuficiente de la inflación de Keynes; el segundo, partiendo del comportamiento racional del individuo, va más allá y proclama la ``Revolución Keynesiana'' como un ``desastre''. En este sentido, la ``hipótesis de expectativas racionales-REH'' y la ``hipótesis de mercado eficiente-EMH'' han cobrado especial importancia, también con consecuencias de gran alcance para el desempeño de las economías y los mercados financieros.

Según el enfoque de Johnson, que mezcla de manera interesante ideología, doctrina y conveniencia, el contraataque monetarista necesitaba encontrar (1) un problema social importante, y (2) una teoría que tuviera que ser ``académica y profesionalmente exitosa en reemplazar al revolucionario anterior. teoría''. El monetarismo ganó fuerza cuando la inflación se convirtió en una seria preocupación social, incluso en los Estados Unidos con la escalada de la guerra de Vietnam; y la teoría cuantitativa del dinero se consideró una explicación científica plausible de la inflación extremadamente alta. Este nuevo interés implicó un reexamen de esa teoría. La figura que estaba asumiendo una posición preeminente era Milton Friedman; y en general la Escuela de Chicago que hemos mencionado en el Capítulo 2 ganó terreno.

La ``teoría cuantitativa'' ---para evitar críticas que habían afectado su formulación anterior: que supondría una tendencia automática al pleno empleo (manifiestamente en conflicto con la experiencia actual) ---fue reinstalada como una forma de dar una explicación, y un política, en lugar del keynesianismo. El monetarismo de Friedman reexaminó críticamente la teoría, seleccionando relaciones cruciales que permitirían obtener resultados a partir de ciertas variables. 59

El presidente del banco central estadounidense, Paul Volcker, proporcionó el ejemplo más sobresaliente de lo que se llama ``monetarismo práctico, a diferencia del teórico'', a fines de la década de 1970, al reducir la cantidad de dinero en presencia de una inflación severa. La inevitable recesión que sobrevino, profunda y duradera, y el comportamiento impredecible de los agregados monetarios, que dificultaron la determinación de una medida de dinero satisfactoria para orientar la política, llevaron a un cambio de política monetaria, no solo en Estados Unidos, hacia control de otras variables, como tasas de interés y metas de inflación. 60

El monetarismo, sin embargo, atrajo una atención renovada con la difusión del pensamiento libertario de Buchanan y otros. Si, dentro del pensamiento de Buchanan, dejamos de lado la teoría de la elección colectiva y sus implicaciones para el correcto funcionamiento de una colectividad, que es la originalidad sobresaliente de su enfoque, el punto relevante que permanece en su trabajo es una confirmación de un arraigado liberalismo. ideas sobre la centralidad del individuo, que es libre de elegir sobre la base de su propia escala subjetiva de preferencias, una escala determinada por la intención de maximizar su utilidad, como quiera que se entienda. Ésta es, de hecho, la sustancia del neoliberalismo, reafirmada en diversas formas por un gran número de pensadores políticos y economistas.

En el campo de la disciplina de la economía, el comportamiento individual se coloca cada vez más en el centro del análisis económico, y el comportamiento de toda la economía se examina sobre la base de las elecciones racionales de los individuos, que se toman utilizando toda la información disponible. Un intento de entender cómo funciona toda la economía ---una economía de libre mercado--- sobre esa base de comportamiento, se hace matemáticamente, de la misma manera que lo hicieron Walras o Pareto un siglo antes. Se construyen modelos econométricos. Un modelo econométrico es un sistema de ecuaciones que conecta ciertas variables consideradas importantes para el funcionamiento de una economía de mercado. Estas conexiones se determinan sobre la base de las elecciones ``racionales'' de los agentes (la racionalidad guiada por motivaciones utilitarias, según la definición de Buchanan). El modelo también tiene en cuenta factores exógenos no de mercado,

Una vez que se resuelven las ecuaciones, el modelo produce un resultado que da un valor a las variables macroeconómicas significativas, como el producto, el empleo, los precios. El modelo finalmente tiene que ser validado por la coherencia de su salida con la experiencia real. Si este es el caso, el modelo es ``científico'' en el mismo sentido que un modelo construido dentro de las ciencias físicas: lógicamente coherente y confirmado experimentalmente. Ciertas proposiciones económicas se vuelven científicamente ``verdaderas'', libres de declaraciones de valor.

Está lejos de nuestra intención dar una descripción de la construcción de estos modelos y de los esquemas teóricos relacionados; más bien es nuestro propósito mostrar que la validez -si la hay- de estos esquemas está condicionada a la aceptación de sus premisas ideológicas y a la existencia de instituciones cuyo funcionamiento es funcional al esquema teórico, de manera que la experiencia pueda validar las predicciones que el esquema pretende ceder. 61

La elección racional de Buchanan puede verse como el trasfondo filosófico de dos esquemas teóricos, recién mencionados: la ``hipótesis de expectativas racionales-REH'' (Robert Lucas y Thomas Sargent se encuentran entre sus principales exponentes) y la ``hipótesis de mercado eficiente-EMH'' (Eugene Fama ), que se han vuelto predominantes tanto en macroeconomía como en la explicación del comportamiento de los mercados financieros: un tema, este último, de abrumadora importancia, dada la expansión de las supraestructuras financieras sobre la estructura real de la economía, y la considerable disrupciones que pueden provenir, y han venido, de mercados financieros disfuncionales.

¿La elección racional de Buchanan depende del supuesto de que la elección se realiza basándose en la experiencia pasada (según la cual las personas asimilan y reaccionan a su experiencia real a lo largo de los años), o en las expectativas racionales actuales de la evolución de las variables que explican los agentes '? ¿interesar? El tema de las expectativas es fundamental para la REH.

Un documento del Banco de la Reserva Federal de Minneapolis 62dio una descripción clara del nuevo enfoque ---expectativas racionales--- mostrando cómo las predicciones, incorporadas a las ecuaciones del modelo econométrico que representan el comportamiento de los individuos, arrojan resultados diferentes de los obtenidos por la metodología anterior, no basados \hspace{0pt}\hspace{0pt}en expectativas. Tomemos las estimaciones de inflación: las personas toman decisiones sobre sus compras no en función de lo que ha sido la inflación en los últimos años, sino de cómo se espera que evolucione la inflación en el futuro y, en estas expectativas, se ven afectadas por cambios en las políticas gubernamentales. ``Para evaluar con precisión los efectos de diferentes políticas económicas, \ldots{} se debe incluir un modelo mucho más sofisticado de las expectativas de las personas en la estructura de los modelos econométricos. El principio de modelado propuesto se basa en la teoría de las expectativas racionales''. 63

El supuesto implícito de que cualquier influencia del gobierno (o del banco central) en el comportamiento espontáneo del mercado perturba las expectativas y, por lo tanto, crea fluctuaciones indebidas en la actividad económica, se explica en un artículo de la misma Revista, unos meses después, escrito por Lucas y Sargent. . 64 Al observar el fracaso de las políticas keynesianas basadas en el uso extensivo de herramientas monetarias y fiscales, quisieron ``reabrir los temas básicos de la economía monetaria'', recordó la teoría neoclásica preexistente, basada en dos postulados conductuales: que ``los mercados claro'' 65y que los agentes actúen, hagan su elección, en su propio interés; pero agregó que se supone que cada agente tiene información limitada. Por tanto, los agentes cometen errores; pero, también, todo el mundo comete el mismo error. (Se puede observar que se trata de una simplificación que supone que todos los individuos son idénticos, o al menos que las diferencias entre ellos se anulan entre sí, de modo que podemos concebir un ``agente representativo''). A partir de sus expectativas racionales. , se determina un cierto nivel de precios y producción. Un observador ajeno al mercado no puede ganarle al mercado y la autoridad (el banco central), como observador ajeno al mercado, no puede actuar de manera diferente. Un cambio inesperado e impredecible en la oferta monetaria, realizado por la autoridad, cambia los niveles de precios y producción con respecto a lo que hubieran sido de otra manera. Este cambio inesperado crea fluctuaciones comerciales y choques imprevistos. Por tanto, lo que se necesita son ``reglas de juego estables, bien comprendidas por los agentes económicos''. El activismo en la política monetaria o el financiamiento del déficit fiscal tiene la ``capacidad de perturbar''. Se debe seguir la regla del X por ciento defendida por Friedman, con respecto a los cambios en el stock de dinero.

Esta es probablemente la conexión más visible y operativa entre el monetarismo de Friedman y la nueva ``hipótesis''.

La hipótesis de las expectativas racionales ve los resultados macroeconómicos como una agregación del comportamiento de todos los agentes. Esta base microeconómica permite remontarnos a los neoclásicos de finales del siglo XIX al XX y al equilibrio de Walras (Capítulo 1 ). La macroeconomía tiene sus raíces en las matemáticas; y en rigurosas micro fundaciones. La Hipótesis es una negación de las relaciones sociales, que no solo arroja sospechas sobre cualquier intervención gubernamental, sino que también niega relevancia a cualquier conexión social entre agentes económicos (individuos o empresas). 66

En los mercados financieros, las expectativas racionales están arraigadas en los precios de los activos financieros. Dado que los rendimientos de los activos son inciertos, la elección racional, que determina el precio del activo, no se puede hacer más que confiando en toda la información disponible (la EMH). Una inferencia que podría extraerse de esta afirmación es que, si la información estuviera completa, el pronóstico de los agentes sería óptimo y los precios serían, por definición, correctos. Dado que la información es más o menos incompleta, los precios pueden ser diferentes de lo que serían de otra manera. La sobrevaloración o subvaloración consecuente implica rendimientos mayores o menores que los resultantes de la previsión óptima. En este caso, sin embargo, aparecerían oportunidades inexplicables de ganancias y el arbitraje haría converger las elecciones de los participantes del mercado en el pronóstico óptimo.

Sin embargo, según tengo entendido, esta no es la inferencia correcta que deduciría un teórico de la EMH. Según él, los precios siempre son correctos, en relación con la información disponible, y las autoridades deben en cualquier caso abstenerse de interferir en su determinación. Ésta es una distinción importante al tratar de explicar las causas de las crisis financieras.

La ``hipótesis de las expectativas racionales'' fue ganando terreno progresivamente y, en asociación con la ``hipótesis del mercado eficiente'', dio un sustento significativo a una interpretación y comprensión específicas del comportamiento de los mercados financieros. Ambos demostraron ser extremadamente exitosos en términos de correspondencia entre las predicciones y los resultados reales. Sobre esta base, se crearon instrumentos financieros innovadores, beneficiando en primer lugar a sus engendradores y usuarios.

Esta correspondencia fue facilitada por los formuladores de políticas, quienes desarrollaron un entorno institucional favorable en el que este enfoque teórico podría tener resultados coherentes, al reducir cualquier interferencia del gobierno que pudiera alterar las expectativas racionales de los individuos y restringir las preferencias espontáneas del mercado financiero. En general, con el fin de aliviar las circunstancias ambientales en las que las expectativas racionales podrían no verse obstaculizadas, se buscó una intervención no gubernamental en la asignación de capital. En el campo de la política monetaria, se siguió una orientación monetaria predecible y los bancos centrales proporcionaron orientación de política futura (información sobre las intenciones futuras de política monetaria 67) fue adoptado. En general, parece seguro que durante las últimas décadas del siglo, el control de la inflación y las finanzas públicas ``sólidas'' ganaron una prioridad cada vez mayor sobre el pleno empleo y la protección social. Relacionado con este cambio estuvo la tendencia de crecimiento de la deuda del sector privado de la economía a niveles sin precedentes: la financiarización de la economía creció fuertemente, en particular su componente privado.

Si aceptamos el punto de vista, mencionado en nuestro Prefacio, de que la economía ``positiva'' (a diferencia de ``normativa'') puede calificar como ciencia en la medida en que su precisión, alcance y conformidad con la experiencia de las predicciones que produce, una pregunta puede ser planteado: ¿hubo algún impedimento que impidiera que la hipótesis de las expectativas racionales ---y la tesis de la hipótesis del mercado eficiente--- pronosticara y, por lo tanto, posiblemente previniera, el colapso de los mercados financieros en 2007-2008? Vale la pena informar el siguiente extracto de un libro publicado justo después de la caída del mercado:

\begin{quote}
``Los mercados financieros privados no pueden funcionar correctamente a menos que exista suficiente información, informes y divulgación tanto a los participantes del mercado como a los reguladores y supervisores relevantes. Cuando los inversores no pueden valorar adecuadamente los nuevos valores complejos, no pueden evaluar adecuadamente las pérdidas generales que enfrentan las instituciones financieras, y cuando no pueden saber quién tiene el riesgo de los llamados desechos tóxicos, esto se convierte en una incertidumbre generalizada \ldots{} Por lo tanto, una vez que la falta de recursos financieros la transparencia del mercado y el aumento de la opacidad de estos mercados se convirtieron en un problema, se sembraron las semillas para un desastre sistémico en toda regla''. 68
\end{quote}

Esto implicaría que, antes del colapso, ``toda la información disponible'' era de hecho extremadamente escasa, por lo que no habría sido posible la autocorrección del mercado.

Pero, según Eugene Fama, este no era el tema en juego. En realidad, la pregunta que planteé anteriormente ---la prevención del colapso del mercado financiero--- probablemente esté mal concebida. Los precios de mercado eran constantemente correctos, eso es coherente con la información disponible, y Fama no se queja de escasez de información. No ve en el colapso del mercado una falla de mercado de la EMH, la EMH sale bastante bien de este episodio; observa que ``los mercados financieros fueron una víctima de la recesión {[}económica{]}, no una causa de ella'', pero lo cierto es que la actividad económica es ``la parte que no entendemos''; culpó a la interferencia del gobierno en el mercado: el gobierno causó la crisis de las hipotecas de alto riesgo 69 y sus remedios ``demasiado grandes para fallar'' para las fallas bancarias respaldaron implícitamente esos precios de mercado.70

Cándido, protagonista de la novela homónima de Voltaire, medio muerto bajo las ruinas de una Lisboa totalmente destruida por un terremoto (1755), es consolado por su compañero, el filósofo Pangloss, que dice que ``todo esto es lo mejor; porque, si hay un volcán en Lisboa, no puede estar en ningún otro lugar; porque es imposible que las cosas no estén donde están; porque todo está bien'' 71 (el volcán en cuestión es la actividad económica).

Cabe señalar que, ya en 1997, un Samuelson desencantado comentaba irónicamente los ``dogmas modernos de Lucas'': ``la historia económica posterior a 1978 {[}el año en que se formuló la REH{]} habla en contra de la confirmación ex post de las especulaciones ex ante de esa Escuela''. 72 No había necesidad de una mayor confirmación, por el colapso financiero de 2008, de una teoría que en verdad era defectuosa.

Dado que el concepto de información parcial disponible parece similar al de ``incertidumbre'', tal como lo considera Hayek (ver Capítulo 2 ), uno puede preguntarse si la perspectiva ideológica es la misma. Hayek no se basa en la racionalidad del mercado, no se puede concebir un sistema racional de preferencias, dadas las piezas de información meramente dispersas que tienen los agentes, pero, sin embargo, piensa que el sistema de precios es la forma más eficiente de conectar esas piezas. Tanto si no podemos confiar en la racionalidad del mercado (Hayek) como si podemos (Fama), la Weltanschauung libertaria es la misma. En una perspectiva opuesta, Keynes consideró la incertidumbre en el contexto del mercado: separó el riesgo, que se puede describir en términos de probabilidades y, por lo tanto, calculable, de la incertidumbre, que no puede. 73 Creía que la incertidumbre es el origen de las fallas del mercado y, por lo tanto, la motivación para la intervención del gobierno.

\hypertarget{una-nueva-situaciuxf3n-cluxe1sica}{%
\section*{¿ Una nueva ``situación clásica''?}\label{una-nueva-situaciuxf3n-cluxe1sica}}
\addcontentsline{toc}{section}{¿ Una nueva ``situación clásica''?}

Hemos hecho esta breve incursión en la teorización sin ninguna intención de suscribir ni de rechazar su contenido, sino de mostrar el estrecho vínculo entre, por un lado, un fundamento ideológico: la confianza en que un enfoque libertario, apoyado en una sociedad organizada a lo largo de Un homo oeconomicus egoísta , y escéptico de la intervención del Estado, puede dar mejores resultados en términos de bienestar (la visión como emergente del neoliberalismo) y, por otro lado, la construcción de modelos que expliquen de manera positiva y neutral el comportamiento real de el sistema económico.

La visión neoliberal no encuentra ninguna inconsistencia entre el colapso del mercado y el núcleo de la teoría de que ``los mercados siempre son correctos''. Quienes apoyan un papel mucho más amplio del gobierno en la estimulación de la economía y la regulación de los mercados financieros tienen una opinión diferente. Sin embargo, tras la perturbación financiera y económica que siguió al colapso del mercado de 2008 y las medidas políticas adoptadas en esta última dirección tras la crisis, una actitud inversa está cada vez más presente en las políticas económicas y la regulación del mercado. Por ejemplo, en el campo de la regulación financiera, vemos en el Reino Unido una tendencia a volver al ``enfoque basado en principios'' anterior a la crisis en contraposición a una regulación prudencial más detallada.

Si nos centramos en el significado polivalente de ``racionalidad'', no es de extrañar que la disciplina de la economía se aleje cada vez más de ser una ciencia social (moral), y se convierta en una ciencia del comportamiento individual, para enmarcarse dentro de la categoría de lo natural. ciencias. El positivismo de Comte parece gozar de una completa venganza. Incluso el ``comportamiento irracional'' (Robert Shiller) puede subsumirse bajo el mismo techo, si lo vemos como un intento de entender un sistema económico a través de las motivaciones del comportamiento humano. Hace muchas décadas, el economista George Shackle observaba: ``es evidente que, por un lado, la economía tiene una frontera con la psicología, o más bien, que entre ellos hay una tierra de nadie clamando por ser explorada y apropiada, que podríamos llamar física económica''. Y da un ejemplo:74 Ahora, el estudio de la neuroeconomía está progresando, quizás fusionando la economía con la ciencia médica; y los restos de la economía política pueden quedar definitivamente abandonados.

Si se puede ver la filosofía del neoliberalismo y la teorización económica que ha estado prevaleciendo en el cambio de siglo actual, lo que se conoce con el nombre de Nueva Economía Clásica (esencialmente, un renacimiento de la Escuela Neoclásica ``Walrasiana'' 75 ). como una ``situación clásica'' en un sentido schumpeteriano (este es nuestro punto {[}c{]}, planteado anteriormente), es muy dudoso, al menos por un par de razones:

\begin{quote}
\begin{itemize}
\item
  Se siguen aplicando nuevos enfoques keynesianos. Según una versión keynesiana, dada la rigidez (o rigidez) de los salarios, que evita su caída incluso cuando los recursos laborales están desempleados, sería suficiente eliminar esa rigidez, y la economía funcionaría de manera eficiente, de manera similar al modelo neoclásico. . El papel del gobierno sería, en esta versión, activo, pero principalmente dirigido a restaurar el buen funcionamiento del modelo neoclásico.
\item
  Según otra versión keynesiana, posiblemente más cercana a la ``original'', la rigidez salarial puede ser útil y ayudar a la estabilización económica: bajar los salarios para restablecer el equilibrio significaría recortar el gasto de los consumidores, exacerbar cualquier recesión, provocar deflación, y específicamente quiebras si, como fue el caso en las circunstancias de principios de la década de 2000, hay demasiada deuda en el sistema económico.
\item
  Un eco de la polémica entre el monetarismo, o en general la Nueva Economía Clásica por un lado, y el keynesianismo de segunda versión por el otro, se puede escuchar en el debate actual, dentro de la Unión Europea, entre una política monetaria ``orientada al Bundesbank'' y las políticas anti-austeridad, gestionadas por la demanda, invocadas para superar el estancamiento actual. La pandemia parece fortalecer estas últimas políticas.
\item
  Después de la crisis financiera de 2008 y la Gran Recesión que siguió, la secuencia que vincula la ideología, las nuevas teorías económicas, las políticas económicas y el colapso financiero y la recesión económica no se puede descartar fácilmente. Desde allí, hasta detectar una conexión causal en lugar de una simple secuencia de eventos, el escalón no fue difícil de escalar. ``No fue un accidente que aquellos que defendieron las reglas que llevaron a la calamidad estuvieran tan cegados por su fe en el libre mercado que no pudieron ver los problemas que estaba creando. La economía había pasado \ldots{} de ser una disciplina científica a convertirse en la mayor animadora del capitalismo de libre mercado''. 76
\end{itemize}
\end{quote}

Desde estos eventos, ha habido una fuerte tendencia no solo a restablecer a Keynes en su papel apropiado tanto en la historia del pensamiento económico como, quizás más importante, en una visión verdaderamente liberal de la sociedad, sino también a invocar nuevamente el políticas económicas que defendió con tanto éxito durante un período de tiempo relativamente largo. El puesto más destacado al respecto se encuentra en el libro Keynes. El regreso del maestro , donde Robert Skidelsky, ``el'' biógrafo de Keynes, no duda en definirlo, con cierto énfasis, como ``el pensador económico más importante del mundo''. 77Mi visión relativista de toda la disciplina de la economía, que habrá surgido de estas páginas (y más de eso en el siguiente capítulo final), me incomoda con este tipo de afirmaciones. Pero antes de eso, no puedo evitar ocuparme de lo que parece ser una especie de subproducto de la Gran Recesión y preguntar qué tipo de filosofía económica, si la hay, se esconde detrás del ``populismo''.

Mientras tanto, podemos decir con seguridad que la disciplina de la economía parece incapaz de captar de manera adecuada todas las características complejas de nuestra sociedad: todavía no tenemos las líneas perfectas de un templo griego a la vista. Esta disciplina hace todo lo posible para merecer su denominación de ``ciencia lúgubre''.

\hypertarget{el-populismo-como-subproducto-del-neoliberalismo}{%
\section*{El populismo como subproducto del neoliberalismo}\label{el-populismo-como-subproducto-del-neoliberalismo}}
\addcontentsline{toc}{section}{El populismo como subproducto del neoliberalismo}

Puede parecer extraño que un ensayo dedicado a las filosofías económicas deba ocuparse del populismo. La respuesta inmediata a una pregunta sobre qué tipo de pensamiento económico tiene en mente un populista sería un rotundo ``ninguno''. Pero el populismo no debe ser descartado meticulosamente, y una respuesta más articulada requiere que miremos:

A. el rasgo principal que diferencia al populismo de la democracia liberal: que es, sobre todo, la forma en que se adquiere y se ejerce el poder político. En general, el primero invoca alguna forma de democracia directa, mientras que el segundo se basa en la democracia representativa;
B. las raíces ideológicas del populismo;
C. las motivaciones que se esconden detrás del auge del populismo;
D. el impacto de las redes sociales en el populismo.

Se puede inferir una postura económica populista una vez considerados estos temas.

\begin{quote}
\begin{itemize}
\tightlist
\item
  Populismo y democracia liberal
\end{itemize}
\end{quote}

Sobre la adquisición y el ejercicio del poder político, los argumentos tanto de los liberales como de los populistas se centran en las élites, pero el papel de las élites se ve de otra manera:

\begin{quote}
\begin{itemize}
\tightlist
\item
  Según los populistas, las élites se oponen necesariamente al pueblo. Sus miembros utilizan su poder para promover sus intereses personales, con el objetivo de obtener una ganancia a expensas de la gente. Así es como hoy en día los populistas utilizan y entienden principalmente la palabra ``élite''. Casi por definición, el comportamiento de las élites no puede ser otro que ``malo'': son una minoría, pero tienen el poder, y explotan al pueblo económicamente como de otras formas. Este comportamiento es visto a la luz de una teoría de la conspiración, y esta visión se refuerza a sí misma: la democracia representativa esconde complots contra el pueblo. El remedio a esta situación es el derrocamiento de las élites y una transferencia del poder directamente al pueblo. En términos constitucionales, el resultado de este proceso es eliminar gradualmente, o al menos marginar, representación del pueblo a través de un órgano electo (el parlamento) y encomendar directamente al pueblo las decisiones pertinentes. La distinción entre funciones ejecutivas y legislativas acaba siendo al menos borrosa.
\end{itemize}

Si las decisiones políticas las toma directamente el pueblo mismo o si se confían a un líder designado por el pueblo, es una cuestión que merece un examen más detenido. Si bien el populismo puede enmarcarse dentro del concepto de democracia directa, la propia idea de democracia directa se torna vaga cuando el líder termina desapegado de la voluntad del pueblo. Sobre la base de un mandato supuestamente fiduciario que le dio el pueblo, el líder podría creer implícitamente que es superior a los demás en saber lo que es ``bueno'' para la comunidad y decidir en consecuencia.

\begin{itemize}
\item
  La visión liberal también ve a las élites como un grupo que ejerce el mando, pero también las considera como personas que, debido a su selección por parte del pueblo a través de un proceso electoral, o dada su experiencia específica en ciertos campos, están constitucionalmente encargadas de la gobernanza del gobierno. organismo público. Institucionalmente, el parlamento es visto como un organismo elegido democráticamente por los ciudadanos, no simplemente como un ejecutivo de decisiones ya tomadas directamente en otros lugares. Las funciones legislativas y ejecutivas están separadas, y cualquier posible intrusión en cada uno de los otros campos está limitada por ``controles y contrapesos''. El parlamento está a cargo de redactar y aprobar las leyes, que el poder ejecutivo del gobierno tiene la tarea de observar e implementar a través de la acción política. Además, la competencia particular requerida para determinadas tareas pertenecientes a la esfera pública, implica encomendar a determinadas personas, que se supone que tienen esa competencia, la ejecución de esas tareas: pensemos en el poder judicial o en la banca central, por ejemplo. Si el parlamento, el gobierno o las autoridades independientes han cumplido o no, en un caso específico o en un conjunto de circunstancias, con sus tareas debidamente, en perjuicio del pueblo, es un asunto que debe analizarse dentro del sistema constitucional actual. de la democracia representativa.
\item
  Raíces ideológicas de los movimientos populistas
\end{itemize}
\end{quote}

Las formas de populismo tienen importantes padres filosóficos. Podemos llevar a dos pensadores, muy lejanos en tiempo y lugar (Jean-Jacques Rousseau y James Buchanan). Con ellos se invoca la democracia directa radical, aunque con una implicación muy diferente del Estado en la vida de la sociedad. De hecho, estos dos nombres son relevantes porque nos llevan a las dos corrientes de pensamiento en las que se basó originalmente toda la construcción de la economía política: la primera, basada en la centralidad de la sociedad en su conjunto, o del Estado; y el segundo, apoyado en el punto de vista individualista, centrado en el individuo como agente racional. Además, me viene a la mente un tercer pensador: Carl Schmitt, y su identificación del líder como dictador: su nombre resuena cada vez más en los debates actuales sobre el populismo.

En la teoría de Rousseau, la Voluntad General es la expresión del cuerpo político, el Estado, que es una entidad moral y el fundamento de la convivencia humana. La volonté généralsignifica que la soberanía se encuentra en el pueblo en su conjunto, y que la soberanía es indivisible. Rousseau rechazó la moderación política, el relativismo, el tradicionalismo y el entusiasmo por el modelo parlamentario británico, expresado por pensadores tan diversos como Montaigne, Voltaire o Hume. Los adversarios de Rousseau pensaban, por el contrario, que la democracia pura y directa era lo más parecido a la anarquía. En la ``incesante batalla entre la Revolución de la Razón y la Revolución de la Voluntad'', la corriente principal del pensamiento de Rousseau, la ``revolución de la voluntad'', lo convirtió en un paria en su sociedad, y se opuso a los republicanos democráticos que hacer la revolución francesa de 1789-1793. 78Al mismo tiempo, Rousseau veía la dictadura solo como un remedio extremo que la democracia directa puede adoptar en circunstancias excepcionales; sólo más tarde Marat y Robespierre, que tenían a Rousseau en alta estima, darían un giro populista a la visión de Rousseau. 79

La Voluntad General, siendo ella misma la esencia del Estado, entra en consideración cuando el asunto a resolver involucra el interés de toda la comunidad y, definido como antes, debe distinguirse de la voluntad de todos. El primero tiene en cuenta sólo el interés común, mientras que el segundo tiene en cuenta el interés privado y no es más que la suma de voluntades particulares, según Rousseau. 80

El Discurso sobre economía política de Rousseau , que sigue al Contrato social , ayuda a comprender mejor cómo operaría la Voluntad General en interés de toda la comunidad. En este sentido, define tres reglas de economía política. El primero dice que la Voluntad General debe seguirse como una especie de solución por defecto en la que el legislador debe apoyarse. El legislador, escribe Rousseau, tiene una infinidad de detalles de administración y economía que cuidar, pero debe seguir ``dos reglas infalibles'': el espíritu de la ley {[}implicidad, la Voluntad General que es el origen de cualquier ley{]} debe decidir en casos particulares que no pudieran preverse; y se debe consultar explícitamente al testamento general siempre que falle esta prueba.

La segunda regla es fundamental para comprender mejor la autoridad sobre la que descansa la Voluntad General y enfatiza la centralidad de la educación pública. El Estado tiene el rol de educador público de la ciudadanía: ``formar ciudadanos no es un trabajo de un día, y para tener hombres es necesario educarlos cuando sean niños'' 81 , para que ``tengan en cuenta su individualidad sólo en su relación con el cuerpo del Estado y ser conscientes, por así decirlo, de su propia existencia meramente como parte de la del Estado''. 82De esta manera, podrían llegar a identificarse en algún grado con este gran conjunto, a sentirse miembros del país. Más allá de cualquier forma de educación privada, a partir de lo que los niños puedan recibir de sus padres, la educación es ``aún más importante para el Estado''. 83

La tercera regla, que afecta más directamente al campo de la economía política, es que la provisión de necesidades públicas es una inferencia obvia de la Voluntad General, y el tercer deber esencial del gobierno. ``Si a un rico le roban, toda la fuerza policial se pone inmediatamente en movimiento\ldots{} qué diferente es el caso del pobre. Cuanto más le debe la humanidad, más se lo niega la sociedad''. 84 Para recaudar impuestos de manera verdaderamente equitativa y proporcionada, la imposición no debe ser ``en proporción simple a la propiedad de los contribuyentes, sino en proporción compuesta a la diferencia de sus condiciones y lo superfluo de sus posesiones''. 85

Estas palabras hicieron de Rousseau, durante mucho tiempo, el héroe insuperable simultáneamente de izquierda y derecha, un estatus que ningún otro pensador había alcanzado. El enfoque de Rousseau en el Estado como entidad suprema, del cual la Voluntad General es la expresión, significó una aversión a las ideas de cosmopolitismo, universalismo y la búsqueda de la paz universal, que fueron componentes básicos de los filósofos británicos de su tiempo.

Diferente, y expuesta dos siglos después, es la visión de Buchanan, 86 que se basa en gran medida en la doctrina del constitucionalismo que hemos mencionado en la Secta. 4.3 . Critica el sistema representativo cuando otorga amplios poderes de discreción a la asamblea electa, el parlamento: esto es típico de la situación en la que un sistema político se basa en la votación por mayoría, con una agenda abierta. El peligro es que un mayoritarismo abierto es muy vulnerable a la manipulación demagógica, a lo que podríamos llamar la dictadura de la mayoría. Otro peligro deriva de la ``agenda abierta'', es decir, del campo de acción extremadamente amplio abierto a las decisiones parlamentarias. 87

Más bien, Buchanan piensa que lo que se necesita es un acuerdo general entre todos los ciudadanos sobre la necesidad de ``limitar la agenda para la acción colectiva'': el parlamento debería poder aprobar leyes solo en una gama limitada de asuntos. Esto no requiere ningún acuerdo de preferencias entre los votantes, significa lo contrario: dado que no se puede alcanzar un consenso sustancial sobre la mayoría de los posibles temas de elección, cada persona querrá estar de acuerdo con reglas que limitan la acción política y, por lo tanto, el rango de coerción que la política implica necesariamente. Esta es esencialmente una forma a través de la cual los individuos obtienen protección contra la extensión potencial de la coerción colectivizada. ``{[}E{]} l constitucionalista se apoya exclusivamente en el demos'', 88 en el pueblo, y en ese sentido se afirma la democracia directa.

Como se mencionó anteriormente, adoptar solo decisiones políticas que sean aprobadas por consentimiento unánime equivale en principio a aceptar la distribución actual de la riqueza: significa adoptar solo políticas que representen ``mejoras de Pareto''.

La democracia representativa e indirecta tiene que ser de hecho limitada en opinión de Buchanan, restringida a los estrechos límites impuestos por la acción colectiva al gobernante: ``se politizarían menos actividades bajo la democracia directa que bajo la indirecta''. 89 Por ejemplo ---agrega--- este sistema reduciría de manera plausible la ``legislación de barril de cerdo'', que es la tendencia a adecuar el gasto público a los intereses locales, utilizando los ingresos de todos los contribuyentes. El problema de la relación principal-agente, que es típico de la democracia representativa indirecta, se minimizaría con el sistema que acabamos de describir.

Pero Buchanan plantea una pregunta adicional: ¿cómo aplicar las reglas de la democracia directa en colectividades donde ya existe ``una constitución'', con características históricamente determinadas, muy alejadas de los principios que él defiende? 90 En esta situación, la democracia directa puede asumir un significado diferente del modelo estilizado que acabamos de describir. Abogó por ``disposiciones para iniciativas populares y referendos {[}que{]} puedan funcionar para prevenir acciones colectivas que de otro modo podrían implementarse''. 91

Buchanan está firmemente convencido de que esta forma de democracia directa es consistente con la verdadera esencia del liberalismo clásico, que requiere que se minimice el tamaño del sector público en la interacción económica y social. Es crítico de la democracia liberal generalmente defendida por otros liberales, promulgada a través de un sistema político representativo. Toma el ejemplo de un posible debate entre una propuesta de aprobación por supramayoría en el parlamento de aumentos de impuestos, y una propuesta alternativa que sometería esta decisión a un referéndum popular: los socialistas de cualquier tipo se opondrían a la primera propuesta sobre principios supuestamente democráticos. porque se violarían los derechos de las minorías; pero no pudieron oponerse a la segunda, por estar basada en el propio electorado.

Está claro que las posiciones intelectuales que se basan, respectivamente, en las filosofías de Rousseau y Buchanan se traducen en ideologías económicas de un tipo muy diferente. El primer teórico se basa en una visión global de la identificación del individuo con el Estado. ¿Estamos lejos de la afirmación de Hegel ---varias décadas después de Rousseau--- de que es a través del Estado como el individuo disfruta de su libertad? (Capítulo 1 ). Lo que surge del contrato social de Rousseau y el discurso de la economía política No es solo la centralidad del Estado en la educación pública, sino también un sistema económico orientado a la tributación progresiva y al igualitarismo, y una perspectiva política y económica basada en las necesidades e intereses nacionales, y desconfiado de cualquier forma de globalización. Buchanan, por el contrario, ve la democracia directa ligada a su constitucionalismo, y como la verdadera encarnación de un genuino pensamiento liberal, basado en un papel extremadamente reducido del Estado y, paralelamente, en un amplio territorio para la expansión de la libre iniciativa. , ya que las pocas limitaciones derivadas de la coacción por parte del Estado son en sí mismas el resultado de la elección colectiva del hombre.

La ilustración de diferentes enfoques filosóficos de la democracia directa podría terminar con Rousseau y Buchanan. Pero, como se señaló anteriormente, el nombre de Carl Schmitt también se menciona con frecuencia en este debate. ``Los juristas chinos, los nacionalistas rusos, la extrema derecha de EE. UU. Y Alemania, así como la extrema izquierda de Gran Bretaña y Francia, se basan en el trabajo del principal teórico del derecho de la Alemania nazi'', durante mucho tiempo. considerado ``como más allá de lo pálido''. 92

El argumento que puede vincular a Schmitt con Rousseau y Buchanan es que los tres exaltan la democracia directa sobre la democracia representativa. Pero Schmitt va mucho más allá de esta contraposición, porque el resultado de su esquema intelectual es una forma de gobierno que abandona el concepto mismo de democracia, al recortar cualquier mandato permanente que el pueblo le dé al líder. Al comienzo del Tercer Reich, escribe que la nueva Ley de Habilitación de 1933, que marca el comienzo del Reich de Hitler, aunque se presenta formalmente como un cambio a la anterior constitución de Weimar, débil y ``neutral'', representa un cambio radical: la La ley ha sido decidida por el parlamento sólo en obediencia a la voluntad del pueblo expresada en las elecciones políticas que acaban de celebrarse; en realidad es un referéndum popular, un plebiscito, que reconoce a Hitler como líder político del pueblo alemán. 2 .

La dictadura implica un ``estado de excepción'' 93 , que es la suspensión de la ley y la limitación de la libertad individual. En el estado de excepción, soberano es quien decide sobre la excepción; la excepción separa la norma de su aplicación, para preservar su sustancia y hacerla efectiva. La dictadura soberana es el órgano de un poder constituyente. Por ejemplo, Schmitt vio este poder constituyente en la Revolución inglesa del siglo XVII, cuando Cromwell estableció una dictadura militar que no dependía de ningún organismo superior, y la transformó en un poder soberano genuino, ya no delegado ni provisional, sino permanente y absoluto. 94Schmitt teorizó la dictadura como un régimen anónimo (un régimen fuera de la ley): su estado de excepción significa la suspensión del estado legal, acompañada de restricciones a la libertad personal y la eliminación de ciertos derechos fundamentales, para establecer un nuevo orden. 95

Schmitt opone la legalidad a la legitimidad. El primero alcanzó su máxima expresión en el liberalismo del siglo XIX, pero no tiene un contenido efectivo, ya que resultó impotente en la República de Weimar en Alemania. El Estado ya no podía limitarse a la aplicación de la ley, sino que exigía decisiones urgentes y contundentes que solo podía tomar un líder que ejerciera una dominación carismática. Su acción encontraría en sí misma legitimidad: ``el ethos del derecho tenía que dar paso al patetismo de la acción''. 96

El choque entre un liberal que apoya la democracia directa y un populista extremo que al final aísla la autoridad del dictador del control de la gente que lo había elegido, es bien visible en la crítica de Buchanan a los dictadores populistas, 97 es decir, a ``aquellos que, en Al mismo tiempo, pretenden ser defensores de la democracia, en cierto sentido electoral, y temerosos del demos. Las personas que pertenecen a este grupo se opondrán con vehemencia a la democracia directa en todas sus formas, y querrán restringir el papel del pueblo a la selección de los gobernantes \ldots{} Una vez elegidos electoralmente, no hay pretensión de que el gobernante esté `representando' al pueblo en todo \ldots{} Y dado que cualquier gobernante está implícitamente modelado como haciendo el bien, no debería haber una base razonada para imponer límites o restricciones a su acción''. 98 Esta visión de Buchanan puede verse como un rechazo sin reservas a cualquier régimen fascista, si tenemos en cuenta el origen electoral de los sistemas de gobierno de Mussolini o Hitler, o de los más recientes.

\hypertarget{causas-del-aumento-del-populismo}{%
\subsection*{Causas del aumento del populismo}\label{causas-del-aumento-del-populismo}}
\addcontentsline{toc}{subsection}{Causas del aumento del populismo}

Si pasamos del territorio de la filosofía política a las motivaciones que están detrás del auge del populismo, su predominio está estrictamente conectado a fases de descontento, empobrecimiento y antagonismo de clase. En el siglo XX, vemos que en tiempos caracterizados por dificultades económicas y sociales, provocadas por guerras o crisis económicas profundas, amplios estratos de la población pudieron encontrar una doctrina, un credo que supuestamente liberaría al pueblo de las penurias y guiaría a la población. personas hacia niveles más altos de bienestar, justicia social y una afirmación del orgullo nacional. Este credo se encontró en el nacionalismo, y el nacionalismo fue personificado por un líder ``schmittiano'' que eliminaría a las élites liberales anteriores de sus puestos de mando y dejaría de referirse al pueblo por su autoridad continua. No es necesario recordar aquí detenidamente el nacimiento del fascismo en Italia y del nacionalsocialismo en Alemania. Ambos se basaron principalmente enEl apoyo de la pequeña burguesía , una clase fuertemente golpeada por la inflación y la pérdida de posición social, y desmoralizada por las dificultades de la Primera Guerra Mundial y sus secuelas: una clase que esperaba recuperar, a través de un dictador, una posición primaria en sus respectivas naciones.

El socialismo soviético puede verse en sí mismo como una versión extrema de una dictadura de tipo schmittiano. Si consideramos sus estructuras políticas, la principal diferencia con respecto a los otros dos regímenes es que el socialismo soviético fue el resultado de una revolución obrera en un país todavía semifeudal, donde las estructuras liberales estaban en su infancia: un país golpeado por pérdidas masivas de guerra, mientras que la propaganda bolchevique logró convertir el conflicto en una vergonzosa guerra imperial. Pero el líder indudablemente tenía una posición schmittiana.

De manera igualmente incuestionable, ninguna democracia de ningún tipo podría atribuirse a ninguna de estas estructuras políticas. Solo los países que se apoyaban en una tradición liberal de larga data, que se habían movido gradualmente hacia sistemas constitucionales liberal-democráticos, pudieron resistir estas tendencias populistas.

Para trasladarnos al presente, ¿qué queda del pensamiento de esos dos teóricos de la democracia directa, en la ola populista generalizada de nuestros días? ¿Y atrae la doctrina schmittiana a los populistas actuales? En primer lugar, parece que las formas de democracia directa teorizadas por Buchanan ---democracia directa como instrumento para lograr un ``gobierno pequeño'' y devolver al individuo el poder de decisión--- encuentran una audiencia muy restringida. Lo que podría haberse teorizado en el apogeo del neoliberalismo, como un paso más para liberar a las personas y las empresas del Leviatán, ha perdido cada vez más atractivo en grandes estratos de la población. Una predicción formulada por los críticos marxistas cuando se publicó el ensayo de Fukuyama parece plausible: que ``es poco probable que el actual triunfo global del capitalismo liberal sea un asunto duradero.99

Hay, más bien, un atractivo genérico para el Estado, que se basa sólo en unos pocos factores: ingresos estancados o en declive; globalización; frustración de la clase media. Como resultado, el populismo económico toma una forma definida por Barry Eichengreen en los siguientes términos: ``un enfoque de la economía que enfatiza la distribución mientras resta importancia a los riesgos para la estabilidad económica de los fuertes aumentos en el gasto público, las finanzas inflacionarias y las intervenciones gubernamentales que anulan las funcionamiento del mercado''. 100 Dentro de la Unión Europea, los populistas han cuestionado las políticas que se inclinan hacia la austeridad y posiblemente la deflación. 101Particularmente en la eurozona, un amplio programa de inversiones, basado en títulos de deuda con garantía conjunta, ha despegado muy recientemente en medio de la pandemia; y queda por ver si seguirá siendo una medida única.

Consideremos los factores que están detrás del populismo actual, partiendo del estancamiento de los ingresos y su supuestamente desigual distribución. La cola de la reciente Gran Recesión sigue moviéndose, no solo en la Unión Europea, sino en todas las democracias liberales en general, con políticas de austeridad y deflacionarias que impiden una recuperación sustancial, o más bien contribuyen al estancamiento o disminución de los ingresos, en varios países. Esta tendencia hacia una producción estancada o decreciente va acompañada de lo que varias estadísticas indican que es una desigualdad creciente, que ha revertido una tendencia anterior y opuesta a principios del siglo XX. 102

No es sorprendente que las formas de democracia directa, que podrían recordar la visión rousseauniana de la Voluntad General, parezcan estar en sintonía con los sentimientos de grandes estratos de la población. Sin embargo, uno puede dudar en hablar, al referirse a los desarrollos actuales, de una visión ideológica ``estatista'' bien articulada como subyacente a las tendencias populistas actuales. Más allá de la crítica confusa de la democracia representativa, ¿hay algún indicio de esquemas ideológicos de algún tipo que puedan apoyar iniciativas políticas coherentes basadas en formas directas de democracia? Es casi en broma que se pueda citar el hecho de que la plataforma electrónica a través de la cual los miembros del movimiento ``5 estrellas'' en Italia expresan su voto, lleva el nombre de ``Rousseau'' (el filósofo podría estar revolcándose en su tumba).

En el ámbito internacional, el populismo puede verse como un subproducto de la globalización. La globalización crea un ``espacio monetario descentralizado'', no organizado ni controlado por una autoridad central. Este espacio tiene dimensiones geográficas, económicas, competitivas y financieras: el dinero se mueve con poca fricción y bajos costos; las personas y las empresas pueden comprar bienes y servicios en todo este espacio con barreras limitadas o nulas; en consecuencia, el mercado global está abierto a la competencia más feroz; y los activos financieros se pueden comprar a través de las fronteras. 103

La globalización, con estas características, ha abrumado política, social y económicamente a los países capitalistas occidentales. En referencia a este último punto, ha llevado al emprendimiento de esos países a trasladar enormes inversiones directas hacia áreas que cuentan con una fuerza laboral relativamente calificada, pero que aún se encuentran en un estado atrasado en términos de niveles de ingresos y salarios (me viene a la mente el continente asiático). . Los bienes producidos en esas áreas se exportan a los países avanzados de Occidente, a precios bajos. En consecuencia, en estos países, el tamaño de la economía, en particular del sector industrial, afectado por las enormes importaciones, no puede crecer y, a menudo, se contrae; y el poder de negociación de su mano de obra y sindicatos se ve afectado en consecuencia. La reducción relativa del sector industrial no va acompañada de una reducción paralela del sector de servicios, menos afectado por la competencia extranjera, estando más orientado hacia el interior; pero, aquí, es la estructura descentralizada del sector de servicios lo que hace a los sindicatos intrínsecamente más débiles (como ejemplos extremos, ver ``Uber'' o ``Deliveroo'').

Cualquier impulso por mejores niveles de vida, tanto en el sector industrial como en el agrícola, se ve frustrado por la presencia potencial de trabajadores que vienen de las regiones más pobres, principalmente de África. Esta mano de obra de bajo costo podría posiblemente ser considerada como un factor positivo por los industriales para recuperar la competitividad en el mercado global (y, más aún, por los compasivos humanitarios de izquierda). Pero, al mismo tiempo, la presión de la inmigración actúa como una especie de tapa sobre las demandas de cualquier mano de obra local por salarios reales más altos. De ahí que gane terreno la fuerte postura antiinmigratoria adoptada por esta fuerza de trabajo y el nacionalismo, que exige el levantamiento de nuevas barreras, tanto a los bienes como a los trabajadores.

Tales fuerzas pueden desacreditar la democracia representativa y producir un llamado a un líder - ¿un líder schmittiano? - que podría proteger al país de enemigos opuestos: las limitaciones de las políticas económicas dirigidas a la estabilidad y sostenibilidad de la deuda y, a nivel mundial, la erosión de las cuotas de mercado para la producción nacional y la presión de la inmigración masiva. Ambos conducen a una mayor atención al interés nacional: el primero, desde un punto de vista de izquierda, en el sentido de una mayor intervención del Estado en la economía; los segundos, desde una perspectiva de derecha, buscando proteccionismo y barreras a la inmigración. Me viene a la mente el gran proteccionista del siglo XIX, Friedrich List (Capítulo 1 ), así como el ordoliberalismo alemán (Capítulo 2 ), en particular con referencia a la política económica de la UE.

Un tercer factor del auge del populismo mencionado anteriormente es la frustración de la clase media. También deben tenerse en cuenta factores no económicos. ``Las antiguas distinciones de intereses económicos y de clase no han desaparecido, pero están cada vez más superpuestas por una más amplia y flexible: entre la gente que ve el mundo desde cualquier lugar y la gente que lo ve desde algún lugar''. 104En otras palabras, a menudo existe la sensación de haber sido dejado atrás por aquellos que han podido progresar socialmente. Esta sensación de haber quedado atrás, se extiende a personas que no tienen motivos concretos para quejarse desde el punto de vista económico. Las personas de ingresos medios, que tienden a diferenciarse de la ``clase trabajadora'', han desarrollado un sentimiento de ira hacia las élites, de exclusión de los grupos sociales y culturales que realmente importan en la vida de una comunidad: no lo es, estrictamente hablando. , cuestión de ingresos y riqueza, pero de ascenso al éxito social y la meritocracia; es una especie de ansiedad por el estatus social. Nuestra sociedad neoliberal, al exaltar ``el éxito'' y poner a ciertas personas en lo más alto de la escala social, una escalera, vale la pena repetirlo, que no necesariamente está vinculado a clases sociales bien definidas, ha generado inexorablemente un sentimiento de frustración en esos grandes grupos de población que, aunque lejos de la base de la escala económica, tienen un sentimiento de no ``pertenencia''. Esa gente ve la meritocracia como un fraude a expensas de los excluidos.105 Una vez más, este no es un hecho sin precedentes. Como se mencionó anteriormente, si retrocedemos en la historia y pensamos en el fascismo italiano y el nazismo alemán, y en supequeñoapoyoburgués, es razonable establecer comparaciones convincentes. Y nuevamente, un territorio creciente aparece cubierto por la larga y siniestra sombra de Carl Schmitt. Esta búsqueda de un líder que pueda dar a la gente un sentido de satisfacción social y orgullo nacional es más fuerte en países donde la tradición liberal de democracia parlamentaria es débil o inexistente.

\hypertarget{el-impacto-de-las-redes-sociales}{%
\subsection*{El impacto de las redes sociales}\label{el-impacto-de-las-redes-sociales}}
\addcontentsline{toc}{subsection}{El impacto de las redes sociales}

Pero hay otro aspecto del populismo que merece consideración, cuyas implicaciones económicas y sociales pueden ser de mayor alcance que las mencionadas anteriormente. Supongamos que, de alguna manera, el populismo puede generar una suerte de Voluntad General, aparentemente en el sentido rousseauniano. Un factor importante que ha recibido mucha atención recientemente es el impacto de las redes sociales. Pueden plantearse dos preguntas al respecto: cómo se forma el consenso en la era actual de las redes sociales; y si podría haber un líder ``schmittiano'' oculto que podría conducir este consenso en direcciones que están lejos de los impulsos nacionalistas, y más bien señalar serias distorsiones en el sistema de mercado y el proceso político.

El embrutecimiento de la mente mediante el uso de breves mensajes electrónicos, de ``tweets'', de opiniones o insultos demasiado claros (a menudo anónimos), de ``me gusta'', de imágenes de cualquier tipo, de emojis, está enmascarado por sus Aspectos muy entretenidos y de rápido alcance: parece que nunca hubiéramos tenido tal nivel de libertad de expresión, de satisfacción instantánea y un enorme poder para influir ---todos juntos--- en cómo se desarrollan las cosas en nuestro mundo: un sueño de democracia directa. Los nombres fantasiosos e infantiles que se le dan a estas redes sociales no hacen más que aumentar su atractivo: ¡todo es tan simple, como un juego de niños! Y las utilizamos incluso cuando hablamos de los temas más complejos que tenemos ante nosotros, ya sean políticos, sociales o de contenido altamente técnico, y, en general, en el total desconocimiento del tema específico en cuestión, cualquier resultado es posible. La falta de experiencia y competencia específicas es una cuestión de orgullo, más que una desventaja (un político británico, un ex miembro del gabinete, ha dicho recientemente que la gente ``ha tenido suficiente de expertos''). Paralelamente va la circulación decreciente de los periódicos y otros medios: como sabemos, pueden estar disponibles en Internet, pero sus historias son demasiado largas y elaboradas, y nos obligan a dedicar un tiempo a ejercitar la mente: es mucho más sencillo. y más rápido para confiar en un ``tweet''.

Esto sería suficiente para motivar una visión crítica de estas redes: la costumbre que tenemos ahora de simplificar los temas más complicados que tenemos ante nosotros, y por tanto polarizar, en extremo, diferentes posiciones (incluso sin considerar el tiempo perdido en interminables charlas). ). Debe intentarse un análisis de costo-beneficio. Los indudables beneficios de estas redes deben evaluarse teniendo en cuenta los aspectos críticos antes mencionados. Admito que este análisis nunca se llevará a cabo: incluso si ignoramos su enorme impopularidad, ¿quién debería ser el juez imparcial de los resultados?

En un discurso pronunciado en octubre pasado en la Universidad de Georgetown, 106 Mark Zuckerberg, CEO de Facebook, presentó su filosofía a la audiencia. Colocó su red en la corriente principal de la democracia estadounidense, con apelaciones apasionadas a la Ilustración, la Primera Enmienda, los padres fundadores, Martin Luther King, un fallo de la Corte Suprema, The New York Timesy \#MeToo (que ``se volvió viral en Facebook''). Quiere que ``más personas compartan sus experiencias\ldots{} dando voz a todos''. ``Muchas de las historias que la gente ha compartido hubieran sido ilegales incluso si las hubiera escrito''. Esto ``empodera a los que no tienen poder y empuja a la sociedad a mejorar con el tiempo \ldots{} con Facebook, más de dos mil millones de personas ahora tienen una mayor oportunidad de expresarse y ayudar a otros \ldots{} a escala. {[}Esto{]} es un nuevo tipo de fuerza en el mundo \ldots{} un quinto poder junto con las otras estructuras de la sociedad. La gente ya no tiene que depender de los guardianes tradicionales de la política y los medios para hacer oír su voz''.

Facebook ---continúa Zuckerberg--- coloca a las personas en la mejor posición posible para enfrentar las tensiones sociales vinculadas a los cambios económicos masivos que surgen de la globalización y las nuevas tecnologías, las consecuencias de la crisis financiera de 2008 y la reacción polarizada ante una mayor migración. Facebook hace todo lo posible para contrarrestar el terrorismo, la pornografía, la violencia organizada, la interferencia en las elecciones políticas y no fomentar los contenidos polarizados, que conducen a comunidades antagónicas (agrega: ``los votantes más polarizados en las últimas elecciones presidenciales fueron las personas menos propensas a usar Internet'').

Quiere evitar el riesgo de ``poner en peligro a la gente''. Los usuarios están protegidos por un equipo de 35.000 empleados, dotados de sistemas de IA que pueden detectar el riesgo de autolesión en minutos, con especial foco en el bienestar de las personas; mientras que una Junta de Supervisión Independiente está disponible para apelar ``nuestras decisiones de contenido''. Su conclusión: ``Creo en las personas''.

Ese discurso expone claramente las dudas sobre la democracia representativa tradicional. Tampoco menciona nunca la ``democracia directa'', sino que parece hablar de la suma de determinadas voluntades. Esto es diferente de la Voluntad General de Rousseau, que mediante una interpretación de ``demos'' puede identificarse con el Estado (ver arriba). Por otro lado, no se sugiere ningún apoyo para un liberalismo libertario inspirado en Buchanan.

Hace algunos años, una escritora británica, Eliane Glaser, planteó una perspectiva crítica, generalmente en relación con Internet. 107Ella, distanciándose de la opinión predominante, según la cual los años posteriores al ensayo de Francis Fukuyama habían dado evidencia de que el pronóstico de Fukuyama era rotundamente erróneo (que en realidad la historia no puede terminar), se preguntó si Fukuyama, paradójicamente, tenía razón. Según ella, en lugar de considerar la prevalencia de la democracia liberal occidental, Fukuyama en realidad estaba presentando ``una forma de encubrir la política de derecha con un disfraz benignamente incontestable''. Señaló el papel decisivo de Internet. En su opinión, ``el capitalismo pretende amar el libre mercado; en realidad, manipula mercados para las élites \ldots{} {[}L{]} a derecha ha construido sistemáticamente un movimiento ideológico que se presenta como todo menos sistemático, todo menos ideológico''. La política se presenta como una cuestión de optimización tecnológica, de hacer un buen trabajo con lo que nos ofrece la tecnología en constante avance. ``En una era post-ideológica\ldots{} ¿Internet es un síntoma o una causa? Cuando cada persona en un vagón de tren está mirando un pequeño dispositivo iluminado, es casi una visión de mal gusto de la distopía \ldots{} el consumismo digital nos vuelve demasiado pasivos para rebelarnos \ldots{} Si lo aceptamos como inevitable, de hecho conducirá al final de la historia, En más de un sentido''.

En la medida en que esta visión pueda conducir a una sociedad despolitizada, este tipo de sociedad podría terminar siendo un instrumento de un dictador schmittiano: no más feliz de tener frente a él grandes y vociferantes reuniones de personas que lo aplauden, como en los regímenes fascistas de nuestro pasado---, pero muchedumbres silenciosas que siguen en sus pequeños dispositivos sus direcciones y favorecen inconscientemente sus propios intereses.

Este modelo de sociedad se ve claramente en las reflexiones de David Runciman. En una visión aún más sombría y extrema, Runciman, en How Democracy Ends---Presenta al lector la guerra que una democracia liberal tiene que librar constantemente contra los excesivos poderes corporativos del mercado y su tendencia a mezclarse con las instituciones políticas. Con el tiempo, y particularmente en nuestros días, esta lucha ha cambiado y se ha vuelto más difícil. Observa que, para las instituciones de la democracia, el peligro de perder el control de los gigantes corporativos del pasado es menos crítico con referencia a los gigantes corporativos de hoy, representados por las redes sociales. Escribe que ``son bestias muy diferentes de Standard Oil. Monopolizan muchas cosas a la vez. Producen cosas de las que hemos llegado a depender de nuestra vida diaria; influyen en lo que nos decimos, dando forma a lo que vemos y oímos''. 108

Si esta interpretación extrema fuera correcta (un fuerte ``si'', lo admito), la estructura económica de la sociedad se basaría en un sistema cuasi-monopolista, en el que pocos gigantes corporativos de las redes sociales tendrían el poder de abordar nuestras elecciones de alguna manera aparentemente imparciales pero sustancialmente guiados por ellos mismos. Esto sería cierto no solo con referencia a las elecciones de nuestros consumidores, sino también a nuestras preferencias políticas.

En el capítulo 2 de este ensayo, mientras se abordan las ``externalidades'' de Pigou, el hecho de que los servicios de redes sociales se proporcionen casi de forma gratuita a sus consumidores se menciona como una especie de ``externalidad positiva'', el excedente de un consumidor, medido por la precio de mercado que los consumidores pagarían de otro modo. La objeción a este razonamiento es que, como en muchos otros casos, las necesidades de los consumidores son creadas artificialmente por las mismas empresas que proporcionan el producto adecuado, y que, por otro lado, existe un excedente del productor generado por la enorme cantidad de datos que están disponibles. a él.

Una de las principales tareas que tiene ante sí la clase política, tal como lo expresan nuestras instituciones representativas, es hacer frente a este inminente desarrollo económico y social, y dar evidencia de que miran el problema, cuidando eficazmente la democracia liberal que les ha tocado. están llamados a defender, y evitando una simple impresión de colusión con los poderes que están llamados a regular. Ésta es la única forma de rescatar a una democracia representativa de todas las acusaciones ---a menudo, justamente levantadas--- de haber ``traicionado'' al pueblo. Lo que viene inmediatamente a la mente son algunos temas que son fundamentales para la supervivencia, tanto política como económica, de una sociedad liberal, y deberían llamar la atención de los consumidores y los reguladores sobre problemas que ya están maduros para ser resueltos:

\begin{quote}
mala conducta en materia de competencia y legislación antimonopolio;

protección de la privacidad, que es un enfoque sin escrúpulos de los derechos de privacidad;

la elusión fiscal mediante la explotación de paraísos fiscales;

papel de las redes sociales como editores. Este problema se deja de lado al describir la red como una ``plataforma''. Lo que esta palabra significa en terminología jurídica no me queda claro. Como cualquier editor, las redes sociales no deben ignorar la responsabilidad de lo que se hace público en la red.
\end{quote}

Pero, sobre todo, está el problema de la enorme concentración de poder económico y político, como lo atestigua la abrumadora participación de la valoración del capital de estas empresas de tecnología sobre el valor total del mercado de valores.

\hypertarget{notas-2}{%
\section*{Notas}\label{notas-2}}
\addcontentsline{toc}{section}{Notas}

\begin{enumerate}
\def\labelenumi{\arabic{enumi}.}
\item
  Véase De Cecco, capítulo I.
\item
  Streek2016, págs. 1-3).
\item
  Véase, por ejemplo, Johnson (1965) y Levi-Faur (1997).
\item
  Johnson, págs.172 y 183.
\item
  Fukuyama (1989).
\item
  pag. 3.
\item
  Si la admiración de Hegel por Napoleón, al derrotar a Prusia en la batalla de Jena, significa que vio ese evento como ``el fin de la historia'', es discutible. Fukuyama escribió como activista político para el Departamento de Estado de EE. UU. Más que como historiador.
\item
  págs. 9 y 14-15.
\item
  De todos modos, Schumpeter, escribiendo en 1954, no usa este término con referencia a la Teoría General de Keynes .
\item
  Schumpeter (1954, pag. 754). Es en este sentido que tenemos que entender el adjetivo ``clásico'': nada que ver específicamente con la ``Escuela Clásica'' de Adam Smith y otros economistas (incluso si, de hecho, la suya era, de una manera schumpeteriana, una ``escuela clásica'' situación").
\item
  Se trata de ``situaciones clásicas'', según Schumpeter.
\item
  Ibídem.
\item
  Lucas y Sargent (1979, pag. 1).
\item
  Keynes, Teoría general , pág. 378.
\item
  Este factor es enfatizado por Heilbroner y Milberg (1995, cit, pág. 57).
\end{enumerate}

dieciséis.
Teoría general , pág. 291.

\begin{enumerate}
\def\labelenumi{\arabic{enumi}.}
\setcounter{enumi}{16}
\item
  Ver, sobre este último punto: Leijonhufvud (1983) (el autor es un keynesiano medio arrepentido).
\item
  pag. 5.
\item
  La ``curva de Phillips'' se definió como ``un hallazgo empírico en busca de una teoría'' (James Tobin, citado por Heilbroner y Milberg, cit, p.~52).
\item
  Johnson (1971).
\item
  págs. 5-6.
\item
  Por ejemplo, uno de los keynesianos estadounidenses, Alvin Hansen, planteó la teoría del ``estancamiento secular'', para enfatizar que una insuficiencia de demanda efectiva siempre prevalecería, sería ``estructural''.
\item
  Johnson, pág. 6.
\item
  En Italia, por ejemplo, donde la inflación cruzó en un punto el nivel del 20\%, un comentario generalizado fue que la democracia no podría sobrevivir si la inflación se mantuvo por mucho tiempo por encima de esa tasa.
\item
  Harvey (2007, pag. 2).
\item
  Samuelson (1997). De alguna manera más amable que Samuelson, George Stigler escribió: ``La competencia es una mala hierba, no una flor delicada''.
\item
  ``Luego hibernando, como una secta exótica, en los Estados Unidos y Gran Bretaña'' (Streek, cit, p.~154). Pero el británico Lionel Robbins es uno de ellos.
\item
  Collier2018, pag. 13).
\item
  Deaton2020, pag. 2).
\item
  Enciclopedia de Filosofía de Stanford ( plato.stanford.edu/contractarianism ).
\item
  ``La causa utilitaria fue promovida por economistas; la causa de los derechos fue promovida por abogados'', escribe Collier, p.~13. Pero esto no es del todo cierto: Richard Posner, un jurista, estaba muy cerca de las opiniones libertarias de Buchanan; Joseph Stiglitz, solo por nombrar un economista, está más cerca de Rawls.
\item
  Kant1981 {[}1785{]}, págs. 12-13).
\item
  Rawls (1971).
\item
  pag. 17.
\item
  pag. 13.
\item
  pag. 13.
\item
  Pero de esta manera descuidan, al menos, a GF Hegel.
\item
  Buchanan y Tullock (1965 {[}1962{]}, págs. 111-112).
\item
  pag. 266.
\item
  Véase, por ejemplo, Boettke (2011).
\item
  Buchanan y Tullock hablan de la ``naturaleza interdisciplinaria del libro'' (p.~VI).
\item
  Hay un acento benthamiano en esto (ver Capítulo 1 ).
\item
  Ibidem, págs. 13-14.
\item
  Buchanan1999 {[}1969{]}, págs. 41-42).
\item
  Buchanan1990).
\item
  Buchanan y Tullock, cit, pág. 20.
\item
  Buchanan1990, cit, págs. 4-7).
\item
  Hobbes's (1909 {[}1651{]}, pag. 132).
\item
  Buchanan1990, pag. 12).
\item
  Ibidem, pág. 12.
\item
  Buchanan1959).
\item
  pag. 127.
\item
  pag. 128.
\item
  Musgrave1969).
\item
  Buchanan, pág. 130.
\item
  Posner1987, pag. 21).
\item
  pag. 22.
\item
  Posner2009).
\item
  Milton Friedman escribió que el desempeño de la economía como ciencia debe ser juzgado por la ``conformidad con la experiencia de las predicciones que produce''. Se reformuló la teoría cuantitativa enfatizando el dinero como un activo que se puede comparar con otros activos, dentro de un ``análisis de cartera'', un análisis del balance de las personas, es decir, del tipo de activos que quieren tener. Ver Friedman (1968 {[}1964{]}, págs. 357-359).
\item
  Ver Axelrod (2011, capítulo 5) y Volcker (2018).
\item
  Según la definición de Milton Friedman de la economía como una disciplina positiva (ver Prefacio).
\item
  Los individuos se asumieron implícitamente como ``demasiado tontos''. Ver Anderson (1978).
\item
  pag. 5.
\item
  Lucas y Sargent, cit.
\end{enumerate}

sesenta y cinco.
Es decir, la oferta de cualquier bien encuentra una correspondencia exacta con la demanda.

\begin{enumerate}
\def\labelenumi{\arabic{enumi}.}
\setcounter{enumi}{65}
\item
  Heilbroner, Milberg, cit, págs.81 y 83.
\item
  Según la definición del Banco Central Europeo.
\item
  Acharia y col.~(2009, pag. 5).
\item
  La crisis de los préstamos se extendió a personas con historial crediticio empañado.
\item
  Cassidy2010).
\item
  de Voltaire1937 {[}1759{]}, pag. 22).
\item
  Samuelson, cit.
\item
  Keynes (1937).
\item
  Grillete (1953, pag. 227).
\item
  Porque, en realidad, poco tiene que ver con la Escuela Clásica de Smith o Ricardo.
\item
  Stiglitz (2009, pag. 238).
\item
  Skidelsky2009).
\item
  Israel2014, pag. 21). Véase también Kelly (2015).
\item
  Israel, cit, págs.23, 216, 348 y 358.
\item
  Rousseau, JJ: El contrato social , cit, págs. 22-23.
\item
  pag. 267.
\item
  pag. 268.
\item
  pag. 269.
\item
  pag. 280.
\item
  pag. 281.
\item
  Sobre la democracia directa de Buchanan como antítesis de la de Rousseau, véase Shearmur (2010).
\item
  Buchanan2001).
\item
  pag. 237.
\item
  pag. 238.
\item
  Este es un tema que ya hemos tocado en la Secta. 4.2 , cuando se trata de la preocupación de Buchanan por una constitución históricamente creada, cuyo contenido real puede estar lejos de los principios de una economía de libre mercado.
\item
  pag. 239.
\item
  Rachman2019).
\item
  Schmitt (2013 {[}1921{]}).
\item
  Traverso (2016, págs. 95-96).
\item
  pag. 238.
\item
  págs. 228-230.
\item
  Sin embargo, Buchanan no menciona explícitamente a Schmitt (quizás, en el momento de escribir, 2001, el nombre de Schmitt no estaba tan de moda ).
\item
  Buchanan, Democracia directa, liberalismo clásico y estrategia constitucional, cit, p.~236.
\item
  Marxism Today , noviembre de 1989.
\item
  Eichengreen (2018, pag. 5).
\item
  Vale la pena recordar que, en la Alemania de entreguerras, las políticas liberales ortodoxas del canciller Brüning favorecieron implícitamente la tendencia del electorado a inclinarse hacia la extrema derecha.
\item
  Desigualdad de ingresos, medida con el coeficiente de Gini
\end{enumerate}

Reino Unido

EE.UU

Francia

Italia

Alemania

2010

33,66

45,60

30.30

34,70

28.00

1980

25,70

37,85

32,56 (1979)

32,50

24,73 (1978)

Fuente ourworldindata.org

\begin{enumerate}
\def\labelenumi{\arabic{enumi}.}
\setcounter{enumi}{102}
\item
  Pringle2020, pag. 113).
\item
  Goodhart (2017, págs. 3-4).
\item
  Kuper (2020).
\item
  Zuckerberg (2019).
\item
  Glaser (2014).
\item
  Runciman (2018, págs. 132-133).
\end{enumerate}

\hypertarget{part-visiones-alternativas}{%
\part{Visiones alternativas}\label{part-visiones-alternativas}}

\hypertarget{como-lo-veo-yo}{%
\chapter*{Como lo veo yo}\label{como-lo-veo-yo}}
\addcontentsline{toc}{chapter}{Como lo veo yo}

El último capítulo investiga el vínculo entre liberalismo e historicismo (Benedetto Croce), en la interpretación de la actividad económica. Significa que las elecciones, incluso las económicas, que tenemos que hacer, son históricamente específicas y requieren continuamente nuevos enfoques y soluciones. Las acciones individuales, así como las instituciones y políticas, se desarrollan históricamente y ninguna de ellas debe tomarse como valores a mantener indefinidamente. ``Una sociedad libre permite una gran variedad de opiniones encontradas'' (Isaiah Berlin). Un sistema económico puede considerarse liberal si es coherente con la afirmación de la libertad individual, como emergente en circunstancias históricas específicas. Las creencias anteriores, vistas como la respuesta final a los problemas económicos, han sido decepcionadas: una sociedad capitalista es liberal aunque adaptativa. El equilibrio entre la libertad negativa, que da prioridad al interés personal irrestricto y los fuertes derechos de propiedad del individuo, y la libertad positiva, que reserva un amplio papel al Estado, debe ser reevaluada adecuada y continuamente de acuerdo con las circunstancias históricas cambiantes. La disciplina de la economía probablemente se beneficiaría de las inyecciones de historicismo e institucionalismo, saliendo de esquemas de teorización demasiado abstractos, aunque formalmente perfectos.

Palabras clave

\begin{itemize}
\tightlist
\item
  Historicismo
\item
  Libertad positiva y negativa
\item
  Institucionalismo
\end{itemize}

\hypertarget{una-visiuxf3n-liberal}{%
\section*{Una visión liberal}\label{una-visiuxf3n-liberal}}
\addcontentsline{toc}{section}{Una visión liberal}

Hemos visto en el capítulo 2 que el filósofo italiano Benedetto Croce quería reconstruir los cimientos del liberalismo, convencido como estaba de la insuficiencia de la teoría liberal convencional, tal como maduraba en el siglo XIX. Esta teoría se basó en el orden natural de la Ilustración o en los principios utilitarios del positivismo.

Croce reaccionó contra lo que llamó ``la orgía de la regularidad abstracta'' y los ``equilibrios más perfectos de la mecánica social'', que estaban detrás de la construcción teórica de los científicos sociales y, en el campo de la economía, de los economistas de la economía clásica y Escuelas neoclásicas: esquemas que se consideraron válidos en cualquier lugar y siempre, según los primeros; o como ``leyes'' que responden a regularidades inmutables del mundo físico, según este último. 1

Croce creía que en el historicismo se podía encontrar una base nueva y persuasiva para el liberalismo. Sin embargo, lo que tenía en mente era el historicismo ni del hegeliano dialéctico ni del marxista materialista. Ambas versiones quieren conducir a ciertos objetivos definidos que se realizan a través del proceso histórico: la afirmación del Estado como encarnación de la racionalidad y la libertad, o el advenimiento de una sociedad sin clases gracias a la revolución proletaria. Y, por eso, ambos son deterministas, porque consisten en una predicción histórica de lo que necesariamente va a suceder: una aproximación a las ciencias sociales que asume la predicción histórica como su objetivo principal. 2

Lejos de cualquier determinismo de este tipo, el historicismo significa, según Croce, que no podemos esperar una organización social específica que será la etapa final del progreso humano, una especie de equilibrio permanente tanto en la vida social como económica.

Su ``historicismo ideal'' significa que las situaciones problemáticas que tenemos frente a nosotros, las elecciones que tenemos que hacer en cualquier campo de actividad, son históricamente específicas y siempre requieren nuevos enfoques y soluciones. Este punto de vista puede verse como ---y, en cierto modo, es--- relativista. Como tal, aunque lejos del determinismo, parece también socavar cualquier confianza en los estándares permanentes suprahistóricos, que nos guían en nuestras decisiones: estándares permanentes que a menudo se consideran esenciales para el liberalismo.

El punto central de las reflexiones de Croce es que los desarrollos históricos se realizan a través de instituciones, políticas y acciones individuales, pero ninguno de ellos debe tomarse como una fuente absoluta de valores que deben mantenerse indefinidamente. El liberalismo de Croce se centra en la supremacía del individuo, en su conciencia y capacidad moral para actuar de conformidad con su sentido de la justicia (esto nos recuerda a John Rawls, que escribe sobre los hombres como seres racionales morales, con fines propios pero capaces de un sentido de justicia). 3Detrás de nuestro comportamiento concreto, siempre hay ``una voz que nos dirige hacia lo que debemos hacer, cuál es nuestra misión y nuestro deber: una voz que puede diferir para cada uno de nosotros, porque la historia necesita pensamientos diferentes y opuestos, que la historia misma cuidar de mediar y armonizar''. 4 Este punto de vista está en sintonía con la definición de sentido común del hombre liberal, como paradójicamente la describe el poeta Robert Frost: ``un liberal es un hombre demasiado amplio para ponerse de su lado en una discusión''. 5

Esto significa reconocer, a la luz de la conciencia histórica, la diversidad y los desacuerdos políticos, considerándolos como una fuente de cambio y crecimiento. De esta manera, ``Liberty - escribe Croce - no está ligada a ningún entorno particular de instituciones o tradiciones o condiciones económicas o cualquier otra cosa; todo esto {[}la libertad{]} puede utilizar para sus propios fines, según lo sugiera la situación y el proceso histórico''. 6 ``La verdad nunca es definitiva, porque toda verdad pone la premisa de nuevas posiciones intelectuales y, con ellas, de nuevas dudas, problemas y nuevas verdades''. 7Un liberal nunca alentará la esperanza de lograr soluciones permanentes: ``Quien refuta o satiriza a los apóstoles de la paz y la igualdad universales, lo hace para oponerse a los medios ingenuos e inadecuados que utilizan por la vaguedad e imposibilidad de sus supuestos; pero no refuta ni satiriza el trabajo honesto, perseguido por los buenos gobernantes en todo momento, para reducir en sociedades específicas desigualdades específicas; o por políticos sabios, para evitar conflictos armados y guerras''. 8

¿Qué pueden significar estas palabras abstractas, aparentemente abstrusas, en términos políticos? En las circunstancias históricas que habían acompañado décadas de debates intelectuales sobre el liberalismo, y en los tiempos turbulentos del siglo, ``Croce buscó restablecer las bases del pluralismo liberal y la tolerancia, y mostrar, en respuesta a la arrogancia del totalitarismo, por qué una medida de humildad debe rodear el compromiso político''. 9 En la misma línea, Isaiah Berlin escribió, muchos años después: ``La idea de que puede haber dos caras de una pregunta, que puede haber dos o más respuestas incompatibles, cualquiera de las cuales podría ser aceptada por hombres racionales y honestos - esa es una noción muy reciente''. ``El mérito de una sociedad libre es que permite una gran variedad de opiniones encontradas sin necesidad de supresión''. 10

¿Cómo puede esta teorización ser relevante para el polémico campo de las ideologías y visiones económicas como se discute en este ensayo? Como enfatizamos en el Capítulo 2 , Croce trató de elevar el liberalismo por encima de cualquier visión que pueda prevalecer históricamente en un momento y lugar específicos, y observó que un sistema económico específico puede considerarse liberal si es consistente con la necesidad de afirmar la libertad individual y el pluralismo. en diferentes circunstancias históricas.

Por lo tanto, Croce pensó que la cuestión de la libertad económica en un sistema capitalista (liberalismo económico, o ``liberismo'', como lo hemos llamado antes) podría, o no, ser consistente con el liberalismo tout court : la respuesta proviene solo de las circunstancias históricas particulares. vivimos con. En las circunstancias de su propio tiempo, ``trató de flexibilizar el liberalismo {[}económico{]} socavando su persistente antiestatismo y, en particular, su vínculo de larga data con la economía del laissez-faire''. 11

El choque entre el economista ``liberalista'' Luigi Einaudi y Benedetto Croce sobre el tema del liberalismo es un ejemplo interesante de un difícil diálogo entre el filósofo liberal y el economista liberal. Este debate se prolongó durante la década de 1930 y se hizo más directo y vehemente en las páginas de la Rivista di storia economica a principios de la década de 1940. Croce destacó la diferencia entre liberalismo y liberismo, que veía como la doctrina del laissez-faire. Los sistemas económicos son históricamente específicos y ninguno de ellos tiene derecho a pretender ser moralmente privilegiado e implícitamente superior. La teoría económica no puede ser, per se , una teoría de una sociedad liberal.

¿Qué inferencias podemos extraer de la adopción de una perspectiva histórica liberal, como la que acabamos de mencionar, sobre los problemas económicos concretos que tenemos ante nosotros hoy? Puede ser útil, en este punto, recordar la distinción, descrita al principio de este ensayo, entre una visión centrada en el individuo como agente racional y una visión del Estado como encarnación de la racionalidad, una visión en la que el individuo siempre ocupa el segundo lugar. La segunda visión está incrustada en ese tipo de historicismo que Croce quería rechazar, y será evidente que un enfoque croceano lo descartaría como antiliberal, porque niega la supremacía del individuo en todos los aspectos de la vida social y económica.

¿Dónde se puede encontrar un equilibrio? Berlín hizo una útil distinción entre libertad negativa (que significa: ¿cuántas puertas están abiertas para mí?) Y libertad positiva (que significa: quién está a cargo, para que mi desarrollo personal y participación plena en la vida de la colectividad ¿Se puede realizar mediante la intervención del Estado?). La libertad negativa se mide por la ausencia de obstáculos para hacer lo que quiero hacer, mientras que la libertad positiva se mide por el alcance de la intervención del Estado en la vida social y económica de un país y por los usos que se le da a dicha intervención. ``El ejercicio incontrolado de una libertad destruye la otra''. En términos de libertad negativa, es decir, sin ninguna intervención del Estado, esta libertad ilimitada puede ser tergiversada y puede, en extremo,12 La libertad positiva, igualmente, puede torcerse y conducir a la sociedad autoritaria o totalitaria. En estos dos extremos, el liberalismo está completamente descartado. Evitar estos extremos determina el grado de liberalismo en la sociedad. ``Debe haber un equilibrio entre las dos {[}libertades{]}, sobre el cual no se pueden enunciar principios claros''. 13 Y, en opinión de Berlin, el concepto de libertad positiva ---aunque, en sí mismo, esencial ``para una existencia decente'' - ha sido históricamente más abusado y pervertido que el de libertad negativa en el mundo moderno.

Volviendo a Croce, su rechazo a una sociedad estatista ---ya sea representada por el Estado ético o por el Estado proletario marxista--- significaría una firme oposición a la versión extrema de la libertad positiva, que necesariamente va acompañada de una coacción totalitaria. Pero si nos inclinamos hacia la libertad negativa, el problema es hasta qué punto podemos inclinarnos hacia ella.

Herbert Stein, escribiendo en 1990 y en cierto modo celebrando el colapso del comunismo, enfatizó que ``hay que traer una nota de realismo'' a la celebración. El núcleo central del capitalismo es la libertad, pero la libertad absoluta es imposible, y nuestras adaptaciones {[}del capitalismo estadounidense{]} no han ido todas en la misma dirección, algunas han sido guiadas por políticas públicas, otras por comportamiento privado. ``La genialidad del sistema es que ambos han tenido la libertad de adaptarse''. 14 Con un enfoque similar, Douglass North estudió las interacciones de los actores (individuos o grupos) y los arreglos institucionales (formas organizacionales) al considerar estas organizaciones como una parte integral del análisis económico en lugar de una adición descriptiva al análisis. 15

La mayoría de los pensadores políticos y economistas liberales, cualquiera que sea su filosofía política y económica, definitivamente estarían de acuerdo con la idea de una ``sociedad adaptativa''. La sociedad necesita diferentes corrientes de pensamiento, y en la disciplina de la economía, el énfasis puede ponerse en diferentes momentos y lugares en una u otra vertiente:

\begin{quote}
\begin{itemize}
\item
  Por un lado, la ambición individual se considera la fuerza impulsora de las economías de libre mercado. La orientación estatal obstaculizaría y limitaría el individualismo. Esto significa que debe darse al individuo una amplia posibilidad de perseguir sus propios objetivos, de modo que pueda tener éxito en sus esfuerzos y su recompensa económica pueda maximizarse, sin verse obstaculizado por las limitaciones públicas, a costa de no dar prioridad a los menos favorecidos. gente. Este pensador liberal cree en los incentivos del mercado como un poderoso instrumento de crecimiento. Opinará que un gran énfasis en la libertad ``negativa'' puede conducir a un nivel de crecimiento que eleve sustancialmente el nivel de vida de todos, a pesar del hecho de que persisten o incluso aumentan grandes brechas en la distribución del ingreso y la riqueza. También podría pensar que la intervención del Estado es a menudo ineficaz,
\item
  Por otro lado, otro pensador liberal, particularmente cuando los tiempos son duros, observará los estratos desfavorecidos de la población y observará que grandes grupos de pobreza y desigualdades cada vez mayores exigen una intervención estatal más profunda, con el fin de elevar su nivel de protección, en El costo de la adopción de políticas que reserven un amplio papel al Estado: una realización más completa de la libertad ``positiva'' a través de formas que pueden incluir políticas fiscales redistributivas, regulaciones más estrictas, propiedad pública de los medios de producción y ampliaciones de la seguridad pública. neto.
\end{itemize}
\end{quote}

Este tipo de opciones bien pueden permanecer dentro del territorio del liberalismo. La democracia representativa, en el aspecto político, es el arreglo institucional que puede asegurar adecuadamente la libertad de tomar estas decisiones económicas fundamentales, pero diferentes.

\hypertarget{hacia-duxf3nde-iruxeda-ahora-un-liberal}{%
\section*{¿ Hacia dónde iría ahora un liberal?}\label{hacia-duxf3nde-iruxeda-ahora-un-liberal}}
\addcontentsline{toc}{section}{¿ Hacia dónde iría ahora un liberal?}

Teniendo en cuenta esta distinción entre libertad negativa y positiva, cabe preguntarse si el neoliberalismo, como se discutió en el capítulo anterior, puede haber conducido a un abuso de la libertad negativa. Creo que de hecho lo ha hecho.

Recordemos algunas características que hemos considerado inherentes al neoliberalismo. Son: relaciones laborales en pie de igualdad, independientemente de la diferente fuerza contractual del empleador y del trabajador; una organización de mercado darwiniana, que posiblemente conduzca a la supresión de la competencia de facto y al surgimiento de posiciones de renta, y una concentración extrema del poder económico; globalismo en el comercio y los movimientos de capital que explota el arbitraje de los costos laborales y posiblemente perjudique la inversión y la producción a nivel regional; neutralidad de las políticas fiscal y monetaria con respecto al comportamiento de los agregados macroeconómicos. En general, los sistemas económicos se basan en estructuras reguladoras más suaves y en la reducción del sector público.

Agreguemos que, como es bastante obvio, en la mayoría de los países avanzados estas características del fundamentalismo de mercado no han destruido estructuras existentes y bien consolidadas, particularmente en lo que se refiere a la denominación general de Estado de Bienestar. Un neoliberal tendría razón al decir que el impacto de las características antes mencionadas ha sido diferente de un país a otro y, en general, exagerado. Pero la penetrante ``ansiedad visceral'', que afecta tanto a los trabajadores como a los empresarios (según las fuertes palabras utilizadas por Paul Samuelson hace más de veinte años, véase el capítulo 4 ), no ha hecho más que aumentar.

Isaiah Berlin, mirando hacia atrás en el siglo XX, tenía suficiente terreno, como hemos visto, para comentar los abusos de la libertad positiva. El nacionalismo y el totalitarismo fueron experiencias bastante recientes e, incluso cuando los gobiernos democrático-liberales estaban a cargo, una sobreextensión de la red de seguridad pública era un tema de debate y preocupación. Hoy, los temas en juego son diferentes. La pregunta es si errar por el lado de la libertad negativa ---con énfasis en fuertes derechos de propiedad privada, mercados libres y libre comercio--- ha ido demasiado lejos, en la dirección de lo que Berlín llamaría ``libertad pseudo-negativa''. Este tipo de libertad termina alejándose del liberalismo, aunque adoptemos una definición de liberalismo bastante amplia y que abarque diferentes perspectivas.

La relevancia de esta pregunta surge incluso cuando, dejando de lado el aspecto ético del liberalismo, miramos la experiencia económica real y nos preguntamos si se ajusta a las predicciones de la Nueva Economía Clásica (que podemos ver como el trasfondo teórico del neoliberalismo). Si se tiene en cuenta que el desempeño de la economía ``positiva'', es decir, de una ciencia económica que pretende ser independiente de las posiciones éticas, debe ser juzgado por la conformidad de la experiencia con las predicciones que produce (según Milton Friedman, ver el Prefacio de este ensayo), es claro que la experiencia económica y financiera real, consistente con los esquemas adoptados por las teorías de expectativas racionales y mercados eficientes, no estaba en conformidad con las predicciones que arrojaban estas teorías.

Las políticas neoliberales que han prevalecido durante mucho tiempo podrían haber encontrado una legitimidad plausible si sus resultados hubieran implicado una mejora general de las condiciones económicas y sociales. Aparentemente, no ocurrió nada, en presencia de tasas de crecimiento muy modestas, o estancamiento o declive a menudo en la mayoría de los países avanzados, y de desigualdades en ingresos y riqueza que solo han empeorado. Incluso en el campo que puede verse como un ``territorio natural'' para una verificación experimental de las teorías de eficiencia del mercado, es decir, el buen funcionamiento de los mercados financieros, fallaron, con el colapso de los mercados con consecuencias bien conocidas en términos de crecimiento y empleo (el hecho de que otros países, en su mayoría no occidentales, hayan obtenido mejores resultados en términos económicos no debe atribuirse a esas proposiciones teóricas. Al menos en un caso importante, el de China,

Es posible decir que no solo se ha abandonado la visión de Adam Smith de la confianza como la base de un sistema de libre mercado que funciona bien, sino que también los principios libertarios del liberalismo de Hayek o incluso Buchanan se han visto privados de toda motivación ética. Aquí, el problema no es el enfoque liberal en el individualismo; Keynes también vio las ambiciones individuales como una fuerza impulsora de la actividad económica y, de hecho, su teoría, como se mencionó anteriormente (Capítulo 2 ), asignó un papel principal al empresario privado para promover la eficiencia económica. Se trata más bien de la visión general, o conjunto de supuestos, que subyace a las teorías de expectativas racionales y mercados eficientes, y sobre la adopción de arreglos institucionales y políticas regulatorias consistentes con ellos. En este sentido, esas teorías aparecen como un ejercicio de lógica abstracta, cuyo supuesto básico es el agente económico individual guiado por el egoísmo y la codicia desenfrenada: un supuesto que es una parodia de la visión de los liberales clásicos.

Como se observó en el capítulo anterior, podemos leer la situación actual a través de lo que puede verse como un subproducto importante del neoliberalismo, el éxito generalizado de los movimientos populistas.

La participación decreciente del sector industrial y la expansión paralela del sector de servicios en los países avanzados ha ido acompañada de una amplia desunionización de la mano de obra y de un número creciente de trabajadores explotados, especialmente en algunos sectores de la economía, como ciertos servicios y la agricultura. . La desindustrialización relativa es en parte el resultado del arbitraje de los costos laborales y la consiguiente deslocalización de la producción. El neoliberalismo ha generado un resentimiento fuerte y arraigado contra las élites políticas y las clases empresariales emergentes que han puesto en práctica el neoliberalismo. Este resentimiento es la fuente y el motor del populismo y de su llamado genérico a un mayor estatismo y, a menudo, a políticas económicas insostenibles.

Si la libertad negativa se ha convertido en una pseudo-libertad, se necesita un cierto reequilibrio para volver a entrar en el campo del liberalismo. La pseudo-libertad negativa tiene que ser restringida si se quiere realizar suficientemente la libertad positiva. 16 Me parece que un equilibrio apropiado y correcto entre la libertad negativa y la positiva requiere, particularmente en las circunstancias actuales ---escribo estas líneas mientras la pandemia aún está entre nosotros--- abordar cuidadosamente los puntos específicos que hemos mencionado como características del neoliberalismo. y preguntarse qué tipo de enfoques podrían ser compatibles con una visión liberal.

Acerca de las relaciones laborales, ``{[}g{]} a globalización ha movido las fábricas de explotación que Marx y Engels y los inspectores de fábrica de la 19 ª siglo se encontraba en Manchester a la periferia capitalista \ldots{} Así que los trabajadores sudado de hoy {[}en la periferia capitalista{]} y la media los trabajadores de clase en los países del capitalismo avanzado, tan alejados unos de otros espacialmente que nunca se encuentran, \ldots{} nunca experimentan juntos la comunidad y la solidaridad que se derivan de la acción colectiva conjunta''. 17 Solo hay que agregar que los ``trabajadores de clase media'' de las economías avanzadas incluyen también a los empleados inseguros de la economía de los gig.

Todo esto puede sonar a viejo marxismo (``los trabajadores del mundo se unen'' 18 ), pero en términos reales significa que, habiendo trasladado la producción industrial en gran parte a la ``periferia'', las economías de los países avanzados se han convertido, en gran parte, en economías de servicios, donde la agrupación densa de trabajadores en una misma gran fábrica es cada vez menos frecuente. Es de interés para una economía liberal sólida que se reconozcan y protejan los derechos de la mano de obra. El liberal admite que existe un conflicto de intereses inherente entre las dos partes del contrato de trabajo, y favorece los sindicatos de libre creación como instrumento común de cooperación y defensa, y de confrontación con la parte contraria (como el ``liberalista'' Luigi El propio Einaudi destacó en su Le lotte del lavoro ). 19 El hombre liberal, sobre todo, desconfía de cualquier forma de corporativismo, preocupado de que pueda favorecer al empresario bajo la bandera suprema del interés nacional.

En cuanto a los mercados abiertos y la globalización, debe reconocerse que el origen de la globalización debe encontrarse en la oportunidad prevista para liberalizar el comercio y los movimientos de capital, mejorar el espíritu de empresa individual, abrir oportunidades e ir más allá de los mercados protegidos con el fin de reforzar la competencia y el crecimiento económico. El espectacular aumento de las transacciones internacionales que se produjo debe considerarse un factor positivo para un bienestar más generalizado y, en cierto modo, puede ser motivo de queja por el hecho mismo de que muchas zonas del mundo se hayan quedado atrás. por actitudes proteccionistas.

Sin embargo, la apertura del mercado ha involucrado inevitablemente áreas que tienen niveles notablemente diferentes de crecimiento económico, estructuras y regulación del mercado, habilidades laborales, salarios y protección laboral; y gobernado por sistemas políticos profundamente diferentes. Así, la globalización ha contribuido a la adopción de políticas deflacionarias en sentido amplio para mantener la competitividad en países que, gracias a su mayor nivel de ingresos y salarios y de una protección laboral más estricta, están expuestos a desequilibrios en sus cuentas exteriores. Por otro lado, ha fomentado la deslocalización de la producción por razones opuestas. La adopción de políticas deflacionarias y la deslocalización de la inversión son dos caras diferentes de un mismo fenómeno,20

Como sucede cuando se desarrolla una crisis importante en un sistema abierto, piense en la década de 1930, la reacción es un retorno a la protección del interés nacional (sin embargo, debe recordarse que incluso los artífices liberales de mente amplia (Keynes es un ejemplo notable) apoyaron barreras al comercio al comienzo de la Gran Depresión, ver Capítulo 2 : este es un tema que no puede verse solo en términos de rabia populista).

No debe tomarse como una desventaja necesaria que una de las posibles consecuencias de las dificultades actuales de mover mercancías por el mundo con la facilidad a la que todos estábamos acostumbrados (dificultades que se atribuyen tanto a las tensiones geopolíticas como a la pandemia actual ) es el intento de las empresas de ``repoblar'', es decir, acortar y diversificar las largas cadenas de suministro, dondequiera que vendan su producto final: ya sea más cerca de su propio mercado local o del país lejano que era su proveedor pero que ahora se ha convertido en también un mercado de consumo.

Sobre los roles respectivos, del Estado y del individuo, en el sistema económico, en las circunstancias específicas actuales, una pregunta que se ha planteado es si el rol mayor del primero es una emergencia temporal o una tendencia a largo plazo, y cómo debería ser. evaluado. Un papel ampliado del Estado ya está con nosotros, impulsado por motivaciones urgentes, ya sea en forma de subsidios o reducciones de impuestos para el sector privado, o de una mayor propiedad de los medios de producción, a través de medidas de rescate. El liberal rechaza cualquier visión estatista del sistema económico, mira con recelo estas políticas a menudo inevitables, es consciente de que esta no es la mejor manera de lidiar con las empresas ``zombis'' (insolventes). Pero, más allá de la emergencia, se debe buscar una presencia pública más estable en circunstancias específicas, como los monopolios naturales o algunos servicios públicos.

El hombre liberal también abordará el tema de la disciplina de la economía, y en particular de las políticas macroeconómicas, liberándose de lo que pueden parecer ahora más como dogmas que como intentos de conducir hacia un bienestar más equilibrado. Vale la pena repetir que Croce fue un filósofo, no un economista. 21 Y sería engañoso identificar su ``historicismo ideal'' con cualquiera de los Weltanshauungen económicos en los que se metamorfoseó el liberalismo en el siglo XX y hasta nuestros días (Capítulos 2 y 4 ). Sin embargo, recordar el liberalismo de Croce y el peso de su historicismo ideal puede ser fundamental para dar más relevancia a una corriente de pensamiento, que actualmente está recibiendo una atención renovada, que recibe el nombre de ``institucionalismo'': una corriente de pensamiento que, por ampliar el campo de investigación del economista, tal vez podría conducir a una nueva ``situación clásica'' schumpeteriana, que todavía falta, como hemos visto.

Heilbroner y Milberg, 22 que escribieron en la década de 1990, observaron que la ``alta teorización'' de esos años alcanzó un alto grado de irrealidad, construyendo modelos sin fundamentos. Escribieron que ``La mayoría de los economistas actuales\ldots{} se concentran en el capitalismo como un sistema de mercado, con la consecuencia de enfatizar sus aspectos funcionales más que institucionales o constitutivos''. Creo que es justo decir que tenían razón. Estos economistas -observaron- no utilizan el lienzo amplio, que es el cuadro de las propiedades políticas y culturales distintivas, así como económicas que caracterizan al capitalismo en un contexto histórico, que es un rasgo tan marcado en los escritos de figuras como Smith, Mill, Marx, Veblen, Schumpeter y Weber o también Braudel 23(Se podrían agregar otros autores, incluido el propio Keynes, o Hyman Minsky, o Douglass North). La importancia de las instituciones es enfatizada por autores que piensan que los relatos puramente teóricos, no institucionales y no históricos privan a la economía de una comprensión más profunda de la economía de libre mercado y su éxito. El institucionalismo económico es, de hecho, la rama de la disciplina que mira los fundamentos de un sistema económico, con especial énfasis en su evolución histórica. Mirar el lado institucional significa indagar sobre la historia, es decir, mirar a la economía como un proceso en el tiempo histórico, por lo tanto, a la incertidumbre fundamental que caracteriza nuestro futuro. Esto encaja bien con la responsabilidad de nuestras elecciones individuales, históricamente específicas, enfatizado por Croce.

En cierto modo, todo economista que mira más allá del funcionamiento cotidiano de un mercado es, per se , un institucionalista, porque tiene que considerar el contexto general en el que tiene lugar cualquier intercambio de mercado; tanto más si pasamos de la microeconomía a la macroeconomía. El propio Keynes ---como se mencionó anteriormente--- puede leerse desde una perspectiva institucional, porque él, como cualquier institucionalista, ve ``la economía como una ciencia social y cultural de amplia base más que como un cuerpo de análisis `matemático-lógico'\,''. 24En segundo lugar, los macroeconomistas e institucionalistas reconocen la necesidad de utilizar el razonamiento deductivo e inductivo, rechazando los límites arbitrarios entre la teoría pura y el análisis empírico. En tercer lugar, también ven un papel para el gobierno y están de acuerdo en que el gobierno juega un papel integral en la respuesta institucional a los problemas del mundo real. Keynes y los institucionalistas tienen una teoría de la estructura capitalista y el cambio institucional, que falta en muchos análisis económicos contemporáneos.

Queda mucho por ver si el liberalismo económico puede reevaluarse de acuerdo con las líneas que hemos descrito, y si la disciplina económica se moverá dando más espacio a factores como la historia y las instituciones. Es posible que eventos como la crisis financiera y la Gran Recesión, o la pandemia actual, lleven a repensar la economía y las políticas económicas, de la misma manera que largos períodos de éxito y prosperidad pueden conducir al estancamiento intelectual. Estos tiempos tan inciertos dejan abierta la puerta a cualquier tipo de desarrollo, y la resistencia del liberalismo se verá sometida a una dura prueba.

\hypertarget{notas-3}{%
\section*{Notas}\label{notas-3}}
\addcontentsline{toc}{section}{Notas}

\begin{enumerate}
\def\labelenumi{\arabic{enumi}.}
\item
  Croce1922). Acerca de este artículo, consulte también el Capítulo 2 , sobre la relación de Croce con Keynes. Keynes fue el editor de la serie Reconstrucción en Europa, donde se insertó este artículo).\\
\item
  Este es el significado diminutivo que Karl Popper le da al historicismo (2002). Justamente escribe que ``la creencia en el destino histórico es pura superstición''. Véanse las páginas IX y 3.
\item
  Consulte el Capítulo 4 de este ensayo.
\item
  Croce1922).
\item
  Según lo citado por Rachman (2020).
\item
  Croce1949, págs. 93-94).
\item
  Croce1922).
\item
  Ibídem.
\item
  Roberts1987, pag. 216).
\item
  Berlín y Jahanbegloo (1991, pag. 43).
\item
  Roberts, pág. 221.
\item
  O la libertad ilimitada para los propietarios de fábricas o los padres permitirá que ambos empleen niños en las minas de carbón.
\item
  Berlín, págs. 40--43.
\item
  Stein1990, págs. 5-6).
\item
  Norte (1971).
\end{enumerate}

dieciséis.
Berlín, pág. 41.

\begin{enumerate}
\def\labelenumi{\arabic{enumi}.}
\setcounter{enumi}{16}
\item
  Streek2016, pag. 25)
\item
  Manifiesto comunista .
\item
  Vea el Capítulo 2 .
\item
  Streek2016, pag. 23).
\item
  ``Intentar leer como economista los escritos de Benedetto Croce sobre la `ciencia económica' es un ejercicio vergonzoso, temeroso de ser irreverente hacia un autor que fue el `mayor ídolo controvertido' \ldots{} de toda la cultura italiana del siglo pasado'', escribió Giorgio Lunghini, un economista marxista (2003, pag. 185).
\item
  Heilbroner y Milberg (1995).
\item
  Heilbroner1988, pag. 50).
\item
  Keller (1983, pag. 1088). Véase también Weinstein (2007) y Whelan (2012).
\end{enumerate}

\hypertarget{part-economuxeda-de-la-complejidad}{%
\part{Economía de la complejidad}\label{part-economuxeda-de-la-complejidad}}

\hypertarget{fundamentos-luxf3gicos-y-filosuxf3ficos-de-la-complejidad}{%
\chapter*{Fundamentos lógicos y filosóficos de la complejidad}\label{fundamentos-luxf3gicos-y-filosuxf3ficos-de-la-complejidad}}
\addcontentsline{toc}{chapter}{Fundamentos lógicos y filosóficos de la complejidad}

Hay al menos 45 definiciones de complejidad según Seth Lloyd, como se informa en The End of Science (Horgan, 1997, págs. 303-305). Rosser Jr.~(1999) defendió la utilidad en el estudio de la economía de una definición que él llamó complejidad dinámica que fue originada por Day (1994). Se trata de que un sistema económico dinámico no logra generar convergencia a un punto, un ciclo límite o una explosión (o implosión) de forma endógena a partir de sus partes deterministas. Se ha argumentado que la no linealidad era una condición necesaria pero no suficiente para esta forma de complejidad, y que esta definición constituía una ``gran carpa'' suficientemente amplia para abarcar las ``cuatro C'' de la cibernética , la catástrofe , el caos y la ``pequeña carpa''.''(Ahora más conocidos como agentes heterogéneos ) complejidad .

\hypertarget{formas-de-complejidad}{%
\section*{Formas de complejidad}\label{formas-de-complejidad}}
\addcontentsline{toc}{section}{Formas de complejidad}

Hay al menos 45 definiciones de complejidad según Seth Lloyd, como se informa en The End of Science (Horgan,1997, págs. 303-305). Rosser Jr.~(1999) defendió la utilidad en el estudio de la economía de una definición que llamó complejidad dinámica que fue originada por Day (1994). 1 Se trata de que un sistema económico dinámico no logra generar convergencia hasta un punto, un ciclo límite o una explosión (o implosión) de forma endógena a partir de sus partes deterministas. Se ha argumentado que la no linealidad era una condición necesaria pero no suficiente para esta forma de complejidad, 2 y que esta definición constituía una ``gran tienda'' suficientemente amplia para abarcar las ``cuatro C'' 3 de la cibernética , la catástrofe , el caos y la " pequeña carpa``. carpa''(ahora más conocidos como agentes heterogéneos ) complejidad .

Norbert Wiener (1948) fundó la cibernética, que se basaba en simulaciones por computadora y fue popular entre los planificadores centrales e informáticos soviéticos mucho después de que no fuera tan admirada en Occidente. Jay Forrester (1961), inventor del simulador de vuelo, fundó la dinámica de su sistema rival , argumentando que los sistemas dinámicos no lineales pueden producir resultados ``contrarios a la intuición''. Probablemente su aplicación más famosa fue The Limits to Growth (Meadows et al.1972), eventualmente criticada por su excesiva agregación. Podría decirse que ambos provienen de la teoría general de sistemas (von Bertalanffy,1950, 1974), que a su vez se desarrolló a partir de la tectología , la teoría general de la organización debida a Bogdanov (1925-29).

La teoría de la catástrofe se desarrolló a partir de la teoría de la bifurcación más amplia, que se basa en suposiciones sólidas para caracterizar patrones de cómo el cambio suave de las variables de control puede generar cambios discontinuos en las variables de estado en valores críticos de bifurcación (Thom, 1975), con Zeeman (1974) modelo de mercado de valores colapsa el primer uso del mismo en economía. Los métodos empíricos para estudiar tales modelos dependen de estadísticas multimodales (Cobb et al.1983; Guastello 2011a, b). Debido a las estrictas suposiciones en las que se basa, se desarrolló una reacción violenta contra su uso, aunque Rosser Jr.~(2007) argumentó que esto se volvió exagerado. 4

Si bien la teoría del caos se remonta a Poincaré (1890), se hizo prominente después de que el climatólogo Edward Lorenz (1963) descubrió la dependencia sensible de las condiciones iniciales , también conocido como ``el efecto mariposa''. Las aplicaciones en economía siguieron las sugerencias hechas por May (1976). Los debates sobre la medición empírica y los problemas asociados con la predicción han reducido su aplicación en economía (Dechert,1996). 5 Es posible desarrollar modelos que exhiban fenómenos combinados catastróficos y caóticos como en la histéresis caótica , 6 primero mostrado como posible en un modelo macroeconómico por Puu (1990), con Rosser Jr.~et al.~(2001) estimando tales patrones de inversión en la Unión Soviética en el período posterior a la Segunda Guerra Mundial.

El tipo de complejidad dinámica de carpa pequeña o agentes heterogéneos no tiene una definición precisa. De manera influyente, Arthur et al.~(1997a) argumentan que tal complejidad exhibe seis características: (1) interacción dispersa entre agentes heterogéneos que interactúan localmente en algún espacio, (2) ningún controlador global que pueda explotar las oportunidades que surgen de estas interacciones dispersas, (3) organización jerárquica transversal con muchos enredos interacciones, (4) aprendizaje y adaptación continuos por parte de los agentes, (5) novedad perpetua en el sistema a medida que las mutaciones lo llevan a desarrollar nuevos nichos ecológicos, y (6) dinámicas fuera de equilibrio con muchos equilibrios o ninguno y poca probabilidad de un estado global óptimo emergente. Muchos apuntan a Thomas Schelling (1971) estudio sobre un tablero Go 19 por 19 7 sobre el surgimiento de la segregación urbana debido a los efectos del vecino más cercano como un ejemplo temprano.

Otras formas de complejidad dinámica no lineal observadas en modelos económicos incluyen atractores extraños no caóticos (Lorenz 1983), límites de cuencas fractales (Lorenz 1983; Abraham et al.1997), atractores de llamaradas (Hartmann y Rössler1998; Rosser Jr.~y col.2003a), y más.

Otros enfoques de complejidad no dinámica utilizados en economía han incluido estructuras (Pryor1995; Stodder1995), 8 jerárquicos (Simon1962), informativo (Shannon1948). Algorítmico (Chaitin1987), estocástico (Rissanen1986) y computacional (Lewis1985; Albin con Foley1998; Velupillai2000).

Aquellos que defienden el enfoque en la complejidad computacional incluyen Velupillai (2005a, B) y Markose (2005), quienes dicen que este último concepto es superior por su fundamento en ideas más definidas, como la complejidad algorítmica (Chaitin1987) y complejidad estocástica (Rissanen1989, 2005). Estos se consideran fundados más profundamente en el trabajo de entropía informacional de Shannon (1948) y Kolmogorov (1983). Mirowski2007) argumenta que los mercados mismos deben verse como algoritmos que están evolucionando a niveles más altos en un Chomsky (1959) jerarquía de los sistemas computacionales, especialmente a medida que se llevan cada vez más a través de las computadoras y se resuelven a través de sistemas programados de doble subasta y similares. McCauley (2004, 2005) e Israel (2005) argumentan que las ideas de complejidad dinámica como la emergencia son esencialmente vacías y deberían abandonarse por otras más basadas en la computación o más basadas en la física, las últimas basándose especialmente en conceptos de invariancia .

En el nivel más profundo, la complejidad computacional involucra el problema de la no computabilidad. En última instancia, esto depende de una base lógica, la de la no recursividad debido a la incompletitud en el sentido de Gödel (Church1936; Turing1937). En los programas informáticos reales, esto se manifiesta más claramente en la forma del problema de la detención (Blum et al.1998). Esto equivale a que el tiempo de parada de un programa sea infinito y se vincula estrechamente con otros conceptos de complejidad computacional, como la complejidad algorítmica de Chaitin. Tales problemas de incompletitud presentan problemas fundamentales para la teoría económica (Rosser Jr.2009a, 2012a, B; Landini y col.2020; Velupillai2009).

En contraste, la complejidad dinámica y conceptos tales como emergencia son útiles para comprender los fenómenos económicos y no son tan incoherentes e indefinidos como se ha argumentado. Un subtema de parte de esta literatura, aunque no toda, ha sido que los modelos o argumentos de base biológica son fundamentalmente incorrectos matemáticamente y deben evitarse en una economía más analítica. En cambio, tales enfoques pueden usarse junto con el enfoque de complejidad dinámica para explicar la emergencia matemáticamente y que tales enfoques pueden explicar ciertos fenómenos económicos que no se pueden explicar fácilmente de otra manera.

\hypertarget{fundamentos-de-la-economuxeda-de-la-complejidad-computacional}{%
\section*{Fundamentos de la economía de la complejidad computacional}\label{fundamentos-de-la-economuxeda-de-la-complejidad-computacional}}
\addcontentsline{toc}{section}{Fundamentos de la economía de la complejidad computacional}

Velupillai2000, págs. 199-200) resume los fundamentos de lo que ha denominado economía computable 9 a continuación.

\begin{quote}
La computabilidad y la aleatoriedad son las dos nociones epistemológicas básicas que he usado como bloques de construcción para definir la economía computable. Ambas nociones pueden ponerse en práctica para formalizar la teoría económica de manera eficaz. Sin embargo, solo se pueden hacer sobre la base de dos tesis: la tesis de Church-Turing y la tesis de Kolmogorov-Chaitin-Solomonoff.
\end{quote}

Iglesia (1936) y Turing (1937) se dieron cuenta de forma independiente de que varias clases amplias de funciones podían describirse como ``recursivas'' y eran ``calculables'' (las computadoras programables aún no se habían inventado). Turing (1936, 1937) fue el primero en darse cuenta de que Gödel (1931) El teorema de la incompletitud proporcionó una base para la comprensión cuando los problemas no eran ``calculables'', llamados ``efectivamente computables'' ya que Tarski (1949). El análisis de Turing que introduce el concepto generalizado de la máquina de Turing , ahora visto como el modelo de un agente económico racional dentro de la economía computable (Velupillai2005b, pag. 181). Si bien el teorema de Gödel original se basó en una prueba diagonal de Cantor que surge de la autorreferencia, la manifestación clásica de no computabilidad en la programación es el problema que se detiene : que un programa simplemente se ejecutará para siempre sin llegar nunca a una solución (Blum et al.1998).

Gran parte de la economía computable reciente ha implicado demostrar que cuando se intenta poner partes importantes de la teoría económica estándar en formas que puedan ser computables, se descubre que no son computables de manera efectiva en ningún sentido general. Estos incluyen los equilibrios walrasianos (Lewis1992), Equilibrios de Nash (Prasad 1991; Tsuji y col.1998), aspectos más generales de la macroeconomía (Leijonufvud 1993), y si un sistema dinámico será caótico o no (da Costa et al.~2005). 10

De hecho, lo que se considera como complejidades dinámicas puede surgir de problemas de computabilidad que surgen al saltar de un marco de números reales clásico y continuo a un marco de solo números racionales digitalizado. Un ejemplo es la curiosa ``función financiera'' de Clower y Howitt (1978) en el que las variables de solución saltan hacia adelante y hacia atrás en grandes intervalos de forma discontinua a medida que las variables de entrada van de números enteros a racionales no enteros a números irracionales y viceversa. Velupillai2005b, pag. 186) señala el caso de un misil Patriot que falló su objetivo por 700 my mató a 28 soldados como ``fuego amigo'' en Dhahran, Arabia Saudita en 1991 debido a un ciclo sin interrupción de una computadora a través de una expansión binaria en una fracción decimal. Finalmente, el descubrimiento de la dependencia sensible caótica de las condiciones iniciales por Lorenz (1963) debido al error de redondeo de la computadora es famoso, un caso que es computable pero indecidible.

En realidad, existen varias definiciones de complejidad basadas en la computabilidad, aunque Velupillai (2000, 2005a, B) sostiene que pueden vincularse como parte de la base más amplia de la economía computable. El primero es el Shannon (1948) medida del contenido de la información, que puede interpretarse como un intento de observar la estructura en un sistema estocástico. Por tanto, se deriva de una medida de entropía en el sistema, o de su estado de desorden. Por lo tanto, si p ( x ) es la función de densidad de probabilidad de un conjunto de K estados denotados por valores de x , entonces la entropía de Shannon está dada por

\[H(X)=-\sum_{x=1}^{K} \ln (p(x))\]

De aquí es trivial obtener el contenido de información de Shannon de X = x como

\[\mathrm{SI}(x)=\ln (1 / p(x))\]

Se llegó a entender que esto equivale al número de bits en un algoritmo que se necesitan para calcular este código. Esto llevaría a Kolmogorov (1965) para definir lo que ahora se conoce como complejidad de Kolmogorov como el número mínimo de bits en cualquier algoritmo que no prefija ningún otro algoritmo a ( x ) que una Máquina de Turing Universal (UTM) requeriría para calcular una cadena binaria de información, x , o,

\[\mathrm{K}(x)=\min |a(x)|\]

donde │ │ denota la longitud del algoritmo en bits. 11 Chaitin (1987) descubriría y ampliaría de forma independiente este concepto de longitud mínima de descripción (MDL) y lo vincularía con los problemas de incompletitud de Gödel, su versión se conoce como complejidad algorítmica , que sería retomada más tarde por Albin (mil novecientos ochenta y dos) 12 y Lewis (1985, 1992) en contextos económicos. 13

Si bien estos conceptos vincularon de manera útil la teoría de la probabilidad y la teoría de la información con la teoría de la computabilidad, todos comparten el desafortunado aspecto de no ser computable. Esto se remediaría con la introducción de la complejidad estocástica por Rissanen (1978, 1986, 1989, 2005). La intuición detrás de la modificación de Rissanen de los conceptos anteriores es centrarse no en la medida directa de la información, sino en buscar una descripción o modelo más breve que describa las ``características regulares'' de la cuerda. Para Kolmogorov, un modelo de cadena es otra cadena que contiene la primera cadena. Rissanen (2005, págs. 89-90) define una función de verosimilitud para una estructura dada como una clase de funciones de densidad paramétricas que pueden verse como modelos respectivos, donde θ representa un conjunto de k parámetros yx es una cadena de datos dada indexada por n :

\[M_{k}=\left\{f\left(x^{n}, \theta\right): \theta \in \mathbf{R}^{k}\right}\]

Para una f dada , con f ( y n ) un conjunto de ``cadenas normales'', la función de máxima verosimilitud normalizada estará dada por

\[f^{*}\left(x^{n}, M_{k}\right)=f\left(x^{n}, \theta^{*}\left(x^{n}\right)\right) /\left[\int_{\theta(m)} \mathrm{f}\left(y^{n}, \theta\left(y^{n}\right)\right) \mathrm{d} y^{n}\right]\]

donde el denominador del lado derecho se puede definir como C n , k .

A partir de esto, la complejidad estocástica viene dada por

\[-\ln f^{*}\left(x^{n}, M_{k}\right)=-\ln f\left(x^{n}, \theta^{*}\left(x^{n}\right)\right)+\ln C_{n, k}\]

Este término puede interpretarse en el sentido de que representa ``la `longitud de código más corta' para los datos x n que se puede obtener con la clase de modelo M k''. (Rissanen2005, pag. 90). Con esto tenemos una medida computable de complejidad derivada de las ideas más antiguas de Kolmogorov, Solomonoff y Chaitin. La conclusión de la complejidad de Kolmogorov es que un sistema es complejo si no es computable. Los partidarios de estos enfoques para definir la complejidad económica (Israel2005; Markose2005; Velupillai2005a, B) señalan la precisión que dan estas medidas en contraste con muchas de las alternativas.

Sin embargo, la complejidad algorítmica de Chaitin (1966, 1987) introduce un límite a esta precisión, una aleatoriedad subyacente última. Consideró el problema de que un programa se haya iniciado sin que uno sepa qué es y, por lo tanto, se enfrenta a una probabilidad de que se detenga, lo que etiquetó como Ω. Vio esta aleatoriedad como subyacente a todos los ``hechos'' matemáticos. De hecho, este Ω en sí mismo, en general, no es computable (Rosser Jr.2020a).

Un ejemplo de esto involucra un teorema de Maymin (2011) que se extiende a ambos lados del límite del problema profundo sin resolver de si P (polinomio) es igual a NP (no polinomio) en los programas, 14 por lo que tiene un Ω desconocido. Este teorema muestra que bajo ciertas condiciones de información los mercados son eficientes si P = NP, lo que pocos creen. En el borde de este da Costa y Doria (2016) usa el O'Donnell (1979) algoritmo que es exponencial y, por lo tanto, no P pero que crece lentamente hasta ``casi P'' para establecer una función de contraejemplo para el problema P = NP. El algoritmo de O'Donnell se cumple si P \textless NP es probable para cualquier teoría estrictamente más fuerte que la aritmética recursiva primitiva, incluso si eso no puede probarlo. Tales problemas pueden aparecer como en el problema del viajante de comercio computacionalmente complejo. Da Costa y Doria establecen que en estas condiciones el algoritmo de O'Donnell se comporta como un sistema ``casi P'' que implica un resultado de ``mercados casi eficientes''. Este es un resultado que camina al borde de lo desconocido, si no de lo incognoscible.

Una cuestión lógica más profunda que subyace a la complejidad computacional y la economía implica debates fundamentales sobre la naturaleza de las matemáticas en sí. Las matemáticas convencionales asumen axiomas etiquetados como el sistema Zermelo-Fraenkel- {[}Axiom of{]} Choice, o ZFC. Pero algunos de estos axiomas han sido cuestionados y se han hecho esfuerzos para desarrollar sistemas matemáticos axiomáticos que no los utilicen. Los axiomas que han sido cuestionados han sido el axioma de la elección, el axioma del infinito y la ley del medio excluido. Un término general para estos esfuerzos ha sido matemática constructivista , con sistemas que enfatizan particularmente no depender de la Ley del Medio Excluido, lo que significa que no hay uso de prueba por contradicción, se ha conocido como intuicionismo , inicialmente desarrollado por Luitzen Brouwer (1908) de la fama del teorema del punto fijo. 15 En particular, las demostraciones estándar del teorema de Bolzano-Weierstrass utilizan la prueba por contradicción, con este Lema de Sperner subyacente, que a su vez subyace en las demostraciones estándar de los teoremas de punto fijo de Brouwer y Kakutani utilizados en las pruebas de existencia de equilibrio general y de Nash (Velupillai2006, 2008). dieciséis

Para los matemáticos, si no para los economistas, el más importante de estos axiomas discutibles es el axioma de elección, que permite la ordenación relativamente fácil de conjuntos infinitos. Esto sustenta las demostraciones estándar de los principales teoremas de la economía matemática, con Scarf (1973) probablemente el primero en notar estos posibles problemas. El axioma de elección es especialmente importante en topología y partes centrales del análisis real. Por un lado, Specker (1953). Pero una forma de solucionar algunos de estos problemas es utilizar un análisis no estándar que permita números reales infinitos e infinitesimales (Robinson1966), lo que permite evitar el uso del axioma de elección para demostrar algunos teoremas importantes.

La cuestión del axioma del infinito quizás esté más estrechamente ligada a las cuestiones sobre la complejidad computacional. La idea filosófica profunda detrás de estos enfoques constructivistas es que las matemáticas deberían tratar con sistemas finitos que son más realistas y más fáciles de calcular. En contra de esto, fue la introducción de Cantor de niveles de infinito en las matemáticas, una innovación que llevó a Hilbert a elogiar a Cantor por ``traer a los matemáticos al paraíso''. Pero los críticos de la computabilidad argumentan que la economía matemática debe ajustarse al mundo real de una manera creíble, con esfuerzos en curso para construir dicha economía basada en una base constructivista (Velupillai2005a, B, 2012; Bartholo y col.2009; Rosser Jr.2010a, 2012a).

\hypertarget{epistemologuxeda-y-complejidad-computacional}{%
\section*{Epistemología y complejidad computacional}\label{epistemologuxeda-y-complejidad-computacional}}
\addcontentsline{toc}{section}{Epistemología y complejidad computacional}

En cuanto a la complejidad computacional, Velupillai (2000) proporciona definiciones y discusión general y Koppl y Rosser Jr.~(2002) proporcionan una formulación más precisa del problema, basándose en los argumentos de Kleene (1967), Binmore (1987), Lipman (1991) y enlatado (1992). Velupillai define la complejidad computacional directamente como ``intratabilidad'' o insolubilidad. Detener problemas como el estudiado por Blum et al.~(1998) proporcionan excelentes ejemplos de cómo puede surgir tal complejidad, con este problema estudiado por primera vez para sistemas recursivos por Church (1936) y Turing (1936, 1937).

En particular, Koppl y Rosser reexaminaron el famoso problema ``Holmes-Moriarty'' de la teoría de juegos, en el que dos jugadores que se comportan como máquinas de Turing contemplan un juego entre ellos que implica una regresión infinita del pensamiento sobre lo que está pensando el otro (Morgenstern 1935). Esencialmente, este es el problema del juego de n niveles con n sin límite superior (Bacharach y Stahl2000). Esto tiene un equilibrio de Nash, pero las máquinas de Turing ``hiperracionales'' no pueden llegar a saber que tienen esa solución o no debido al problema de la detención. Que las mejores funciones de respuesta no sean computables surge del problema de autorreferencia involucrado fundamentalmente similar a las que subyacen al Teorema de incompletitud de Gödel (Rosser Sr1936; Kleene1967, pag. 246). Aaronson (2013) ha mostrado vínculos entre estos problemas en la teoría de juegos y el problema N = P de complejidad computacional. Tales problemas se extienden también a la teoría del equilibrio general (Lewis1992; Richter y Wong1999; Landini y col.2020).

Binmore's (1987, págs. 209-212), la respuesta a tal indecidibilidad en los sistemas de autorreferencia invoca una forma ``sofisticada'' de actualización bayesiana que implica un grado de mayor ignorancia. Koppl y Rosser están de acuerdo en que los agentes pueden operar en tal ambiente aceptando límites en el conocimiento y operar en consecuencia, tal vez sobre la base de la intuición o ``espíritus animales keynesianos'' (Keynes1936). Los agentes hiperracionales no pueden tener un conocimiento completo, esencialmente por la misma razón por la que Gödel demostró que ningún sistema lógico puede ser completo en sí mismo.

Sin embargo, incluso para la solución propuesta por Binmore también existen límites. Así, Diaconis y Freedman (1986) han demostrado que el teorema de Bayes no se sostiene en un espacio de dimensión infinita. Puede haber una falla para converger en la solución correcta a través de la actualización bayesiana, especialmente cuando la base es discontinua. Puede haber convergencia en un ciclo en el que los agentes están saltando de una probabilidad a otra, ninguna de las cuales es correcta. En el ejemplo simple del lanzamiento de una moneda, es posible que estén saltando de un lado a otro asumiendo a priori de 1/3 y 2/3 sin poder nunca converger en la probabilidad correcta de 1/2. Nyarko (1991) ha estudiado este tipo de dinámicas cíclicas en situaciones de aprendizaje en modelos económicos generalizados.

Koppl y Rosser comparan este tema con el problema de Keynes (1936, Cap. 12) del concurso de belleza. En esto, se supone que los participantes ganan si adivinan con mayor precisión las conjeturas de los otros participantes, lo que potencialmente implica un problema de regresión infinita con los participantes tratando de adivinar cómo los otros participantes adivinarán sobre sus conjeturas, etc. Esto también puede verse como un problema de reflexividad (Rosser Jr.2020b). Una solución viene al optar por ser algo ignorante o limitadamente racional y operar en un nivel particular de análisis. Sin embargo, como no hay forma de determinar racionalmente el grado de acotación, lo que en sí mismo implica un problema de regresión infinita (Lipman1991), esta decisión también implica en última instancia un acto arbitrario, basado en espíritus animales o lo que sea, una decisión que finalmente se toma sin pleno conocimiento.

Un punto curiosamente relacionado aquí está en los resultados posteriores (Gode y Sunder 1993; Mirowski2002) sobre el comportamiento de los traders de inteligencia cero. Gode \hspace{0pt}\hspace{0pt}y Sunder han demostrado que en muchas configuraciones de mercado artificial, los comerciantes de inteligencia cero que siguen reglas muy simples pueden converger en equilibrios de mercado que incluso pueden ser eficientes. No solo puede ser necesario limitar el conocimiento de uno para comportarse de una manera racional, sino que uno puede ser capaz de ser racional en algún sentido sin tener conocimiento alguno. Mirowski y Nik-Kah (2017) argumentan que esto completa una transformación del tratamiento del conocimiento en economía en la era posterior a la Segunda Guerra Mundial, pasando de suponer que todos los agentes tienen conocimiento completo a todos los agentes que tienen conocimiento cero.

Otro punto sobre esto es que hay grados de complejidad computacional (Velupillai 2000; Markose2005), con Kolmogorov (1965) proporcionando una definición ampliamente aceptada de que el grado de complejidad computacional viene dado por la duración mínima de un programa que se detendrá en una máquina de Turing. Hemos estado considerando los casos extremos de no detenerse, pero de hecho existe una jerarquía aceptada entre los niveles de complejidad computacional, y las dificultades de conocimiento experimentan cambios cualitativos a través de ellos. Se considera que esta jerarquía consta de cuatro niveles (Chomsky1959; Wolfram1984; Mirowski2007). En el nivel más bajo se encuentran los sistemas lineales, de fácil resolución, con un nivel tan bajo de complejidad computacional que podemos verlos como no complejos. Por encima de ese nivel se encuentran los problemas polinomiales (P) que son sustancialmente más complejos desde el punto de vista computacional, pero que en general se pueden resolver. Por encima de eso están los problemas exponenciales y otros problemas no polinomiales (NP) que son muy difíciles de resolver, aunque aún no se ha demostrado que estos dos niveles sean fundamentalmente distintos, uno de los problemas no resueltos más importantes de la informática. Por encima de este nivel se encuentra el de complejidad computacional total asociado donde la longitud mínima es infinita, donde los programas no se detienen. En este caso, los problemas de conocimiento solo pueden resolverse volviéndose efectivamente menos inteligente.

\hypertarget{fundamentos-de-la-economuxeda-de-la-complejidad-dinuxe1mica}{%
\section*{Fundamentos de la economía de la complejidad dinámica}\label{fundamentos-de-la-economuxeda-de-la-complejidad-dinuxe1mica}}
\addcontentsline{toc}{section}{Fundamentos de la economía de la complejidad dinámica}

En contraste con las medidas definidas computacionalmente descritas anteriormente, la definición de complejidad dinámica se destaca curiosamente en cuanto a su negatividad: sistemas dinámicos que no generan de manera endógena y determinista ciertos resultados de ``buen comportamiento''. La acusación de que no es precisa tiene peso. Sin embargo, la virtud de la misma es precisamente su generalidad garantizada por su vaguedad. Puede aplicarse a una amplia variedad de sistemas y procesos que muchos han descrito como ``complejos''. Por supuesto, los computacionalistas argumentan con razón que son capaces de subsumir porciones sustanciales de dinámica no lineal con su enfoque, como por ejemplo con el resultado ya mencionado sobre la no computabilidad de la dinámica caótica (Costa et al.2005).

Sin embargo, la mayor parte de este debate y discusión reciente, especialmente por Israel (2005), McCauley (2005) y Velupillai (2005b, 2005c) se ha centrado en un resultado particular que está asociado con algunos modelos de agentes que interactúan dentro de la parte de complejidad de la tienda más pequeña (agentes que interactúan heterogéneos) del concepto más amplio de complejidad dinámica de la tienda grande. Esta propiedad o fenómeno es emergencia . Fue muy discutido por cibernéticos y teóricos de sistemas generales (von Bertalanffy1974), incluso bajo la etiqueta anagenesis (Boulding1978; Jantschmil novecientos ochenta y dos), aunque inicialmente fue formalizado por Lewes (1875) y ampliado por Morgan (1923), basándose en la idea de leyes heteropáticas debidas a Mill (1843, Libro III). Gran parte de la discusión reciente se ha centrado en Crutchfield (1994) porque lo ha asociado más claramente con procesos dentro de sistemas computarizados de agentes heterogéneos en interacción y lo ha vinculado a conceptos de computabilidad de longitud mínima relacionados con la idea de Kolmogorov, lo que facilita el manejo de los computacionalistas. En cualquier caso, la idea es la aparición dinámica de algo nuevo de forma endógena y determinista del sistema, a menudo también denominado autoorganización . 17

Además, todos los citados aquí agregarían otro elemento importante, que aparece en un nivel superior dentro de un sistema jerárquico dinámico como resultado de procesos que ocurren en niveles inferiores del sistema. Crutchfield1994) permite que lo que está involucrado son bifurcaciones que rompen la simetría, lo que lleva a McCauley (2005, pp.77-78) para ser especialmente despectivo, identificándolo con modelos biológicos (Kaufmann 1993) y declarando que ``hasta ahora nadie ha presentado un ejemplo claro empíricamente relevante o incluso teóricamente claro''. Los críticos se quejan del holismo implícito e Israel lo identifica con el de Wigner (1960) Alienación ``mística'' de la visión sólidamente fundamentada de Galileo.

Ahora bien, la queja de McCauley equivale a una aparente falta de invariancia , una falta de ergodicidad o equilibrios de estado estacionario, con simetrías claramente identificables cuya ruptura provoca estas reorganizaciones o transformaciones de nivel superior.

\begin{quote}
Podemos entender cómo una célula muta a una nueva forma, pero no tenemos un modelo de cómo un pez se convierte en pájaro. Eso no quiere decir que no haya sucedido, solo que no tenemos un modelo que nos ayude a imaginar los detalles, los cuales deben basarse en complicadas interacciones celulares que no se comprenden. (McCauley2005, pag. 77) 18
\end{quote}

Si bien probablemente tenga razón en que los detalles de estas interacciones no se comprenden completamente, una nota al pie en la misma página apunta en la dirección de algún entendimiento que ha aparecido, no vinculado directamente a Crutchfield o Kaufmann. McCauley señala el trabajo de Hermann Haken (1983) y sus ``ejemplos de bifurcaciones para la formación de patrones a través de la ruptura de la simetría''. En este punto se sugieren varios enfoques posibles.

Un enfoque es el de la sinergia debido a Haken (1983), aludido anteriormente. Esto trata más directamente con el concepto de arrastre de oscilaciones a través del principio esclavista (Haken1996), que opera según el principio de aproximación adiabática . Un sistema complejo se divide en parámetros de orden que se supone que se mueven lentamente en el tiempo y ``esclavizan'' variables o subsistemas que se mueven más rápido. Si bien puede ser que los parámetros de orden estén operando en un nivel jerárquico más alto, lo que sería consistente con muchas generalizaciones hechas sobre patrones relativos entre tales niveles (Allen y Hoekstra1990; Gritando1992; Radner1992), Este no es necesariamente el caso. Las variables pueden ser completamente equivalentes en una sola jerarquía plana, como con las variables de control y de estado en los modelos de teoría de catástrofes. Las perturbaciones estocásticas pueden provocar cambios estructurales cerca de los puntos de bifurcación.

Si la dinámica lenta viene dada por el vector F , la dinámica rápida generada por el vector q , siendo A , B y C matrices, y ε un vector de ruido estocástico, entonces una versión linealizada localmente viene dada por

\[\mathrm{d} \mathbf{q}=\mathbf{A q}+\mathbf{B}(\mathbf{F}) \mathbf{q} \mathbf{C}(\mathbf{F})+\varepsilon\]

La aproximación adiabática viene dada por

\[\mathrm{d} \mathbf{q}=-(\mathbf{A}+\mathbf{B}(\mathbf{F}))^{-1} \mathbf{C}(\mathbf{F})\]

La dependencia de la variable rápida de las variables lentas viene dada por A + B (F) . Los parámetros de orden son los de menor valor absoluto.

La bifurcación de ruptura de simetría ocurre cuando los parámetros de orden se desestabilizan al obtener autovalores con partes reales positivas, mientras que las ``variables esclavas'' exhiben lo contrario. El caos es un resultado posible. Sin embargo, la situación más dramática es cuando las variables esclavas se desestabilizan y se ``rebelan'' (Diener y Poston1984), con la posibilidad de que los roles cambien dentro del sistema y los antiguos esclavos reemplacen a los antiguos ``jefes'' para convertirse en los nuevos parámetros de orden. Un ejemplo en la naturaleza de tal arrastre emergente y autoorganizado podría ser la aparición periódica y coordinada del moho de limo a partir de amebas separadas, que luego se desintegra de nuevo en sus células aisladas (Garfinkel1987). Un ejemplo en las sociedades humanas puede ser el estallido de la Gran Plaga de mediados del siglo XIV en Europa, cuando la hambruna acumulada y la inmunodeficiencia explotó en un colapso poblacional masivo (Braudel1967).

Otro enfoque se encuentra en Nicolis (1986), derivado del trabajo de Nicolis y Prigogine (1977) sobre el arrastre de frecuencia. Rosser Jr.~(1994) han argumentado que esto puede servir como un posible modelo para el momento anagenético, o el surgimiento de un nuevo nivel de jerarquía. Sea n niveles bien definidos de la jerarquía, con L 1 en la parte inferior y L n en la parte superior. Un nuevo nivel, L n +1 , o estructura disipativa , puede emerger en una transición de fase con un grado suficiente de arrastre de las oscilaciones en ese nivel. Sea k variables oscilantes, x j y z i ( t) ser un proceso estocástico exógeno distribuido de forma independiente e idéntica con media cero y varianza constante, entonces la dinámica viene dada por las ecuaciones diferenciales no lineales acopladas de la forma

\[\mathrm{d} x_{i} / \mathrm{d} t=f_{i}\left(x_{j}, t\right)+z_{i}(t)+\sum_{j=1}^{k} \int_{1}^{\mathrm{k}} x_{j}\left(t^{\prime}\right) \mathrm{w}_{i j}\left(t^{\prime}+\tau\right) \mathrm{d} t^{\prime}\]

con w ij representando un operador de matriz de correlación cruzada. El tercer término es la clave, ya sea ``activado'' o ``desactivado'', y el primero muestra el arrastre de frecuencia. Nicolis (1986) ve esto en términos de un modelo de neuronas, con un oscilador maestro no lineal duro que se enciende mediante una ruptura de simetría del operador de matriz de correlación cruzada cuando la distribución de probabilidad de las partes reales de sus valores propios es superior a cero. 19 Entonces emergerá un nuevo vector variable en el nivel L n +1 que es y j , que amortiguará o estimulará las oscilaciones en el nivel L n , dependiendo de si la suma sobre ellas está por debajo o por encima de cero. 20 Un ejemplo podría ser el surgimiento de un nuevo nivel de jerarquía urbana (Rosser Jr.1994).

Con respecto a la relación entre complejidad dinámica y emergencia, otra perspectiva sobre esto ha venido de la Escuela Austriaca de economía (Koppl 2006, 2009; Luis2012; Rosser Jr.2012a), con la idea de que los sistemas económicos de mercado emergen espontáneamente, una de sus ideas más profundas, que extrajeron de la Ilustración escocesa de Hume y Smith, así como de pensadores como Mill (1843) y Herbert Spencer (1867-1874) que escribió sobre la evolución y la sociología económica (Rosser Jr.~2014b). Este enlace se puede encontrar en el trabajo de Carl Menger (1871/1981), fundador de la Escuela Austriaca. Menger planteó esto de la siguiente manera en términos de lo que la investigación económica debería descubrir (Menger1883/1985, pag. 148):

\begin{quote}
\ldots{} Cómo las instituciones que sirven al bienestar común y son extremadamente importantes para su desarrollo surgen sin una voluntad común dirigida a establecerlas.
\end{quote}

Menger (1892) luego planteó el surgimiento espontáneo de dinero mercantil en sociedades primitivas sin un papel fiduciario de los estados como un ejemplo importante de esto.

Varios seguidores de Menger no siguieron este enfoque con fuerza, muchos enfatizaron enfoques de equilibrio no muy diferentes de la visión neoclásica emergente, que era una idea que se podía encontrar en el trabajo de Menger, quien es ampliamente visto como uno de los fundadores del enfoque marginalista neoclásico. junto con Jevons y Walras. La figura crucial que revivió el interés por el surgimiento entre los austriacos y lo desarrolló mucho más fue Friedrich A. Hayek (1948, 1967). 21 Hayek se basó en los resultados de incompletitud de Gödel, consciente del papel de la autorreferencia en esto, y de cómo la superación de las paradojas de la incompletitud puede implicar el surgimiento de un nivel superior que pueda comprender el nivel inferior. Curiosamente, su conciencia de esto provino originalmente de su trabajo en psicología en su The Sensory Order de 1952 (pp, 188-189):

\begin{quote}
Aplicar los mismos principios generales al cerebro humano como aparato de clasificación. Parecería significar que, aunque comprendamos su modus operandi en términos generales, o, en otras palabras, poseamos una explicación del principio sobre el que opera, nunca, por ningún medio del mismo cerebro, podremos entenderlo. llegar a una explicación detallada de su funcionamiento en circunstancias particulares, o ser capaz de predecir cuáles serán los resultados de sus operaciones. Para lograr esto, sería necesario un cerebro de una complejidad de orden superior, aunque aún podría estar construido sobre los mismos principios. Tal cerebro podría explicar lo que sucede en nuestro cerebro, pero a su vez sería incapaz de explicar sus propias operaciones, y así sucesivamente.
\end{quote}

Koppl (2006, 2009) argumenta que este argumento se aplica también a la larga oposición de Hayek a la planificación central, con un planificador central que enfrenta este problema cuando intenta comprender el efecto en la economía que está tratando de planificar de sus propios esfuerzos de planificación. 22 Esta visión de la importancia de la complejidad y el surgimiento llegaría a tener una gran influencia en la economía austriaca desde que Hayek presentó sus argumentos y sigue siéndolo (O'Driscoll y Rizzo1985; Lachmann1986; Lavoie1989; Horwitz1992; Wagner2010).

\hypertarget{conocimiento-y-complejidad-dinuxe1mica}{%
\section*{Conocimiento y complejidad dinámica}\label{conocimiento-y-complejidad-dinuxe1mica}}
\addcontentsline{toc}{section}{Conocimiento y complejidad dinámica}

En sistemas dinámicamente complejos, el problema del conocimiento se convierte en el problema epistemológico general. Considere el problema específico de poder conocer las consecuencias de una acción tomada en tal sistema. Sea G ( x t ) el sistema dinámico en un espacio de n dimensiones. Deje que un agente posee un conjunto de acciones A . Dejemos que una acción dada por el agente en un momento particular sea dada por un él . Por el momento, no especifiquemos ninguna acción de ningún otro agente, cada uno de los cuales también posee su propio conjunto de acciones. Podemos identificar una relación en la que x t = f ( a it). El problema de conocimiento para el agente en cuestión se convierte así, ``¿Puede el agente conocer el sistema reducido G ( f ( a it ) cuando este sistema posee una dinámica compleja debido a la no linealidad''?

En primer lugar, es posible que el agente pueda comprender el sistema y saber que lo comprende, al menos hasta cierto punto. Una razón por la que esto puede suceder es que muchos sistemas dinámicos no lineales complejos no siempre se comportan de manera errática o discontinua. Muchos sistemas fundamentalmente caóticos exhiben transitoriedad (Lorenz1992). Un sistema puede entrar y salir de comportarse de manera caótica, pasando largos períodos durante los cuales el sistema se comportará efectivamente de una manera no compleja, ya sea siguiendo un equilibrio simple o siguiendo un ciclo límite fácilmente predecible. Si bien el sistema permanece en este patrón, las acciones del agente pueden tener resultados predecibles fácilmente, y el agente puede incluso tener confianza en su capacidad para manipular el sistema de manera sistemática. Sin embargo, esto esencialmente evita la pregunta.

Consideremos cuatro formas de complejidad dinámica: dinámica caótica, límites de cuencas fractales, transiciones de fase discontinuas en situaciones de agentes heterogéneos y modelos teóricos de catástrofes relacionados con sistemas de agentes heterogéneos. Para el primero de ellos existe un problema claro para el agente, la existencia de una dependencia sensible de las condiciones iniciales. Si un agente pasa de la acción a it a la acción a jt , donde \textbar{} a it - a jt \textbar{} \textless{} ε \textless1, entonces no importa cuán pequeño sea ε , existe un m tal que \textbar{} G ( f ( a it + t ′ ) - G ( f ( a jt + t ′ ) \textbar\textgreater{} m para algún t ′ para cada ε . A medida que ε se acerca a cero, m / ε se acercará al infinito. Será muy difícil para el agente predecir el resultado de cambiar Este es el problema del efecto mariposa o la dependencia sensible de las condiciones iniciales. Más particularmente, si el agente tiene una conciencia imperfectamente precisa de sus acciones, con la zona de borrosidad superior a ε, el agente se enfrenta a un potencial amplio rango de incertidumbre con respecto al resultado de sus acciones. En Edward Lorenz (1963) estudio original de este asunto cuando ``descubrió el caos'', cuando reinició su simulación de un sistema de dinámica de fluidos de tres ecuaciones a la mitad, el error de redondeo que desencadenó una divergencia dramática posterior era demasiado pequeño para que su computadora lo ``percibiera'' ( en el cuarto decimal).

Hay dos elementos de compensación para la dinámica caótica. Aunque un conocimiento exacto es efectivamente imposible, requiriendo un conocimiento esencialmente infinitamente preciso (y conocimiento de ese conocimiento), un conocimiento aproximado más amplio a lo largo del tiempo puede ser posible. Por lo tanto, los sistemas caóticos generalmente están limitados y, a menudo, son ergódicos (aunque no siempre). Si bien las trayectorias relativas a corto plazo para dos acciones ligeramente diferentes pueden divergir bruscamente, las trayectorias volverán en algún momento posterior hacia la otra, acercándose arbitrariamente entre sí antes de volver a divergir. No sólo se pueden conocer los límites del sistema, sino que se puede conocer el promedio a largo plazo del sistema. Todavía hay límites, ya que uno nunca puede estar seguro de no estar lidiando con un largo transitorio del sistema, posiblemente moviéndose posteriormente a un modo de comportamiento sustancialmente diferente. Pero la posibilidad de un grado sustancial de conocimiento, incluso con cierto grado de confianza con respecto a ese conocimiento, no está fuera de discusión para los sistemas caóticamente dinámicos.

Con respecto a los límites de las cuencas fractales, identificados por primera vez para los modelos económicos por Hans-Walter Lorenz (1992) en el mismo artículo en el que discutió el problema de la fugacidad caótica. Mientras que en un sistema caótico puede haber solo una cuenca de atracción, aunque el atractor sea fractal y extraño y, por lo tanto, genere fluctuaciones erráticas, el caso del límite de la cuenca fractal involucra múltiples cuencas de atracción, cuyos límites entre sí toman formas fractales. El atractor para cada cuenca puede ser tan simple como ser un solo punto. Sin embargo, los límites entre las cuencas pueden encontrarse arbitrariamente cerca unos de otros en ciertas zonas.

En tal caso, para el caso puramente determinista, una vez que uno es capaz de determinar en qué cuenca de atracción se encuentra, puede resultar un grado sustancial de previsibilidad. Sin embargo, puede existir el problema de la dinámica transitoria, con el sistema tomando una ruta larga y tortuosa antes de comenzar a acercarse al atractor, incluso si el atractor es simplemente un punto al final. El problema surge si el sistema no es estrictamente determinista, si G incluye un elemento estocástico, por pequeño que sea. En este caso, uno puede ser empujado fácilmente a través del límite de una cuenca, especialmente si uno se encuentra en una zona donde los límites se encuentran muy cerca unos de otros. Por lo tanto, puede haber cambios discontinuos repentinos y muy difíciles de predecir en la trayectoria dinámica a medida que el sistema comienza a moverse hacia un atractor muy diferente en una cuenca diferente.

Sin embargo, también en este caso puede haber algo similar al tipo de dispensación a largo plazo que observamos para el caso de la dinámica caótica. Incluso si la predicción exacta en el caso caótico es casi imposible, es posible discernir patrones, límites y promedios más amplios. Asimismo, en el caso de los límites de cuencas fractales con un elemento estocástico, con el tiempo se debe observar un salto de una cuenca a otra. Algo parecido al patrón de dinámica de juegos evolutivos a largo plazo estudiado por Binmore y Samuelson (1999), uno puede imaginar a un observador haciendo un seguimiento de cuánto tiempo permanece el sistema en cada cuenca y eventualmente desarrollando un perfil de probabilidad del patrón, con el porcentaje de tiempo que el sistema pasa en cada cuenca posiblemente acercándose a los valores asintóticos. Sin embargo, esto depende de la naturaleza del proceso estocástico, así como del grado de complejidad del patrón fractal de los límites de la cuenca. Un proceso estocástico no ergódico puede hacer muy difícil, incluso imposible, observar la convergencia en un conjunto estable de probabilidades de estar en las respectivas cuencas, incluso si estas son pocas en número con atractores simples.

Para el caso de las transiciones de fase en sistemas de agentes heterogéneos que interactúan localmente, el mundo de la llamada ``complejidad de carpa pequeña''. Brock y Hommes (1997) han desarrollado un modelo útil para comprender tales transiciones de fase, basado en la mecánica estadística. Este es un sistema estocástico y está impulsado fundamentalmente por dos parámetros clave, una fuerza de interacciones o relaciones entre agentes vecinos y un grado de voluntad de cambiar patrones de comportamiento por parte de los agentes. Para su modelo, el producto de estos dos parámetros es crucial, con una bifurcación que ocurre para su producto. Si el producto está por debajo de un cierto valor crítico, habrá un solo estado de equilibrio. Sin embargo, una vez que este producto exceda un valor crítico particular, surgirán dos equilibrios distintos. Efectivamente, los agentes saltarán de un lado a otro entre estos equilibrios en los patrones de pastoreo. Para modelos de mercados financieros (Brock y Hommes1998) esto puede parecerse a oscilaciones entre mercados alcistas optimistas y mercados bajistas pesimistas, mientras que por debajo del valor crítico, el mercado tendrá mucha menos volatilidad, ya que rastrea algo que puede ser un equilibrio de expectativas racionales.

Para este tipo de configuración, existen esencialmente dos problemas graves. Uno es determinar el valor del umbral crítico. El otro es comprender cómo los agentes saltan de un equilibrio a otro en la zona de equilibrio múltiple. Ciertamente, el segundo problema se parece un poco a la discusión del caso anterior, si no involucra un conjunto tan dramático de posibles cambios discontinuos.

Por supuesto, una vez que se pasa un umbral de discontinuidad, puede ser reconocible cuando se vuelve a acercar. Pero antes de hacerlo, puede ser esencialmente imposible determinar su ubicación. El problema de determinar un umbral de discontinuidad es mucho más amplio y molesta a los legisladores en muchas situaciones, como intentar evitar umbrales catastróficos que pueden provocar el colapso de una población de especies o de todo un ecosistema. No se quiere cruzar el umbral, pero sin hacerlo no se sabe dónde está. Sin embargo, para situaciones menos peligrosas que involucran irreversibilidades, es posible determinar la ubicación del umbral a medida que uno se mueve hacia adelante y hacia atrás a través de él.

Por otro lado, en tales sistemas es muy probable que la ubicación de tales umbrales no permanezca fija. A menudo, tales sistemas exhiben un patrón evolutivo autoorganizado en el que los parámetros del sistema en sí mismos quedan sujetos a cambios evolutivos a medida que el sistema se mueve de una zona a otra. Tal falta de ergodicidad es consistente no solo con la incertidumbre del estilo keynesiano, sino que también puede llegar a parecerse a la complejidad identificada por Hayek (1948, 1967) en sus discusiones sobre la autoorganización dentro de sistemas complejos. Por supuesto, para las economías de mercado, Hayek mostró optimismo con respecto a los resultados de tales procesos. Incluso si los participantes del mercado no pueden predecir los resultados de tales procesos, el patrón de autoorganización será, en última instancia, muy beneficioso si se deja por sí solo. Aunque a menudo se considera que los keynesianos y los austriacos hayekianos están en profundo desacuerdo, algunos observadores han notado las similitudes de puntos de vista con respecto a estos fundamentos de la incertidumbre (Shackle1972; Loasby1976; Rosser Jr.2001a, B). Además, este enfoque conduce a la idea de la apertura de los sistemas que se vuelve consistente con el enfoque realista crítico de la epistemología económica (Lawson1997).

Considerar este problema de umbrales importantes nos lleva a la última de nuestras formas de complejidad dinámica para considerar aquí, las interpretaciones de la teoría de catástrofes. El problema del conocimiento es esencialmente el que se señaló anteriormente, pero se escribe más claramente en términos generales, ya que es más probable que las discontinuidades involucradas sean tan grandes como los estallidos de las grandes burbujas especulativas. El modelo de Brock-Hommes y sus descendientes pueden verse como una forma de lo que está involucrado, pero el enfoque de la teoría de catástrofes original saca a relucir cuestiones clave con mayor claridad.

La primera aplicación de la teoría de las catástrofes en economía por Zeeman (1974) de hecho consideraron las caídas de los mercados financieros en una formulación simplificada de dos agentes: fundamentalistas que estabilizaron el sistema comprando bajo y vendiendo caro y ``chartistas'' que persiguen tendencias de una manera desestabilizadora comprando cuando los mercados suben y vendiendo cuando caen. Al igual que en la formulación de Brock-Hommes, permite que los agentes cambien sus roles en respuesta a la dinámica del mercado, de modo que a medida que el mercado sube, los fundamentalistas se vuelven cartistas, acelerando la burbuja, y cuando llega el colapso, vuelven a ser fundamentalistas, acelerando el colapso. Rosser Jr.~(1991) proporciona una formalización ampliada de esto en términos de teoría de catástrofes que lo vincula con el análisis de Minsky (1972) y Kindleberger (2001), retomado además en Rosser Jr.~et al.~(2012) y Rosser Jr.~(2020c). Esta formulación implica una formulación catastrófica de cúspide en la que las dos variables de control son las demandas de las dos categorías de agentes, y la demanda de los cartistas determina la posición de la cúspide que permite las caídas del mercado.

El problema del conocimiento aquí involucra algo que no se modeló específicamente en Brock y Hommes, aunque tienen una versión del mismo. Se trata de las expectativas de los agentes sobre las expectativas de los demás agentes. Este es efectivamente el tema del ``concurso de belleza'' discutido por Keynes en el Capítulo 12 de esta Teoría General (1936). El ganador del concurso de belleza en un concurso de periódicos no es quien adivina a la chica más bonita, sino quien adivina mejor las suposiciones de los demás participantes. Keynes notó que uno podría comenzar a jugar a adivinar las expectativas de los demás en sus conjeturas de las conjeturas de los demás, y que esto podría ir a niveles más altos, en principio, una regresión infinita que conduce a un problema de conocimiento imposible. Por el contrario, el enfoque de Brock y Hommes simplemente tiene agentes que cambian de estrategia después de observar lo que hacen los demás. Estos problemas de nivel potencialmente superior no entran en juego. Este tipo de problemas reaparecen en los problemas asociados con la complejidad computacional.

\hypertarget{conocimiento-y-ergodicidad}{%
\section*{Conocimiento y ergodicidad}\label{conocimiento-y-ergodicidad}}
\addcontentsline{toc}{section}{Conocimiento y ergodicidad}

Un tema controvertido que involucra el conocimiento y la complejidad involucra las fuentes profundas de la idea de Keynes-Knight de la incertidumbre fundamental (Keynes 1921; Caballero1921). Ambos dejaron en claro que para la incertidumbre no existe una distribución de probabilidad subyacente que determine eventos importantes sobre los que los agentes deben tomar decisiones. La formulación de Keynes de esto ha provocado mucha discusión y debate sobre por qué vio surgir esta falta de distribución de probabilidad.

Una teoría que ha recibido mucha atención, debido a Davidson (1982-83), es que si bien ni Keynes ni Knight lo mencionaron nunca, lo que puede provocar tal incertidumbre, especialmente para la comprensión de Keynes, es la aparición de noergodicidad en los procesos dinámicos subyacentes a la realidad económica. Al hacer este argumento, Davidson citó específicamente los argumentos de Paul Samuelson (1969, pag. 184) en el sentido de que ``la economía como ciencia asume el axioma ergódico''. Davidson se basó en esto para afirmar que el fracaso de este axioma es una cuestión ontológica que es fundamental para comprender la incertidumbre keynesiana, cuando el conocimiento se rompe. Muchos han repetido desde entonces este argumento, aunque Álvarez y Ehnts (2016) argumentan que Davidson malinterpretó a Samuelson, quien en realidad rechazó este punto de vista ergódico por estar vinculado a un punto de vista clásico más antiguo que él no aceptó.

El argumento de Davidson ha sido criticado más recientemente por varios observadores, quizás más enérgicamente recientemente por O'Donnell (2014-15), quien argumenta que Davidson ha tergiversado la hipótesis ergódica, que Keynes nunca la consideró y que la incertidumbre keynesiana es más una cuestión de inestabilidades a corto plazo que deben entenderse utilizando la economía del comportamiento en lugar de los elementos asintóticos vinculados a la ergodicidad. Un argumento importante de O'Donnell es que incluso en un sistema ergódico que va a pasar a un estado estacionario a largo plazo, puede estar fuera de ese estado durante un período de tiempo tan largo que uno no podrá determinar si es ergódico o no. Este es un fuerte argumento al que Davidson no ha logrado responder completamente (Davidson2015).

Para esto es fundamental comprender la hipótesis ergódica en sí misma y su desarrollo y límites, así como su relación con los propios argumentos de Keynes, que resulta algo complicado, pero de hecho vinculado a preocupaciones centrales de Keynes de manera indirecta, especialmente dado que nunca lo mencionó directamente. La mayoría de los economistas que discuten este asunto, incluidos Davidson y O'Donnell, han aceptado como definición de un sistema ergódico que a lo largo del tiempo (asintóticamente) sus ``promedios espaciales son iguales a sus promedios temporales''. Esta formulación se debió a Ehrenfest y Ehrenfest-Afanessjewa (1911), con Paul Ehrenfest alumno de Ludwig Boltzmann (1884) quien expandió el estudio de la ergodicidad (y acuñó el término) como parte de su largo estudio de la mecánica estadística, particularmente cómo un promedio agregado a largo plazo (como la temperatura) podría emerger de un conjunto de partes dinámicamente estocásticas (movimientos de partículas). Resulta que a pesar de su amplia influencia, la formulación precisa de los Ehrenfests era inexacta (Uffink2006). Pero esto reflejaba que había múltiples vertientes en el significado de ``ergodicidad''.

De hecho, existe un debate en curso sobre cómo Boltzmann acuñó el término en primer lugar. Su alumno, Ehrenfest, afirmó que era por combinar el griego ergos (``trabajo'') con hodos (``camino''), mientras que Gallavotti (1999) que vino de él usando su propio neologismo, monode , que significa una distribución estacionaria, en lugar de hodos . Esto encaja con la mayoría de las primeras formulaciones de ergodicidad que lo analizaron dentro del contexto de distribuciones estacionarias.

Las discusiones posteriores sobre la ergodicidad se basarían en dos teoremas complementarios probados por Birkhoff (1931) y von Neumann (1932), aunque este último se probó primero y enfatiza la preservación de la medida, mientras que la variación de Birkhoff fue más geométrica y relacionada con las propiedades de recurrencia en sistemas dinámicos. Ambos implican convergencia a largo plazo, y la formulación de Birkhoff mostró no solo la preservación de la medida, sino que para un sistema ergódico estacionario una indecomponibilidad métrica tal que no solo se llena el espacio correctamente, sino que es imposible dividir el sistema en dos que también lo harán completamente. llenar el espacio y preservar la medida, resultado que extiende la obra fundamental de Poincaré (1890) sobre cómo la recurrencia y el llenado del espacio ayudan a explicar cómo pueden surgir dinámicas caóticas en la mecánica celeste.

En von Neumann's (1932) formulación sea T una transformación que preserva la medida en un espacio de medida con para cada función f integrable al cuadrado en ese espacio, ( Uf ) ( x ) = f ( Tx ), entonces U es un operador unitario en el espacio. Para cualquier operador unitario U en un espacio de Hilbert H , la secuencia de promedios:

\[(1 / n)\left(f+U f+\cdots+U^{n-1} f\right)\]

está fuertemente convergente para cada f en H . Observamos que estos son espacios de medida finita y que esto se refiere a sistemas estacionarios, al igual que con Boltzmann.

Birkhoff's (1931), a veces llamado el ``teorema ergódico individual'', modifica la secuencia anterior de promedios para ser:

\[(1 / n)\left(f(x)+f(T x)+\cdots+f\left(T^{n-1 x}\right)\right)\]

que convergen para casi todas las x . Estos teoremas complementarios se han generalizado a los espacios de Banach y a muchas otras condiciones. 23 Fue a partir de estos teoremas que evolucionaría la próxima ola de desarrollos en Moscú y en otros lugares. 24 Este era el estado de la teoría ergódica cuando Keynes tuvo su debate sobre la econometría a fines de la década de 1930 con el alumno de Paul Ehrenfest, Jan Tinbergen.

El vínculo entre la estacionariedad y la ergodicidad llegaría a debilitarse en un estudio posterior, con Malinvaud (1966) mostrando que un sistema estacionario podría no ser ergódico, con un ciclo límite como ejemplo, con Davidson consciente de este caso desde el comienzo de sus discusiones. Sin embargo, se siguió creyendo que los sistemas ergódicos debían ser estacionarios, y esto siguió siendo una clave para Davidson, además de ser aceptado por la mayoría de sus críticos, incluido O'Donnell. Sin embargo, resulta que esto puede romperse en sistemas caóticos ergódicos de dimensión infinita, que pueden no ser estacionarios (Shinkai y Aizawa2006), que recupera el papel de la dinámica caótica en socavar la capacidad de lograr el conocimiento de un sistema dinámico, incluso uno que es ergódico.

Dadas estas complicaciones, vale la pena volver a Keynes para comprender cuáles eran sus preocupaciones, que salieron más claramente en sus debates con Tinbergen (1937, 1940; Keynes,1938) sobre cómo estimar econométricamente modelos para pronosticar la dinámica macroeconómica. Una profunda ironía aquí es que Tinbergen fue alumno de Paul Ehrenfest y, por lo tanto, fue influenciado por sus ideas sobre la ergodicidad, incluso cuando Keynes no abordó directamente este asunto. En cualquier caso, lo que Keynes objetó fue la aparente ausencia de homogeneidad, esencialmente una preocupación de que el modelo en sí cambia con el tiempo. La solución de Keynes a esto fue dividir una serie de tiempo en submuestras para ver si se obtienen las mismas estimaciones de parámetros que se obtienen para toda la serie de tiempo. La homogeneidad no es estrictamente idéntica ni a la estacionariedad ni a la ergodicidad, pero es probable que en ese momento Tinbergen, siguiendo a Ehrenfest, supusiera probablemente que las tres eran válidas para los modelos que estimó. Por lo tanto, se asumió que la hipótesis ergódica era válida para estos primeros modelos econométricos, mientras que Keynes se mostró escéptico de que existiera una homogeneidad suficiente para suponer que se sabía lo que estaba haciendo el sistema a lo largo del tiempo (Rosser Jr.2016a).

\hypertarget{reflexividad-y-unificaciuxf3n-de-conceptos-de-complejidad}{%
\section*{Reflexividad y unificación de conceptos de complejidad}\label{reflexividad-y-unificaciuxf3n-de-conceptos-de-complejidad}}
\addcontentsline{toc}{section}{Reflexividad y unificación de conceptos de complejidad}

Estrechamente relacionada con la autorreferencia está la idea de reflexividad . Este es un término sin una definición acordada, y se ha utilizado en una amplia variedad de formas (Lynch2000). Se deriva del latín reflectere , que generalmente se traduce como ``inclinarse hacia atrás'', pero puede referirse a ``reflejo'' como en una sacudida de la rodilla cuando se toca, no lo que se quiere decir aquí, o de manera más general está vinculado a ``reflexión'' como en una imagen que se refleja, posiblemente de un lado a otro muchas veces, como en la situación de dos espejos enfrentados. Este último es más el enfoque aquí y más el tipo que está conectado con la autorreferencia y todo lo que implica. Alguien que hizo ese vínculo con fuerza fue Douglas Hofstadter (1979) en su Gödel, Escher, Bach: An Eternal Golden Braid y aún más tarde (Hofstadter2006). Para Hofstadter, la reflexividad está ligada a los cimientos de la conciencia a través de lo que llama ``bucles extraños'' de autorreferencia indirecta, que para él destacan ciertos grabados de Maurits C. Escher, en particular sus ``Dibujando manos'' y también su ``Galería de impresiones,''Con muchos comentaristas sobre la reflexividad citando`` Drawing Hands '', que muestra dos manos dibujándose entre sí (Rosser Jr.2020b). 25 Hofstadter sostiene que el fundamento de su teoría es el Teorema de incompletitud de Gödel, con su profunda autorreferencia, junto con ciertas piezas de JS Bach, así como estos grabados de Escher.

El término probablemente ha sido el más utilizado y con la mayor variedad de significados en sociología (Lynch 2000)). Su uso académico fue iniciado por el destacado sociólogo Robert K. Merton (1938), que lo utilizó para plantear el problema de los sociólogos pensando en cómo sus estudios y cavilaciones encajan en el marco social más amplio, tanto en cómo ellos mismos son influenciados por ese marco en términos de sesgos y paradigmas, pero también en términos de cómo sus estudios y la forma en que hacen sus estudios podría reflejarse también para influir en la sociedad. Entre los sociólogos, los usos más radicales del concepto implicaron una aguda autocrítica en la que uno deconstruye el paradigma e influye en el que está operando hasta el punto de que apenas se puede hacer ningún análisis (Woolgar1991), y muchos se quejan de que esto conduce a un callejón sin salida nihilista. Los primeros usos del término por los economistas siguieron esta línea particular de analizar cómo los economistas particulares están operando dentro de ciertos marcos metodológicos y cómo llegaron a hacerlo a partir de influencias sociales más amplias y cómo su trabajo puede luego reflejarse para influir en la sociedad, a veces incluso a través de determinadas influencias. políticas o incluso formas de recopilar y reportar datos relevantes para las políticas (Hands2001; Davis y Klaes2003).

Merton1948) también usaría la idea para proponer la idea de la profecía autocumplida , una idea que ha sido ampliamente aplicada en economía como con el concepto de equilibrio de manchas solares (Azariadis1981), y muchos ven esto como derivado originalmente de Keynes (1936, Cap. 12) y su análisis del comportamiento del mercado financiero basado en los concursos de belleza de los periódicos británicos de principios del siglo XX. En esos concursos, los periódicos publicaban fotos de mujeres jóvenes y pedían a los lectores que las calificaran por su presunta belleza. El ganador de tal concurso no fue la persona que adivinó qué mujer joven era objetivamente la más hermosa, sino cuál recibió la mayor cantidad de votos. Esto significaba que un jugador astuto de un juego de este tipo estaba realmente tratando de adivinar las conjeturas de los otros jugadores, y Keynes comparó esto con los mercados financieros donde el fundamental subyacente de un activo es menos importante para su valor de mercado de lo que los inversores creen que es. Esto llevó a Keynes incluso a notar que este tipo de razonamiento puede moverse a niveles más altos, tratando de pensar lo que otros piensan que piensan los demás. y a niveles aún más altos en una regresión infinita potencial, un reflejo infinito clásico en un programa que no se detiene. Esta idea de concurso de belleza de Keynes ha llegado a ser vista como una pieza central de su visión filosófica, lo que implica en última instancia no solo reflexividad sino también complejidad (Davis2017).

George Soros (1987), quien luego también argumentaría que el análisis era parte de la economía de la complejidad (Soros 2013). Soros ha argumentado durante mucho tiempo que pensar en esta versión de reflexividad inspirada en concursos de belleza ha sido clave para su propia toma de decisiones en los mercados financieros. Él lo ve como una explicación de los ciclos de auge y caída en los mercados como en la burbuja inmobiliaria estadounidense de principios de la década de 2000, cuyo declive desencadenó la Gran Recesión. Obtuvo el término por primera vez como estudiante de Karl Popper en la década de 1950 (Popper1959), con Popper también una influencia en Hayek (1967) en conexión con estas ideas (Caldwell 2013). Por lo tanto, la idea de reflexividad con vínculos a argumentos sobre la incompletitud y las regresiones infinitas asociadas con la autorreferencia se ha vuelto muy influyente entre los economistas y financieros que estudian la dinámica del mercado financiero y otros fenómenos relacionados.

Ahora vemos la posibilidad de vincular nuestras principales escuelas de complejidad a través del sutil y extraño bucle involucrado en la autorreferencia indirecta en el corazón de una forma más profunda de reflexividad. La autorreferencia indirecta en el corazón del teorema de incompletitud de Gödel está profundamente ligada a la complejidad computacional ya que conduce a los bucles do infinitos del nivel más alto de complejidad computacional en los que un programa nunca se detiene. La salida de la incompletitud implica, en efecto, lo que Davis y Klaes invocaron: pasar a un nivel jerárquico superior en el que un agente o programa exógeno determina qué es verdadero o falso, aunque esto abre la puerta a la incoherencia (Landini et al.2020). La autorreferencia indirecta abre la puerta a la complejidad dinámica en sus implicaciones para la dinámica del mercado, lo que también se vincula a la complejidad jerárquica a medida que se pueden generar nuevos niveles de jerarquía. Consideremos brevemente cómo surge esto del fundamental Gödel (1931) teorema.

El teorema de Gödel es en realidad dos teoremas. El primero es el de incompletitud: cualquier sistema formal consistente en el que se pueda realizar aritmética elemental 26 es incompleto; hay enunciados en el lenguaje del sistema formal que no pueden probarse ni refutarse dentro del sistema formal. El segundo aborda el problema de la coherencia 27: para cualquier sistema formal consistente en el que se pueda realizar aritmética elemental, la consistencia del sistema formal no puede probarse dentro del sistema formal mismo. Entonces, la coherencia implica lo incompleto, pero cualquier intento de superar lo incompleto moviéndose a un nivel superior implica que uno no pueda probar la consistencia de este sistema de nivel superior, y ambas partes de esta falla se deben a las paradojas de la autorreferencia (reflexiva) que conduce a paradojas.

Hofstadter (2006) proporciona una excelente discusión de la naturaleza del carácter indirecto involucrado en la demostración de la parte principal del teorema, que implica el uso de ``números de Gödel''. Estos son números asignados a enunciados lógicos, y su uso puede conducir a la creación de enunciados paradójicos autorreferenciados incluso dentro de un sistema especialmente diseñado para evitar tales enunciados autorreferenciales. El sistema al que Gödel sometió este tratamiento eventualmente genera un enunciado equivalente a ``Esta oración es indemostrable'' fue el sistema lógico desarrollado por Whitehead y Russell (1910-13) específicamente para proporcionar una base formal consistente para las matemáticas sin paradojas lógicas. Russell, en particular, estaba muy preocupado por la posibilidad de paradojas en la teoría de conjuntos, como las que involucran conjuntos de autorreferencia. El problema clásico era ``¿El conjunto de todos los conjuntos que no se contienen a sí mismos se contiene a sí mismo?'' Una famosa versión simple de esto implica ``¿Quién afeita al barbero en una ciudad donde el barbero solo afeita a los que no se afeitan ellos mismos?'' Ambos implican bucles de trabajo interminables similares que surgen de su autorreferencia. Whitehead y Russell intentaron eliminar estas molestias desarrollando la teoría de tipos que establecía jerarquías de conjuntos para evitar que se refirieran a sí mismos. Pero luego Gödel hizo su truco de establecer sus números, que aplicó al sistema de Whitehead y Russell de manera indirecta para generar una declaración autorreferencial que implicaba una paradoja irresoluble dentro del sistema. Es más bien como el agujero que hizo Escher en medio de su ``Galería de Grabados'' permitió que el hombre mirara un grabado en una pared en una galería de una ciudad que contiene la galería en la que está parado mirándola.

Por lo tanto, no es sorprendente que el problema de la autorreferencia haya sido el núcleo de gran parte del pensamiento sobre la reflexividad desde un punto temprano, y que este pensamiento adquirió un tono más agudo cuando varias figuras pensaron en el teorema de Gödel, o incluso antes, en el teorema de Gödel. las paradojas consideradas por Bertrand Russell. Vincular esto a la comprensión de la complejidad proporciona una base para una complejidad reflexiva que abarca todas las formas principales de complejidad.

\hypertarget{observaciones-adicionales}{%
\section*{Observaciones adicionales}\label{observaciones-adicionales}}
\addcontentsline{toc}{section}{Observaciones adicionales}

En los sistemas computacionalmente complejos el problema de comprenderlos está relacionado con la lógica, los problemas de regresión infinita e indecidibilidad asociados con la autorreferencia en sistemas de máquinas de Turing. Esto puede manifestarse como el problema de la detención, algo que puede surgir incluso para una computadora que intenta calcular con precisión incluso un sistema dinámicamente complejo como, por ejemplo, la forma exacta del conjunto de Mandelbrot (Blum et al.1998). Una máquina de Turing no puede comprender completamente un sistema en el que su propia toma de decisiones es una parte crucial. Sin embargo, el conocimiento de tales sistemas puede obtenerse por otros medios.

En la medida en que los modelos tienen fundamentos axiomáticos en lugar de ser meramente ad hoc, como muchos de ellos en última instancia lo son, estos fundamentos están estrictamente dentro del modo matemático clásico no constructivista, asumiendo el axioma de elección, la ley del medio excluido y otros caballos de batalla de los matemáticos y economistas matemáticos cotidianos. En la medida en que brinden información sobre la naturaleza de la complejidad económica dinámica y el problema especial de la emergencia (o anagénesis), no lo hacen basándose en fundamentos axiomáticos 28eso pasaría bien con los constructivistas e intuicionistas de principios y mediados del siglo XX, mucho menos con sus discípulos más recientes, que siguen la esperanza ideal de que ``El futuro es una minoría; el pasado y el presente son mayoría'', para citar a Velupillai (2005b, pag. 13), él mismo parafraseando a Shimon Peres de una entrevista sobre las perspectivas de paz en Oriente Medio.

Existe una gama considerable de modelos disponibles para contemplar o modelar fenómenos emergentes que operan en diferentes niveles jerárquicos. Un área interesante para ver cuál de los enfoques podría resultar más adecuado podría ser el estudio de la evolución de los procesos del mercado a medida que ellos mismos se vuelven más informatizados. Este es el enfoque de Mirowski (2007) que llega a afirmar que fundamentalmente los mercados son algoritmos. El tipo simple de precio publicado: mercado al contado en el que la mayoría de la gente tradicionalmente ha comprado cosas se encuentra en la parte inferior de una jerarquía chomskyiana de complejidad y control autoreferenciado. Así como los algoritmos más nuevos pueden contener algoritmos más antiguos dentro de ellos, la aparición de tipos más nuevos de mercados puede contener y controlar los tipos más antiguos a medida que avanzan hacia niveles más altos en esta jerarquía chomskyiana. Los mercados de futuros pueden controlar los mercados al contado, los mercados de opciones pueden controlar los mercados de futuros, y el orden cada vez más alto de estos mercados y su creciente automatización empuja al sistema a un nivel más alto hacia el ideal inalcanzable de ser una Máquina Universal de Turing en toda regla (Cotogno2003).

Mirowski aporta argumentos más recientes en biología con respecto a la coevolución, señalando que el espacio en el que los agentes y sistemas están evolucionando cambia con su evolución. En la medida en que el sistema de mercado se asemeje cada vez más a un conjunto gigantesco de algoritmos que interactúan y evolucionan, tanto la biología como el problema de la computabilidad se manifestarán e influirán mutuamente (Stadler et al.2001). Al final, la distinción entre los dos puede volverse irrelevante.

En el gran contraste de la complejidad computacional y dinámica, vemos superposiciones cruciales que involucran cómo las paradojas que surgen de la autorreferenciación de la complejidad computacional subyacente pueden implicar el surgimiento tan profundamente asociado con la complejidad dinámica. Estas interrelaciones pueden volverse más manifiestas al contemplar el mundo espejo de la reflexividad y sus interminables concatenaciones. Estas son algunas de las muchas consideraciones que se encuentran en los cimientos de la economía de la complejidad.

\hypertarget{notas-4}{%
\section*{Notas}\label{notas-4}}
\addcontentsline{toc}{section}{Notas}

\begin{enumerate}
\def\labelenumi{\arabic{enumi}.}
\item
  Velupillai2011) ha etiquetado esta visión de la complejidad dinámica como complejidad ``Day-Rosser''.
\item
  Estrictamente hablando, esto es incorrecto. Goodwin1947) mostraron tales patrones dinámicos endógenos en sistemas lineales acoplados con rezagos. Sistemas similares fueron analizados por Turing (1952) en su trabajo que ha sido visto como el fundamento de la teoría de la morfogénesis, un fenómeno de complejidad por excelencia. Sin embargo, la inmensa mayoría de estos sistemas dinámicamente complejos implican cierta no linealidad, y el equivalente normalizado desacoplado del sistema lineal acoplado no es lineal.
\item
  Esta moneda vino de Horgan (1997, Cap. 11) quien etiquetó con desdén las cuatro C para representar la caoplexidad , que él consideraba una burbuja intelectual o una moda pasajera. Rosser Jr.~(1999) argumentó que se trataba de una acuñación como ``Impresionismo'' que inicialmente fue un insulto pero que puede verse como una caracterización útil.
\item
  Arnol'd (1993) proporciona una discusión clara de los problemas matemáticos involucrados mientras evita las controversias.
\item
  Para una discusión más detallada de las controversias matemáticas subyacentes que involucran la teoría del caos, ver Rosser Jr.~(2000b, Apéndice matemático).
\end{enumerate}

6 .
Este término fue acuñado por Abraham y Shaw (1987) y Abraham (1985) también concibió el fenómeno combinado relacionado de caostrofe .

7 .
A menudo se ha afirmado incorrectamente que Schelling utilizó un tablero de ajedrez para este estudio.

8 .
La complejidad estructural parece al final equivaler a una ``complejidad'', que Israel (2005) argumenta es meramente un concepto epistemológico más que ontológico, con ``complejidad'' y ``complicación'' provenientes de diferentes raíces latinas ( complecti , ``captar, comprender o abrazar'' y complicare , ``doblar, envolver''), incluso si muchos confundiría los conceptos (incluso von Neumann1966). Rosser Jr.~(2004) argumenta que la complejidad como tal plantea problemas epistemológicos esencialmente triviales, cómo descifrar muchas partes diferentes y sus vínculos.

9 .
La ``economía computable'' fue neologizada por Velupillai en 1990 y se distingue de la ``economía computacional'', simbolizada por el trabajo que se encuentra en las conferencias de la Association for Computational Economics y su revista Computational Economics. El primero se centra más en los fundamentos lógicos del uso de las computadoras en la economía, mientras que el segundo tiende a centrarse más en aplicaciones y métodos específicos.

10 .
Otro tema principal de la economía computable implica considerar qué partes de la teoría económica pueden probarse cuando se relajan axiomas lógicos clásicos como el axioma de elección y la exclusión del medio. Bajo tales matemáticas constructivas pueden surgir problemas para probar los equilibrios walrasianos (Pour-El y Richards1979; Richter y Wong1999; Velupillai2002, 2006) y equilibrios de Nash (Prasad 2005).

11 .
Debe entenderse que mientras que, por un lado, el primer trabajo de Kolmogorov axiomatizaba la teoría de la probabilidad, sus esfuerzos por comprender el problema de la inducción lo llevarían a argumentar más tarde que la teoría de la información precede a la teoría de la probabilidad (Kolmogorov 1983). McCall (2005) proporciona una discusión útil de esta evolución de las opiniones de Kolmogorov.

12 .
A Albin le gustó el ejemplo del problema de agregación de capital planteado por Joan Robinson (1953-54) que para agregar capital es necesario conocer ya el producto marginal del capital para determinar la tasa de descuento para calcular los valores presentes, mientras que al mismo tiempo ya se necesita conocer el valor del capital agregado para determinar su valor marginal. producto. La economía convencional intenta escapar de este bucle do potencialmente infinito simplemente asumiendo que todos estos se resuelven convenientemente simultáneamente en un gran equilibrio general.

13 .
Estrechamente relacionado estaría el prior universal de Solomonoff (1964) que coloca el concepto MDL en un marco bayesiano. De aquí surge la idea bastante intuitiva de que el estado más probable también tendrá la longitud más corta de algoritmo para describirlo. El trabajo de Solomonoff también se desarrolló de forma independiente, basándose en la teoría de la probabilidad de Keynes (1921).

14 .
El problema P = NP fue identificado por primera vez por John Nash Jr.~(1955) en una carta a la Agencia de Seguridad Nacional de EE. UU. que analiza los métodos de cifrado en el criptoanálisis, que se clasificó hasta 2013. Nash dijo que pensaba que era cierto que P no era igual a NP, pero señaló que no pudo probarlo, y aún no se ha comprobado que este día.

15 .
Irónicamente, la prueba original de Brouwer de su teorema del punto fijo se basó en los axiomas de ZFC, y él solo proporcionó una alternativa intuicionista mucho más tarde (Brouwer 1952).

16 .
Para discusiones lógicas autorizadas de los temas involucrados en general en estas alternativas constructivistas, ver Kleene y Vesley. 1965; Kleene1967; obispo1967).

17 .
Este término se ha asociado especialmente con Bak (1996) y su criticidad autoorganizada , aunque no fue el primero en discutir la autoorganización en estos contextos.

18 .
El argumento de McCauley se basa en Moore (1990, 1991a,B) estudio de mapas iterados de baja dimensión que son máquinas de Turing sin atractores, propiedades de escala o dinámica simbólica. McCauley sostiene que este punto de vista proporciona una base para la complejidad como sorpresa final e imprevisibilidad.

19 .
En un modelo relacionado, Holden y Erneux (1993) muestran que el cambio sistémico puede tomar la forma de un paso lento a través de una bifurcación de Hopf supercrítica, lo que conduce a la persistencia durante un tiempo del estado anterior incluso después de que se ha pasado el punto de bifurcación.

20 .
Otro enfoque más involucra la idea del hiperciclo debido a Eigen y Schuster (1979), discutido en el próximo capítulo.

21 .
Ver Vaughn (1999), Vriend (2002) y Caldwell (2004) para discutir cómo llegó Hayek a sus puntos de vista sobre la complejidad y la emergencia y cómo encajan con sus otros puntos de vista.

22 .
La oposición a la planificación centralizada y el apoyo al surgimiento espontáneo de sistemas de mercado de abajo hacia arriba se muestra en un largo debate entre filósofos sobre si el surgimiento solo funciona de abajo hacia arriba o si puede involucrar una causalidad de arriba hacia abajo. Van Cleve (1990) introduce la supervención al permitir esta causalidad de arriba hacia abajo en los sistemas emergentes, mientras que Kim (1999) sostiene que los procesos emergentes deben ser fundamentalmente de abajo hacia arriba. Luis (2012) argumenta que Hayek se movió hacia el punto de vista de la supervención en sus escritos posteriores que también enfatizaban los procesos evolutivos grupales (Rosser Jr.~2014b).

23 .
Ver Halmos (1958) sobre cómo estos teoremas vinculan la teoría de la medida con la teoría de la probabilidad.

24 .
Velupillai2013, págs. 432-433, n8) muestra que, si bien la mayoría de la teoría ergódica ha seguido una formulación frecuentista, la Escuela de Moscú se basaría en las ideas de Keynes en su enfoque de estos temas.

25 .
A menudo se piensa que los ejemplos de reflexividad en el arte implican el efecto Droste , en el que una obra contiene una imagen de sí misma dentro de sí misma, claramente una cuestión de autorreferencia. Entre los primeros ejemplos conocidos se encuentra un cuadro de Giotto de 1320, El tríptico de Stefaneschi , en el que en el panel central se representa al cardenal Stefaneschi arrodillado ante San Pedro y presentándole el tríptico mismo. No hace falta decir que, incluso si dejan de representarse después de una secuencia finita de imágenes, tales obras de arte que exhiben este efecto Droste implican una regresión infinita de imágenes cada vez más pequeñas que contienen imágenes cada vez más pequeñas (Rosser Jr.2020b).

26 .
Por ``aritmética elemental'' se entiende aquello que puede derivarse del conjunto de axiomas de Peano asumiendo una lógica estándar del tipo Zermelo-Frankel con el axioma de elección (ZFC).

27 .
Cabe señalar que en su teorema original, Gödel solo pudo demostrar que estaba incompleto para una forma limitada de consistencia ω. Rosser Sr.~(1936) que usó la ``Oración de Rosser'' (o ``truco''): ``Si esta oración es demostrable, entonces hay una prueba más corta de su negación''. Esto ha llevado a algunos a referirse al teorema combinado como el ``Teorema de Gödel-Rosser''.

28 .
Si bien este movimiento se centra en refinar los fundamentos axiomáticos, en última instancia busca ser menos formalista y bourbakiano. Esto es consistente con la historia de la economía matemática, que primero se movió hacia una mayor axiomatización y formalismo dentro del paradigma matemático clásico, solo para alejarse de él en años más recientes (Weintraub2002).

\hypertarget{fundamentos-de-la-economuxeda-conductual-compleja}{%
\chapter*{Fundamentos de la economía conductual compleja}\label{fundamentos-de-la-economuxeda-conductual-compleja}}
\addcontentsline{toc}{chapter}{Fundamentos de la economía conductual compleja}

Herbert A. Simon desarrolló la idea de racionalidad limitada a partir de sus primeros trabajos (Simon 1947, 1955a, 1957), que se considera la base de la economía conductual moderna.. La economía del comportamiento contrasta con la economía más convencional en que no asume la racionalidad de la información completa por parte de los agentes económicos en su comportamiento. En este sentido, se basa en conocimientos sobre el comportamiento humano de otras disciplinas de las ciencias sociales como la psicología y la sociología, entre otras. Sin lugar a dudas, se pueden encontrar economistas anteriores que sostenían que la gente está motivada por algo más que una mera maximización egoísta. De hecho, desde los inicios de la economía con Aristóteles, quien puso las consideraciones económicas en un contexto de filosofía moral y conducta adecuada, a través del padre de la economía política, Adam Smith en su Teoría de los sentimientos morales.(1759), a economistas institucionales posteriores como Thorstein Veblen (1899) y Karl Polanyi (1944), quienes vieron la conducta económica de las personas como incrustada en contextos sociales y políticos más amplios. Sin embargo, fue Simon quien acuñó ambos términos y estableció la economía del comportamiento moderna.

\hypertarget{resumen}{%
\section*{Resumen}\label{resumen}}
\addcontentsline{toc}{section}{Resumen}

Herbert A. Simon desarrolló la idea de racionalidad limitada desde sus primeros trabajos (Simon1947, 1955a, 1957), que se considera la base de la economía del comportamiento moderna . La economía del comportamiento contrasta con la economía más convencional en que no asume la racionalidad de la información completa por parte de los agentes económicos en su comportamiento. En este sentido, se basa en conocimientos sobre el comportamiento humano de otras disciplinas de las ciencias sociales como la psicología y la sociología, entre otras. Sin lugar a dudas, se pueden encontrar economistas anteriores que sostenían que la gente está motivada por algo más que una mera maximización egoísta. De hecho, desde los inicios de la economía con Aristóteles, quien puso las consideraciones económicas en un contexto de filosofía moral y conducta adecuada, a través del padre de la economía política, Adam Smith en su Teoría de los sentimientos morales (1759), a economistas institucionales posteriores como Thorstein Veblen (1899) y Karl Polanyi (1944) que veían la conducta económica de las personas como incrustada en contextos sociales y políticos más amplios. Sin embargo, fue Simon quien acuñó ambos términos y estableció la economía del comportamiento moderna.

Las iniciativas de Simon llevaron a una oleada de actividad e investigación durante las siguientes décadas, muchas de las cuales se hicieron más influyentes en las escuelas de negocios y los programas de gestión a medida que la revolución de las expectativas racionales conquistó la mayor parte de la economía durante las décadas de 1970 y 1980. Asumir una racionalidad limitada por parte de los agentes económicos lo llevó al concepto de satisfacción , que si bien las personas no maximizan, se esfuerzan por lograr los objetivos establecidos dentro de las limitaciones. Esto se aceptó en las escuelas de negocios, ya que a los gerentes se les enseñó a lograr niveles de ganancias aceptables para los propietarios.

También surgió de su descubrimiento de la racionalidad limitada su interés en investigar más profundamente cómo las personas piensan y comprenden como parte de su toma de decisiones. Esto lo llevó a considerar cómo se podría estudiar esto mediante el uso de computadoras. Esto lo llevó a convertirse en uno de los fundadores del campo de la inteligencia artificial (Simon1969), y Simon en general es considerado como uno de los primeros líderes de la informática en general. Pero fue su preocupación por las implicaciones de la racionalidad limitada lo que lo llevó a este campo naciente.

Simon también se convertiría en una figura destacada en el desarrollo temprano de la teoría de la complejidad, particularmente de la teoría de la complejidad jerárquica (Simon 1962), aunque solo hizo un vínculo indirecto entre esto y la racionalidad acotada. Sin embargo, los teóricos de la complejidad modernos están mucho más dispuestos a ver un vínculo estrecho y directo entre la complejidad de un tipo u otro y la racionalidad limitada y, por lo tanto, también con la economía del comportamiento (Velupillai2019). 1 De hecho, la complejidad puede verse como una, si no la base fundamental, de por qué las personas tienen una racionalidad limitada. La complejidad se encuentra en el corazón mismo de la economía del comportamiento desde este punto de vista, y Simon trató de comprender cómo las personas deciden frente a una complejidad tan ineludible.

\hypertarget{herbert-simon-y-la-racionalidad-limitada}{%
\section*{Herbert Simon y la racionalidad limitada}\label{herbert-simon-y-la-racionalidad-limitada}}
\addcontentsline{toc}{section}{Herbert Simon y la racionalidad limitada}

El difunto Herbert A. Simon es ampliamente considerado como el padre de la economía del comportamiento moderna , al menos fue su trabajo al que se aplicó por primera vez esta frase. También fue uno de los primeros teóricos de la economía de la complejidad, si no el padre per se, y también fue uno de los fundadores del estudio de la inteligencia artificial en la informática. De hecho, fue un erudito que publicó más de 900 artículos académicos en numerosas disciplinas, y aunque ganó el Premio Nobel de Economía en 1978 por su desarrollo del concepto de racionalidad limitada., su doctorado fue en administración pública y nunca estuvo en un departamento de economía. Debemos utilizar el término ``moderno'' antes que ``economía del comportamiento'' porque se puede considerar que bastantes economistas anteriores se centran en el comportamiento humano real, asumiendo que la gente no se comporta plenamente en lo que ahora llamaríamos una manera ``económicamente racional'' (Smith1759; Veblen1899).

En este punto debemos tener claro que con la ``economía del comportamiento'' no asumimos una visión similar a la de la ``psicología del comportamiento'' del tipo defendido o practicado por Pavlov o BF Skinner (1938). Este último no considera que el estudio de lo que está en la mente o la conciencia de las personas sea de utilidad o interés. Todo lo que importa es cómo se comportan, particularmente cómo responden para responder a los estímulos repetidos en su comportamiento. Esto es más parecido a la economía neoclásica estándar, que también pretende estudiar cómo se comporta la gente con poco interés en lo que sucede dentro de sus cabezas. La principal diferencia entre estos dos es que la economía convencional hace una fuerte suposición sobre lo que está sucediendo dentro de la mente de las personas: que están maximizando racionalmente las funciones de utilidad individuales derivadas de sus preferencias utilizando información completa. En contraste, la economía del comportamiento no asume que las personas sean completamente racionales y particularmente no asume que estén completamente informadas. Lo que está pasando dentro de sus cabezas es importante,economía de la felicidad (Easterlin2017) son temas legítimos para la economía del comportamiento.

En cualquier caso, desde el comienzo de su investigación con su pionera disertación de doctorado que salió como libro en 1947, Comportamiento administrativo y a través de importantes artículos y libros en la década de 1950 (Simon1955a, 1957), Simon consideraba que las personas estaban limitadas tanto en su conocimiento de los hechos como en su capacidad para calcular y resolver los problemas difíciles asociados con el cálculo de soluciones óptimas a los problemas. Se enfrentan a límites inevitables a su capacidad para tomar decisiones completamente racionales. Por lo tanto, las personas viven en un mundo de racionalidad limitada , 2 y fue esta comprensión lo que lo llevó al estudio de la inteligencia artificial en la informática como parte de su estudio de cómo piensan las personas en un mundo así (Simon1969).

Esto llevó a Simon al concepto de satisfacción . Las personas establecen objetivos que buscan alcanzar y luego no realizan más esfuerzos para mejorar las situaciones una vez que se han alcanzado estos objetivos, si es que lo están. Por lo tanto, una empresa no maximizará las ganancias, pero sus gerentes buscarán lograr un nivel aceptable de ganancias que mantendrá a los propietarios suficientemente felices. Esta idea de satisfacción se convirtió en la clave central del estudio conductual de la empresa (Cyert y March1963) y entró en la literatura sobre administración, donde probablemente se volvió más influyente que en la economía, durante bastante tiempo.

Algunos economistas, en particular Stigler (1961), han tomado la posición de Simon y han argumentado que en realidad es un partidario de la racionalidad económica total, pero solo agregan otro aspecto a optimizar, a saber, minimizar los costos de la información. La gente todavía está optimizando, pero tiene en cuenta los costes de la información. Sin embargo, el argumento de Stigler enfrenta un problema inevitable e ineludible: la gente no sabe ni puede saber cuáles son los costos totales de la información. En este sentido se enfrentan a un problema potencial de regresión infinita (Conlisk1996). Para conocer los costos de la información, deben determinar cuánto tiempo deben dedicar a este proceso de aprendizaje; deben aprender cuáles son los costos de aprender cuáles son los costos de la información. Esto luego conduce al siguiente problema de orden superior de aprender cuáles son los costos de aprender cuáles son los costos de la información, y esta regresión en principio no tiene fin. 3 Al final, deben usar el tipo de dispositivos heurísticos (o ``regla empírica'') que Simon propone que las personas que enfrentan una racionalidad limitada deben usar para responder a la pregunta. La plena racionalidad es imposible, y la ubicuidad de la complejidad es una razón central por la que este es el caso.

Simón (1976) Distingue sustantivo racionalidad de procedimiento racionalidad. La primera es el tipo de racionalidad asumida tradicionalmente por la mayoría de los economistas en la que las personas pueden lograr una optimización total en su toma de decisiones. Esto último implica que seleccionen procedimientos o métodos mediante los cuales puedan ``hacer lo mejor'' en un mundo en el que esa optimización total es imposible, las heurísticas con las que se manejan en un mundo de racionalidad limitada. En este sentido, no es el caso de que Simon vea a las personas como completamente irracionales o locas. Tienen intereses y generalmente saben cuáles son y los persiguen. Sin embargo, están inevitablemente limitados en su capacidad para hacerlo plenamente, por lo que deben adoptar varios métodos esencialmente ad hoc para lograr sus objetivos satisfactorios.

Entre estas heurísticas que Simon defendía para lograr la racionalidad procesal estaban el ensayo y error, la imitación, el seguimiento de la autoridad, la búsqueda desmotivada y el seguimiento de corazonadas. Pingle y día (1996) utilizó experimentos para estudiar la efectividad relativa de cada uno de estos, ninguno de los cuales claramente puede lograr resultados completamente óptimos. Su conclusión fue que cada uno de estos puede ser útil para mejorar la toma de decisiones, sin embargo, ninguno de ellos es claramente superior a los demás. Es aconsejable que los agentes realicen varios de estos y se muevan de uno a otro en diferentes circunstancias, aunque, como se señaló anteriormente, puede ser difícil saber cuándo hacerlo y exactamente cómo hacerlo. 4

\hypertarget{imitaciuxf3n-e-inestabilidad-de-los-mercados}{%
\section*{Imitación e inestabilidad de los mercados}\label{imitaciuxf3n-e-inestabilidad-de-los-mercados}}
\addcontentsline{toc}{section}{Imitación e inestabilidad de los mercados}

Si bien esta lista de procedimientos que pueden respaldar una búsqueda estrictamente racional de la racionalidad procesal es razonable, un punto que no se ha señalado claramente es que un enfoque excesivo en uno de estos en lugar de otros puede generar problemas. Es evidente que seguir la autoridad puede generar problemas cuando la autoridad es defectuosa, como han demostrado muchos ejemplos desafortunados en la historia. Cualquiera de estos puede generar problemas si se sigue con demasiada intensidad, pero uno que ha jugado un papel particularmente desafortunado en los mercados es la imitación, a pesar de que es un método ampliamente utilizado por muchas personas con una larga historia de éxito evolutivo. El problema es particularmente agudo en los mercados de activos, donde la imitación puede generar burbujas especulativas que desestabilizan los mercados y pueden generar problemas mucho más amplios en la economía, como lo demuestra claramente la crisis de 2008.

Una extensa literatura (MacKay 1852; Baumol1957; Zeeman1974; Rosser Jr.1997) ha reconocido que, si bien los agentes que se centran en los valores fundamentales de los activos a largo plazo tienden a estabilizar los mercados vendiéndolos cuando sus precios superan estos fundamentales y comprando cuando están por debajo de ellos, los agentes que persiguen tendencias pueden desestabilizar los mercados comprando cuando los precios están subiendo, por lo que haciendo que suban más, y viceversa. Cuando aparece una tendencia de precios al alza, los perseguidores de tendencias obtendrán mejores rendimientos que los fundamentalistas y la imitación de aquellos que lo hagan bien llevará a los agentes que podrían haber seguido estrategias fundamentalistas estabilizadoras a seguir estrategias de persecución de tendencias desestabilizadoras, que tenderán a empujar el precio aún más hacia arriba. Y cuando una burbuja finalmente alcanza su punto máximo y comienza a caer, los cazadores de tendencias pueden bajar el precio más rápidamente a medida que se siguen unos a otros en un pánico de ventas.

Smith et al.~(1988), con muchos estudios posteriores que apoyan esta observación. 5 Incluso en situaciones con un horizonte temporal finito y un pago claramente identificado que establece el valor fundamental del activo que se negocia, en mercados experimentales se ha demostrado repetidamente que aparecerán burbujas incluso en estos casos simplificados y bien definidos. La gente tiene una fuerte tendencia a especular y a seguirse unos a otros en una especulación tan desestabilizadora a través de la imitación. Los procedimientos que pueden respaldar la racionalidad procesal en un mundo de racionalidad limitada pueden conducir a malos resultados si se aplican con demasiada fuerza.

Observamos que tales patrones toman regularmente tres patrones diferentes. Una es que el precio suba a un pico y luego caiga abruptamente después de alcanzar el pico. Otra es que el precio suba a un pico y luego disminuya de una manera más gradual de una manera razonablemente simétrica. Finalmente, vemos burbujas subiendo a un pico, luego disminuyendo gradualmente por un tiempo, finalmente colapsando en un choque impulsado por el pánico. Manías, pánicos y bloqueos clásicos de Kindleberger (2001) muestra en su Apéndice B que de 47 burbujas especulativas históricas, cada una de las dos primeras tiene cinco ejemplos, mientras que el resto, la gran mayoría, sigue el patrón final, que requiere agentes heterogéneos que no son completamente racionales para que ocurra (Rosser Jr.~1997). Esto muestra que la complejidad está profundamente involucrada en la mayoría de las burbujas especulativas.

Los gráficos 2.1 , 2.2 y 2.3 muestran la trayectoria temporal de los precios de tres burbujas antes, durante e inmediatamente después de la crisis de 2008. Muestran los tres patrones descritos anteriormente, tomados de Rosser Jr.~et al.~(2012). La primera es para el petróleo, que alcanzó un máximo de 147 dólares por barril en julio de 2008, el precio nominal más alto jamás observado, y luego se desplomó con fuerza a poco más de 30 dólares por barril en noviembre siguiente. Parece que es más probable que los productos básicos sigan este patrón que otros activos (Ahmed et al.2014).

\textbf{Figura 2.1} Precios del petróleo, 2000-2011

\textbf{Figura 2.2} Precios de la vivienda en EE. UU., Índice de Case-Shiller, 1987--2013

\textbf{Figura 2.3} Patrón de precios del mercado de valores de EE. UU., 2000-2011

El segundo patrón fue seguido por la burbuja inmobiliaria, que alcanzó su punto máximo a mediados de 2006 según este gráfico, que muestra dos índices diferentes, el de 10 ciudades de Case-Shiller y el de 20 ciudades también. Mirando de cerca, se puede ver un poco de aspereza alrededor del pico que lo hace parecer casi como el tercer patrón, mientras que, de hecho, si uno mira los mercados de la vivienda en ciudades individuales, se ven como los postula este patrón, con esta aspereza a nivel nacional reflejando que diferentes ciudades alcanzaron su punto máximo en diferentes momentos, con una ronda final de ellas en enero de 2007 antes de que todas declinaran.

Históricamente, este tipo de patrón se ve a menudo con las burbujas del mercado inmobiliario. El declive más gradual que en los otros patrones, casi simétrico con el aumento, refleja ciertos fenómenos de comportamiento. Las personas se identifican de manera muy personal e intensa con sus hogares y, como resultado, tienden a no aceptar fácilmente que su hogar ha perdido valor cuando intentan venderlo durante una recesión. Como resultado, tienden a ofrecer precios demasiado altos y luego se niegan a bajarlos fácilmente cuando no venden. El resultado es una disminución más dramática en el volumen de ventas en la fase descendente en comparación con los otros patrones, ya que la gente se aferra y se niega a bajar los precios.

El tercer caso muestra el mercado de valores estadounidense como lo muestra el promedio Dow-Jones, que alcanzó su punto máximo en octubre de 2007, para luego colapsar en septiembre de 2008. Tales patrones parecen ser más comunes en los mercados de activos financieros. Tales patrones muestran heterogeneidad de agentes con diferentes patrones de imitación, un grupo más inteligente (o más afortunado) que sale antes en la cima, seguido de un grupo menos inteligente (o menos afortunado) que espera que el precio vuelva a subir, solo que entrar en pánico más tarde en masa por cualquier motivo.

Finalmente, la Figura 2.4 muestra cómo este patrón con su período de dificultades financieras (Minsky1972) se puede modelar en un modelo basado en agentes que tiene agentes que cambian de una estrategia a otra en función de sus éxitos relativos, aunque no instantáneamente (Gallegati et al.~2011). Este modelo se basa en ideas de Brock y Hommes (1997, 1998) que subyacen al llamado modelo bursátil de Santa Fe (Arthur et al.~1997b). Lo que desencadena el colapso retrasado son los agentes que se encuentran con restricciones financieras, como sucede cuando las personas deben cumplir con las llamadas de margen en los mercados de valores. La curva más alta muestra el patrón cuando los agentes se imitan entre sí con más fuerza, como en un modelo de mecánica estadística cuando hay una interacción más fuerte entre las partículas.

\textbf{Figura 2.4} Patrón de angustia financiera simulado

\hypertarget{complejidad-jeruxe1rquica-y-la-cuestiuxf3n-del-surgimiento}{%
\section*{Complejidad jerárquica y la cuestión del surgimiento}\label{complejidad-jeruxe1rquica-y-la-cuestiuxf3n-del-surgimiento}}
\addcontentsline{toc}{section}{Complejidad jerárquica y la cuestión del surgimiento}

Si bien podemos ver el descubrimiento de Herbert Simon de la racionalidad limitada como una afirmación indirecta de ser un ``padre de la complejidad'', su afirmación más directa, reconocida por Seth Lloyd en su famosa lista, es su artículo de 1962 para la American Philosophical Society sobre ``The Architecture de Complejidad''. En este ensayo transdisciplinario se ocupa de todo, desde las jerarquías organizativas a través de las evolutivas hasta las que involucran ``sistemas químico-físicos''. Está muy preocupado por el problema de la descomponibilidad de los sistemas de orden superior en sistemas de nivel inferior, señalando que los de producción, como los de relojería, así como los organizativos, funcionan mejor cuando existe dicha descomponibilidad, que depende de la estabilidad y funcionalidad de los sistemas de nivel inferior. 6

Sin embargo, reconoce que muchos de estos sistemas implican una casi descomponibilidad , quizás una complejidad jerárquica equivalente a la racionalidad limitada. En la mayoría de ellos existen interacciones entre los subsistemas, dependiendo la evolución más amplia del sistema de fenómenos agregados. Simon ofrece el ejemplo de un edificio con muchas habitaciones. La temperatura en una habitación puede cambiar eso en otra, aunque sus temperaturas no converjan. Pero las temperaturas generales que están involucradas en estas interacciones están determinadas por la temperatura agregada de todo el edificio.

Simon también se ocupa de lo que muchos consideran el problema más fundamental que involucra la complejidad, a saber, el de la emergencia. Su discusión más seria sobre el surgimiento de niveles más altos de estructura jerárquica a partir de niveles más bajos involucra la evolución biológica, donde estos temas han sido discutidos más intensamente durante mucho tiempo. Argumenta que la forma en que surgieron estos niveles superiores no ha reflejado procesos teleológicos sino procesos estrictamente aleatorios. También argumenta que incluso en sistemas cerrados, no es necesario que haya cambios en la entropía en el agregado cuando los subsistemas emergen dentro de ese sistema. Pero también reconoce que los organismos son sistemas energéticamente abiertos, por lo que ``no hay forma de deducir la dirección, y mucho menos la velocidad, de la evolución a partir de consideraciones termodinámicas clásicas'' (Simon1962, pag. 8). Sin embargo, el desarrollo de formas intermedias estables es la clave para el surgimiento de formas aún más elevadas.

Simon no cita esta literatura más antigua, pero este tema fue central para la literatura británica ``emergentista'' que surgió del siglo XIX para convertirse en el discurso dominante en la década de 1920 con respecto a la historia más amplia de la evolución biológica, todo integrado dentro de una visión más amplia que encajaba. esto dentro del surgimiento de sistemas físicos y químicos desde partículas a través de moléculas a niveles tan superiores por encima de la evolución biológica en términos de conciencia humana, sistemas sociales y aún sistemas superiores (Lewes 1875; Morgan1923) Simon se ocupó de esta multiplicidad de procesos sin establecer su interconexión tan estrechamente como lo hicieron estas figuras anteriores. En la década de 1930 con la síntesis neodarwiniana (Fisher1930; Wright1931; Haldane1932), el énfasis volvió a un proceso darwiniano casi continuo de cambios graduales que surgen del nivel de cambios probabilísticos que surgen de mutaciones a nivel genético, con el gen como el foco último de la selección natural (Dawkins 1976; Rosser Jr.2011a, B).

Si bien Simon evitó abordar este tema de la emergencia en la evolución biológica en 1962, cuando la síntesis reduccionista neodarwiniana estaba en el nivel más alto de su influencia, pronto la visión de la emergencia reaparecería, basada en un proceso evolutivo multinivel (Crow 1955; Hamilton1964; Precio1970). Esto se desarrollaría aún más con el estudio de la dinámica no lineal y la complejidad en tales sistemas, con figuras como Stuart Kaufffmann (1993) y James Crutchfield (1994, 2003), quienes se basan en modelos computacionales para sus descripciones de la autoorganización en sistemas evolutivos biológicos.

Figura 2.5 de Crutchfield (2003, pag. 116) describe cómo una mutación de nivel genético inicial puede conducir a efectos emergentes en niveles más altos. En el lado derecho están los genotipos que se mueven hacia arriba de una cuenca de atracción a otra, mientras que en el lado izquierdo los fenotipos también lo hacen en un patrón paralelo. Introduce el concepto de mesoescalas para tales procesos, que siguen claramente la advertencia de Simon sobre la necesidad de que surjan sistemas intermedios estables para apoyar la aparición de otros de orden superior.

\textbf{Figura 2.5} Emergencia evolutiva

Este punto de vista sigue siendo cuestionado por muchos evolucionistas (Gould 2002). Si bien la tradición que atraviesa la teoría de la catástrofe de D'Arcy Thompson (1917) ha defendido durante mucho tiempo la forma que surge de estructuras profundas en la evolución orgánica, los críticos han argumentado que tales procesos de autoorganización son, en última instancia, procesos teleológicos que replican viejas perspectivas teológicas pre-evolutivas como la de Paley (1802) en el que todas las cosas están en su lugar, como debe ser debido a la voluntad divina. Otros han criticado que tales procesos carecen de principios de invariancia (McCauley2005). Otros argumentan una base más computacional para tales procesos (Moore1990). No hay una resolución fácil de este debate, e incluso aquellos que defienden la importancia de la autoorganización emergente reconocen el papel de la selección natural. Así, Kaufffmann (1993, pag. 644) ha declarado, ``La evolución no es solo `casualidad atrapada en un ala'. No es solo un retoque ad hoc, de bricolaje, de artilugio. Es un orden emergente honrado y perfeccionado por selección''.

Si bien los mecanismos no son los mismos, los problemas de la autoorganización emergente también se aplican a los sistemas socioeconómicos. El enfoque de Simon tendía a estar en las organizaciones y sus jerarquías. Si bien bien pudo haberse alineado con los sintetizadores neodarwinianos más tradicionales en lo que respecta al surgimiento de estructuras de orden superior en la evolución biológica, el papel de la conciencia humana dentro de los sistemas socioeconómicos humanos significa que las reglas son diferentes allí, y la formación de Las estructuras de orden superior pueden convertirse en una cuestión de voluntad y planificación conscientes, no mera aleatoriedad.

\hypertarget{racionalidad-limitada-y-aprender-a-creer-en-el-caos}{%
\section*{Racionalidad limitada y aprender a creer en el caos}\label{racionalidad-limitada-y-aprender-a-creer-en-el-caos}}
\addcontentsline{toc}{section}{Racionalidad limitada y aprender a creer en el caos}

Una de las mayores ironías con respecto a la racionalidad limitada es que fueron colegas de Herbert Simon en Carnegie-Mellon, particularmente John Muth (1961), quien desarrolló la idea de expectativas racionales mientras estudiaba las implicaciones de la racionalidad limitada. Muth, en particular, vio el supuesto de expectativas racionales como una solución a los problemas planteados por la racionalidad limitada. Sin embargo, Herbert Simon nunca tendría nada que ver con este desarrollo, viéndolo como un repudio a la racionalidad limitada. La idea de que la gente no solo sabe cuál es el verdadero modelo de la economía, sino que su visión subjetiva de la distribución de probabilidad del ruido exógeno en el sistema se corresponde con la distribución de probabilidad objetiva de dicho ruido, que también era convenientemente gaussiana, simplemente no era válida. aceptable en su opinión. Aparte de la incapacidad de los agentes racionales limitados para discernir el ``verdadero modelo de la economía'', nunca aceptaría la idea de que el ruido fuera gaussiano. Por supuesto,1955b), por lo que no se unió a sus colegas en su júbilo por el desarrollo de esta idea.

Dicho esto, en determinadas circunstancias puede suceder que los comportamientos heurísticos simples como regla empírica pueden funcionar bien en un mundo de dinámicas no lineales complejas para ayudar a los agentes racionales limitados a imitar dinámicas subyacentes que incluso pueden ser caóticas. Esto puede surgir si los agentes pueden lograr expectativas consistentes o CEE (Hommes y Sorger1998), una idea derivada del trabajo de Grandmont (1998) que se había hecho antes, a pesar de que solo se publicó el mismo año que el suyo. Un ejemplo de esto fue estudiado por Hommes y Rosser Jr.~(2001) para la dinámica de las pesquerías cuando estas pueden presentar patrones caóticos. Tales patrones pueden surgir debido a la tendencia de las pesquerías a exhibir curvas de oferta que se inclinan hacia atrás debido a los límites de capacidad de carga de la mayoría de las pesquerías. Cuando los precios superan cierto nivel que es consistente con el rendimiento máximo sostenido, la cantidad de pescado disminuirá y se capturará menos.

De Rosser Jr.~(2001b), X es la biomasa de los peces en la pesquería, siendo F ( X ) la tasa de crecimiento de X , que a su vez es igual a los rendimientos de cosecha en estado estacionario de la pesquería, h , que a su vez es igual a Q en el diagrama de oferta-demanda en la parte superior derecha de la figura. La porción bionómica está en la parte inferior derecha del diagrama y refleja un Schaeffer (1957) función de rendimiento, siendo r la tasa de crecimiento natural ilimitado de la población de peces y K la capacidad de carga de la pesquería:

\[Q=h=F(X)=r X(1--X / K)\]

Esta logística es bien conocida por ser capaz de exhibir dinámicas caóticas cuando de forma discreta a partir del trabajo de May (1976). Siguiendo a Gordon (1954) con E = esfuerzo de captura medido por el tiempo que los barcos están fuera, q = capturabilidad por barco por día, C = costo, con costo marginal constante = c , p = precio del pescado y δ la tasa de descuento en el tiempo, entonces el costo viene dado por

\[C=c / q X\]

y la función básica de recolección puede ser dada por

\[h(X)=q E X\]

Sobre la base de Clark (1990), Hommes y Rosser Jr.~(2001) derivó una curva de oferta completa que varía con δ . Esto se inclina hacia arriba para δ = 0, acercándose asintóticamente al nivel de producción asociado con el rendimiento máximo sostenido, pero se dobla hacia atrás para δ \textgreater{} 0.02, alcanzando una curva hacia atrás máxima en δ = ∞, en cuyo punto la curva de oferta es idéntica al equilibrio de acceso abierto debido a Gordon1954) dada por

\[S(p)=r c / p q(1--c / p q H)\]

con curva de demanda lineal dada por

\[D(p)=A--B p\]

Hommes y Rosser Jr.~(2001) describen la dinámica de la telaraña de una pesquería de este tipo bajo expectativas adaptativas por medio de una función discreta

\[P_{t}=\left[A--S_{\delta}\left(p_{t-1}\right)\right] / B\]

Hommes y Rosser Jr.~(2001) muestran que esto puede ser caótico para valores dados de δ ya que S varía con él. Esto ocurrirá cuando S esté retrocediendo en esas porciones, lo que también puede conducir a resultados catastróficos a medida que cambia la demanda (Copes1970). 7

La cuestión de los pescadores racionales delimitados surge si les permitimos basar sus expectativas en una simple heurística, p e que representa el precio esperado, de un proceso autorregresivo de un período dado por

\[P^{e}(t)=\alpha+\beta\left(p_{t-1}-\alpha\right)\]

Este proceso AR (1) puede cambiar de acuerdo con el aprendizaje de autocorrelación de la muestra en el que los agentes a lo largo del tiempo ajustan los dos parámetros de control, α y β , en función del desempeño de los pescadores. Basado en el ECE y asumiendo que la dinámica caótica subyacente para la optimización de la pesquería viene dada por un mapa asimétrico de tiendas de campaña, Hommes y Rosser Jr.~(2001) muestran que estos parámetros pueden converger en valores tales que esta simple heurística AR (1) reproducirá la dinámica caótica subyacente, que será una CEE.

Esto se muestra en la Fig.9 de Hommes y Rosser Jr.~(2001), donde los pescadores comienzan capturando un nivel dado de X asumiendo una p constante , pero a medida que β en particular cambia inicialmente, aparece un movimiento de dos períodos, que luego se vuelve caótico después de que se produce el ajuste posterior de ambos parámetros. A este proceso se le ha llamado aprender a creer en el caosObservamos que esta dinámica permanece acotada como todas las dinámicas caóticas, evitando así un colapso catastrófico, un caso de caos que evita la catástrofe. Si bien esto replica en cierta medida las cifras estándar que muestran bifurcaciones que duplican el período hacia el caos, esta no es una de las que involucran la variación de un parámetro de crecimiento. Más bien se trata de un proceso de convergencia en un patrón de comportamiento basado en parámetros autorregresivos que se ajustan en tiempo real, no es lo mismo, incluso si se parece a él.

\hypertarget{economuxeda-del-comportamiento-e-incertidumbre-keynesiana}{%
\section*{Economía del comportamiento e incertidumbre keynesiana}\label{economuxeda-del-comportamiento-e-incertidumbre-keynesiana}}
\addcontentsline{toc}{section}{Economía del comportamiento e incertidumbre keynesiana}

Herbert Simon evitó en gran medida abordar directamente las implicaciones macroeconómicas de sus ideas, más allá de expresar su desaprobación de la hipótesis de expectativas racionales que muchos afirmaban derivada de su trabajo, y esto incluso se afirmó como algo tan fundamental que era axiomático y no podía ser cuestionado por una teoría teórica profunda. y razones filosóficas, a pesar de su evidente y bien conocido incumplimiento de la realidad empírica, un punto del que Simon era plenamente consciente. Dado que su concepto de racionalidad limitada viola las expectativas racionales completas, y también la conexión profunda con la complejidad dinámica no lineal que se ha presentado anteriormente en este capítulo, aunque no tan completamente como podría haber sido, surge la pregunta, empujando más allá de la racionalidad limitada para la economía del comportamiento de manera más amplia,8 idea de incertidumbre fundamental?

La visión convencional es que en 1921 Frank Knight y John Maynard Keynes publicaron libros que establecían la distinción entre riesgo e incertidumbre , y Knight había acuñado claramente esta distinción, pero con el trabajo de Keynes explorando la distinción más profundamente al adoptar la misma terminología más tarde ( Keynes1936; Rosser Jr.2001a). El ``riesgo'' es cuantificable en términos de poder identificar una distribución de probabilidad que sea relevante para comprender un problema. ``Incertidumbre'' significa que no existe tal distribución de probabilidad identificable. A diferencia de Knight, Keynes era más consciente de la posibilidad de varias posibilidades intermedias que surgen de la incapacidad de estimar la medida cuantitativa por la disponibilidad de datos o por otras razones, además de reconocer la dificultad de separar una variedad de distribuciones de probabilidad posiblemente apropiadas. Este último es un tema que se ha debatido más intensamente, especialmente desde la crisis financiera de 2008, ya que el papel de la curtosis o ``colas gordas'' en los rendimientos financieros se ha vuelto más publicitado.

Observadores como Nassim Taleb (2010) Que distingue cisnes grises de cisnes negros .Las primeras involucran distribuciones de probabilidad que muestran colas gruesas y son conocidas, lo que potencialmente puede explicar resultados extremos en los mercados financieros y otras situaciones. Estos últimos involucran una verdadera incertidumbre keynesiana / knightiana, donde es imposible asignar una distribución de probabilidad, y donde los eventos descritos ``surgen de la nada'' sin ninguna posibilidad de pronosticarlos o esperarlos. Al respecto, Taleb argumentó que la crisis de 2008 fue un simple cisne gris, un desenlace extremo, que sin embargo obviamente estaba llegando y era esperado por cualquier observador razonable, en contraste con el desplome del mercado de valores del 19 de octubre de 1987, 22\% para el promedio Dow-Jones, hasta el día de hoy, la mayor disminución de un día jamás, que no fue predicha por nadie y no tuvo una causa obvia, que ``salió de la nada'', y que fue un verdadero cisne negro,9

Rosser Jr.~(1998, 2006) ha argumentado que la complejidad proporciona una base fundamental para la realidad de la incertidumbre fundamental. Paul Davidson (1996) ha argumentado que este no es el caso, que no solo la complejidad, sino nociones como la racionalidad limitada simoniana no son fundamentos propios o fundamentales de la incertidumbre fundamental. Distingue la incertidumbre ontológica de la epistemológica , argumentando que la verdadera incertidumbre keynesiana es la primera basada en la realidad de la no ergodicidad en la mayoría de las relaciones dinámicas en el mundo real (Davidson1982-83). Por el contrario, considera que la racionalidad limitada y las diversas variabilidades que surgen de la dinámica compleja no lineal son meramente epistemológicas. Si las personas tuvieran conocimientos y sistemas de predicción realmente precisos y precisos, podrían superar estas dificultades. El énfasis de Simon en las limitaciones del conocimiento y las limitaciones computacionales por parte de los individuos es objeto de especial escrutinio y crítica en este sentido. La base de la racionalidad limitada (y la complejidad) no es la incertidumbre fundamental, sino la mera incapacidad para calcular y saber. Si tan solo tuviéramos supercomputadoras con superconocimiento, todo estaría bien.

No existe una resolución definitiva de este debate, aunque debe tenerse en cuenta que una fuente importante de falta de ergodicidad dentro de muchos sistemas es la no linealidad de las relaciones dinámicas subyacentes que conduce a la complejidad. Pero como es bien sabido en el estudio econométrico de la dinámica caótica, es profundamente difícil distinguir la dinámica caótica determinista del ruido aleatorio (Dechert1996). Este debate se enfrenta a esta profunda incertidumbre propia.

Tal como está, mientras que la economía del comportamiento puede o no ser la base de la verdadera incertidumbre keynesiana / knightiana, cisnes negros talebianos, pero puede proporcionar una forma posible de lidiar con la política en un mundo sujeto a tal incertidumbre de cualquier fuente. Por lo tanto, aunque muchos macroeconomistas la ignoran de manera absurda, la de George Akerlof (2002) La macroeconomía del comportamiento casi con certeza está afectando fuertemente a los responsables de la formulación de políticas en la práctica, incluso si no hablan abiertamente de su influencia. Los banqueros centrales del mundo real y otros formuladores de políticas macroeconómicas están siguiendo patrones de comportamiento heurísticos como lo recomendó el fallecido Herbert A. Simon, incluso si pocos de ellos admitirán hacerlo.

\hypertarget{economuxeda-del-comportamiento-y-la-complejidad-de-la-evoluciuxf3n-institucional}{%
\section*{Economía del comportamiento y la complejidad de la evolución institucional}\label{economuxeda-del-comportamiento-y-la-complejidad-de-la-evoluciuxf3n-institucional}}
\addcontentsline{toc}{section}{Economía del comportamiento y la complejidad de la evolución institucional}

El vínculo entre la economía institucional y la economía evolutiva se remonta al trabajo de Thorstein Veblen (1898). Es en gran parte en reconocimiento de este hecho que la primera organización en los Estados Unidos dedicada al estudio de la economía institucional se llama Association for Evolutionary Economics, 10 con nombres similares que se utilizan en otras naciones para dicho estudio, incluso en Japón (Shiozawa et al.~Alabama.2019). Si bien no fue reconocido en ese momento y sigue siendo poco conocido, Veblen no solo pidió que la economía sea una ciencia evolutiva , sino que introdujo ciertas ideas que desde entonces han demostrado ser importantes para comprender la naturaleza de la complejidad en la economía, particularmente la de la causalidad acumulativa. , a menudo, 11 que muchos piensan que fueron introducidos más tarde por Allyn Young (1928) o Gunnar Myrdal (1957), y este último ha hecho que el término sea ampliamente conocido entre los economistas. Entre las diversas formas de complejidad que son relevantes para la economía, la causalidad acumulativa está más obviamente ligada a la complejidad dinámica , que conduce a rendimientos crecientes, equilibrios múltiples y una variedad de bifurcaciones en los sistemas dinámicos económicos. Sin embargo, se puede ver que también está relacionado con la complejidad computacional , el principal rival de la complejidad dinámica en el análisis económico.

Un tema importante para la cuestión de cómo la teoría evolutiva se relaciona con la economía institucional en su formulación temprana involucra las relaciones de Veblen con John R. Commons y Joseph Schumpeter. Veblen desarrolló ideas de la economía evolutiva darwiniana a principios del siglo XX en los Estados Unidos, mientras que Schumpeter es ampliamente visto como un firme partidario de un enfoque evolutivo del desarrollo económico, particularmente en lo que respecta a la evolución de la tecnología, incluso cuando criticó la economía institucional y la aplicación. de ideas biológicas (Rosser Jr.~y Rosser2017). Tampoco es muy conocido, Commons (1924) también apoyaba una visión evolutiva, aunque tenía una perspectiva más teleológica sobre eso que Veblen o Schumpeter, quienes no veían la dirección necesaria para la evolución y el cambio tecnológico (Papageorgiou et al.~2013). Al tratar con un tema de complejidad, Schumpeter abogó firmemente por una visión discontinua o saltacionalista de la evolución (Schumpeter1934; Rosser Jr.1992), con el que Veblen estuvo de acuerdo con respecto al cambio tecnológico. Con respecto a la evolución institucional, Veblen la vio en su mayoría proceder de una manera más continua a través de la causalidad acumulativa, por lo que se acercó un poco más a Commons en ese asunto, incluso cuando argumentó que era fundamentalmente inestable y que experimentaría crisis y colapsos.

Un tema central para la economía institucional es la distinción entre instituciones y organizaciones (North 1990). Esto se vuelve central para el papel de la evolución en la economía, en particular, cuál es el meme que es el lugar de la selección natural evolutiva. En la literatura más antigua, el énfasis estaba más en las organizaciones, como con Commons (1934) que vieron organizaciones compitiendo entre sí, un tema también recogido por Alchian (1950), incluso cuando Commons enfatizó las estructuras más profundas de las instituciones en los sistemas legales. Mientras las organizaciones compiten, los economistas cada vez más evolucionistas se han centrado en las prácticas y rutinas como los memes más cruciales, siendo este un tema especial entre los seguidores neo-schumpeterianos como Nelson y Winter (mil novecientos ochenta y dos).

Un elemento importante de los procesos evolutivos es el surgimiento de estructuras de nivel superior a partir de estructuras de nivel inferior y más simples. Esto es más obvio en términos de organizaciones, pero en la evolución institucional el papel de los memes se vuelve crucial. Esto encaja con el tema de la evolución multinivel, controvertido durante mucho tiempo en la teoría de la evolución (Heinrich2004). Dentro de los sistemas humanos, esto se vincula a la cooperación, con Ostrom (1990) desarrollando cómo esa cooperación puede surgir a través de instituciones particulares. Este proceso de emergencia está ligado a conceptos profundos de complejidad, con Simon (1962) un desarrollador crucial de esta línea de pensamiento.

La comprensión de la compleja dinámica de la evolución institucional puede generar una posible reconciliación o incluso una síntesis entre la vieja y la nueva economía institucional. Coase (1937) reconoció que Commons originó la idea de la importancia de los costos de transacción, la pieza central de la nueva economía institucional (Williamson 1985). Mikami (2011) que ha argumentado que el esfuerzo por minimizar el costo de las transacciones puede conducir a una dinámica evolutiva compleja. Esto puede involucrar la causalidad acumulativa de Veblen, reconociendo cómo esto puede vincularse a una emergencia evolutiva compleja.

En el momento en que Thorstein Veblen estaba escribiendo sus obras más importantes, cuando el siglo XIX se convertía en el XX, no había una conciencia clara o general de lo que ahora llamamos complejidad , incluso aunque muchas ideas con las que ahora lo asociamos habían estado flotando en el mundo. varias disciplinas durante muchos años, especialmente en matemáticas e incluso algo en economía (Rosser Jr.2009b). No tenemos ninguna razón para creer que Veblen fuera particularmente consciente de estos aspectos, aunque la evolución misma se considera ahora como un proceso de complejidad por excelencia (Hodgson y Knudsen2006), que Veblen abogaría firmemente. 12 En cualquier caso, un aspecto central del enfoque de Veblen sobre la evolución económica fue su invocación de la idea de causalidad acumulativa , que fue el primero en introducir. 13 Debemos señalar que la causalidad acumulativa puede conducir a complejidades dinámicas a través de rendimientos crecientes, que Brian Arthur (1989, 1994) ha argumentado que es la clave central para comprender la complejidad, y que Veblen reconoció como presente en la tecnología industrial.

\hypertarget{el-debate-de-la-discontinuidad-en-la-teoruxeda-evolutiva}{%
\section*{El debate de la discontinuidad en la teoría evolutiva}\label{el-debate-de-la-discontinuidad-en-la-teoruxeda-evolutiva}}
\addcontentsline{toc}{section}{El debate de la discontinuidad en la teoría evolutiva}

Fue Leibniz quien inicialmente acuñó la frase natura non facit saltum , o ``la naturaleza no da un salto''. Sería recogido por el propio Darwin quien lo repitió y lo aplicó a su teoría de la selección natural, y Marshall seguiría a Darwin al aplicarlo a la economía, repitiéndolo en los Prefacios de las ocho ediciones de sus Principios de Economía . Para Darwin (1859, págs. 166-167):

``Natura non facit saltum\ldots{} ¿Por qué la naturaleza no debería dar un salto de estructura en estructura? En la teoría de la selección natural podemos entender claramente por qué no debería hacerlo: porque la selección natural sólo puede actuar aprovechando ligeras variaciones sucesivas; nunca puede dar un salto, sino que debe avanzar con los pasos más cortos y lentos''.

Esta fue una declaración contundente para Darwin dado que no entendía los fundamentos de cómo funcionaba el proceso de mutación a través de cambios en los genes, pero de hecho, muchos teóricos de la evolución desde Darwin han quedado impresionados por la idea de que solo pueden ocurrir cambios menores en los genes. en un momento para que las especies sean viables y sobrevivan y se reproduzcan, estableciendo así que al menos la mayoría de los procesos evolutivos sean lentos y graduales, como afirma Darwin. Sin embargo, hasta que la comprensión de la genética se integró completamente en la teoría darwiniana con la síntesis neodarwiniana en la década de 1930, hubo una mayor apertura para un cambio discontinuo más notable en la perspectiva lamarckiana que permitió la herencia de características adquiridas y, por lo tanto, más cambio evolutivo rápido.

Después de la década de 1930, la reafirmación más dramática de la posibilidad de un cambio rápido en forma de equilibrio puntuado vendría con Eldredge y Gould (1972), cuyos argumentos siguen siendo controvertidos entre los biólogos evolutivos. Sin embargo, la base de sus argumentos se estableció en el desarrollo de la síntesis neodarwiniana en sí durante la década de 1930, incluso si no fue claramente reconocida en ese momento. Una parte central de la síntesis neodawiniana, especialmente tal como la formuló Fisher (1930), implicaba centrarse en el gen, con la selección natural operando al nivel del gen, lo que contrastaba con las teorías que veían la selección natural operando a niveles más altos en totalidades. Los cambios en el nivel de un gen deben ser bastante pequeños para que sean viables, pero un método para estudiar esto a través de paisajes de aptitud como lo introdujo Sewall Wright (1932) abrió la puerta a una perspectiva más amplia, que puede trasladarse al estudio de la evolución institucional (Mueller 2015).

Una pieza de base siempre presente con respecto al marco del paisaje de aptitud de Wright que abrió la puerta a tales discontinuidades o puntuaciones saltacionalistas fue que Wright desde el principio permitió múltiples óptimos o equilibrios locales dentro de esos paisajes. Si bien él mismo no vio discontinuidades dramáticas en el nivel genético, reconoció que los cambios ambientales rápidos podrían cambiar los paisajes de modo que un antiguo pico podría convertirse con bastante rapidez en un valle y el pico más cercano al que se puede llegar por un gradiente podría estar a cierta distancia, lo que implicaría una evolución rápida, si no necesariamente discontinua en genotipo y fenotipo. 14 La Figura 2.6 muestra la descripción original de Wright de los paisajes de fitness y ciertos casos que podrían suceder (Wright1932, reproducido en Wright 1988, pag. 110), con el recuadro C que muestra el caso que se acaba de describir, un cambio de paisaje debido a algún cambio ambiental, que podría ocurrir de forma bastante repentina.

\textbf{Figura 2.6} Paisajes de fitness de Sewall Wright

Con respecto a la aplicación de estas ideas a la evolución económica y más específicamente a la evolución institucional, se acepta generalmente que si bien Marshall pudo haber estado de acuerdo con Leibniz y Darwin en que natura non facit saltum , Veblen tendió a aceptar la idea de que la evolución institucional podría ser discontinua, o en al menos, los equilibrios institucionales no eran estables y podían cambiar repentinamente. Así declaró (Veblen1919, pag. 242--243):

\begin{quote}
``No sólo la conducta del individuo está rodeada y dirigida por sus relaciones habituales con sus compañeros del grupo, sino que estas relaciones, al ser de carácter institucional, varían a medida que varía el escenario institucional. Los deseos y anhelos, el fin y el fin, los caminos y los medios, la amplitud y deriva de la conducta del individuo son funciones de una variable institucional de carácter altamente complejo e inestable''.
\end{quote}

Curiosamente, aunque Schumpeter apoyó firmemente la idea del cambio tecnológico discontinuo y utilizó el lenguaje de la evolución en el contexto del desarrollo económico, rechazó el uso de analogías biológicas en tales discusiones, declarando que (Schumpeter 1954, pag. 789), ``ninguna apelación a la biología sería de la más mínima utilidad''. Descartó los mecanismos selectivos ya sean de tipo darwiniano o lamarckiano, usando la palabra ``evolución'' de una manera simplemente evolutiva (Hodgson1993a, B).

Si bien Wright no lo deletreó, una clave para la existencia de múltiples equilibrios locales en sus paisajes de fitness es la presencia de algún tipo de rendimientos crecientes. Esto trae a Arthur's (1994) énfasis en los rendimientos crecientes y su vínculo con la existencia de equilibrios múltiples y complejidad dinámica, 15 que se traslada a la evolución institucional. Minniti1995) utilizó una variación de Arthur et al.~(1987) urn modelo para mostrar cómo pueden surgir equilibrios de delincuencia alta y baja en una sociedad, con interacciones sociales que proporcionan retroalimentaciones positivas como la clave para tal resultado, con posibles discontinuidades que surgen a medida que la cantidad de delincuencia puede cambiar muy repentinamente de un estado a otro. Esto se muestra en la figura 2.7, donde el eje horizontal es el porcentaje de la población que es delincuente, mientras que el eje vertical muestra la probabilidad de que un nuevo miembro de la sociedad sea un delincuente. Rosser Jr.~y col.~(2003b) aplicaron este modelo de economías informales en economías en transición, existiendo también equilibrios múltiples, como se aprecia en las grandes diferencias en esta variable entre las economías en transición de Europa del Este, con el grado de desigualdad jugando un papel importante como se analiza en el próximo capítulo.

\textbf{Figura 2.7} Equilibrios sociales múltiples

\hypertarget{instituciones-organizaciones-y-el-lugar-de-la-evoluciuxf3n-econuxf3mica}{%
\section*{Instituciones, organizaciones y el lugar de la evolución económica}\label{instituciones-organizaciones-y-el-lugar-de-la-evoluciuxf3n-econuxf3mica}}
\addcontentsline{toc}{section}{Instituciones, organizaciones y el lugar de la evolución económica}

Si las economías son sistemas evolutivos, entonces la cuestión de cuál es el lugar de esa evolución es importante. Hodgson y Knudsen (2006) argumentan que hay tres características cruciales involucradas en la evolución verdaderamente darwiniana: variabilidad, selección natural y herencia. Para que algo califique como un locus de evolución, debe exhibir los tres. En la evolución biológica, el gen ciertamente cumple con todos estos: la mutación proporciona variabilidad aleatoria, la selección natural determina si un organismo que contiene un gen sobrevivirá o no, y los genes pasan de un organismo a otro a través de la reproducción si el organismo es capaz de sobrevivir y atraer parejas. para efectuar esto. Los críticos de la economía evolutiva argumentan que no existe una unidad o elemento definitivo en las economías que cumplan con los tres, incluso si muchos cumplen con algunos de ellos.

Dada la larga defensa de los seguidores institucionalistas de Veblen para hacer de la economía una ciencia evolutiva, estos temas han sido centrales en los debates dentro de esta área. El enfoque en las organizaciones ha atraído la atención durante mucho tiempo, y esto posiblemente sea más importante para Commons que para Veblen. Para Commons, la selección dirigida o artificial era más importante que la selección natural estrictamente aleatoria, y señaló que el propio Darwin pasó mucho tiempo discutiendo tanto la selección natural aleatoria como la reproducción artificial (Commons1934, pag. 657; Vanberg1997). 16 Commons consideraba que las organizaciones estaban sujetas a una dirección y, por lo tanto, objetos apropiados para este tipo de evolución dirigida, que tenía como objetivo el mejoramiento humano general. En su argumento a favor de la evolución como fuerza fundamental en microeconomía, Armen Alchian (1950) enfatizó la competencia de las empresas, y la supervivencia del más apto implica qué empresas pueden acercarse más a maximizar las ganancias, incluso si no saben con precisión cómo lo están haciendo, siendo las empresas claramente el lugar de la evolución.

Una crítica a la idea de que las empresas, o de manera más general, las organizaciones, que sirven como el lugar clave de la evolución en la economía, es que, si bien están sujetas a una variabilidad aleatoria a medida que experimentan los choques del sistema, la selección natural opera claramente en su competencia entre sí. , con empresas no rentables que no logran sobrevivir, la pieza que falta es la herencia. Las empresas y organizaciones no se reproducen esencialmente a sí mismas. Todo lo que hacen es sobrevivir, aunque pueden cambiar mientras lo hacen. Estos cambios pueden reflejar estas fuerzas evolutivas de la selección natural, pero el elemento hereditario de hacerlo debe estar operando en un nivel más bajo que el de la empresa u organización en sí.

La principal alternativa para servir como el meme evolutivo 17 son los hábitos o prácticas dentro de una organización. Si bien no se vieron impulsados \hspace{0pt}\hspace{0pt}a este argumento al tratar de encajar la nueva economía institucional en un marco evolutivo per se, así es como North (1990) y Williamson (2000) definir instituciones. Son hábitos o prácticas, no organizaciones. Esto es también lo que Nelson y Winter (mil novecientos ochenta y dos) llegaron en su búsqueda de la clave de la economía evolutiva, aunque etiquetaron estos memes como ``rutinas''. Pero antes de cualquiera de estos y antes de Commons y su énfasis en las organizaciones, Veblen identificó los hábitos, incluidos los hábitos de pensamiento, como el lugar central de evolución en las instituciones económicas, declarando (Veblen1899, págs.190-191):

\begin{quote}
``La situación de hoy da forma a las instituciones del mañana a través de un proceso coercitivo y selectivo, actuando sobre la visión habitual de las cosas de los hombres y, por lo tanto, alterando o fortaleciendo un punto de vista o una actitud mental transmitida desde el pasado''.
\end{quote}

Dado que, como él mismo dijo, la conducta del individuo está ``rodeada por sus relaciones habituales con sus compañeros del grupo'', con estas relaciones de ``carácter institucional'', son los hábitos y las relaciones habituales las que están en la base de la evolución de la sociedad. instituciones, incluso si él ve estas instituciones como estructuras sociales de orden superior. Son los hábitos los que están en la base, y los hábitos pueden cambiar, dando lugar a nuevos hábitos que pueden ser heredados por las personas y organizaciones que los utilizan. 18

\hypertarget{emergencia-y-evoluciuxf3n-multinivel}{%
\section*{Emergencia y evolución multinivel}\label{emergencia-y-evoluciuxf3n-multinivel}}
\addcontentsline{toc}{section}{Emergencia y evolución multinivel}

Entre las ideas más fuertemente asociadas con la complejidad está la de emergencia , que una entidad de orden superior surge de una de nivel inferior que no es simplemente la suma de las partes de la de nivel inferior, que la entidad emergente es algo cualitativamente diferente. Si bien la idea de que un todo es mayor que la suma de sus partes ha existido durante mucho tiempo, una formalización científica de la misma probablemente se deba a John Stuart Mill (1843) en sus discusiones de lógica en las que caracterizó situaciones en las que algo cualitativamente diferente de sus partes aparece como representación de leyes heteropáticas . Sus ejemplos originales involucraron la química, como la forma en que aparece la sal cuando se combina el sodio con el cloro, y la sal no se parece en nada a ninguno de los dos por separado. Lewes1875) aplicó el término emergencia a tales fenómenos. Esto condujo a la escuela de pensamiento ``emergente británica'' que, especialmente en la década de 1920 (Morgan1923) aplicaría este concepto a la evolución, en particular a problemas tales como cómo los organismos multicelulares surgen de los unicelulares. Se aplicaría a cómo los grupos sociales más grandes se organizarían para actuar juntos a partir de grupos separados previamente más pequeños, una idea claramente importante en la evolución de las instituciones (McLaughlin1992).

En la teoría de la evolución biológica, este punto de vista cayó en desgracia en la década de 1930 con el surgimiento de la síntesis neodarwiniana, que puso el foco en el gen como el locus de la evolución, el meme, como Dawkins (1976) lo etiquetó. La idea de que la selección natural se produjo en niveles superiores al gen, en el nivel de ``totalidades'' o grupos, fue rechazada específicamente (Williams1966). El contrario obvio a esto en la evolución biológica involucra a los insectos sociales (Wilson2012), en el que los individuos se subordinan al bien de la colonia, convirtiéndose la colonia en vehículo de evolución. La mayoría atribuye la comprensión matemática de cómo esto puede surgir al trabajo de Price (1970) y Hamilton (1964, 1972). Sin embargo, de hecho, la formalización original de este entendimiento en términos de selección dentro del grupo versus entre grupos se debió a Crow (1955).

Sea B w la regresión génica dentro del grupo sobre el valor de aptitud del rasgo definido por Wright (1951); B b sea \hspace{0pt}\hspace{0pt}la regresión génica entre grupos al valor de aptitud; V w es la varianza entre los individuos dentro de un grupo y V b es la varianza entre las medias entre los grupos. Para un gen altruista, uno esperaría que B w sea \hspace{0pt}\hspace{0pt}negativo (que el comportamiento dentro del grupo dañe al individuo), mientras que B b sería positivo (el comportamiento del individuo ayuda al grupo). A partir de esto, una condición suficiente para que el gen altruista aumente en frecuencia viene dada por

\[B_{\mathrm{b}} /\left(-B_{\mathrm{w}}\right)>V_{\mathrm{w}} / V_{\mathrm{b}}\]

Dentro de la biología se ha argumentado ampliamente que esta condición rara vez se cumple. Sin embargo, también se ha reconocido que parece ser válido para los insectos sociales, y como Wilson (2012) argumenta, esto implica que aunque solo una minoría de especies muestran esta característica, terminan constituyendo una gran porción de la biomasa animal en la tierra (especialmente si se incluye a los seres humanos en ese cálculo).

De hecho, esta formulación se puede trasladar a los humanos para resolver el problema de la cooperación frente a las trampas dentro del contexto de la teoría del juego del dilema del prisionero (Heinrich 2004). El problema específico para el ser humano pasa a ser el de reconocer quién es un cooperador y quién no dentro de los grupos sociales, siendo el hacerlo con éxito la condición para que se produzca la cooperación y una coordinación de mayor nivel. Considerar en detalle cómo esa cooperación puede surgir en numerosos contextos para tratar con recursos de propiedad común fue el enfoque central del trabajo de Ostrom (1990). En general, esto puede verse como una condición para el surgimiento de instituciones de nivel superior a partir de instituciones de nivel inferior. 19

Algo paralelo a esto es una formulación de emergencia en la evolución biológica debido a Eigen y Schuster (1979) conocido como hiperciclo , que implica la conservación y transmisión de información, vinculándolo más a formas computacionales de complejidad. ``El sistema más simple que puede permitir la evolución de enlaces reproducibles'' (Eigen y Schuster1979, pag. 87). Definen un umbral de contenido de información, que si se excede para un sistema conducirá a una degeneración de la información debido a una catástrofe de error . Por encima de una catástrofe de errores hay una ``desintegración de la información debido a una acumulación constante de errores'' (Eigen y Schuster1979, pag. 25).

Sea V m el número de símbolos, σ m \textgreater{} 1 el grado de superioridad de la ventaja selectiva de la ``copia maestra'' y q m la calidad de la copia de símbolos. El umbral viene dado por

\[V_{\mathrm{m}}<\ln \sigma_{\mathrm{m}} /\left(1-q_{\mathrm{m}}\right)\]

Tal formación de hiperciclo ha sido simulada por Mosekilde et al.~(1983), y el concepto ha sido aplicado a la evolución de las estructuras de mercado basadas en tasas diferenciales de aprendizaje entre empresas por Silverberg et al.~(1988). También se ha relacionado con el concepto de autopoesis , definido como la reproducción estable de una estructura espacio-temporal (Varela et al.1974).

Esto puede verse como vinculado a la autoorganización tal como la formuló Turing (1952) en forma de morfogénesis . Cuando tal morfogénesis implica la emergencia a un nivel superior, se convierte en morfogénesis hipercíclica (Rosser Jr.1991, Cap. 6), o el momento anagenético de Rosser Jr.~et al.~(1994). Radzicki (1990) aplicó tales argumentos a la cuestión de la formación de instituciones a partir de dinámicas caóticas subyacentes. 20 Dentro de la evolución, el surgimiento de niveles jerárquicos superiores también fue el foco central de Simon (1962).

Esto plantea paralelos dentro de los modelos teóricos de juegos evolutivos del tema de la evolución multinivel (Heinrich2004), con las ecuaciones de Price-Hamilton proporcionando condiciones suficientes para que esto ocurra, aunque la versión original se debió a Crow (1955). Para su población, B w y B b son regresiones genéticas de aptitud dentro y entre grupos sobre el valor del rasgo, V w y V b son las varianzas genéticas dentro y entre grupos, con W la aptitud media de la población, luego

\[\Delta C=\left(B_{\mathrm{w}} V_{\mathrm{w}}+B_{\mathrm{b}} V_{\mathrm{b}}\right) / W\]

Esto permite una declaración de Hamilton (1972) condición para que un rasgo altruista aumente (el equivalente a la cooperación en un nivel superior) como

\[B_{\mathrm{w}} /\left(B_{\mathrm{b}-}-B_{\mathrm{w}}\right)<r_{,}\]

donde r es el coeficiente de relación de Sewall Wright (Crow y Aoki1984). El lado izquierdo se puede interpretar como una relación entre el costo de la aptitud y el beneficio menos la aptitud.

Otra línea de procesos evolutivos emergentes está asociada con la visión neo-schumpeteriana fuertemente asociada con Nelson y Winter (mil novecientos ochenta y dos) y su estudio de cuáles son los memes clave en la economía evolutiva. Son conocidos por defender la idea de que las rutinas son el meme clave que es el lugar de tales desarrollos evolutivos. Los mismos Nelson y Winter estaban menos enfocados en este asunto de los órdenes superiores emergentes que se convierten en el lugar de la evolución, pero algunos de sus seguidores han perseguido tales ideas. En particular ha sido el desarrollo de la idea de mesoeconomía por Dopfer et al.~(2004), originalmente debido a Ng (1986). Este es un nivel de economía que es intermedio entre la microeconomía de la empresa donde presumiblemente operan principalmente los procesos de Nelson y Winter y el nivel de macroeconomía totalmente agregado. El nivel mesoeconómico es más a nivel de industria o sector donde un meme puede haberse difundido entre empresas dentro de un sector o incluso en un conjunto de sectores relacionados. Estos desarrollos pueden llevar a que ésta sea la parte más importante de la economía desde el punto de vista del crecimiento y el desarrollo evolutivo.

En términos de evolución institucional que opera en niveles más altos de estructuras emergentes, un partidario posiblemente sorprendente de este punto de vista es el economista austriaco Friedrich Hayek. Esto parecería estar asociado, al menos en parte, con su abierta aceptación de la complejidad (Hayek1967) y especialmente en conexión con esto el concepto de emergencia, que se remonta abiertamente a los emergentistas británicos de la década de 1920. Su apertura a esta línea de pensamiento provino de sus primeros trabajos en psicología que culminaron en su The Sensory Order (Hayek1952). En este trabajo, vio específicamente la conciencia humana como una propiedad emergente que surge del sistema nervioso y el cerebro (Lewis2012). Crucial en su formulación fue la influencia de la teoría de sistemas desarrollada por Ludwig von Bertalanffy (1950), quien a su vez fue influenciado por la cibernética de Norbert Wiener (1948), considerada por muchos como otra forma temprana de complejidad dinámica. Más profundamente detrás de la cibernética fue el desarrollo del ``sistema universal de organizaciones'' o tectología de AA Bogdanov (1925-29), posiblemente una forma de economía institucional evolutiva que enfatiza el surgimiento. 21

De hecho, Hayek (1988) en su trabajo final, The Fatal Conceit , aplicó su visión de la complejidad emergente que involucra la evolución en una forma de orden superior, con tales estructuras institucionales emergentes compitiendo entre sí y evolucionando como todos compitiendo entre sí y sobreviviendo o no a través de un proceso sistémico. seleccion natural. Algunos argumentarían que esta aceptación de la selección natural que opera en el nivel de totalidades sociales de orden superior constituía una contradicción con el individualismo metodológico de la escuela austriaca, aunque de hecho en esto se remonta a las ideas evolutivas del fundador de esa escuela, Carl Menger. (1923) que, al igual que Hayek, se desarrolló completamente al final de su carrera.

\hypertarget{economuxeda-institucional-antigua-y-nueva-desde-una-perspectiva-evolutiva-compleja}{%
\section*{Economía institucional antigua y nueva desde una perspectiva evolutiva compleja}\label{economuxeda-institucional-antigua-y-nueva-desde-una-perspectiva-evolutiva-compleja}}
\addcontentsline{toc}{section}{Economía institucional antigua y nueva desde una perspectiva evolutiva compleja}

Los viejos y nuevos enfoques de la economía institucional han sido vistos durante mucho tiempo como en profundo conflicto, con el enfoque evolutivo derivado especialmente de Veblen del viejo punto de vista en conflicto con la mayor aceptación de la economía neoclásica afirmada por el nuevo, comenzando con Coase (1937). De hecho, fue Veblen quien inicialmente acuñó la frase ``economía neoclásica'', que usó de manera peyorativa para criticar el enfoque de equilibrio de Alfred Marshall y otros, por lo que la aceptación de Coase de este enfoque y el esfuerzo por encajar en él la nueva economía institucional parece ser un conflicto profundo difícil de superar. El vínculo entre la idea de Veblen de causalidad acumulativa y la teoría de la complejidad dinámica moderna parecería simplemente reforzar este desacuerdo entre los enfoques.

El concepto unificador central de la nueva economía institucional es el de costo de transacción y minimizarlo es el núcleo central de cómo se forman y se desarrollan las instituciones y organizaciones. Si una empresa subcontrata una actividad o la lleva a cabo dentro de sí misma, está determinado por cuál de estos minimizará sus costos de transacción, como argumentó inicialmente Coase (1937), con esto llevado adelante por Williamson (1985) y North (1990) en su formulación más explícita del nuevo enfoque de la economía institucional. Debemos señalar que Coase en particular, algo así como Schumpeter, rechazó específicamente la aplicación directa de ideas biológicas o evolutivas a su visión de la economía.

Incluso cuando Coase se opuso a la visión evolutiva de la vieja economía institucional de Veblen, reconoció vínculos con partes de sus opiniones. En particular, la idea de que los costos de transacción son importantes fue algo que obtuvo inicialmente de Commons (1934), y Williamson también reconoció más tarde esta fuente. Como ya se señaló, Commons adoptó una visión de la evolución institucional que enfatizaba su dirección y su sujeción a las decisiones humanas conscientes, al igual que con los criadores de animales estudiados por Darwin y Sewall Wright. Las instituciones pueden ser creadas conscientemente por personas sin que simplemente aparezcan o emerjan de algún misterioso proceso dinámico más allá del control humano. Mikami (2011) quien sostiene que incluso si a Coase no le gustaba la biología, sus puntos de vista simpatizan con la sociobiología y que el esfuerzo por minimizar los costos de transacción puede conducir a un proceso dinámico que es complejo.

\hypertarget{resumiendo}{%
\section*{Resumiendo}\label{resumiendo}}
\addcontentsline{toc}{section}{Resumiendo}

Herbert A. Simon fue el ``padre de la economía del comportamiento'' que formuló el concepto de racionalidad limitada a partir de eso. También fundó la noción jerárquica de complejidad que atraviesa los límites disciplinarios, lo que tiene implicaciones para el surgimiento evolutivo en estructuras de nivel superior en la naturaleza. Esto va más allá de la biología a una visión más amplia del universo, con un proceso evolutivo emergente que se extiende desde la aparición de átomos a partir de partículas subatómicas hasta la conciencia humana y más allá.

Para comprender la compleja evolución de las instituciones económicas es fundamental comprender plenamente las implicaciones de las ideas del fundador de la economía evolutiva y la economía institucional, Thorstein Veblen. Particularmente importante fue su formulación del concepto de causalidad acumulativa, que luego retomaron de manera más prominente figuras como Young, Myrdal y Kaldor. Esto se vincula con la teoría de la complejidad dinámica moderna a través de rendimientos crecientes, lo que conduce a múltiples equilibrios y dinámicas de desequilibrio complejas. La visión de Veblen era completamente darwinista en el sentido de que no propuso ninguna evolución teleológica dirigida de la manera que más favorecía el economista institucional John R. Commons.

De las ideas de Veblen sobre la evolución institucional surge también la posibilidad de la emergencia compleja de órdenes superiores de instituciones basadas en la cooperación, vinculadas a las ideas de Herbert Simon, así como basándose en la teoría de la evolución multinivel desarrollada por biólogos como Crow, Hamilton. , y el precio. La existencia y competencia entre instituciones económicas jerárquicas también implica problemas de complejidad computacional, nuevamente sin una dirección definida o un resultado probable. Esto revela las profundas relaciones entre la complejidad y la economía del comportamiento.

\hypertarget{notas-al-pie}{%
\section*{Notas al pie}\label{notas-al-pie}}
\addcontentsline{toc}{section}{Notas al pie}

\begin{enumerate}
\def\labelenumi{\arabic{enumi}.}
\tightlist
\item
  Los problemas que surgen de la complejidad dinámica, como las discontinuidades repentinas y la dependencia sensible de las condiciones iniciales, implican una dificultad extrema para que los agentes se formen expectativas racionales con respecto a eventos futuros, mucho menos información completa y una racionalidad completa en su toma de decisiones. Otra fuente de sesgo es la inconsistencia temporal que implica el descuento hiperbólico (Gowdy et al.2013).
\item
  Podría decirse que Simon era paralelo en esto con Broadbent (1950), quien inició estudios sobre cómo los límites en la cognición conducen a la fatiga de la carga de trabajo.
\item
  Los planificadores centrales enfrentaron este problema en cuanto a cuánto tiempo y cómo dedican a pensar en cómo planificar. En la literatura francesa y rusa esto llegó a conocerse como planificación , el proceso de ``planificar cómo planificar'', aunque este término a veces se utilizaba para la planificación en general, así como para abordar el problema de la agregación de planes de nivel micro en macro coherente. unos (Rosser Jr.~y Rosser,2018, pag. 11).
\item
  Se pueden encontrar estudios más detallados sobre este tema en Allen et al.~(2011).
\end{enumerate}

5 .
Este resultado contrasta con el trabajo anterior de Vernon Smith (1962) que muestra cómo con los mercados de doble subasta los mercados libres convergen rápidamente hacia los equilibrios.

6 .
Ver Rosser Jr.~et al.~(1994) para la discusión de diferentes formas de relaciones jerárquicas y emergencia. Rosser Jr.~(2010b) proporciona una discusión sobre las relaciones entre multidisciplinar, interdisciplinario y transdisciplinario .

7 .
Rosser Jr.~y Rosser (2006) consideran los problemas de la gestión de tales resultados catastróficos dentro de un marco institucionalista.

8 .
En contraste, la economía post-walrasiana (Colander, 2006) critica e intenta ir más allá del marco walrasiano, mientras que la economía poskeynesiana (también llamada ``poskeynesiana'') tiende a admirar las ideas de Keynes en diversos grados entre la variedad de escuelas de pensamiento poskeynesiano, con Harcourt y Kreisler (2013a, B) proporcionando una descripción general de estas escuelas.

9 .
No existe una separación definitiva de estos casos, ya que incluso las distribuciones gaussianas permiten resultados extremos, aunque con menos frecuencia que los que exhiben colas de grasa kurtótica. En Tom Stoppard's (1967) Rosencrantz y Guildentstern están muertos la secuencia de apertura tiene a los personajes finalmente condenados discutiendo sobre lanzar monedas cuando uno de ellos sigue lanzando cara ``contra todo pronóstico'' 92 veces seguidas, un resultado permitido por distribuciones de probabilidad donde la probabilidad de una cara es uno. la mitad por cada lanzamiento de moneda justa. Incluso Keynes aceptó ese resultado y señaló que las compañías de seguros obtienen ganancias al apostar en distribuciones de probabilidad identificables y medibles, incluso cuando defendió la incertidumbre fundamental para muchas situaciones.

10 .
En los EE. UU., Esta sociedad ha estado estrechamente asociada con la llamada ``vieja economía institucional'', mientras que puede ser que un enfoque evolutivo que tenga en cuenta la complejidad pueda unir los enfoques antiguos y nuevos.

11 .
Es un debate abierto si Veblen vio o no la causalidad acumulativa como necesariamente implicando economías de escala, aunque era consciente de la importancia de las economías de escala en los sistemas industriales (Veblen, 1919). Setterfield (1997) reconoce la prioridad de Veblen al introducir el concepto, pero sostiene que Young (1928) y Kaldor (1972) lo relacionó más claramente con el fenómeno de los rendimientos crecientes.

12 .
Veblen no fue el primer economista en defender la utilidad para la economía de la teoría evolutiva, y tanto Marx como Marshall lo hicieron antes que él, incluso cuando lo hicieron desde perspectivas muy diferentes. El factor más complicado en todo esto es el hecho de que el propio Darwin fue influenciado de manera crucial por el trabajo de Malthus sobre la población cuando desarrolló su teoría de la selección natural (Rosser Jr.,1992).

13 .
Que esto no es ampliamente conocido se puede ver en que Business Dictionary identifica al creador del término como Allyn Young (1928) {[}www.businessdictionary.com/definition/cumulative-causation.html{]} y Wikipedia identifica a su autor (en realidad, ``causalidad acumulativa circular'') como Gunnar Myrdal (1957) {[} \url{https://en.wikipedia.org/wiki/Circular_cumulative_causation} {]}. Ciertamente, el uso del término por Myrdal recibió una amplia atención.

14 .
Claramente, no existe un límite definitivo al observar lo que son datos esencialmente discretos entre lo que es continuo y lo discontinuo. En la evolución biológica, se observan diferentes individuos a través de generaciones y, con la excepción de gemelos o clones idénticos, el genotipo de cada individuo es discretamente distinto del de los demás. Del mismo modo, con los fenotipos, se podrían representar las posibles variaciones de una determinada característica física en una escala continua, pero los individuos seguirán teniendo diferencias discretas con otros individuos en tales características, incluso si son muy pequeñas. Por tanto, la distinción se vuelve arbitraria. En el nivel más bajo vemos una granularidad discontinua, pero en un nivel más alto, los defensores de la continuidad ven solo cambios graduales, especialmente en los promedios de población,2000a, Cap. 1, para un análisis más detallado de la distinción entre formas continuas y discontinuas).

15 .
La formulación original del modelo de Arthur fue de Arthur et al.~(1987) y su estudio de las urnas Polya.

16 .
Curiosamente, Sewall Wright también se centró en la cría de animales debido a que trabajó para el Departamento de Agricultura de EE. UU. En la década de 1920, donde su pensamiento al respecto lo llevó a algunas de sus ideas, como la deriva aleatoria, también conocida como ``el efecto Sewall Wright'', a veces visto como una violación de la selección natural estricta en cómo se pueden formar nuevas especies, aunque la separación de subgrupos genéticamente distintos de una población puede ocurrir al azar en la naturaleza o mediante el control y la dirección consciente de los humanos como en la cría de animales.

17 .
El término ``meme'' como lugar de evolución se debe a Dawkins (1976), quien también propuso por primera vez la idea de ``darwinismo universal''.

18 .
Torgler estudia cómo la evolución de hábitos y normas determina el comportamiento fiscal en las sociedades (2016).

19 .
Sethi y Somanathan (1996) muestran que en tales juegos hay múltiples equilibrios de Nash, con algunos de apoyo y algunos de destrucción de los equilibrios cooperativos que son consistentes con el desarrollo sostenible. Rosser Jr.~y Rosser (2006) amplían este argumento.

20 .
Algunos de los primeros modelos de formación de hiperciclo requirieron la ausencia de parásitos. Sin embargo, con una mezcla adecuada pueden ser estables frente a los parásitos (Boerlijst y Hogeweg,1991).

21 .
Véase también Stokes (1995).

\hypertarget{la-compleja-dinuxe1mica-de-las-interacciones-sociales}{%
\chapter*{La compleja dinámica de las interacciones sociales}\label{la-compleja-dinuxe1mica-de-las-interacciones-sociales}}
\addcontentsline{toc}{chapter}{La compleja dinámica de las interacciones sociales}

Qué tan grande es la economía no observada (NOE) y qué determina su tamaño en diferentes países y regiones del mundo es una cuestión muy estudiada (Schneider y Enste, 2000, 2002). El tamaño de este sector en una economía tiene ramificaciones importantes. Afecta negativamente la capacidad de una nación para recaudar impuestos para apoyar a su sector público, lo que puede llevar a que más agentes económicos se muevan hacia el sector no observado (Johnson et al.~1997). Cuando este sector se asocia con actividades delictivas o corruptas, puede socavar el capital social y la cohesión social más amplia (Putnam et al.~1993), lo que puede dañar el crecimiento económico (Knack y Keefer, 1997; Zak y Knack, 2001). Además, dado que los programas de ayuda internacional están vinculados a medidas oficiales del tamaño de las economías, estas pueden verse distorsionadas por amplias variaciones en los tamaños relativos de la NOE entre diferentes países, especialmente entre las economías en desarrollo.

\hypertarget{introducciuxf3n}{%
\section*{Introducción}\label{introducciuxf3n}}
\addcontentsline{toc}{section}{Introducción}

Qué tan grande es la economía no observada (NOE) y qué determina su tamaño en diferentes países y regiones del mundo es una cuestión muy estudiada (Schneider y Enste, 2000, 2002). 1 El tamaño de este sector en una economía tiene ramificaciones importantes. Afecta negativamente la capacidad de una nación para recaudar impuestos para apoyar a su sector público, lo que puede llevar a que más agentes económicos se muevan hacia el sector no observado (Johnson et al.1997). Cuando este sector está asociado con actividades delictivas o corruptas, puede socavar el capital social y la cohesión social en general (Putnam et al.1993), lo que puede dañar el crecimiento económico (Knack y Keefer, 1997; Zak y Knack,2001). Además, dado que los programas de ayuda internacional están vinculados a medidas oficiales del tamaño de las economías, estas pueden verse distorsionadas por amplias variaciones en los tamaños relativos de la NOE entre diferentes países, especialmente entre las economías en desarrollo.

Los primeros estudios (Guttman, 1977; Feige,1979; Tanzi,1980, Frey y Pommerehne, 1984) enfatizó el papel de los altos impuestos y los grandes sistemas del estado de bienestar para empujar a las empresas y sus trabajadores hacia el sector no observado. Aunque algunos estudios más recientes han encontrado lo contrario, que impuestos más altos y gobiernos más grandes en realidad pueden estar relacionados negativamente con el tamaño de este sector (Friedman et al.2000), otros continúan encontrando la relación más tradicional (Schneider, 2002; Schneider y Klinglmair,2004). 2 Se ha encontrado que varios otros factores están relacionados con la ENO a nivel mundial, incluidos los grados de corrupción, los grados de sobrerregulación, la falta de un sistema legal creíble (Friedman et al.2000), el tamaño del sector rural y el grado de fragmentación étnica (Lassen, 2007).

Un factor que a menudo se ignora en esta combinación es la desigualdad de ingresos. Los primeros artículos publicados que tratan empíricamente de esta posible relación se centraron en esta relación dentro de las economías en transición (Rosser Jr.~et al.2000, 2003b). 3 Para un grupo importante de economías en transición, encontraron una relación positiva fuerte y robusta entre la desigualdad de ingresos y el tamaño de la economía no observada. El primero de estos también encontró una relación positiva entre los cambios en estas dos variables durante el período inicial de transición, mientras que el segundo solo encontró que la relación de niveles aún se mantenía significativamente después de tener en cuenta varias otras variables. La otra variable significativa más importante fue una medida de inestabilidad macroeconómica, específicamente la tasa máxima anual de inflación que un país había experimentado durante la transición.

Aquí, la hipótesis de una relación entre el grado de desigualdad de ingresos y el tamaño de la economía no observada se extiende al conjunto de datos globales estudiado por Friedman et al.~(2000). Se consideran variables macroeconómicas que no incluyeron y también un índice de confianza como medida de capital social. Una conclusión principal es que el hallazgo de estudios anteriores se traslada al conjunto de datos globales: la desigualdad de ingresos y el tamaño de la economía no observada poseen una correlación positiva fuerte, significativa y robusta. Ninguna otra variable aparece con una relación similar de manera uniforme, aunque un índice de corrupción sí lo hace para algunas especificaciones. Sin embargo, la inflación no está significativamente correlacionada para el conjunto de datos globales, en contraste con los hallazgos para los países en transición, y tampoco lo está el PIB per cápita. A diferencia de Friedman et al, las medidas de carga regulatoria y la falta de aplicación de los derechos de propiedad tienen una correlación negativa débil con el tamaño de la economía no observada, pero no de manera significativa. Sin embargo, la falta de observancia de los derechos de propiedad está fuertemente correlacionada negativamente con la corrupción, y la carga regulatoria también está bajo algunas especificaciones. El hallazgo de Friedman et al.~(2000) que las tasas impositivas están correlacionadas negativamente con el tamaño de la economía no observada se mantiene solo de manera insignificante en regresiones múltiples.

Además, se consideran qué variables están correlacionadas en regresiones múltiples con la desigualdad de ingresos, los niveles de corrupción y la confianza. En una formulación general, las dos variables que se correlacionan significativamente con la desigualdad de ingresos son una relación positiva con el tamaño de la economía no observada y la carga regulatoria, con una relación negativa con tasas impositivas significativas al nivel del diez por ciento. En cuanto al índice de corrupción, las variables que se correlacionan significativamente con él son las relaciones negativas con la observancia de los derechos de propiedad y la confianza. La confianza está significativamente relacionada negativamente con la corrupción, pero contrariamente a la intuición está relacionada positivamente con el tamaño de la economía no observada, aunque su relación bivariada es negativa.

Más allá de estos hallazgos empíricos más específicos (y las implicaciones políticas relacionadas), hay una cuestión metodológica más general a considerar. Contribuye al paradigma emergente que enfatiza el papel de las interacciones sociales de agentes heterogéneos en sistemas económicos complejos como algo importante a considerar además del análisis más convencional que se enfoca únicamente en los incentivos individuales. Que una implicación tan clara del enfoque convencional como que los impuestos más altos deberían estar asociados con una mayor participación en la economía no observada puede ser anulada por el efecto de tales interacciones sociales es una fuerte evidencia de esta conclusión.

\hypertarget{rendimientos-laborales-en-la-economuxeda-no-observada}{%
\section*{Rendimientos laborales en la economía no observada}\label{rendimientos-laborales-en-la-economuxeda-no-observada}}
\addcontentsline{toc}{section}{Rendimientos laborales en la economía no observada}

Mientras que Friedman et al.~(2000) se centran en las decisiones tomadas por los líderes empresariales, consideremos las decisiones tomadas por los trabajadores con respecto a qué sector de la economía desean suministrar mano de obra. Esto nos permite ver claramente el tema de las interacciones sociales involucradas en la formación de la economía no observada que tienden a quedar fuera en tales discusiones. Centrarse en las decisiones de los líderes empresariales no explica por qué la distribución del ingreso podría entrar en el asunto, y puede ser que el uso de este enfoque en mucha literatura anterior explique por qué los investigadores han evitado la hipótesis que consideramos tan convincente. Sin embargo, factores como el capital social y la cohesión social parecen estar relacionados con el grado de desigualdad de ingresos y, por lo tanto, deben reconocerse.

Necesitamos aclarar el uso de la terminología. Como se señaló en la nota al pie 1 anterior, la mayor parte de la literatura en este campo no ha distinguido entre términos como ``informal, clandestino, ilegal, en la sombra'', etc. al referirse a actividades económicas que no se informan a las autoridades gubernamentales (y por lo tanto no aparecen en las cuentas oficiales nacionales y del producto de la renta, aunque algunos gobiernos se esfuerzan por estimar algunas de estas actividades e incluirlas). En Rosser Jr.~et al.~(2000, 2003b) se utilizaron los términos ``informal'' y ``no oficial'', respectivamente, y se argumentó que todas estas etiquetas significaban lo mismo. Sin embargo, debe reconocerse allí que existen diferentes tipos de actividades de este tipo y que tienen diferentes implicaciones sociales, económicas y políticas, algunas claramente indeseables y otras potencialmente deseables desde ciertas perspectivas, por ejemplo, empresas que solo pueden operar de esa manera. debido a una excesiva regulación de la economía (Asea,1996). 4

Rosser Jr.~y col.~(2007) utilizó el término ``Economía No Observada'' (NOE), que se utilizará aquí y que se introdujo en el Sistema de Cuentas Nacionales (SNA) de las Naciones Unidas en 1993 (Calzaroni y Rononi, 1999), y que se ha aceptado en los debates sobre políticas dentro de la OCDE (Blades y Roberts, 2002) y otras instituciones internacionales. El SNA subdivide a la NOE en tres categorías amplias: ilegal , clandestina e informal (Calzaroni y Rononi,1999). Hay más subdivisiones de estos con respecto a si su estado se debe a errores estadísticos, subregistro o falta de registro, que no discutiremos más.

El sector ilegal consiste en actividades que serían en sí mismas ilegales si se denunciaran oficialmente, por ejemplo, asesinatos, robos, sobornos, etc. Parte de la corrupción entra en esta categoría, pero no toda. En general, estas actividades se consideran inequívocamente indeseables desde el punto de vista social, económico y político. Las actividades subterráneas son aquellas que no son ilegales per se, pero que no se informan al gobierno para evitar impuestos o regulaciones. Por lo tanto, se vuelven ilegales, pero solo por esta falta de denuncia. Muchos de estos pueden ser deseables hasta cierto punto desde el punto de vista social y económico, incluso si no informar de ellos reduce los ingresos fiscales y puede contribuir a un entorno económico más corrupto. Finalmente, las actividades informales son aquellas que tienen lugar dentro de los hogares y no involucran intercambios de mercado por dinero. Por lo tanto, por definición, no entrarían en las cuentas nacionales de ingresos y productos, incluso si fueran declarados. En general, se cree que ocurren con mayor frecuencia en las zonas rurales de los países menos desarrollados y que son en gran medida beneficiosos social y económicamente. Aunque las implicaciones más amplias de estos diferentes tipos de actividad económica no observada varían considerablemente, todas dan como resultado que no se paguen impuestos al gobierno sobre ellas.

Aunque no es necesario para las relaciones positivas entre nuestras principales variables, la desigualdad de ingresos, la corrupción y el tamaño de la NOE, las condiciones bajo las cuales surgen los equilibrios múltiples como se discute en Rosser Jr.~et al.~(2003b) son de interés. Esta idea se basa en una considerable literatura, gran parte de ella en sociología y ciencias políticas, que enfatiza las retroalimentaciones positivas y los umbrales críticos en los sistemas que involucran interacciones sociales. Schelling (1978) en economía y Granovetter (1978) en sociología notó tales fenómenos, con Crane (1993) discutir casos que involucran conductas sociales negativas que se propagan rápidamente después de que se cruzan los umbrales críticos. Putnam y col.~(1993) sugirió posibles equilibrios múltiples al discutir el contraste entre el norte y el sur de Italia en términos de capital social y desempeño económico. Aunque Putnam enfatiza la participación en actividades cívicas como clave para medir el capital social, otros se enfocan más en medidas de confianza generalizada, que se encuentra fuertemente correlacionada con el crecimiento económico a nivel nacional (Knack y Keefer,1997; Zak y Knack,2001; Svendsen,2002). Dado que Coleman (1990) define el capital social como la fuerza de los vínculos entre las personas en una sociedad, puede estar relacionado con la cohesión social y los costos de transacción potencialmente más bajos en la actividad económica.

El concepto de capital social es controvertido. Los primeros defensores de la idea incluyeron a Bourdieu (1977) y Loury (1977). Se pueden encontrar descripciones generales importantes en Woolcock (1998), Dasgupta (2000), Svendsen y Svendsen (2004), con Durlauf y Fafchamps (2005) proporcionando una perspectiva más crítica. Los últimos señalan que diferentes observadores proporcionan definiciones contradictorias del concepto con medidas y estimaciones econométricas confusas. Destacan especialmente el problema del ``capital social negativo'', que los fuertes vínculos dentro de ciertos subgrupos, como la mafia, pueden ser contrarios al crecimiento económico. Putnam (2000) distingue entre capital social ``puente'' y capital social ``vinculante''. El primero consiste en vínculos en toda la sociedad en general, del tipo que presumiblemente reduce los costos de transacción de la actividad económica. Estos últimos son entre individuos dentro de un subgrupo de sociedad, el tipo que podría ser contrario al crecimiento económico general, aunque no necesariamente a los ingresos de los miembros del grupo y podría corresponder más al capital social negativo de Durlauf y Fafchamps. 5 Supondremos que las medidas de confianza generalizada sirven como sustitutos del capital social puente, más económicamente productivo.

Dasgupta (2000, págs. 395-396) proporciona tres conceptualizaciones alternativas a nivel agregado para la operación del capital social, que él identifica con la confianza. El primero lo tiene operando a través de la productividad total de los factores.

\[Y=A f\left(K_{1} A\right)\]

donde Y es la producción total, A es la productividad total de los factores, K es el capital físico agregado y N es la fuerza de trabajo. A es una función positiva del capital social puente, visto como una reducción de los costos de transacción a través de la confianza generalizada. Dasgupta encuentra que la evidencia de esto es débil, al menos para el este de Asia. El segundo enfoque distingue el capital humano, H, y lo ve influenciado junto con el capital físico por la reducción de los costos de transacción a través del capital social.

\[Y=A f(B(K, H), N)\]

donde B ahora captura las externalidades de red social del capital social. Dasgupta también informa para esto que la evidencia es débil de que B contribuya sustancialmente al crecimiento económico en los países de reciente industrialización. Finalmente Dasgupta postula que el capital social funciona tanto a través del capital humano como del trabajo a través de C ,

\[Y=A f(K, C N(H, N))\]

Dasgupta luego argumenta que no es posible distinguir claramente entre estas hipótesis. Sin embargo, aquí consideraré que ( 3.3 ) es la representación más apropiada y una consideración adicional supondrá que el elemento de externalidad social operará directamente a través de su impacto sobre el trabajo (no nos preocuparemos directamente por el capital físico).

Rosser Jr.~y col.~(2000, 2003b) argumentan que el vínculo entre la desigualdad de ingresos y el tamaño de la ENO es una relación causal bidireccional, que se ejecuta principalmente a través de rupturas de la cohesión social y el capital social. La desigualdad de ingresos conduce a la falta de estos, lo que a su vez conduce a una mayor tendencia a abandonar la economía observada debido a la alienación social. Zak y Feng (2003) encuentran más fáciles las transiciones a la democracia con una mayor igualdad. En sentido contrario, el gobierno más débil asociado con una gran NOE reduce los mecanismos redistributivos y tiende a agravar la desigualdad de ingresos. 6 Llevar la corrupción a esta relación simplemente la refuerza en ambas direcciones. Aunque nadie antes de Rosser Jr.~et al.~(2000) vincularon directamente la desigualdad de ingresos y el NOE, algunos lo hicieron de manera indirecta. Así, Knack y Keefer (1997) señaló que tanto la igualdad de ingresos como el capital social estaban vinculados al crecimiento económico y, por lo tanto, presumiblemente entre sí. Putnam (2000) muestra entre los estados de los Estados Unidos que el capital social está vinculado positivamente con la igualdad de ingresos, pero está vinculado negativamente con las tasas de criminalidad.

El argumento formal en Rosser Jr.~et al.~(2003b) se basó en un modelo de participación en la actividad mafiosa debido a Minniti (1995). Ese modelo, a su vez, se basó en ideas de retroalimentación positiva en modelos de urnas Polya debido a Arthur et al.~(1987; ver también Arthur,1994). La idea básica es que los retornos al trabajo de participar en la actividad de la NOE aumentan durante un tiempo a medida que aumenta el tamaño relativo de la NOE y luego disminuyen más allá de cierto punto. Esto puede generar un umbral crítico que puede generar dos estados de equilibrio estable distintos, uno con un sector NOE pequeño y otro con un sector NOE grande. En el modelo de la actividad delictiva, el argumento es que la ley y el orden comienzan a romperse y luego se rompen sustancialmente en cierto punto, lo que coincide con una aceptabilidad social sustancialmente mayor de la actividad delictiva. Sin embargo, eventualmente se produce un efecto de saturación y los delincuentes simplemente compiten entre sí, lo que conduce a rendimientos decrecientes. Dado que dos de las principales formas de actividad NOE son ilegales por una razón u otra, se pueden imaginar tipos de dinámicas similares.

Sea N la fuerza laboral; N NOE sea la proporción de la fuerza de trabajo en el sector NOE; r j será el retorno esperado de la actividad laboral en el sector NOE menos el de trabajar en el sector observado para el individuo j , y a j será la diferencia debida únicamente a las características personales para el individuo j de los retornos del trabajo en el NOE menos los de trabajando en la economía observada, capturando tanto los efectos del capital humano como del capital social en el individuo. Supongamos que esta variable se distribuye uniformemente en el intervalo unitario, j ∈ {[}0, 1{]}, con una j aumentando comoj aumenta, que van desde un mínimo en una o y un máximo en un 1 . Además, esta diferencia de rendimiento entre los sectores sigue una función cúbica. Con todos los parámetros asumidos como positivos, esto da la vuelta al trabajo en el sector NOE para el individuo j como

\[r_{j}=a_{j}+\left(-\alpha N_{\text {noe }}^{3}+\beta N_{\text {noe }}{ }^{2}+\gamma N_{\text {noe }}\right)\]

con el término entre paréntesis en el lado derecho igual a f ( N μ ). La Figura 3.1 muestra esto para tres personas, cada una con una propensión personal diferente a trabajar en el sector NOE.

\textbf{Figura 3.1} Rendimientos relativos del trabajo en el sector no observado para tres individuos separados (eje vertical) como una función del porcentaje de la economía en el sector no observado (eje horizontal)

El equilibrio del mercado laboral más amplio se obtiene considerando la dinámica estocástica de la toma de decisiones de los potenciales nuevos trabajadores que ingresan. Sea N ′ = N + 1; q ( noe ) = probabilidad de que un nuevo participante potencial trabaje en el sector NOE, 1 - q (noe) = probabilidad de que un nuevo participante potencial trabaje en el sector observado, con λ noe = 1 con probabilidad q ( noe ) y λ noe = 0 con probabilidad 1 - q (noe). Esto implica que

\[q(n o e)=\left[a_{1}--f\left(N_{\text {noe }}\right)\right] /\left(a_{1}-a_{0}\right)\]

Por lo tanto, después del cambio en la fuerza laboral, la proporción de NOE será

\[N_{\text {noe }}^{\prime}=N_{\text {noe }}+(1 / N)\left[q(\text { noe })--N_{\text {noe }}\right]+(1 / N)\left[\lambda_{\text {noe }}-q(n 0 e)\right]\]

El tercer término de la derecha es el elemento estocástico y tiene un valor esperado de cero (Minniti, 1995, pag. 40). Si q (noe)\textgreater{} N noe , entonces el valor esperado de N ′ noe \textgreater{} N noe . Esto implica la posibilidad de tres equilibrios, siendo los dos externos estables y el intermedio inestable. Esta situación se muestra en la Figura 3.2 .

\textbf{Figura 3.2} Probabilidad de que un nuevo participante en el mercado laboral trabaje en el sector no observado, q ( u ), (eje vertical) en función del porcentaje de mano de obra en el sector no observado (eje horizontal)

El argumento se puede resumir postulando que la ubicación del intervalo {[} a 0 , a 1{]} aumenta con un aumento en el grado de desigualdad de ingresos, en el nivel de corrupción en la sociedad, o en un aumento en la brecha entre capital social puente y vinculante. Tal efecto tenderá a aumentar la probabilidad de que una economía esté en el equilibrio superior en lugar de en el equilibrio inferior y, si no se mueve de menor a mayor, se moverá a un valor de equilibrio más alto. En otras palabras, esperaríamos que una mayor desigualdad de ingresos o más corrupción resulten en una mayor proporción de la economía en la parte no observada. Sin embargo, al utilizar la confianza como principal indicador del capital social, la relación es ambigua, ya que dependerá del tipo de capital social que refleje. Si refleja un capital social puente, entonces esperaríamos que una mayor confianza conduzca a una menor actividad en la NOE,

Además, se puede esperar que haya interacciones mutuas entre varios de ellos. Se puede esperar que la economía no observada aumente la desigualdad al reducir los ingresos fiscales disponibles para la redistribución. También esperamos una fuerte retroalimentación de la misma a la corrupción, con todos estos potencialmente afectando el capital social de varias maneras.

Finalmente, se deben considerar otras variables que pueden interactuar con estas y entre sí, incluidos los factores institucionales, de política o macroeconómicos más amplios que se describen a continuación.

\hypertarget{variables-y-fuentes-de-datos}{%
\section*{Variables y fuentes de datos}\label{variables-y-fuentes-de-datos}}
\addcontentsline{toc}{section}{Variables y fuentes de datos}

Aquí revisaré parte del análisis empírico de Rosser Jr.~et al.~(2007), en el que se consideran ocho variables: una medida de la participación del sector NOE en cada economía, una medida del índice de Gini del grado de desigualdad del ingreso en cada economía, un índice del grado de corrupción en cada economía, real per cápita ingresos en cada economía, tasas de inflación en cada economía, una medida de la carga fiscal en cada economía, una medida del cumplimiento de los derechos de propiedad, una medida del grado de regulación en cada economía y un grado de confianza generalizada. 7 Este conjunto de variables produjo ecuaciones para todas las variables dependientes con altos grados de significación estadística basadas en la prueba F. Los resultados para 1992-1993 y 2000 se estimaron utilizando estimaciones de MCO. Hay problemas para medir cada una de estas variables.

Sin duda, el más difícil de medir es la participación relativa de una economía que no se observa. La esencia del problema es que uno está tratando de observar lo que, en general, la gente no desea haber observado. Por lo tanto, existe una incertidumbre inherentemente sustancial con respecto a cualquier método o estimación, y hay mucha variación entre los diferentes métodos de estimación. Schneider y Enste (2000) proporcionan una discusión de los diversos métodos que se han utilizado. Sin embargo, argumentan que para las economías capitalistas de mercado desarrolladas, el método más confiable es el que se basa en el uso de estimaciones de la demanda de divisas. Se hace una estimación de la relación entre el PIB y la demanda de divisas en un período base, luego se miden las desviaciones de los pronósticos de este modelo. Este método, debido a Tanzi (1980), se utiliza ampliamente en muchos países de ingresos altos para medir la actividad delictiva en general. Dado que la mayoría de los modelos de demanda de divisas asumen que las tasas impositivas miden el efecto de la economía sumergida, esto complica su uso para probar esa variable. 8

Schneider y Enste recomiendan el uso de modelos de consumo de electricidad para economías en transición, un método originado por Lizzera (1979) debido a la inestabilidad de las relaciones financieras durante la transición económica. Kaufmann y Kaliberda (1996) y también Lackó (2000) han realizado tales estimaciones para las economías en transición, que proporcionan la base para el trabajo anterior de Rosser Jr.~et al.~(2003b). Las estimaciones de Kaufmann y Kaliberda tienen un método similar al de la demanda de moneda, excepto que se estima una relación entre el PIB y el uso de electricidad en un período base, y las desviaciones más tarde proporcionan la participación estimada de la NOE. El enfoque de Lackó difiere en que modela las relaciones de consumo de electricidad de los hogares en lugar del uso de electricidad a nivel agregado. Por supuesto, muchas formas de actividad económica clandestina no implican el uso de electricidad, y la tecnología de producción de electricidad puede cambiar con el tiempo y complicar tales estimaciones.

Otro enfoque es MIMIC, o indicador múltiple de causa múltiple, utilizado por primera vez en este contexto por Frey y Pommerehne (1984) y utilizado por Loayza (1996) para realizar estimaciones para varias economías latinoamericanas. Este método implica derivar la medida a partir de un conjunto de vínculos entre presuntas variables subyacentes y presuntos indicadores. Este método tiene el problema de que, en efecto, ya supone saber cuáles son las relaciones, por lo que se obtendrán resultados sesgados al probarlo en cualquiera de las presuntas variables subyacentes. 9

Un método más consiste en observar las discrepancias en los datos de las cuentas de producto y el ingreso nacional entre las estimaciones del PIB y las estimaciones del ingreso nacional. Schneider y Enste enumeran varios otros métodos que se han utilizado. Sin embargo, estos cuatro son los que subyacen a las cifras que utilizamos en nuestras estimaciones.

Si bien se utilizan algunas alternativas a algunas de sus otras variables, las medidas de la NOE que Friedman et al.~(2000) se utilizan para las estimaciones de 1992-1993 que son más directamente comparables con su estudio. Estos, a su vez, se toman de tablas que aparecen en una versión anterior de Schneider y Enste (2000). Tienen 69 países enumerados y para muchos países proporcionan dos estimaciones diferentes. En general, para los países de la OCDE utilizan estimaciones de demanda de divisas, principalmente debido a Schneider (1997) o Williams y Windbeck (1995) o Bartlett (1990), con promedios de las estimaciones proporcionadas cuando hay más de uno disponible. Para las economías en transición se utilizan modelos de consumo de electricidad, principalmente de Kaufmann y Kaliberda, con algunos de Lackó. Los modelos de consumo de electricidad también se utilizan para las estimaciones más dispersas para África y Asia, y la mayoría de estas estimaciones se basan en el trabajo de Lackó, como se informa en Scheider y Enste. Para América Latina, la mayoría de las estimaciones provienen de Loayza (1996) que utilizó el método MIMIC. Sin embargo, para algunos países, los números de modelo de consumo de electricidad están disponibles, debido a Lackó y reportados por Schneider y Enste. Por último, el enfoque de discrepancia de las cuentas de ingresos y productos nacionales fue la fuente de un país, Croacia, también como se informa en Schneider y Enste. Aquí, la estimación se selecciona de las disponibles en función de los argumentos anteriores sobre cuál se esperaría que fuera más preciso. La mayoría de estas cifras corresponden a principios y mediados de la década de 1990.

Para 2000 números proporcionados por Schneider y Klinglmair (2004) son usados. Una parte sustancial de estos números se basan en la extensión DYMIMIC del método MIMIC. Esto dificulta la comparación de nuestros resultados para los dos puntos de datos diferentes y para cualquier estudio de relaciones dinámicas entre ellos, que generalmente mostraron resultados en su mayoría no significativos. 10 Desafortunadamente, hubo menos números de países disponibles para este año, y el conjunto consistió principalmente en los de la OCDE y las economías en transición. Esta variable se convirtió en la principal limitación para el conjunto de datos de 2000, que tenía solo 21 países para todas las variables.

Aunque no es tan difícil de medir como la NOE, la desigualdad de ingresos es una variable algo difícil de medir, con varios enfoques en competencia. El coeficiente de Gini es el número más ampliamente disponible en diferentes países, aunque no está disponible para todos los años en la mayoría de los países. Además, existen diferentes fuentes de datos que subyacen a las estimaciones de la misma, y \hspace{0pt}\hspace{0pt}las encuestas en los países de ingresos más altos generalmente reflejan los ingresos, mientras que en los países más pobres a menudo reflejan solo patrones de consumo. Para la mayoría de los países en transición para 1992--93, las estimaciones elaboradas por Rosser Jr.~et al.~(2000) se utilizan, sin embargo, para los demás países se utilizan números proporcionados por el Informe de Desarrollo Humano de las Naciones Unidas para 2002 o 2003, que también corresponden a varios años de la década de 1990. De los 69 países estudiados en Friedman et al.~(2000) hay tres para los que no se dispone de datos sobre el coeficiente de Gini: Argentina, Chipre y Hong Kong. Por tanto, no se incluyen en estas estimaciones.

La medida de corrupción es un índice utilizado por Friedman et al.~(2000) que proviene de Transparencia Internacional (1998). Cabe señalar que la escala utilizada para este índice tiene un valor más alto para las naciones menos corruptas y varía de uno a diez. Esto contrasta con nuestros números de coeficiente de Gini y NOE, que aumentan con más NOE y más desigualdad. Por lo tanto, una relación positiva entre la corrupción y cualquiera de esas otras dos variables se mostrará como una relación negativa para nuestras variables. Para 2000 se utilizan números actualizados de la misma fuente.

Las cifras reales del PIB per cápita provienen del Informe de Desarrollo Humano de la ONU para 2001 y son para el año 2000. La tasa de inflación estimada es de la misma fuente pero es un promedio para el período 1990-2000. La medida de la carga tributaria proviene del Índice de Libertad Económica 2001 de Heritage Foundation (O'Driscoll Jr.~et al.2001). Esto combina una estimación basada en la tasa impositiva marginal máxima sobre la renta, la tasa impositiva marginal que enfrenta el ciudadano medio y la tasa impositiva corporativa máxima y oscila entre uno (carga impositiva baja) y 5 (carga impositiva alta). Este número aumenta a medida que aumenta la carga fiscal. La medida de la aplicación de los derechos de propiedad proviene de O'Driscoll Jr.~et al.~(2001) y varía de uno (alto cumplimiento de los derechos de propiedad) a cinco (bajo cumplimiento de los derechos de propiedad). La medida de la carga regulatoria también es de O'Driscoll Jr.~et al.~(2001) y varía de uno (carga reglamentaria baja) a cinco (carga reglamentaria alta). Obviamente, hay una cantidad considerable de subjetividad involucrada en muchas de estas estimaciones. Después de tener en cuenta estas variables hasta ahora, el conjunto de datos utilizables se reduce de 69 a solo 52.

Finalmente, la medida de confianza para 1992-1993 es el índice utilizado en la Encuesta Mundial de Valores (Inglehart et al.~1998), que varía de cero a 100, y un valor más alto significa más confianza. Aunque estudian 43 ``sociedades'', muchas de estas son subsecciones de las naciones que se observan aquí, como la ciudad de Moscú e Irlanda del Norte. Al final, cuando los números de este tipo se combinan con los enumerados anteriormente, solo quedan 32 de los 69 países originales, con el conjunto fuertemente dominado por la OCDE y los países en transición. Por lo tanto, con el fin de capturar una visión más amplia, se consideran regresiones con y sin la variable de confianza. Para 2000, los números utilizados para este índice fueron proporcionados personalmente por Ronald Inglehart, para el cual se dispuso de estimaciones anuales para muchos más países. 11

\hypertarget{hallazgos-empuxedricos}{%
\section*{Hallazgos empíricos}\label{hallazgos-empuxedricos}}
\addcontentsline{toc}{section}{Hallazgos empíricos}

Antes de las regresiones múltiples de MCO para los datos de 1992--93, la matriz de correlación para estas nueve variables generalmente presagia los resultados de la regresión, con algunas excepciones. Utilizando el conjunto más grande de 52 naciones sin confianza, para cada una de las otras tres variables dependientes principales, las variables independientes que resultan ser estadísticamente significativas en las regresiones MCO también tienen un valor absoluto alto en la matriz de correlación con la variable dependiente. Las dos excepciones son que la falta de aplicación de los derechos de propiedad y la carga regulatoria parecen estar fuertemente correlacionadas con el NOE, pero no así en la regresión múltiple. Pero sus relaciones con la corrupción son las correlaciones bivariadas más altas en la matriz, presagiando que la corrupción puede tener su efecto en algunas regresiones múltiples. El principal valor atípico se produce cuando aportamos confianza y el conjunto de datos se reduce a 32 países. La confianza está correlacionada negativamente con la NOE en la matriz de correlación, pero parece estar relacionada positivamente con ella en la regresión múltiple al nivel del diez por ciento.

En la regresión MCO sin la variable de confianza en la que la medida de la economía no observada es la variable dependiente y las otras siete variables son las independientes. El más significativo desde el punto de vista estadístico es el índice de corrupción, es decir, al nivel del 5 por ciento, siendo el más fuertemente correlacionado en la matriz de correlación. Se mantiene la relación positiva esperada entre estos dos (mostrada por un signo negativo). La otra variable significativa al nivel del 5 por ciento es el coeficiente de Gini. Los resultados cualitativos que se ven aquí se muestran consistentemente en otras regresiones con estas y otras variables en varias combinaciones.

Otro muestra la misma regresión pero con la variable confianza incluida como variable independiente y con el número de observaciones reducido en 20 debido a la indisponibilidad del índice de confianza para esos países. El índice de Gini sigue siendo significativo, incluso con más fuerza que en la regresión anterior. La corrupción ya no es significativa, aunque lo es casi al nivel del diez por ciento. Sin embargo, un resultado peculiar es que la confianza está relacionada positivamente con la NOE y significativamente al nivel del diez por ciento. Esto podría deberse a que el número de confianza está aumentando el capital social ``vinculante'' y ``puente'', posiblemente en consonancia con este resultado.

Siguiendo los argumentos de McCloskey y Ziliak (1996) el tamaño de los coeficientes para estas dos variables estadísticamente significativas es lo suficientemente grande como para ser también económicamente significativo. En la regresión más grande, las presuntas relaciones ceteris paribus serían que un aumento del 10 por ciento en el coeficiente de Gini estaría asociado con un aumento del 6 por ciento en la participación del PIB en la economía no observada, mientras que un aumento del 10 por ciento en la tasa de La corrupción (cambio en el valor del índice de un punto) estaría asociada con un aumento del 4 por ciento en la participación del PIB en la economía no observada. Estas son relaciones notables económicamente, aunque hay que tener cuidado al hacer extrapolaciones como estas. 12

Sin embargo, un hallazgo de las naciones en transición no se traslada al conjunto de datos globales. Esta es la relación estadísticamente significativa entre la inflación y el tamaño de la NOE, que incluso se trasladó al crecimiento de la NOE. Una posible explicación de esto es que durante el período de observación, las economías en transición experimentaron una inflación mucho más alta que la mayoría del resto del mundo, y Ucrania alcanzó una tasa anual máxima de más del 10,000 por ciento. Esta alta inflación estuvo fuertemente relacionada con el proceso general de colapso y colapso institucional que sucedió en esos países en ese momento.

Un hallazgo de Friedman et al.~(2000) no se confirma, su conclusión de que la carga tributaria se correlaciona negativamente con el tamaño de la NOE de manera significativa. La matriz de correlación muestra una correlación bivariada negativa de −0,45, pero en la regresión más grande esto se convierte en una relación débilmente positiva y estadísticamente insignificante, mientras que en la Tabla en una regresión adicional es una relación débilmente negativa pero insignificante. La probable explicación del contraste entre este hallazgo y el de Friedman et al.~(2000) es que existe una fuerte relación negativa entre la carga tributaria y la desigualdad de ingresos, al menos en el conjunto de datos más amplio, como se ve en la regresión más grande. En la regresión múltiple esto domina. El factor más importante aquí es la desigualdad de ingresos, y cuando aparece en una ecuación, la significación estadística (e incluso el signo encontrado) desaparece. Por tanto, Friedman et al.~(2000) excluyeron la distribución del ingreso en sus diversas estimaciones parece haber distorsionado profundamente sus hallazgos. La relación no es estadísticamente significativa en ninguna dirección en un modelo más especificado.

Luego están los resultados de la regresión MCO para el conjunto más pequeño de variables pero con el coeficiente de Gini como variable dependiente. El tamaño del NOE es estadísticamente significativo al nivel del 5 por ciento, aunque no al nivel del 1 por ciento. Aún más significativo estadísticamente, manteniéndose fuertemente en el nivel del 1 por ciento, es la carga tributaria, que está correlacionada negativamente. Parecería que estas cargas fiscales dan como resultado una redistribución de ingresos notable, o si no es así, entonces las naciones con distribuciones de ingresos más equitativas están más dispuestas a tolerar tasas impositivas más altas. Como en regresiones anteriores, la medida de la inflación tampoco se muestra tan estadísticamente significativa como ocurre con las otras variables.

En cuanto a la importancia económica, la relación de la NOE con la desigualdad de ingresos parece ser algo más débil que en el sentido contrario. Por lo tanto, un aumento del 10 por ciento en la participación de la economía no observada en el PIB solo estaría asociado con un aumento de alrededor del 2 por ciento en el coeficiente de Gini. La carga tributaria parece ser económicamente significativa, con un aumento del 20 por ciento en la carga tributaria que lleva a una disminución del 40 por ciento en el coeficiente de Gini.

Otra regresión aporta confianza a esta estimación para el conjunto de datos más pequeño de 32 países. Si bien el NOE sigue siendo una variable significativa, la tributación ahora solo es significativa al nivel del diez por ciento, y la carga regulatoria ahora se vuelve significativa al nivel del 5 por ciento, con una correlación negativa con la desigualdad. Además, nuestras variables macroeconómicas vuelven a entrar en juego de alguna manera, siendo el deflactor significativo al nivel del 10 por ciento y correlacionado positivamente con la desigualdad.

Luego, considere los resultados de confianza como la variable dependiente, que solo está disponible para el conjunto de datos de 32 observaciones. La variable más significativa es la corrupción al nivel del 1 por ciento, que tiene el signo esperado. Un resultado anómalo es que la NOE es significativa al nivel del 10 por ciento, pero con un signo positivo inesperado, volcando la relación bivariada entre estas dos variables en la matriz de correlación. Un resultado sorprendente es que la hipótesis de que la igualdad impulsaría la confianza no se sostiene por completo. El signo es el esperado, pero simplemente falta ser significativo al nivel del 10 por ciento. Por lo tanto, curiosamente, la desigualdad parece estar más directamente relacionada con la NOE que con el intermediario hipotético, el capital social medido por la confianza, aunque esto puede deberse al menor conjunto de datos disponibles con la variable confianza.

Luego considere la matriz de correlación para la variable establecida para 2000, con resultados generalmente similares en comparación con el período anterior. Hay regresiones MCO sobre el conjunto de variables completo para cada una de las principales variables dependientes, y solo se muestra una, dado que la variable límite para este período es la variable NOE. Desafortunadamente, solo hay 21 países en este conjunto de datos, confinados a la OCDE y las economías en transición.

El que probablemente sea de mayor interés tiene a NOE como variable dependiente. Los resultados son razonablemente consistentes con las estimaciones anteriores de 1992-93, pero con algunas variables adicionales significativas. Por lo tanto, la desigualdad vuelve a ser significativa al nivel del 5 por ciento con nuestro signo positivo esperado, y la confianza vuelve a ser significativa con un signo positivo y al nivel del 1 por ciento. Como antes, este último deshace el signo observado en la matriz de correlaciones. Las dos variables adicionales que son significativas son la corrupción, que está relacionada positivamente como se esperaba y al nivel de significancia del 10 por ciento, junto con la inflación, que está relacionada negativamente de manera contradictoria con la NOE y significativa al nivel del 5 por ciento, lo que contrasta fuertemente con los hallazgos solo para las economías en transición.

Para uno con el coeficiente de Gini como variable dependiente, la historia básica de la relación bidireccional entre el NOE y la desigualdad continúa manteniéndose, con el NOE positivo y significativo al nivel del 5 por ciento. Además, la influencia de la inflación es aún más fuerte y está relacionada positivamente al nivel del 1 por ciento. A diferencia del conjunto de datos anterior, la confianza es ahora una variable significativa, relacionada negativamente con la desigualdad y significativa al nivel del 1 por ciento. También se diferencia de la estimación anterior que las variables de fiscalidad y regulación ya no son significativas, aunque la fiscalidad sigue teniendo un signo negativo.

Un problema serio para estas estimaciones es la endogeneidad potencial de varias variables entre sí, con muchas posibilidades disponibles. Un esfuerzo para lidiar con esto involucró varias posibles formulaciones de ecuaciones simultáneas usando mínimos cuadrados de dos etapas. 13 Desafortunadamente, los resultados de estas estimaciones fueron en general débiles, lo que generó dudas sobre la solidez de los hallazgos.

\hypertarget{conclusiones}{%
\section*{Conclusiones}\label{conclusiones}}
\addcontentsline{toc}{section}{Conclusiones}

El hallazgo de Rosser Jr.~et al.~(2000, 2003b, 2007) que parece haber una relación bidireccional significativa entre el tamaño de la economía no observada (o economía informal o no oficial) y la desigualdad de ingresos se confirma tentativamente cuando el conjunto de datos se amplía para incluir naciones que representan una muestra más global basada en en las regresiones MCO, pero no conserva la importancia en las formulaciones de ecuaciones simultáneas o en las estimaciones de cambios en las variables entre los dos períodos de tiempo. El hallazgo de Friedman et al.~(2000) que existe una fuerte relación entre el tamaño de la economía no observada y el nivel de corrupción en una economía se confirma más débilmente, y puede haber una relación bidireccional significativa, aunque algo más fuerte al pasar de la corrupción a la no economía observada que al revés. Esto se debilita en las carreras con confianza que cubren solo 32 países para 1992-93, pero es más fuerte para 2000. Que la tasa máxima de inflación anual sea importante en el tamaño de la economía no observada se mantiene para las economías en transición no Mantener para conjuntos de datos nacionales más grandes.

El hallazgo no confirmado del estudio de Friedman, Johnson, Kaufmann y Zoido-Lobatón es el de una relación negativa entre impuestos más altos y el tamaño de la economía no observada. Estos resultados no encuentran una relación estadísticamente significativa entre esta visión y la visión alternativa más tradicional que sostiene que los impuestos más altos llevan a las personas a la economía no observada. El fracaso de Friedman et al.~(2000) incluir cualquier medida de desigualdad de ingresos puede explicar este contraste y muestra la importancia de las interacciones sociales.

Sus hallazgos de que la economía no observada aumenta con la falta de aplicación de los derechos de propiedad y las cargas regulatorias no se encuentran directamente en ninguno de los períodos de tiempo. Sin embargo, existen fuertes relaciones entre estos y la corrupción para el conjunto de datos más amplio sin la variable de confianza en 1992-93 y para el cumplimiento de los derechos de propiedad con la variable de confianza, con la corrupción fuertemente vinculada con la economía no observada, lo que sugiere que tal vez este sea el vía a través de la cual estas variables tienen su efecto. Sin embargo, estas relaciones no se mantuvieron en absoluto en 2000, aunque estas variaciones pueden reflejar los distintos conjuntos de países utilizados, con el conjunto de 21 países de 2000 limitado a la OCDE y las economías en transición, mientras que el mayor de los de 1992-1993 se ha fijado en 52 países, sin la variable confianza, incluye muchos países menos desarrollados.

El uso de la confianza como nuestra medida de capital social condujo a resultados algo confusos que pueden reflejar conflictos entre la unión del capital social entre subgrupos y la unión del capital social entre grupos, aunque presumiblemente la confianza generalizada debería representar este último. En todo caso tuvo una inesperada relación positiva y significativa con la economía no observada para ambos períodos de tiempo, más acorde con ella como medida de capital social vinculante. Si bien fue insignificante con la desigualdad en 1992-93, se relacionó significativa y negativamente con la desigualdad en 2000, en consonancia con la mayor parte de la literatura. En cuanto a la corrupción, fue significativa en ambos períodos con el signo negativo esperado. La NOE y la corrupción fueron las variables significativas que determinaron la confianza en 1992-1993, conservando sus signos,

Los esfuerzos para probar la solidez de estos resultados utilizando mínimos cuadrados de dos etapas en cada uno de los conjuntos de datos y OLS no se sostienen bien, advirtiendo de una fragilidad encontrada tanto por Durlauf como por Fafchamps (2005) y Breusch (2005) sobre estudios tanto del capital social como de la economía no observada. Los problemas y las incertidumbres con respecto a gran parte de los datos, especialmente para las estimaciones del tamaño de la economía no observada, probablemente contribuyan sustancialmente a esta falta de solidez.

Si bien estos resultados deben usarse con cautela al hacer recomendaciones de políticas, refuerzan la advertencia entregada en Rosser Jr.~y Rosser (2001): las organizaciones internacionales preocupadas por los impactos negativos en la recaudación de ingresos en varios países de tener grandes sectores no observados deben ser cautelosos a la hora de recomendar políticas que conduzcan a aumentos sustanciales en la desigualdad de ingresos. Los programas de austeridad fiscal para reducir los déficits presupuestarios que se centran en reducir los programas de transferencias igualitarias pueden resultar contraproducentes en una situación de ingresos reducidos. La desigualdad agudamente creciente bien puede tener el resultado contraproducente de aumentar el tamaño de la economía no observada y la corrupción, reduciendo así los ingresos fiscales y generando, en general, una disminución del capital social y la cohesión social general, un hallazgo profundo que muestra cómo un resultado esperado convencionalmente puede no se cumple cuando se tienen en cuenta las interacciones sociales dinámicamente complejas.

\hypertarget{notas-al-pie-1}{%
\section*{Notas al pie}\label{notas-al-pie-1}}
\addcontentsline{toc}{section}{Notas al pie}

\begin{enumerate}
\def\labelenumi{\arabic{enumi}.}
\tightlist
\item
  Se han utilizado muchos términos para la economía no observada, incluidos informal, no oficial, en la sombra, irregular, subterráneo, subterráneo, negro, oculto, oculto, ilegal y otros, con gran parte de esta terminología originada en estudios en Italia (Pettinati, 1979) .. Generalmente estos términos se han utilizado indistintamente. Sin embargo, observe aquí las distinciones entre algunos de estos y, por lo tanto, utilizaremos el descriptor más neutral, economía no observada, adoptado para uso formal por el Sistema de Cuentas Nacionales (SCN) de las Naciones Unidas (ver Calzaroni y Rononi,1999; Blades y Roberts,2002).
\item
  Sin embargo, en Schneider y Neck (1993) se argumenta que la complejidad de un código tributario es más importante que su nivel de tasas impositivas. Además, en Schneider y Enste (2002, págs. 97-101) se argumenta que para los países de bajos ingresos las tasas impositivas más altas podrían reducir la participación de la economía sumergida, ya que se necesita algún gobierno para establecer mercados oficiales.
\item
  Lewis Davis (2007) señala el modelo teórico de Rauch (1993) que hipotetiza tal relación en el desarrollo en conjunto con la curva de Kuznets. Durante la etapa intermedia del desarrollo, la desigualdad aumenta a medida que muchos pobres se trasladan a la ciudad y participan en la ``economía informal subempleada'', un concepto que sigue a la discusión de de Soto (1989), aunque esto se asemeja más a la economía ``subterránea'' como se define más adelante aquí. Rauch no proporciona datos empíricos y su modelo teórico difiere del presentado aquí y también involucra un mecanismo diferente. Rosser Jr.~y col.~(2007) inicialmente extendió esto más allá de las economías en transición a un servidor de datos global más amplio.
\item
  Otro aspecto positivo de la actividad económica no observada de cualquier tipo surge de los efectos multiplicadores sobre el resto de la economía que puede generar (Bhattacharya, 1999).
\end{enumerate}

5 .
Lassen (2007) sostiene que las divisiones étnicas destruyen el capital social y pueden abrir la puerta a una economía informal más amplia. Bjørnskov (2006) proporciona un conjunto más completo de elementos involucrados en el capital social.

6 .
Este efecto se ve más allá de los estudios que muestran que el pago de impuestos está vinculado a la confianza general y al capital social. Scholz y Lubell,1998; Slemrod,1998). Anderson y col.~(2004) proporcionan pruebas experimentales de los vínculos entre la igualdad y la voluntad de proporcionar bienes públicos. Aunque no mencionan explícitamente la distribución del ingreso, Schneider y Enste (2002) enfatizan la ``moralidad tributaria'' como un factor en el pago de impuestos, y reconocen que la justicia percibida de un sistema tributario influye en esto. Si la confianza general y la igualdad de ingresos aumentan la moralidad fiscal, entonces podrían incrementar el pago de impuestos.

7 .
Otras variables se han incluido en otras pruebas, incluidas las tasas de desempleo, el PIB agregado, una medida de carga fiscal y un índice de libertad económica general. Sin embargo, ninguna de las dos primeras fue significativa y tampoco lo fueron en otros estudios. El PIB real per cápita presumiblemente es una mejor medida que el agregado de todos modos. En cuanto a la carga fiscal, es la misma que la medida de la carga fiscal, excepto que incluye el nivel de gasto público. La mayor parte de la literatura apoya la idea de que el aspecto fiscal es la parte más importante de esto y nuestros resultados apoyarían esto. Por último, el índice general de libertad económica contiene cinco subíndices, tres de los cuales ya se están utilizando individualmente. Además, un índice que lo incluye es una medida de la ``actividad del mercado negro'', que parece otra medida directa de la actividad económica no observada. o al menos una parte importante. Entonces, esta variable tiene demasiadas correlaciones directas con otras variables para ser de utilidad.

8 .
En la economía actual, observar la demanda de efectivo puede no funcionar tan bien dado el aumento de las criptomonedas y su uso para actividades delictivas (Norgaard, 2020).

9 .
Los creadores del enfoque MIMIC fueron Zellner (1970) y Goldberger (1972). Breusch2005) muestra que su uso para algunas estimaciones de la economía sumergida conduce a resultados muy frágiles, un resultado que puede ser más general que solo para el método MIMIC. MIMIC significa ``múltiples indicadores, múltiples causas'' y DYMIMIC simplemente agrega ``dinámica'' al frente de eso.

10 .
En una comunicación personal (2005), Dominick Enste señala que si bien el método DYMIMIC puede tener ventajas como una estimación de la NOE, la forma en que otras variables entran en su medición puede hacer que sea menos adecuado para su uso en la verificación de la importancia independiente de esas variables para explicar los determinantes de la NOE.

11 .
La discusión con varios interlocutores sugiere que estas estimaciones tienen muchos problemas. Sin embargo, probablemente fueron las mejores cifras disponibles para un conjunto tan amplio de países.

12 .
Ha habido algunos ejemplos espectaculares de naciones que han experimentado incrementos dramáticos tanto en la desigualdad como en el tamaño de su economía no observada, siendo especialmente notable lo que sucedió en Rusia entre 1989 y 1993.

13 .
Ver Rosser Jr.~et al.~(2007) para una mayor discusión de este tema.

\hypertarget{part-econofuxedsica-entropuxeda-y-complejidad}{%
\part{Econofísica, entropía y complejidad}\label{part-econofuxedsica-entropuxeda-y-complejidad}}

\hypertarget{econofuxedsica-entropuxeda-y-complejidad}{%
\chapter*{Econofísica, entropía y complejidad}\label{econofuxedsica-entropuxeda-y-complejidad}}
\addcontentsline{toc}{chapter}{Econofísica, entropía y complejidad}

El término econofísica fue neologizado en 1995 en la segunda conferencia Statphys-Kolkata en Kolkata (antes Calcuta), India, por el físico H. Eugene Stanley, quien también fue el primero en usarlo en forma impresa (Stanley 1996). Mantegna y Stanley (2000, pp.~Viii-ix) definen ``el campo multidisciplinario de la econofísica'' como ``un neologismo que denota las actividades de los físicos que están trabajando en problemas económicos para probar una variedad de nuevos enfoques conceptuales derivados de las ciencias físicas'' Chakrabarti 2005, pág. 225).

\hypertarget{los-oruxedgenes-y-la-naturaleza-de-la-econofuxedsica}{%
\section*{Los orígenes y la naturaleza de la econofísica}\label{los-oruxedgenes-y-la-naturaleza-de-la-econofuxedsica}}
\addcontentsline{toc}{section}{Los orígenes y la naturaleza de la econofísica}

El término econofísica fue neologizado en 1995 en la segunda conferencia Statphys-Kolkata en Kolkata (antes Calcuta), India, por el físico H. Eugene Stanley, quien también fue el primero en usarlo en forma impresa (Stanley et al.~1996a). Mantegna y Stanley (1999, pp.~viii-ix) definen ``el campo multidisciplinario de la econofísica'' como ``un neologismo que denota las actividades de los físicos que están trabajando en problemas económicos para probar una variedad de nuevos enfoques conceptuales derivados de las ciencias físicas'' Chakrabarti 2005, pag. 225).

La lista de tales problemas ha incluido distribuciones de rendimientos en los mercados financieros (Mantegna 1991; Levy y Salomón1997; Bouchaud y Cont1998; Gopakrishnan y col.1999; Lux y Marchesi1999; Sornette y Johansen2001; Farmer y Joshi2002; Li y Rosser2004) la distribución de la renta y la riqueza (Drăgulescu y Yakovenko 2001; Bouchaud y Mézard2000; Chatterjee y col.2007; Yakovenko y Rosser Jr.2009), la distribución de los choques económicos y las variaciones de la tasa de crecimiento (Bak et al.~1993; Canning y col.1998), la distribución del tamaño de las empresas y las tasas de crecimiento (Stanley et al.~1996b; Takayasu y Okuyama1998; Botazzi y Secchi2003), la distribución del tamaño de las ciudades (Rosser Jr 1994; Gabaix1999) y la distribución de descubrimientos científicos (Plerou et al.~1999; Sornette y Zajdenweber1999), entre otros problemas, todos los cuales a veces se considera que no siguen patrones normales o gaussianos que pueden describirse completamente mediante media y varianza. Las principales fuentes de enfoques conceptuales de la física utilizados por los economistas han sido los modelos de mecánica estadística (Spitzer1971), modelos geofísicos de terremotos (Sornette 2003), y modelos de avalanchas de ``pilas de arena'', este último con criticidad autoorganizada (Bak 1996). Uno de los primeros físicos en afirmar la identidad esencial de los métodos estadísticos utilizados en física y ciencias sociales fue Majorana (1942), que ha sido visto por algunos economistas como un precursor.

Un tema común entre quienes se identifican a sí mismos como economistas es que la teoría económica estándar ha sido inadecuada o insuficiente para explicar las distribuciones no gaussianas observadas empíricamente para varios de estos fenómenos, como la asimetría ``excesiva'' y las ``colas gruesas'' leptokurtóticas (McCauley 2004; Chatterjee y Chakrabarti2006; Lux2009). El surgimiento de la econofísica siguió bastante pronto a las influyentes interacciones y discusiones que ocurrieron entre grupos de físicos y economistas en el Instituto Santa Fe (Anderson et al.1988; Arthur y col.1997a), y algunos de los físicos involucrados en estas discusiones también se involucraron en el movimiento de la econofísica.

Ahora llegamos a una gran curiosidad e ironía en este asunto: algunas de las principales técnicas utilizadas por los economistas fueron desarrolladas inicialmente por los economistas (y muchas otras desarrolladas por los matemáticos), y algunas de las ideas asociadas con los economistas fueron desarrolladas por los físicos. Por lo tanto, en cierto sentido, estos esfuerzos de los físicos se asemejan a traer carbón a Newcastle, excepto que debe admitirse que muchos economistas olvidaron o nunca supieron de estos temas o métodos. Esto es cierto para el más canónico de tales modelos, la distribución de Pareto (Pareto1897).

\hypertarget{el-papel-de-la-distribuciuxf3n-de-pareto}{%
\section*{El papel de la distribución de Pareto}\label{el-papel-de-la-distribuciuxf3n-de-pareto}}
\addcontentsline{toc}{section}{El papel de la distribución de Pareto}

Si hay un solo tema que une a los economistas es la insistencia en que muchos fenómenos económicos ocurren de acuerdo con distribuciones que obedecen a leyes de escala más que a la normalidad gaussiana. Ya sean simétricas o sesgadas, las colas son más gruesas o más largas de lo que serían si fueran gaussianas, y parecen ser lineales en las cifras con el logaritmo de una variable representado frente a su distribución de probabilidad acumulada. Buscan procesos físicos, con mayor frecuencia a partir de la mecánica estadística, que puedan generar estas distribuciones no gaussianas que obedezcan las leyes de escala.

La versión canónica (y original) de tal distribución fue descubierta por el economista y sociólogo matemático Vilfredo Pareto, en 1897. Sea N el número de observaciones de una variable que exceden un valor x con A y α constantes positivas. Luego

\[N=A x^{-\alpha}\]

Esto exhibe la propiedad de escala en que

\[\ln (N)=\ln A-\alpha \ln (x)\]

Esto se puede generalizar a una forma estocástica más clara reemplazando N con la probabilidad de que una observación exceda x . Pareto formuló esto para explicar la distribución del ingreso y la riqueza y creía que había un valor universalmente verdadero para α que equivalía a aproximadamente 1,5. Estudios más recientes (Clementi y Gallegati2005) sugieren que son solo los extremos superiores de la distribución de la renta y la riqueza los que siguen a tal propiedad de escala, con los extremos inferiores siguiendo la forma logarítmica normal de la distribución gaussiana que se asocia con el paseo aleatorio, originalmente argumentado para la totalidad de la distribución de la renta por Gibrat (1931), un punto más estudiado por Yakovenko y Rosser Jr (2009), Shaikh (2016), y Shaikh y Jacobo (2020).

El paseo aleatorio y su distribución logarítmica normal asociada es el gran rival de la distribución de Pareto y sus parientes en la explicación de los fenómenos económicos estocásticos. Solo unos años después de que Pareto hiciera su trabajo, se descubrió la caminata aleatoria en un doctorado. tesis sobre mercados especulativos del matemático Louis Bachelier (1900), cinco años antes de que Einstein lo usara para modelar el movimiento browniano, su primer uso en física (Einstein 1905). Aunque la distribución de Pareto tendría sus defensores para explicar la dinámica de precios estocástica (Mandelbrot1963), la caminata aleatoria se convertiría en el modelo estándar para explicar la dinámica de los precios de los activos durante muchas décadas, aunque serían los rendimientos de los activos los que se modelarían en lugar de los precios de los activos directamente como lo hizo Bachelier originalmente. Como ironía adicional, fue un físico, MFM Osborne (1959), quien fue uno de los defensores influyentes del uso de la caminata aleatoria para modelar los rendimientos de los activos. Se suponía que era el paseo aleatorio gaussiano el que subyacía a la dinámica de los precios de los activos cuando se desarrollaban conceptos básicos de economía financiera como la fórmula de Black-Scholes (Black y Scholes1973). Si p es el precio, R es el rendimiento debido a un aumento de precio, B es la deuda y σ es la desviación estándar de la distribución gaussiana, Osborne caracterizó el proceso dinámico de precios por

\[\mathrm{d} p=R p \mathrm{~d} t+\sigma p \mathrm{~d} B\]

Mientras tanto, físicos, matemáticos y economistas realizaron una variedad de esfuerzos durante mucho tiempo para modelar una variedad de fenómenos usando la distribución de Pareto o una de sus parientes o generalizaciones, como el Lévy estable (1925) distribución, antes de la clara aparición de la econofísica. Alfred J. Lotka (1926) vio los descubrimientos científicos siguiendo este patrón. George Zipf (1941) vería los tamaños de las ciudades como si lo hicieran. Benoit Mandelbrot (1963) vio los precios del algodón hacerlo y se inspiró para descubrir la geometría fractal al estudiar las matemáticas de la propiedad de escala (Mandelbrot 1963, 1997). Ijiri y Simon (1977) vieron los tamaños de las firmas también siguiendo este patrón, un resultado confirmado más recientemente por Axtell (2001).

\hypertarget{el-papel-de-la-mecuxe1nica-estaduxedstica}{%
\section*{El papel de la mecánica estadística}\label{el-papel-de-la-mecuxe1nica-estaduxedstica}}
\addcontentsline{toc}{section}{El papel de la mecánica estadística}

Además, los economistas utilizarían modelos de mecánica estadística para estudiar una variedad más amplia de dinámicas económicas antes del surgimiento de la econofísica como tal. Aquellos que lo hicieron incluyeron a Hans Fōllmer (1974), Lawrence Blume (1993), Steven Durlauf (1993), William Brock (1993), Duncan Foley (1994) y Michael Stutzer (1994), con Durlauf (1997) que proporciona una descripción general de un conjunto aún más amplio de aplicaciones. Sin embargo, en 1993 los economistas eran plenamente activos incluso si aún no se habían identificado con este término.

Si bien poco de este trabajo se enfoca explícitamente en generar resultados consistentes con las leyes de escala, ciertamente es razonable esperar que muchos de ellos puedan hacerlo. Es cierto que la visión más tradicional de los mercados eficientes en la que todos los agentes poseen información completa y expectativas racionales sobre un único equilibrio estable no se mantiene en estos modelos y, por lo tanto, la crítica de la econofísica tiene cierto peso. Sin embargo, muchos de estos modelos hacen supuestos de al menos formas de racionalidad y aprendizaje limitados, con la posibilidad de que algunos agentes incluso se ajusten a los supuestos más tradicionales. Stutzer (1994) reconcilia la formulación de máxima entropía de la mecánica estadística de Gibbs con una formulación de economía financiera relativamente convencional de la fórmula de opciones de Black-Scholes, basada en afirmaciones contingentes de Arrow-Debreu (Arrow 1974). Brock y Durlauf (2001) formalizan agentes heterogéneos que interactúan socialmente dentro de un marco de elección discreta que maximiza la utilidad. 1 Ninguno de estos genera específicamente resultados de ley de escala, pero no hay nada que les impida hacerlo potencialmente.

Si bien algunos economistas buscan integrar sus hallazgos con la teoría económica, como se señaló anteriormente, muchos buscan reemplazar la teoría económica convencional, considerándola inútil y limitada. Una ironía en este esfuerzo es que se ha argumentado que la teoría económica neoclásica convencional en sí misma fue sustancialmente el resultado de importar concepciones de la física del siglo XIX a la economía, sin que todos los observadores lo aprobaran (Mirowski1989a). Muchos consideran que la culminación de este esfuerzo son los fundamentos del análisis económico de Paul Samuelson (1947), cuya licenciatura fue en física en la Universidad de Chicago. El propio Samuelson señaló con aprobación que la disertación de Irving Fisher de 1892 (1920) fue supervisado en parte por el pionero de la mecánica estadística, J. Willard Gibbs (1902), y ya en 1801, Nicholas-François Canard (1969) concibió la oferta y la demanda como ``fuerzas'' ontológicamente contradictorias en un sentido físico. De modo que la interacción entre la economía y la física se ha prolongado durante mucho más tiempo y es considerablemente más complicada de lo que se suele concebir.

La mayor parte de este trasfondo histórico más profundo de ir y venir no es conocido por los economistas actuales. Esto ha llevado a veces a que se formulen argumentos que son potencialmente problemáticos por diversos motivos. Estos han sido discutidos en un artículo crítico por Gallegati et al.~(2006) denominado ``Tendencias preocupantes en econofísica''. Las tendencias que identificaron incluyeron una falta de conocimiento de la literatura previa (especialmente en economía), una tendencia a creer que se pueden encontrar regularidades empíricas universales en la economía que probablemente no están allí en contraste con lo que se encuentra en gran parte de la física, una tendencia a utilizar metodologías estadísticas poco rigurosas a veces poco mejores que simplemente mirar cifras y, finalmente, utilizar fundamentos teóricos cuestionables, como asumir principios de conservación en situaciones en las que es poco probable que se mantengan. McCauley (2008) respondió a su crítica, argumentando que la teoría económica es tan defectuosa que simplemente debería ser rechazada en su totalidad a favor de las ideas que provienen de la física. Rosser Jr (2008a, B) consideró este debate y señala que, de hecho, los economistas a menudo hacen suposiciones que no son ciertas, aunque claramente existen límites a lo irreales que pueden ser las suposiciones en un modelo útil. También argumenta que una forma de resolver esto es que los físicos y los economistas lleven a cabo más investigaciones de manera conjunta, y que se haya desarrollado algo de eso.

\hypertarget{econoquuxedmica-y-econobiologuxeda}{%
\section*{Econoquímica y Econobiología}\label{econoquuxedmica-y-econobiologuxeda}}
\addcontentsline{toc}{section}{Econoquímica y Econobiología}

Curiosamente, pero como era de esperar, dada la tremenda atención prestada al nuevo movimiento de la econofísica, ha engendrado imitadores desde 2000 en la forma de la econoquímica y la econobiología , aunque estas no han tenido ni mucho menos el mismo grado de desarrollo. El primer término es el título de un curso de estudio establecido en la Universidad de Ulm por Barbara Mez-Starke y se utilizó para describir el trabajo de Hartmann y Rössler (1998) en una conferencia en 2002 en Urbino, Italia (ver también Padgett et al.~(2003) para un esfuerzo más reciente). El último término apareció por primera vez en Hens (2000), aunque McCauley (2004, págs. 196-199) lo descarta como un competidor no digno de la econofísica. Sin embargo, ha existido una larga tradición entre los economistas de abogar por inspirarse más en la biología que en la física (Hodgson1993a, B), remontándose al menos hasta la famosa declaración de Alfred Marshall de que la economía 2 es ``una rama de la biología interpretada en sentido amplio'' (Marshall1920, pag. 637), incluso cuando se podría decir que el aparato analítico real de Marshall se basó más en la física que en la biología.

\hypertarget{econofuxedsica-y-entropuxeda}{%
\section*{Econofísica y entropía}\label{econofuxedsica-y-entropuxeda}}
\addcontentsline{toc}{section}{Econofísica y entropía}

\begin{quote}
``A lo largo de los años he llegado a tener algo de impaciencia y aburrimiento con aquellos que intentan encontrar un análogo de la entropía de Clausius o Boltzman o Shannon para ponerlo en teoría económica. Es la estructura matemática de la termodinámica clásica (fenomenológica, macroscópica, no estocástica) que tiene isomorfismos con la economía teórica .''- Paul A. Samuelson, (1990, pag. 263)

``\ldots{} a lo largo de su carrera {[}de Samuelson{]} \ldots{} el maestro de la retórica científica, insinuando continuamente los paralelismos entre la teoría neoclásica y la física del siglo XX, y los niega conscientemente, generalmente en el mismo artículo.'' - Philip Mirowski, (1989b, pag. 186{]}
\end{quote}

El papel problemático de la entropía en la econofísica se destaca por las citas presentadas anteriormente: que el economista posiblemente más influyente del siglo XX, Paul A. Samuelson, lo jugó en ambos sentidos con respecto al papel del concepto de entropía en el desarrollo de la teoría económica, y más ampliamente el papel de la física en la economía. Si bien ridiculizaba regularmente las aplicaciones del concepto de entropía en economía, más poderosamente que cualquier otro economista impuso conceptos extraídos de la física en la economía neoclásica estándar, incluida esa parte importante de la economía, la mecánica estadística, una contradicción señalada con tanta fuerza por Mirowski.

El término ``econofísica'' fue introducido verbalmente por H. Eugene Stanley en una conferencia en Calcuta en 1995, y extensamente en forma impresa por Mantegna y Stanley (1999) quien lo identificó con físicos que aplicaban ideas de la física a la economía. Esta formulación se vuelve problemática cuando entendemos que las personas formadas como físicos lo han estado haciendo durante mucho tiempo, siendo el propio Samuelson un ejemplo destacado, junto con el de aquellos que recibieron el Premio Nobel de Economía antes que él, Jan Tinbergen (1937), cuyo profesor principal fue Paul Ehrenfest, quien formuló la ``hipótesis ergódica'' con su esposa (1911), basándose en el trabajo de su profesor principal, Ludwig Boltzmann (1884). 3 Boltzmann vinculó el estudio de la mecánica estadística con el concepto de entropía desarrollado inicialmente por Clausius (1867), quien a su vez se inspiró en el trabajo sobre la termodinámica de las máquinas de vapor de Carnot en 1824. La formulación más simple de la ley de la entropía tomó la forma de la segunda ley de la termodinámica: que en un sistema termodinámico cerrado aumenta la entropía.

Dado que estos préstamos de la física a la economía han sido anteriores al movimiento más reciente de los físicos para aplicar sus modelos a la economía, ampliaremos el concepto de economía independientemente de si estas aplicaciones fueron realizadas por personas que eran principalmente físicos, principalmente economistas, o que fueron posiblemente ambos, con muchos economistas importantes que se habían formado originalmente como físicos, con Tinbergen como estudiante de Ehrenfest como un ejemplo.

Con respecto a la aplicación específica de la idea de la entropía en la economía y, por tanto, como una forma de econofísica, distinguiremos entre dos enfoques básicos. Uno puede ser etiquetado como ontológico mientras que el otro puede verse como metafórico, aunque algunos involucrados en esto a veces han confundido a estos dos, como los energéticos Helm, Winiarski y Ostwald, como lo describe Mirowski (1989a). En la formulación ontológica, se considera que la base de la economía son los procesos físicos y biológicos impulsados \hspace{0pt}\hspace{0pt}por la energía, siendo la Segunda Ley de la Termodinámica un principio organizador clave para esta base, con Georgescu-Roegen (1971) el exponente más influyente de esta idea, aunque también fue un crítico de ella, como señaló Rosser Jr.~(1991). Este punto de vista sigue más la tradición de Carnot y Clausius, pero también depende del trabajo de Boltzmann modificado por Gibbs (1902) en mecánica estadística, con la formulación de Boltzmann-Gibbs de la ley de la entropía. Este enfoque tiene sus mayores defensores entre los economistas ecológicos, algunos de los cuales dicen que este punto de vista representa la ``economía biofísica'' (Christensen1989).

El enfoque metafórico se basa más en la formulación de información de la entropía debida a Shannon y Weaver (1949), con aplicaciones en las finanzas y la teoría del equilibrio, muchas de ellas más estrechamente vinculadas a la economía moderna. En última instancia, estos dos conceptos de entropía comparten matemáticas comunes de distribuciones de probabilidad de logaritmos de productos de estados posibles del mundo, incluso si tienen aplicaciones considerablemente diferentes. Si bien gran parte de la econofísica moderna se preocupa más por otros asuntos, como las distribuciones de variables de la ley de potencia, el concepto de entropía entra en muchas aplicaciones de la econofísica, con importantes nuevos enfoques de la economía que se basan en estas formulaciones más metafóricas. 4

\hypertarget{unidad-de-los-conceptos-centrales-de-la-entropuxeda}{%
\section*{Unidad de los conceptos centrales de la entropía}\label{unidad-de-los-conceptos-centrales-de-la-entropuxeda}}
\addcontentsline{toc}{section}{Unidad de los conceptos centrales de la entropía}

La forma más utilizada de la ecuación de Boltzmann para la entropía está en su tumba, aunque nunca la escribió de esa manera (Uffink 2014). Implica W, la probabilidad termodinámica de un estado agregado de un sistema de moléculas de gas, siendo k la constante de Boltzmann y S la entropía. Toma la forma

\[S=k \ln W\]

Dada N estados microscópicos del sistema, la probabilidad de una molécula de gas está en el estado i-ésimo es N i / N . Entonces W viene dado por (Chakrabarti y Chakraborty2006)

\[W=N ! / \Pi N_{i} !\]

Esto significa que la entropía de Boltzmann se puede reescribir como

\[S=k \ln \left(N ! / \Pi N_{i} !\right) .\]

La entropía básica de Shannon viene dada por H de la distribución de probabilidad de los estados de incertidumbre informativa para el mensaje i. de H ( p 1 \ldots{} p n ). Esto entonces equivale a (Shannon y Weaver1949; Renyi1961)

\[H\left(p_{1} \ldots p_{n}\right)=-k \Sigma p_{i} \ln p_{i}\]

Reconociendo que p i = N i / N , la unidad básica de estos dos conceptos aparece a medida que N aumenta, lo que lleva a la fórmula de Boltzmann en ( 4.6 ) a aproximarse a (Tsallis1988; Thurner y Hanel2012)

\[S=-k N \Sigma p_{i} \ln p_{i}\]

lo que significa que en el límite cuando N se acerca al infinito, la entropía de Boltzmann es proporcional a la entropía de Shannon.

\hypertarget{la-entropuxeda-ontoluxf3gica-y-la-econofuxedsica-como-luxedmite-fundamental-del-crecimiento}{%
\section*{La entropía ontológica y la econofísica como límite fundamental del crecimiento}\label{la-entropuxeda-ontoluxf3gica-y-la-econofuxedsica-como-luxedmite-fundamental-del-crecimiento}}
\addcontentsline{toc}{section}{La entropía ontológica y la econofísica como límite fundamental del crecimiento}

El enfoque ontológico de la econofísica se deriva del papel directo y fundamental de la energía en la economía, no solo para la producción industrial o el suministro de electricidad o transporte, sino a nivel ecológico o biofísico, el de la energía solar que impulsa la biosfera global. Esto es más un regreso a la visión de la termodinámica de Carnot y Clausius, donde la entrada continua de energía solar muestra la apertura del sistema de la tierra que le permite evitar la ley de la entropía mientras dure el sol (Georgescu-Roegen1971; Rosser Jr.1991). 5 Sin embargo, esa energía solar que llega en sí es finita y, por lo tanto, proporciona un límite directo a la actividad económica que depende de los ecosistemas a través de los cuales la energía solar se disipa en las cadenas alimentarias que son impulsadas por esa energía. Además, Georgescu-Roegen extendió este argumento a insumos de recursos materiales más amplios, argumentando que también están sujetos a una forma de la ley de la entropía que también proporciona límites adicionales a la economía. Más ampliamente para él (Georgescu-Roegen1971, pag. 281) ``el proceso económico consiste en una transformación continua de baja entropía en alta entropía, es decir, en desperdicio irrevocable o, con un término de actualidad, en contaminación''.

Si bien las variaciones de este argumento se han vuelto muy influyentes, especialmente en la economía ecológica como con Martinez-Allier (1987), también se ha enfrentado a fuertes críticas. Así, Gerelli (1985) argumenta que la escala de la entrada solar es tal que está en órdenes de magnitud más allá de limitar realmente la economía mundial, con muchas otras limitaciones más mundanas más relevantes en el corto plazo. Nordhaus (1992) estimó que la entropía es hasta 12 órdenes de magnitud por debajo de la tecnología como límite para el crecimiento, con Young (1994) pesando de manera similar. En ese sentido, la reducción de las fuentes de energía almacenadas y sus límites, como ocurre con los combustibles fósiles, puede ser más relevante con la contaminación derivada de su uso, incluso más limitante, como con resultados como el cambio climático derivado de la quema de dichos combustibles que liberan su dióxido de carbono almacenado. Otros críticos han enfatizado el ingenio ilimitado de la mente humana como Julian Simon, quien argumentó que (1981, pag. 347) ``aquellos que ven el universo relevante como ilimitado ven la segunda ley de la termodinámica como irrelevante para la discusión''.

Otra figura importante en esta línea de argumentación fue Alfred J. Lotka (1925), el padre del concepto de ciclos depredador-presa. Lotka argumentó que la ley de la entropía es una fuerza impulsora profunda en la evolución, una fuente de una dirección teleológica del proceso hacia una mayor complejidad. Vio esto como la base física fundamental de la biología que necesitaba ser estudiado matemáticamente y, a su vez, vio que la economía se derivaba del ecosistema, como lo han hecho los economistas ecológicos más recientes. Irónicamente, Lotka fue una tremenda influencia en Paul Samuelson, quien lo citó de manera prominente en su obra maestra, Foundations of Economic Analysis (1947), aunque más por su categorización de las condiciones de estabilidad de los sistemas lineales que por sus argumentos sobre la ley de la entropía o su relación con la economía.

\hypertarget{la-entropuxeda-ontoluxf3gica-y-la-visiuxf3n-energuxe9tica-del-valor-econuxf3mico}{%
\section*{La entropía ontológica y la visión energética del valor económico}\label{la-entropuxeda-ontoluxf3gica-y-la-visiuxf3n-energuxe9tica-del-valor-econuxf3mico}}
\addcontentsline{toc}{section}{La entropía ontológica y la visión energética del valor económico}

Estrechamente relacionada con el argumento de que los flujos de energía que se disipan a medida que opera la ley de la entropía son la base de la economía, está la idea de que la energía o alguna medida de entropía deberían ser la base para medir el valor en una economía. Esto fue propuesto por primera vez por físicos ``energéticos'' de finales del siglo XIX y principios del XX. Así Helm (1887) y Winiarski (1900) argumentó que el oro era ``energía sociobiológica''. Más cerca del argumento de la entropía estaba Ostwald (1908) cuya opinión era que los factores de conversión basados \hspace{0pt}\hspace{0pt}en la disponibilidad física de formas específicas de energía eran la clave para la determinación del valor fundamental. Ampliando esto, Julius Davidson (1919) argumentó que la ley de los rendimientos decrecientes en economía 6 se basaba en última instancia en la ley de la entropía. Mucho más tarde Davis (1941) argumentaría que la utilidad del dinero era una forma de ``entropía económica'', aunque Lisman (1949) señaló que esto no era operacionalmente equivalente a cómo funciona la ley de la termodinámica en física, y Samuelson (1972) simplemente descartó estos argumentos como ``chiflados''.

Curiosamente, algunos de los que apoyaban la idea de que la entropía desempeñaba un papel ontológico fundamental en la economía también tenían problemas con tales enfoques del valor. Lotka (1925, p.~355) señaló que,

``El proceso físico es un caso típico de `acción de disparo' en el que la relación entre la energía liberada y la energía aplicada no está sujeta a ninguna ley general restrictiva (por ejemplo, un toque del dedo sobre un interruptor puede disparar toneladas de dinamita). A diferencia del caso de los factores de conversión de la termodinámica, el factor de proporcionalidad viene determinado aquí por el mecanismo particular empleado''.

Lo mismo ocurre con Georgescu-Roegen (1971), aunque veía la entropía como el límite último del crecimiento, no la veía tan útil para determinar el valor, que en última instancia veía como resultado de la utilidad. Por lo tanto, nadie quiere el hongo venenoso de baja entropía y algunas personas valoran más el huevo batido de alta entropía que el huevo crudo de baja entropía. Estas son cuestiones de utilidad, y aunque Georgescu-Roegen no veía la utilidad (o la utilidad marginal para ser más precisos) como la única fuente de valor como lo hacían los teóricos subjetivistas de la escuela austriaca, ciertamente la veía como muy importante y era un importante desarrollador de la teoría de la utilidad moderna al principio de su carrera. 7

\hypertarget{entropuxeda-metafuxf3rica-y-valor-de-equilibrio-general}{%
\section*{Entropía metafórica y valor de equilibrio general}\label{entropuxeda-metafuxf3rica-y-valor-de-equilibrio-general}}
\addcontentsline{toc}{section}{Entropía metafórica y valor de equilibrio general}

Pasando al corazón de la economía, la entropía se ha propuesto como una alternativa a la explicación convencional del valor de Arrow-Debreu. Esa visión estándar tiene el equilibrio como un vector de precios que son puntos fijos. La alternativa entrópica reconoce la realidad de un mundo estocástico en el que el equilibrio se describe mejor como una distribución de probabilidad de precios, ya que los precios nunca son los mismos en todas partes y en ningún momento para cualquier producto, excepto como medida de accidente cero. Una expresión temprana de esta idea se debe a Hans Fōllmer (1974). Un desarrollo más completo de esto se ha debido a Foley (1994), más tarde ampliado por Foley y Smith (2008).

El Foley básico (1994) El modelo implica supuestos sólidos como que todas las transacciones posibles dentro de una economía tienen la misma probabilidad. Sin embargo, su solución implica una distribución estadística de comportamientos en la economía donde una transacción particular es inversamente proporcional al exponencial de su precio de entropía de equilibrio, y esto proviene de un conjunto de precios sombra de entropía máxima de Botlzmann-Gibbs. El equilibrio general walrasiano es un caso especial de este modelo cuando la ``temperatura'' es cero. La forma más general carece de las implicaciones habituales de bienestar y permite la posibilidad de precios negativos como en el caso de las subastas de Herodotus (Baye et al.2012). 8 Sin embargo, Foley enfatiza el papel crucial de las restricciones en este enfoque, algo que comparte con el modelo Arrow-Debreu.

Sea m mercancías, n agentes de tipo k que logren una transacción x de la cual hay h k {[} x {]} proporción de agentes de tipo k de r que realizan la transacción x de un conjunto de oferta A , de los cuales hay mn . Multiplicidad de una asignación para n agentes asignados a S acciones, cada uno de ellos s , viene dada por:

\[W\left[n_{s}\right]=n ! /\left(n_{1} ! \ldots n_{s} ! \ldots n_{S} !\right)\]

La entropía de Shannon de esta multiplicidad viene dada por:

\[H\left\{h^{k}[x]\right\}=-\Sigma_{k=1}^{r} W^{k} \Sigma_{x \varepsilon A k} h^{k}[x] x=0\]

Maximizar esta formulación entrópica sujeta a las restricciones de viabilidad apropiadas, que si no está vacía, da la solución canónica única de Gibbs:

\[H^{k}[x]=\exp [-\Pi x] / \Sigma_{x} \exp [-\Pi x]\]

donde Π son vectores de los precios sombra de la entropía.

\hypertarget{entropuxeda-entre-econofuxedsica-y-sociofuxedsica}{%
\section*{Entropía entre econofísica y sociofísica}\label{entropuxeda-entre-econofuxedsica-y-sociofuxedsica}}
\addcontentsline{toc}{section}{Entropía entre econofísica y sociofísica}

Otro uso metafórico de los conceptos de entropía ha sido en conjunto con ese pariente cercano de la econofísica, la sociofísica . Acuñado inicialmente por Galam et al.~(mil novecientos ochenta y dos), sigue la sociodinámica del neologismo desarrollada por Weidlich y Haag (1980). Un énfasis importante de esta sociofísica está en el modelado de la dinámica de grupo, incluido el pastoreo. Una solución favorecida por Weidlich y Haag (1983) es la ecuación maestra, utilizada especialmente para estudiar patrones migratorios, entre otros fenómenos. Cuando las restricciones no resuelven de manera única el modelo estocástico de esta ecuación, un proceso de Markov de orden n puede surgir como la única solución de entropía máxima (Lee y Pressé2012).

Si bien no está tan desarrollada como la econofísica, la sociofísica ha seguido su fundación por Galam junto con Weidlich y Haag a lo largo de una variedad de caminos, con Chakrabarti et al.~(2008) proporcionando una excelente descripción de estas investigaciones. Ambas posibilidades de aplicar el concepto de entropía a este enfoque han sido estudiadas en profundidad por Mimkes (2008), quien también se esfuerza por extender su análisis a todas las ciencias sociales. En su formulación vemos un retorno a la cuestión de las aplicaciones ontológicas versus metafóricas del concepto de entropía cuando Mimkes vincula la entropía con la naturaleza fundamental de la función de producción. Si bien esto evoca la visión de Georgescu-Roegen (1971) donde los procesos reales de la economía son fundamentalmente una elaboración de la Segunda Ley de la Termodinámica, Mimkes finalmente se retira a una aplicación más metafórica donde es la formulación matemática de la entropía como un dispositivo descriptivo para los datos sobre los resultados distributivos en la economía que es el foco principal del análisis. Si bien invoca e implica la percepción ontológica más profunda, al final gana el enfoque más metafórico. Sin embargo, no hay ninguna razón por la cual una sociofísica más desarrollada pueda no involucrar aún más seriamente el enfoque ontológico.

\hypertarget{modelado-financiero-entruxf3pico-metafuxf3rico}{%
\section*{Modelado financiero entrópico metafórico}\label{modelado-financiero-entruxf3pico-metafuxf3rico}}
\addcontentsline{toc}{section}{Modelado financiero entrópico metafórico}

En la portada de sus Fundamentos del análisis económico (1947), Paul Samuelson citó a Gibbs diciendo: ``Las matemáticas son un lenguaje''. Eso es ciertamente. Pero en el caso de la entropía de Shannon, así como en los modelos financieros basados \hspace{0pt}\hspace{0pt}en las matemáticas de la entropía, es una metáfora más que una ontología lingüística.

Basándose en muchas discusiones de varios economistas, Schinkus (2009) argumenta que los economistas están más inclinados que los economistas habituales a abordar los datos sin ideas preconcebidas sobre distribuciones o valores de parámetros, aunque pueden estar más inclinados a basarse en ideas de la física, con entropía entre las relacionadas con la modelización financiera. Así, Dionosio et al.~(2009, pag. 161) argumentan que:

\begin{quote}
``La entropía es una medida de dispersión, incertidumbre, desorden y diversificación utilizada en procesos dinámicos, en estadística y teoría de la información, y se ha adoptado cada vez más en la teoría financiera''.
\end{quote}

Las aplicaciones de la ley de la entropía usando entropía de Shannon o distribuciones de Boltzmann-Gibbs encajan fácilmente en distribuciones explicativas o de modelado que se basan en la lognormalidad, que son fácilmente consistentes con los enfoques gaussianos. Si bien sabemos que, en última instancia, estas entropías son esencialmente idénticas matemáticamente, la diferencia real es que creemos que está impulsada a la maximización como una ley de la física, mientras que en las más metafóricas observar un extremo de la entropía es simplemente una condición matemática útil.

Alguien que aproveche las dos medidas principales de entropía para desarrollar la teoría financiera central en la forma de la fórmula de fijación de precios de opciones de Black-Scholes (1973) es Michael J. Stutzer (1994, 2000). En el segundo de estos usó la entropía de Shannon para su generalización del vínculo, después de señalar que Cozzolino y Zahneri (1973) había utilizado la entropía de Shannon para derivar distribuciones logarítmicas normales del precio de las acciones, el mismo año en que Black y Scholes (1973) publicaron su resultado sin depender directamente de ninguna matemática de entropía. Por su generalización Stutzer (2000) planteó el problema en forma discreta al considerar un proceso de precio de mercado de valores dado por

\[\Delta p / p=\mu \Delta t+\sigma \sqrt{\Delta t \Delta z}\]

donde p es el precio, t es el intervalo de tiempo , y el segundo término en el lado derecho es el choque al azar, con estos distribuido \textasciitilde{} N (0, ? t ). Con Q como cantidad, rΔt la tasa libre de riesgo de retorno, y P la distribución real densidad riesgo condicional, un foco central es el riesgo condicional neutral dada por d Q / d P .

A partir de estos, se considera la entropía relativa que minimiza la densidad neutral al riesgo condicional que en efecto maximiza el orden

\[\arg \min _{\mathrm{d} Q / \mathrm{d} P} \int \log \mathrm{d} Q / d P d Q\]

sujeto a una restricción de martingala dada por

\[r \Delta t-E[(\Delta p / p)(d Q / d P]=0\]

A partir de esto, muestra que cuando los rendimientos de los activos son IID con choques distribuidos normalmente como se indica anteriormente, la densidad del producto martingala formada a partir de la entropía relativa que minimiza el riesgo condicional es la que se usa para calcular la fórmula de fijación de precios de opciones de Black-Scholes. Reconoce que esto no se generaliza fácilmente a distribuciones no gaussianas, como las de la ley de potencias, muy estudiadas por los economistas, lo que sugiere un enfoque más débil utilizando procesos heterocedásticos condicional auto regresivos generalizados (GARCH).

\hypertarget{muxe1s-metuxe1fora-el-proceso-anti-entruxf3pico-de-la-burbuja-de-minsky}{%
\section*{Más metáfora, el proceso anti-entrópico de la burbuja de Minsky}\label{muxe1s-metuxe1fora-el-proceso-anti-entruxf3pico-de-la-burbuja-de-minsky}}
\addcontentsline{toc}{section}{Más metáfora, el proceso anti-entrópico de la burbuja de Minsky}

Como se discutió anteriormente, la maximización de la entropía implica una dinámica estocástica gaussiana. Estos no son consistentes con las distribuciones de la ley de energía que se ven en los rendimientos del mercado financiero o en las distribuciones de riqueza. Una fuente probable de esta diferencia es la tendencia a la dinámica de burbujas antientrópicas que puede describirse mediante el proceso de Minsky (Minsky1972, mil novecientos ochenta y dos; Kindleberger2001; Rosser Jr.1991). En lugar de nivelar las irregularidades, una burbuja especulativa puede aumentar las desviaciones de los resultados del equilibrio a largo plazo, ya sea de tipo entrópico estocástico como el modelo de Foley y otros o un equilibrio general determinista walrasiano. La dinámica de retroalimentación positiva que surge del impulso o el ruido de los comerciantes empuja los precios a extremos, alejándolos temporalmente de estos equilibrios, y por lo general terminan con algún tipo de colapso. Estos movimientos extremos conducen a las colas gordas kurtóticas que aparecen en la dinámica de rendimiento de los activos financieros de manera tan ubicua (Lux2009).

Minsky1972) argumentó que estas dinámicas emergen de manera endógena a través de mecanismos psicológicos en los que los agentes se vuelven complacientes con respecto al riesgo durante períodos de equilibrio entrópico con distribuciones gaussianas predominando en respuesta a choques exógenos. Pasan por etapas de creciente asunción de riesgos, en las que los índices de apalancamiento aumentan y emergen burbujas. La etapa final de este proceso involucra la dinámica Ponzi que se ha desquiciado completamente de los fundamentos. La riqueza ha aumentado drásticamente con los precios de la burbuja especulativa, pero al final la burbuja se derrumba, generalmente en un dramático Momento de Minsky cuando el pánico se apodera y los agentes venden el activo en masa (Kindleberger 1972). Con esto, la dinámica devuelve los precios a la zona de equilibrio entrópico a largo plazo, la ``Venganza de la entropía''.

Se ha entendido bien que históricamente tales dinámicas han adoptado generalmente una de tres formas diferentes (Rosser Jr.~1991, Cap. 5). Los tres de estos casos se muestran en este libro en las Figs. 2.1 , 2.2 , 2.3 (Rosser Jr.~et al.~2012) para los activos que exhibieron cada uno de ellos durante el período de la Gran Recesión de 2007-2009, aunque la dinámica se muestra durante más tiempo, con uno de ellos (la vivienda en Estados Unidos) alcanzando su punto máximo antes de la crisis financiera más amplia que marcó el comienzo de la Gran Recesión.

El primer caso involucra un precio que sube de manera acelerada, solo para colapsar repentinamente después de alcanzar su punto máximo en un momento dramático de Minsky. Para el período de la crisis financiera, esto está bien demostrado por los precios del petróleo, que alcanzaron su punto máximo en julio de 2008 a \$ 147 por barril, solo para bajar drásticamente a alrededor de \$ 30 por barril en noviembre de 2008 (ver Figura 2.1 , este libro). Estos patrones se ven a menudo en burbujas especulativas de precios de materias primas.

El segundo caso implica un aumento más gradual de los precios que luego disminuye también de manera gradual una vez alcanzado el pico. Se puede argumentar que tal caso no posee un Minsky Moment propiamente dicho en el sentido de un choque repentino asociado con un pánico, aunque puede haber emociones de pánico involucradas para los agentes en tal burbuja y su declive. El ejemplo de la crisis financiera es que si la vivienda en Estados Unidos, cuyos precios comenzaron a subir en 1998 y luego alcanzaron su punto máximo en 2006, disminuyeron a partir de entonces durante varios años, como se muestra en la figura 2.2 . (este libro). De hecho, el sector inmobiliario parece más propenso a mostrar este patrón, y una explicación que parece ser válida especialmente para el sector inmobiliario residencial es que la gente se niega a vender inmediatamente durante la recesión, creyendo que los precios son ``injustos'' y ``demasiado bajos'', lo que lleva a ellos alquilan su vivienda si deben mudarse o simplemente se niegan a vender. Por lo tanto, tales patrones tienden a mostrar una caída en el volumen de ventas durante la caída más que una rápida caída en el precio, que cae a medida que finalmente la gente se rinde y acepta los precios más bajos.

El tercer caso es históricamente el más común según lo documentado por Kindleberger (2001, Apéndice B). Implica precios que suben a un pico y luego disminuyen por un tiempo de manera gradual durante un ``período de dificultades financieras'' (Minsky1972), luego, en un momento posterior, experimentando el Momento Minsky y chocando con fuerza. Durante la crisis financiera, la mayoría de los mercados de activos financieros mostraron este patrón, y la figura 2.3 (este libro) muestra el ejemplo del mercado de valores estadounidense medido por el promedio industrial Dow-Jones. Alcanzó su punto máximo en octubre de 2007, pero luego se desplomó en septiembre de 2008, 11 meses después de atravesar un período de declive más errático.

Esta dinámica no puede modelarse asumiendo agentes homogéneos. En la cima, los ``iniciados'' inteligentes o afortunados se venden a ``forasteros'' menos inteligentes o afortunados que continúan aferrándose al activo, al igual que los propietarios en el segundo caso se niegan a vender sus casas inicialmente a medida que el precio baja. El Momento Minsky finalmente llega cuando el pánico golpea a este grupo de agentes y venden en masa en el choque. Aunque este es, con mucho, el patrón más común de burbujas especulativas en la historia, ha habido pocos esfuerzos para modelarlo. Gallegati et al.~(2011) en un modelo basado en agentes derivado en última instancia del enfoque de Brock-Hommes (Brock y Hommes 1997). En este marco, el comportamiento de agentes heterogéneos está determinado cualitativamente por un parámetro de contagio y una disposición a cambiar el parámetro de comportamiento. El mecanismo del colapso retrasado después del período de dificultades financieras provino de una restricción de riqueza como la que encuentran los agentes en los mercados de activos con llamadas de margen. Cuando el precio cae por debajo de cierto nivel, pueden verse obligados a vender. La Figura 2.4 (este libro) muestra una simulación que muestra el patrón general y también muestra el impacto de un aumento en la fuerza del parámetro de contagio, que mueve el pico más alto y lo retrasa levemente (Gallegati et al.~2011).

\hypertarget{modelado-de-la-dinuxe1mica-de-distribuciuxf3n-de-la-riqueza-y-la-renta-mediante-la-mecuxe1nica-estaduxedstica}{%
\section*{Modelado de la dinámica de distribución de la riqueza y la renta mediante la mecánica estadística}\label{modelado-de-la-dinuxe1mica-de-distribuciuxf3n-de-la-riqueza-y-la-renta-mediante-la-mecuxe1nica-estaduxedstica}}
\addcontentsline{toc}{section}{Modelado de la dinámica de distribución de la riqueza y la renta mediante la mecánica estadística}

Al estudiar la dinámica de distribución de la riqueza y el ingreso, encontramos que la relación entre las distribuciones de la ley de potencias no basadas en la entropía y las distribuciones de la ley de potencias juega un papel central en el modelado de estos sistemas dinámicos. En particular, parece cada vez más que, si bien la dinámica de la riqueza refleja en gran medida las distribuciones de la ley de poder, la dinámica de distribución del ingreso puede ser una combinación, con distribuciones de Boltzmann-Gibbs relacionadas con la entropía que explican mejor la distribución del ingreso para el 97-98 por ciento más pobre, mientras que una distribución de la ley de poder de Pareto puede funcionar mejor para el nivel superior de ingresos, donde la dinámica de la riqueza puede desempeñar un papel más importante (Drăgulescu y Yakovenko2001; Yakovenko y Rosser Jr.2009).

La conciencia de la posibilidad de utilizar ideas de entropía en la medición de la distribución del ingreso comenzó cuando los economistas buscaban generalizaciones de las diversas medidas en competencia que se han utilizado para estudiar la distribución del ingreso. Así, en 1981, Cowell y Kugal (1981) buscó una formulación axiomática generalizada para medidas aditivas de distribución del ingreso. Descubrieron que al agregar dos axiomas al enfoque habitual, pudieron demostrar que un enfoque de entropía generalizada podría subsumir la medida de Atkinson ampliamente estudiada (1970) y la medida de Theil (Bourguignon 1979). Si bien la medida de Atkinson se ha utilizado más ampliamente y es capaz de distinguir la asimetría de las colas, la de Theil puede tener más generalidad. Bourguignon1979) muestra que es la única medida de desigualdad descomponible ``ponderada por ingresos'' que es homogénea en cero. Cowell y Kugal (1981) muestran que al agregar un axioma de sensibilidad a los demás, el índice de Theil es el único que se puede derivar de un concepto de entropía generalizado.

Estas primeras discusiones también involucraron fuertes reclamos con respecto a las dificultades de vincular las medidas de entropía con las distribuciones de la ley de potencia, reclamos que ahora parecen exagerados hasta cierto punto. Así encontramos a Montroll y Schlesinger (1983, pag. 209) alegando que:

\begin{quote}
``La derivación de distribuciones con colas de potencia inversa a partir del formalismo de máxima entropía sería una consecuencia sólo de una condición auxiliar no convencional que implica la especificación del promedio de una función logarítmica complicada''.
\end{quote}

Esta afirmación puede ser exagerada, aunque de hecho las funciones logarítmicas están involucradas en la relación entre las dos, lo que no es sorprendente dado que las medidas de entropía son esencialmente logarítmicas.

El enfoque de distribución de la ley de poder domina la discusión de la dinámica de distribución de la riqueza, al igual que la dinámica de los mercados financieros. El padre de este enfoque fue Vilfredo Pareto (1897), quien inicialmente se formó como ingeniero, pero luego se convirtió en socio-economista ya que su teoría involucraba la relación entre clases sociales a lo largo del tiempo. Muy apropiadamente, la motivación original y el enfoque de estudio de Pareto fue, de hecho, la distribución del ingreso. Afirmó una verdad universal asociada con un parámetro de distribución de ingresos estimado. Estaba equivocado, especialmente dado que su teoría se ajusta a mejores distribuciones de riqueza en lugar de distribuciones de ingresos, como se señaló anteriormente. Pareto argumentó incorrectamente que su coeficiente supuestamente universal para la explicación de la ley de poder de la distribución del ingreso encajaba en su teoría de la ``circulación de las élites'', en la que no se podía hacer nada para igualar el ingreso porque el proceso político simplemente implicaría sustituir una élite de poder por otro sin cambios apreciables en la distribución del ingreso. Pero debemos reconocer que formuló este punto de vista a fines del siglo XIX, cuando no había habido un siglo de cambios importantes en la estructura socioeconómica en ninguna parte. No hace falta decir que no mucho después hubo grandes cambios en la distribución (Piketty2014), incluso cuando su método pasó a ser ``subterráneo'', solo para ser revivido para otros usos, como describir distribuciones de tamaño metropolitano urbano (Auerbach 1913).

La preocupación moderna por la distribución del ingreso basada en los conceptos de la física de la ley de potencia de Pareto se debió a un sociólogo, John Angle (1986). Después de la aparición de la economía actual, muchos dieron un paso adelante para aplicar las distribuciones de la ley de potencia para estudiar la dinámica de las distribuciones de riqueza. Basándose en el trabajo de Pareto, quien pensó erróneamente que había encontrado un coeficiente universal para la distribución del ingreso, los economistas encontraron que las distribuciones actuales de la riqueza se ajustan al punto de vista de la ley de poder de Pareto (Bouchaud y Mézard2000; Chakraborti y Chakrabarti2000; Salomón y Richmond2002).

En este punto, es necesario considerar la cuestión de si estamos tratando con modelos ontológicos en contraposición a modelos ``meramente'' metafóricos en estos asuntos. Sabemos que existen tendencias estocásticas para la dinámica de la riqueza y la renta, pero no es del todo obvio que los diversos imperativos aparentes para la maximización o minimización de la entropía estén realmente impulsando los resultados. Sin embargo, muchos que estudian estos temas ven procesos termodinámicos subyacentes a las tendencias básicas de la dinámica de distribución de la riqueza y el ingreso. Estos procesos no son tan directos como la dirección ontológica basada en las máquinas de vapor de Carnot, pero derivan de tendencias más amplias de la dinámica de distribución de la riqueza y el ingreso que ocurren en ausencia de cambios sustanciales en las políticas públicas con respecto a las políticas distributivas.

Pareto se equivocó en su propuesta original. Pensó que había encontrado una ley universal de distribución del ingreso que encajaba con su teoría de la ``circulación de las élites'', dentro de la cual no importaba qué grupo de élite gobernara la sociedad, la distribución subyacente del ingreso no cambiaría. Él estaba equivocado. El legado de su enfoque ha estado en el estudio de las distribuciones de riqueza, donde ahora se entiende que su presentación de las leyes de poder explica las distribuciones de riqueza en lugar de las distribuciones de ingresos.

La distribución de Pareto viene dada por:

\[N=A / x^{\alpha}\]

donde N es el número de observaciones por encima de x , y A y α son constantes. Esto incluye como casos especiales una amplia variedad de otras formas que subyacen a muchos modelos económicos. El caso especial cuando α = 1 conduce a la ``Ley de Zipf'', (Zipf1941), ampliamente visto para describir la distribución del tamaño urbano, así como muchos otros, aunque hasta qué punto se aplica esta ``ley'' es un tema de debate continuo.

Yakovenko y Rosser Jr.~(2009) presentan un análisis de distribución de ingresos unificado que combina una formulación entrópica de Boltzmann-Gibbs para la distribución de ingresos más bajos con una distribución de la ley de potencia Paretiana para los niveles más altos de ingresos. El modelo hace una suposición heroica de conservación del dinero o los ingresos o la riqueza, lo que empíricamente no es irrazonable para los Estados Unidos desde mediados de la década de 1970 para los niveles medios, incluso cuando los estratos superiores han experimentado niveles crecientes. Pero esto encaja con la combinación de un modelo entrópico logarítmico normal para la mayoría de la población con respecto al ingreso, incluso cuando el nivel superior de la distribución del ingreso parece seguir una dinámica de riqueza siguiendo una distribución de la ley de poder Paretiana.

Suponiendo una conservación del dinero, m , la distribución de equilibrio de Boltzmann-Gibbs de base entrópica está dada por la probabilidad, P , de que el nivel sea m , dada por:

\[P(m)=c e^{-m / T_{m}}\]

donde c es una constante de normalización y T m es la ``temperatura del dinero'' en términos termodinámicos, que es igual a la oferta monetaria per cápita. Esto describe la porción más baja de la distribución del ingreso.
Suponiendo una tasa fija de transferencias monetarias proporcionales con esto igual a γ, la distribución estacionaria del dinero (ingreso) está relacionada con la forma de distribución Gamma que se diferencia de la de Boltzmann-Gibbs por tener un prefactor de ley de potencia, m β , donde

\[\beta=-1--\ln 2 / \ln (1--\gamma)\]

Esto relaciona la forma de Boltzmann-Gibbs con una ley de potencia equivalente más simplemente de lo que suponen Montroll y Schlesinger (1983). Esta formulación que muestra la conexión entre las dos conceptualizaciones de la distribución de la riqueza y la renta viene dada por:

\[P(m)=c m^{\beta} e^{-m / T}\]

Esto representa la distribución estacionaria, pero permitir que m crezca estocásticamente desconecta el resultado de la solución de máxima entropía (Huang2004). La distribución estacionaria en estas condiciones se convierte en un caso de campo medio gobernado por una ecuación de Fokker-Planck, que no es ni Boltzmann-Gibbs ni Gamma, pero es una versión de una distribución Lotka-Volterra generalizada, con w la riqueza por persona, J es la transferencia promedio entre agentes, y σ es la desviación estándar, y es

\[P(w)=c\left[\left(e^{-J / \sigma \sigma w}\right) /\left(w^{2+J / \sigma \sigma}\right)\right]\]

Por lo tanto, es posible combinar una formulación entrópica de Boltzmann-Gibbs para la parte inferior de la distribución del ingreso con una forma de ley de potencia para su extremo superior, que corresponde a la formulación de la dinámica de la riqueza que se deriva irónicamente de Pareto, dado que originalmente pensó que su conceptualización era una ley universal de distribución del ingreso. Su formulación sería contrarrestada poco después por Bachelier (1900), pero ahora vemos los dos juntos para proporcionar una explicación empírica de la distribución del ingreso que tiene profundas raíces en las formulaciones económicas marxistas y otras clásicas con respecto a la dinámica de clase socioeconómica (Cockshott et al.~2009; Shaikh2016; Shaikh y Jacobo2020).

La Figura 4.1 muestra una distribución de este tipo en su forma logarítmica para la distribución del ingreso de EE. UU. En 1997, con la porción de Boltzmann-Gibbs, que cubre el 97 por ciento inferior de la distribución del ingreso, siendo no lineal en el lado izquierdo, mientras que la porción paretiana es lineal en registros en el lado derecho que cubren el 3 por ciento superior de la distribución del ingreso (Yakovenko2013, Figura 5).

\textbf{Figura 4.1} Distribución log-log del ingreso de Estados Unidos, secciones de Boltzmann-Gibbs y Pareto, 1997, de Yakovenko ( 2013 , Fig.5)

\hypertarget{rompiendo-burbujas-y-la-venganza-de-la-entropuxeda-metafuxf3rica}{%
\section*{Rompiendo burbujas y la venganza de la entropía metafórica}\label{rompiendo-burbujas-y-la-venganza-de-la-entropuxeda-metafuxf3rica}}
\addcontentsline{toc}{section}{Rompiendo burbujas y la venganza de la entropía metafórica}

Ahora consideramos más específicamente cómo interactúa la dinámica del mercado financiero con la dinámica de distribución de la renta y la riqueza en el curso de las burbujas especulativas que siguen un proceso de Minsky. Un aspecto notable de una burbuja importante es que aumenta la riqueza y los ingresos de la parte superior de la jerarquía de distribución de ingresos y riqueza en comparación con el resto. Esto está asociado con la dinámica anti-entrópica del proceso y se revierte cuando la burbuja desaparece en un choque, la ``venganza de la entropía''. 9 Esto debería aparecer durante una burbuja como un movimiento ascendente de la porción Paretiana que también moverá su límite con la porción Boltzmann-Gibbs de la distribución hacia la izquierda.

No tenemos los datos de la crisis financiera más reciente, ni los tenemos de la Gran Depresión, otro período que siguió a un colapso financiero importante que se ha postulado que ha reducido drásticamente la riqueza y algo igualado la distribución del ingreso, aunque los niveles de riqueza declinó sustancialmente, la Gran Depresión provocó el fin de la ``Edad Dorada'' (Smeeding 2012). Los eventos durante la Gran Recesión 2007-2009 son más complicados en parte porque se involucraron varias burbujas diferentes, con el colapso de la burbuja inmobiliaria impactando fuertemente a la clase media mientras que el mercado de valores y los colapsos del mercado de derivados afectaron más fuertemente a los ricos. Por lo tanto, en su punto más bajo en 2009, el mercado de valores estadounidense había caído más de la mitad de su valor. La riqueza total se redujo a finales de 2009 en un 50 por ciento. De esto, la riqueza del 10 por ciento superior cayó un 13 por ciento, mientras que la riqueza del 1 por ciento superior cayó un 20 por ciento (Smeeding2012). Sin embargo, el mercado de valores se recuperó con bastante rapidez, más que en la década de 1930 o incluso después de 2000, mientras que el mercado de la vivienda de EE. UU. Se recuperó mucho más lentamente, lo que llevó a un resultado en el que la desigualdad de riqueza probablemente se redujo durante un período de tiempo durante 2008-- En 2009, es casi seguro que aumentó después de eso, ya que los que estaban en la parte superior se beneficiaron de la recuperación del mercado de valores, mientras que los que estaban en el medio se vieron frenados por los continuos problemas en el mercado de la vivienda de EE. UU. El proceso de Minsky estaba en marcha, pero de una manera más complicada que en otras situaciones históricas.

Sin embargo, la evidencia de apoyo, si es débil, se puede ver al considerar el final de la burbuja de las puntocom en 2000. Esto se puede ver en la Figura 4.2 (Yakovenko2013, Fig. 6), que muestra la relación logarítmica-logarítmica de la distribución del ingreso en los Estados Unidos para los años 1983--2001. En general, se ve poco movimiento de la porción de Boltzmann-Gibbs, pero pequeños cambios anuales de la parte Paretiana, lo que refleja una desigualdad en constante aumento a lo largo del tiempo. Sin embargo, hay una excepción en esta figura, lo que sucedió entre 2000 y 2001, los últimos años mostrados, con 2000 el fin de la burbuja de las puntocom. En este caso, vemos una reversión, con la porción Paretiana de 2001 por debajo de las porciones de 2000. Esto sería coherente con nuestra historia de la venganza de la entropía tras el colapso de la burbuja de las puntocom, bastante sustancial, de finales de la década de 1990.

\textbf{Figura 4.2} Distribución anual logarítmica-logarítmica del ingreso en EE. UU., 1983-2001, de Yakovenko ( 2013 , Fig. 6)

\textbf{Notas al pie}

\begin{enumerate}
\def\labelenumi{\arabic{enumi}.}
\tightlist
\item
  Si bien la mayoría de los modelos económicos financieros realizados por economistas se han basado en modelos derivados de la mecánica estadística, un rival ha sido los modelos basados \hspace{0pt}\hspace{0pt}en modelos geofísicos de terremotos (Sornette, 2003). Véase también Rosser Jr.~(2008b).
\item
  Para una discusión más completa de las relaciones entre la econofísica, la econoquímica y la econobiología dentro de la perspectiva transdisciplinaria, ver Rosser Jr.~(2010b).
\item
  Véase Rosser Jr.~(2016a) para un análisis más detallado del desarrollo de la hipótesis ergódica y su relación con la economía. Rosser Jr.~(2016b) considera el papel de la entropía en econofísica con más detalle.
\end{enumerate}

4 .
Observamos que ahora hay una variedad de extensiones de las versiones más básicas de entropía de Boltzmann-Gibbs y Shannon, incluida Renyi (1961) y Tsallis (1988) (este último más estrechamente relacionado con el estudio de las distribuciones de la ley de potencias), con varios esfuerzos para generalizarlos como lo hicieron Thurner y Hanel (2012). Sin embargo, no nos centraremos en estos y notaremos que la mayoría de ellos se reducen a las formas más simples de forma asintótica a medida que ciertos parámetros modificadores se acercan al infinito, incluso cuando reconocemos que pueden ser útiles para aplicaciones futuras. Véase Rosser Jr.~(2016b) para una mayor discusión.

5 .
Georgescu-Roegen (1971) en particular se basó fuertemente en el argumento de Schrōdinger (1945, Cap. 6) sobre cómo la vida es en última instancia un proceso antientrópico basado en que los organismos son sistemas abiertos capaces de atraer tanto materia como energía mientras viven, y en este sentido la muerte de los organismos representa la victoria final de la entropía. Una alternativa es seguir más directamente a Carnot y Clausius al enfatizar el papel de la máquina de vapor en la economía moderna como en Cockshott et al.2009).

6 .
La ley de los rendimientos o la productividad decrecientes (marginales) es probablemente la única denominada ``ley'' en economía para la que no se ha encontrado ningún contraejemplo.

7 .
Rosser Jr.~(2008a) proporciona más información sobre este debate.

8 .
Herodoto describió una subasta matrimonial en Babilonia con precios descendentes para las posibles novias. Lo más deseable iría por precios positivos, pero la subasta permitió precios negativos para las novias potenciales menos deseables. Esto contrasta con la mayoría de las sociedades donde hay un precio positivo para la novia o un precio positivo para el novio, más a menudo descrito como una ``dote''. El problema de los precios negativos a menudo se confunde al declarar dos mercados separados, como uno para suministrar agua cuando escasea y otro diferente para sacarla cuando hay inundaciones. Pero el mercado de novias babilónico descrito por Herodoto deja en claro que puede haber mercados unificados con precios tanto positivos como negativos.

9 .
Véase Rosser Jr.~(2020c) para una mayor discusión.

image\_3.2.png

\hypertarget{econofuxedsica-y-entropuxeda-en-sistemas-urbanos-regionales-dinuxe1micamente-complejos}{%
\chapter*{Econofísica y entropía en sistemas urbanos / regionales dinámicamente complejos}\label{econofuxedsica-y-entropuxeda-en-sistemas-urbanos-regionales-dinuxe1micamente-complejos}}
\addcontentsline{toc}{chapter}{Econofísica y entropía en sistemas urbanos / regionales dinámicamente complejos}

Desde al menos los primeros esfuerzos de Alan Wilson (1967, 1970), la idea de utilizar la ley de la entropía para ayudar a modelar el desarrollo de patrones estructurales espaciales urbanos y regionales ha sido influyente. Para comprender cómo se ha hecho esto y cuán útil es como enfoque, primero debemos considerar las diversas formulaciones de esa ley que se han hecho. El desarrollo completo de la idea está asociado con la segunda ley de la termodinámica debido principalmente a Boltzmann (1884), aunque se basa en trabajos anteriores de Carnot (1824) y Clausius (1867). Jaynes (1957) preparó este enfoque para su aplicación en economía y Georgescu-Roegen (1971) también proporcionó una perspectiva profunda. Posteriormente, Shannon (1948) ampliaría esto al estudio de los patrones de información. Rosser Jr.~(2016b) argumenta que dentro de los sistemas económicos, el primero es más apropiado cuando las fuerzas termodinámicas ontológicas están impulsando objetivamente la dinámica de un sistema. Este último es más importante como herramienta metafórica cuando surge un patrón matemático similar.

\begin{quote}
``\ldots{} la termodinámica clásica es \ldots{} la única teoría física de contenido universal de la que estoy convencido, dentro del marco del marco de aplicabilidad de sus conceptos básicos, nunca será derrocado.'' - Albert Einstein, citado en Rifkin (1981, pag. 44)
\end{quote}

\hypertarget{observaciones-iniciales}{%
\section*{Observaciones iniciales}\label{observaciones-iniciales}}
\addcontentsline{toc}{section}{Observaciones iniciales}

Dado que al menos los primeros esfuerzos de Alan Wilson (1967, 1970), la idea de utilizar la ley de la entropía para ayudar a modelar el desarrollo de patrones estructurales espaciales urbanos y regionales ha sido influyente. Para comprender cómo se ha hecho esto y cuán útil es como enfoque, primero debemos considerar las diversas formulaciones de esa ley que se han hecho. El desarrollo completo de la idea está asociado con la segunda ley de la termodinámica debida principalmente a Boltzmann (1884), aunque basándose en trabajos anteriores de Carnot (1824) y Clausius (1867). Jaynes1957) preparó este enfoque para su aplicación en economía con Georgescu-Roegen (1971) también proporciona una perspectiva profunda. Más tarde, Shannon (1948) lo ampliaría al estudio de los patrones de información. Rosser Jr.~(2016b) sostiene que dentro de los sistemas económicos, el primero es más apropiado cuando las fuerzas termodinámicas ontológicas están impulsando objetivamente la dinámica de un sistema. Este último es más importante como herramienta metafórica cuando surge un patrón matemático similar. 1

Una forma en la que el primero puede generar patrones estructurales es a través del funcionamiento de la energía en el sistema, dado que la segunda ley de la termodinámica trata sobre cómo la energía se disipa a través de sistemas cerrados. La energía es crucial en el transporte, por lo que no debería sorprender que a medida que los costos de transporte entran en la determinación de tales patrones espaciales, podamos ver la ley de la entropía en su forma objetiva como relevante para dar forma a tales patrones y, de hecho, los costos de transporte se han considerado centrales. en la conformación de patrones espaciales urbanos y regionales que se remontan a von Thūnen (1826). Sobre la base de una propuesta de Reilly (1931) y obra de Weaver (1948). Wilson (1967, 1970, 2010) utilizaría el supuesto de minimizar los costos de transporte para modelar un sistema complejo de distribución espacial de actividades que maximizan la renta. Otro esfuerzo inicial en líneas similares se debió a Medvekov (1967).

Muchas aplicaciones de la entropía o modelos urbanos y regionales seguirían el enfoque metafórico basado en la de Shannon (1948) entropía de información. Uno de los primeros esfuerzos en este sentido se debió a Chapman (1970) para un modelo de concentración espacial o dispersión de actividades y también Batty (1976). Asimismo, esto ha apuntalado modelos de expansión urbana (Cabral et al.2013). Los índices de grados de segregación racial se han basado en tales medidas (Mora y Ruiz-Castillo,2011). Asimismo, las medidas para la diversidad del uso de la tierra se han basado en dicha entropía (Walsh y Webber,1977).

Más bien, volviendo a la formulación termodinámica fundamental, se han realizado esfuerzos para modelar la sostenibilidad ecológica de los sistemas urbanos y regionales basados \hspace{0pt}\hspace{0pt}en sus patrones de uso de energía. La evaluación de la huella de carbono se debe a Wackernagel y Rees (1996). Aplicaciones más directas, incluido el uso del concepto de exergía, se deben a Balocco, Paeschi, Grazzini y Basosi (2000). Marchinetti, Putselli y Tierzi (2006) consideran tales modelos dentro de la dinámica de sistemas complejos de los sistemas disipativos (Prigogine1980).

Una alternativa enfatiza las fuerzas antientrópicas asociadas con la aglomeración para modelar patrones de jerarquía urbana 2 que reflejan las distribuciones de la ley de poder iniciadas por Pareto (1897), con el apoyo de Singer (1936) y Gabaix (1999). Un caso especial es la regla del tamaño de rango debido a Auerbach (1913) y Zipf (1941), con el apoyo de Batten (2001), Nitsch (2005) y Berry y Okulicz-Kozaryn (2012).

Finalmente, muchos, incluidos Papageorgiou y Smith (1983), Weidlich y Haag (1987), Krugman (1996), Portugali (1999), Gabaix y Ioannides, 2004) y Rosser Jr.~(2011a). Estas interacciones pueden desencadenar las irregularidades en los caminos dinámicos que marcan los sistemas dinámicamente complejos, que seguramente son los sistemas urbanos y regionales.

La ley de la entropía, o segunda ley de la termodinámica, se convierte así en que en un sistema cerrado la entropía aumenta, lo cual fue formulado por primera vez por Clausius (1867), quien también estableció la primera ley clásica de la termodinámica de que en un sistema cerrado la cantidad de energía es constante, con esto más desarrollado por Ludwig Boltzmann (1884). La inspiración para este desarrollo provino del estudio de las máquinas de vapor por Sadi Carnot (1824). Hizo la observación crucial inicial de la primera ley, que sería crucial para comprender la imposibilidad de una máquina de movimiento perpetuo. Carnot formuló que el trabajo de una máquina de vapor provenía de la transformación de la energía térmica de una fuente más caliente a un fregadero más frío y reconoció una eficiencia máxima para esta transformación. 3 Fue de este entendimiento que Clausius derivó su conceptualización, más tarde esbozada por Boltzmann.

La variación más importante de esto sería la metafórica que mide la entropía informacional debida a Shannon (1948) y Shannon y Weaver (1949). Si bien estas dos formas de entropía se aplican a situaciones muy diferentes sin una ley ontológica de entropía que opere con respecto a la entropía informativa metafórica de Shannon, están fundamentalmente relacionadas. 4

Esta unidad fundamental se extiende a variaciones y generalizaciones posteriores del concepto de entropía desarrollado por Renyi (1961), Tsallis (1968) y Thurner y Hanel (2012). Este último se vincula con un desarrollo en Rusia de la ``nueva entropía'' que se vincula con la teoría de la ergodicidad donde la entropía se ve como un isomorfismo entre los estados de Bernoulli (Kolmogorov,1958; Sinaí,1959; Ornstein,1970).

\hypertarget{el-modelo-de-wilson}{%
\section*{El modelo de Wilson}\label{el-modelo-de-wilson}}
\addcontentsline{toc}{section}{El modelo de Wilson}

El modelador más influyente de sistemas urbanos y regionales en utilizar el concepto de entropía ha sido Sir Alan G. Wilson (1967, 1969, 1970, 2000, 2010). Su modelo principal original era la distribución espacial de los flujos de la actividad minorista, basado en un modelo de Reilly (1931). El espacio está dividido por orígenes I y destinos j (a menudo un lugar central) de modo que S ij es una matriz de flujos de dinero desde los orígenes I a los sitios de venta minorista j . Entonces, la entropía a maximizar sujeta a las restricciones presupuestarias de los flujos viene dada por

\[\operatorname{Max} S=-\sum S_{j j} \ln S_{i j}\]

donde para los beneficios de un sitio de venta minorista dados por W j y los costos de ir desde un origen a un sitio de venta minorista dados por c ij, esto dará una distribución espacial que maximiza la renta

\[S=\Sigma W_{j} \exp \left(c_{i j}\right)\]

Esto podría modificarse aún más especificando más actividades con niveles de población y tipos de puntos de venta. En principio, esto es ampliamente consistente con el original von Thūnen (1826) modelo de alquiler con patrón de anillos alrededor de un lugar central, aunque Wilson rara vez hizo hincapié en este punto.

Este modelo básico de Wilson ha pasado desde entonces por muchas modificaciones y ampliaciones, incluidas muchas del propio Wilson, a menudo con varios coautores. Por lo tanto, si bien Wilson asumió originalmente que los costos de transporte crecen linealmente con el logaritmo de beneficios, ambos pueden ser logarítmicos, lo que podría ser cierto para un modelo de viajes largos involucrados en el transporte interurbano, con otras formas funcionales posibles a medida que las restricciones se ajustan en consecuencia (Haynes y Phillips ,1987).

El modelo también se ha ampliado a otras aplicaciones. Así, Rees y Wilson (1976) y Rogers (2008) colocó esto en modelos de flujos migratorios. Straussfogel (1991) lo utilizó en estudios de suburbanización. En los modelos de flujos comerciales, las relaciones insumo-producto pueden introducirse en modelos integrados (Kim et al.1983; Roy y Flood,1992).

Mientras que el modelo básico asumió zonas discretas, Angel y Hyman (1976) maximización de la entropía extendida a representaciones espaciales continuas. Los problemas de estimación empírica surgen en relación con la agregación y la estructura espacial en modelos de interacción espacial (Batty y Skildar,mil novecientos ochenta y dos). Se han desarrollado modelos econométricos de autocorrelación espacial en este marco (Berry et al.,2008) así como formas más amplias de interacción espacial (Fischer y Griffith, 2008).

Un mayor énfasis en un enfoque metafórico de la entropía de la información de Shannon se debió a Snickars y Weibull (1977). Fotheringham (1983) aplicó esto para el caso de zonas de destino en competencia. Smith y Hsieh (1997) introdujo un equivalente de Markov. Anas1984) vincula la maximización de la utilidad y la maximización de la entropía en estos modelos mediante un modelo logit multinomial. Wilson (2010) argumenta que estos enfoques son consistentes con la interpretación de la ``complejidad desorganizada'' del enfoque de la entropía de la información de Shannon según lo propuesto por Weaver (1948). Esto contrasta con el enfoque inicial de Wilson (1967, 1970) que persiguió un enfoque de entropía basándose más en Botlzmann.

Una expansión sustancial de este marco dentro del marco de Boltzmann se debió a Harris y Wilson (1978) que introdujo la dinámica lenta en el modelo. Esto tomó la forma de introducir elementos derivados de Lotka (1925) y Volterra (1938), con Wilson (2008) etiquetando el resultado de esta combinación de Boltzmann, Lotka y Volterra como el ``enfoque BLV''. La dinámica lenta permite el crecimiento en función de la rentabilidad de las ubicaciones dadas, siendo la dinámica rápida relacionada la dinámica de ajuste de equilibrio a más corto plazo. Esta configuración proporcionó una base para considerar modelos de bifurcaciones y cascadas catastróficas (Wilson,1981; Chalado,2009) así como dinámicas caóticas (mayo, 1973; Rosser Jr.,1991). 5

Esto eventualmente conduciría a una consideración más amplia de cómo el modelo de Wilson encaja en un marco de complejidad más amplio, especialmente vinculando con Weaver (1948) distinción entre formas de complejidad organizadas y desorganizadas. Para ello, se puede considerar que la entropía proporciona un principio organizativo clave (Wilson,2006) basándose en el enfoque BLV. Esto incluso se ha propuesto para proporcionar una explicación de cómo los modelos entrópicos de flujos de nivel inferior pueden proporcionar una base para las distribuciones de la ley de energía libre de escala de las distribuciones de los tamaños del área de asentamiento (Dearden y Wilson,2009), que consideraremos a continuación como asociado a principios organizativos antientrópicos.

\hypertarget{variaciones-en-los-modelos-de-distribuciuxf3n-espacial-entruxf3pica}{%
\section*{Variaciones en los modelos de distribución espacial entrópica}\label{variaciones-en-los-modelos-de-distribuciuxf3n-espacial-entruxf3pica}}
\addcontentsline{toc}{section}{Variaciones en los modelos de distribución espacial entrópica}

Si bien el trabajo de Wilson inspiró un gran esfuerzo por parte de muchas personas, como se vio en la sección anterior, otros también usaron varias medidas entrópicas para estudiar distribuciones espaciales en sistemas urbanos y regionales de varias cosas. Una línea de investigación se inspiró en la aplicación de Theil (1972), que se basa en la medida de entropía de información de Shannon. Entre los primeros en hacerlo estaba Batty (1974). La versión espacial básica del índice de Theil donde H es el índice, n es el número de zonas y p i es la probabilidad de que la variable x aparezca en la zona I , está dada por

\[H_{n}=\left[\Sigma p_{i} \log \left(1 / p_{i}\right)\right] / \log n\]

Esta medida de entropía puede variar de 0 a 1, indicando este último una distribución completamente igual en las zonas espaciales, en la máxima entropía, y 0 indicando una concentración total en una zona, o un grado máximo de desigualdad y antientropía. Este índice se ha aplicado ampliamente en muchas ciencias sociales y naturales.

Batty's (1974) La variación de esta, a la que llamó entropía espacial , implica considerar lo que sucede a medida que se reduce el tamaño de las zonas, lo que también implica un número creciente de ellas. Si Δ x i es el tamaño de la zona, entonces el índice de entropía espacial Batty viene dado por

\[H=\left(\lim \Delta x_{i} \rightarrow 0\right)--\Sigma p_{i} \log \left(p_{i} / \Delta x_{i}\right)\]

Esta formulación es muy similar a la propuesta por Bailey (1990) para medir la entropía social , centrándose nuevamente en los grados de similitud o igualdad entre grupos o zonas sociales.

Entre las aplicaciones más directas de esto para los sistemas urbanos se encuentra el estudio de la expansión urbana descontrolada (Cabral et al.~2013). Una línea ha sido medir el grado de fragmentación de la propiedad. Miceli y Sirmans (2007) argumenta que esto desalienta el desarrollo, ya que los promotores inmobiliarios prefieren patrones de propiedad menos dispersos. Los patrones dispersos asociados con la expansión urbana conducen a una forma de poder monopólico que se manifiesta a través del problema de la resistencia. En términos más generales, se considera que la expansión urbana descontrolada contribuye a una variedad de problemas sociales y ambientales, con mayores costos de infraestructura e incluso mayores problemas de salud pública (Brueckner,2000; Nechyba y Walsh,2004; Frenkel y Ashkenazi,2007).

Si bien la mayoría de los observadores ven la expansión urbana descontrolada como un problema importante, tiene sus defensores. Así, Wassmer (2008) argumenta que la expansión descontrolada aumenta la satisfacción con la vivienda y las escuelas, reduce las tasas de criminalidad y aumenta la conveniencia de viajar en automóvil, aunque este último es un objetivo de quienes argumentan que la expansión descontrolada agrava los problemas ambientales. Cabral y col.~(2013) ven esto como una cuestión de compensaciones. Los niveles más altos de entropía espacial exigen transporte e infraestructura, mientras que los niveles más bajos aumentan los niveles de desigualdad y fragmentación socioeconómica.

Como era de esperar, se han utilizado medidas de entropía de la información para medir los grados de segregación racial en áreas urbanas tanto para residencias como para escuelas (Mora y Ruiz-Castillo, 2011). Si bien probablemente la medida más utilizada en estos estudios es el índice de Theil que se muestra en la Ec. ( 5.3 ) anterior y propuesto por primera vez para estudiar la segregación escolar por Theil y Finizza (1971), con aplicaciones como el estudio de la segregación en el área de la Bahía de San Francisco (Miller y Quigley, 1990). Sin embargo, Mora y Ruiz-Castillo defienden la superioridad de la forma desnormalizada de este conocido como índice de información mutua, también debido a Theil (1971), que puede ser más útil para estudiar la descomponibilidad en las escuelas.

Sin embargo, otras aplicaciones espaciales incluyen la medición de la diversidad de patrones de uso de la tierra (Walsh y Webber, 1977) y distribuciones de asentamientos espaciales (Medvekov, 1967) así como los patrones espaciales de distribución de la población (Chapman, 1970). Purvis y col.~(2019) proporcionan una descripción general de muchas de estas aplicaciones.

\hypertarget{sistemas-urbanos-regionales-termodinuxe1micamente-sostenibles}{%
\section*{Sistemas urbanos / regionales termodinámicamente sostenibles}\label{sistemas-urbanos-regionales-termodinuxe1micamente-sostenibles}}
\addcontentsline{toc}{section}{Sistemas urbanos / regionales termodinámicamente sostenibles}

La mayoría de los modelos discutidos en las dos secciones anteriores se han basado en el concepto de información metafórica de entropía proveniente de Shannon y Weaver, con la posible excepción del desarrollo de Wilson de la dinámica lenta que se basa más directamente en Boltzmann. Sin embargo, otra vertiente del análisis entrópico de los sistemas urbanos y regionales se basa más en el enfoque ontológico original en el que un sistema urbano o regional se ve impulsado por la termodinámica en su sentido físico original que implica transferencias de energía y transformaciones siguiendo la Segunda Ley de la Termodinámica. Entre los que persiguen este enfoque se encuentran Rees (1992), Balocco et al.~(2004), Zhang y col.~(2006), Marchinetti, Pulselli y Tierzi (2006) y Purvis et al.~(2019).

El enfoque de la mayor parte de esta investigación está particularmente en la sustentabilidad ecológica de los sistemas urbanos y regionales, viéndolos como sistemas disipativos abiertos que experimentan entradas y salidas de energía y materiales (Georgescu-Roegen, 1971; Prigogine,1980). Mientras que para los sistemas cerrados la entropía aumenta, con los sistemas abiertos la entropía puede aumentar o disminuir si la energía y los materiales fluyen hacia el sistema. Este fue de hecho el Schrōdinger (1945) argumento sobre la vida, que implica un proceso antientrópico mediante el cual los seres vivos extraen energía y crean orden y estructura mientras viven. Un término específico para anti-entropía es exergía (Rant,1956).

Distingamos entonces tres conceptos: entropía total o S total , entropía interior o S i , y entropía exterior o S o . Estos se relacionan dinámicamente de acuerdo con

\[\mathrm{d} S_{\text {total }} / \mathrm{d} t=\mathrm{d} S_{i} / \mathrm{d} t+\mathrm{d} S_{0} / \mathrm{d} t, \text { with } \mathrm{d} S_{i} / \mathrm{d} t>0\]

Sin embargo, d S o / d t puede ser positivo o negativo, por lo que si es negativo y tiene un valor absoluto que excede el valor absoluto que excede al de S i , entonces la entropía total puede disminuir a medida que el sistema genera orden a medida que se dibuja. en energía y materiales, solo para exportarlos como desperdicio y desorden, con la entropía aumentando fuera del sistema. Como Wackernagel y Rees (1996) lo expresaron, ``Las ciudades son agujeros negros entrópicos'', lo que plantea serias dudas sobre su sostenibilidad, ya que generan grandes huellas ecológicas.

La exergía se define a menudo como la cantidad máxima de trabajo útil posible para alcanzar un estado de entropía máxima, lo que significa que debe ser cero si se logra un estado de entropía máxima. Rant's (1956) la formulación original estaba en el contexto de la ingeniería química. Si B es exergía, U es energía interna, P es presión, V es volumen, T es temperatura, S es entropía, μ i es el potencial químico del componente i, y N i son los moles del componente i , entonces la formulación de Rant es dada por

\[B=U+P V--T S-\Sigma \mu_{i} N_{i}\]

Esto implica, ceteris paribus, que

\[\mathrm{d} B / \mathrm{d} t \leq 0 \leftrightarrow \mathrm{d} S / \mathrm{d} t \geq 0\]

que destaca la interpretación de la exergía como anti-entropía. 6

Una aplicación de esto usando una modificación de la ecuación de Rant debido a Moran y Sciubba (1994) ha sido realizado por Balocco et al.~(2004). Estudian la exergía involucrada en la construcción de edificios y la depreciación real en la ciudad de Castelnuovo Beardenga cerca de Siena, Italia. Esto también implica el uso de relaciones input-output involucradas con la industria de la construcción. Llegan a la conclusión de que los edificios más recientes no son tan eficientes como los más antiguos, y que los construidos en 1946-1960 proporcionan la mayor sostenibilidad.

Siguiendo a Wackernagel y Rees, así como a Balocco, Paeschi, Grazzini y Basosi, y también a Haken (1988) y Svirizhev (2000), Zhang y col.~(2006) participan en un ambicioso esfuerzo para aplicar conceptos de entropía al estudio del desarrollo sostenible de Ningbo, China, una ciudad de casi 6 millones de habitantes algo al sur de Shanghai en la provincia de Zhejiang. Su esfuerzo combina tanto medidas ontológicas de entropía como de información metafórica, ya que dividen su análisis en cuatro partes. Los dos primeros están vinculados al desarrollo y mantienen la entropía de entrada y la energía de salida impuesta , que están básicamente determinadas por la producción. Los dos segundos se consideran parte del metabolismo del sistema urbano, el metabolismo regenerativo y el metabolismo destructivo . los cuales están ligados a la generación de contaminación y su saneamiento. Esto se convierte en una medida de armonía con el medio ambiente. El resultado del primero da el grado de desarrollo mientras que el segundo da el grado de armonía. Ellos estiman estos para el período 1996-2003 y encuentran que estas dos medidas generalmente iban en direcciones opuestas, con el grado de desarrollo aumentando (asociado con la disminución de la entropía) a medida que disminuía el grado de armonía (asociado con el aumento de la entropía). Esto plantea el problema de la sostenibilidad del desarrollo urbano en China de manera bastante aguda.

Marchinetti, Pulselli y Tierzi (2006) 7 considere este enfoque desde un nivel más general, basándose en ideas debidas a Morin (1995) con respecto a la autonomía versus la dependencia de los sistemas en su entorno, mientras se utiliza el enfoque de estructuras disipativas de los sistemas abiertos asociados con Prigogine (1980). Ven que los sistemas urbanos evolucionan entre los extremos de la autarquía y la globalización. Sin embargo, argumentan que al final ninguno de estos extremos es sostenible.En su defensa de un camino equilibrado, enfatizan cómo los sistemas urbanos y regionales son ecosistemas que operan sobre la base de los flujos de energía (Odum,1969) dentro de un conjunto de totalidades complejas que surgen de un conjunto de componentes de micro-nivel que interactúan (Ulanowicz, 2012).

\hypertarget{procesos-anti-entruxf3picos-en-sistemas-urbanos-regionales}{%
\section*{Procesos anti-entrópicos en sistemas urbanos / regionales}\label{procesos-anti-entruxf3picos-en-sistemas-urbanos-regionales}}
\addcontentsline{toc}{section}{Procesos anti-entrópicos en sistemas urbanos / regionales}

Empujar contra esta versión entrópica de la estructura de los sistemas urbanos y regionales es una versión de ley de poder de dicha estructuración, al menos para ciertos casos y situaciones. Podría decirse que esto se trata en el marco de la entropía, dada la cuestión del equilibrio entre exergía y entropía en los sistemas urbanos y regionales. La mayoría de los sistemas y medidas hasta ahora han involucrado esencialmente relaciones internas o distribuciones dentro de sistemas urbanos o regionales. Pero cuando se consideran sistemas de distribución de nivel superior, la relación de entropía puede romperse o incluso volverse completamente irrelevante.

Una forma en que las fuerzas antientrópicas pueden manifestarse es mediante la aparición de distribuciones de la ley de potencia (Rosser Jr., 2016b), con evidencia sustancial de que el tamaño de las ciudades puede seguir tales distribuciones (Gabaix, 1999). Pareto (1897) identificó el concepto de distribuciones de la ley de potencias. Para P es población, r es rango y A y α son constantes, entonces

\[r P_{r}^{\alpha}=A_{i}\]

que se puede poner en forma logarítmica, que es lineal,

\[\ln r=\ln A-\alpha\left(\ln P_{r}\right)\]

Observamos que para el caso especial de α = 1, la población de la entidad de rango r se convierte en

\[P_{r}=P_{1} / r_{i}\]

La cual fue etiquetada como la regla del tamaño de rango por Auerbach (1913) y más tarde llegaría a ser conocida como la Ley de Zipf, que se argumenta que es válida para muchas distribuciones (Zipf, 1941). 8

La cuestión de si las distribuciones del tamaño de la ciudad siguen o no la ley de Zipf y, por lo tanto, obedecen la regla del tamaño del rango, ha sido un tema de debate continuo desde Auerbach (1913) lo propuso por primera vez y Lotka (1925) lo cuestionó. Algunos, especialmente los geógrafos urbanos (Berry y Okulicz-Kozaryn,2012) han argumentado que se trata de una ley universal. Otros, más a menudo economistas, lo han cuestionado, argumentando que no hay una razón clara por la que deba seguirse, incluso si los tamaños de las ciudades pueden exhibir distribuciones de la ley de potencias (Batten,2001; Fujita y col.1999), aunque Gabaix (1999) argumenta que la Ley de Zipf surge en el límite si la Ley de Gibrat sostiene que las tasas de crecimiento son independientes del tamaño de las ciudades.

Listón2001) en particular muestra las distribuciones del tamaño de las ciudades de EE. UU. que muestran distribuciones de la ley de energía desde 1790 hasta el presente, incluso si no es exactamente la regla del tamaño de rango (con el hecho de que Los Ángeles es sustancialmente más grande que la mitad del tamaño de Nueva York, un ejemplo de por qué podría no retener). Nitsch2005) llevaron a cabo un metaestudio de estudios empíricos pasados, observando una amplia gama de hallazgos en los estudios, pero al mirarlos en conjunto encontraron una media de α = 1.08, bastante cercana al valor de Zipf. Berry y Okulicz-Kozaryn (2012) argumentan que las variaciones en las estimaciones se deben a que no se utilizan medidas coherentes de las regiones urbanas en los estudios, y si las medidas más grandes de este tipo se utilizan en las megalópolis, entonces la ley de Zipf y la regla del tamaño del rango se cumplen plenamente. En cualquier caso, ya sea que lo haga o no, la evidencia es fuerte de que las distribuciones del tamaño de las ciudades están distribuidas por ley de poder, lo que muestra un dominio de las fuerzas antientrópicas para esta parte de los sistemas urbanos y regionales.

Una posible base para estos procesos antientrópicos que pueden generar resultados distributivos de la ley del poder son las economías de escala, conocidas desde hace mucho tiempo por ser una base también de complejidad económica (Arthur, 1994). Los sistemas urbanos en particular pueden exhibir hasta tres tipos diferentes de economías de escala: economías internas a nivel de empresa (Marshall,1879), economías de localización que implican la aglomeración externa entre empresas de una misma industria (Marshall, 1919) y economías de urbanización que involucran economías de aglomeración externa que se extienden a través de industrias (Hoover y Vernon, 1959).

Papageorgiou y Smith (1983) y Weidlich y Haag (1987). Sin embargo, desde entonces, estos modelos han sido reemplazados por los de ``nueva geografía económica'' que enfatizan las economías de escala que surgen dentro de la competencia monopolística, según lo analizado por Dixit y Stiglitz (1977). Mientras que Fujita (1988) comenzó a usar esto para modelar sistemas urbanos y regionales, Krugman (1991) enfoque recibió la mayor atención e influencia (Rosser Jr., 2011a).

\hypertarget{complejidad-entropuxeda-y-autoorganizaciuxf3n-de-sistemas-urbanos-regionales}{%
\section*{Complejidad, entropía y autoorganización de sistemas urbanos / regionales}\label{complejidad-entropuxeda-y-autoorganizaciuxf3n-de-sistemas-urbanos-regionales}}
\addcontentsline{toc}{section}{Complejidad, entropía y autoorganización de sistemas urbanos / regionales}

Esto nos lleva a darnos cuenta de que la interacción entre las fuerzas entrópicas y antientrópicas dentro de los sistemas urbanos y regionales puede generar una complejidad que subyace en el surgimiento de patrones estructurales de orden superior a través de la autoorganización, ya que los puntos de bifurcación se encuentran dentro de la dinámica no lineal de los sistemas que conducen a transformaciones estructurales morfogenéticas (Rosser Jr., 1990, 1991; Krugman,1996; Portugali,1999). Esto puede verse observando cómo operan estos sistemas desde la perspectiva de la complejidad dinámica, qué día (1994) definido como sistemas que no convergen endógenamente en un estado estacionario o crecimiento exponencial. Se sabe que tal complejidad toma cuatro formas: cibernética, teoría de catástrofes, teoría del caos y complejidad basada en agentes (Rosser Jr.,1999). Se puede ver que todas estas formas han operado dentro de los sistemas urbanos y regionales.

El modelo más importante de dinámica urbana basado en una cibernética se debió a Forrester (1961) en su Urban Dynamics , aunque etiquetó su enfoque como parte de la teoría de la dinámica de sistemas . Esto implicó un conjunto de ecuaciones en diferencias no lineales con interconexiones complicadas entre sí que implican efectos de retroalimentación positivos y negativos. Cuando se simuló, exhibió roturas estructurales y cambios repentinos en ciertos puntos, con el sistema demasiado complicado para descubrirlos mediante análisis, en lugar de requerir simulación.

Mucho más difundidos han sido los estudios de cambios estructurales en sistemas espaciales urbanos y regionales y más generales utilizando la teoría de catástrofes. Amson1974) inició el uso de la teoría de la catástrofe en tales sistemas, examinando los determinantes de la renta y la ``opulencia'' (atractivo) de la densidad urbana utilizando un modelo de catástrofe de cúspide. Mees1975) modeló el renacimiento de las ciudades en la Europa medieval como una catástrofe de mariposas. Wilson (1976) modeló la elección modal de transporte como una catástrofe doble, y basándose en el modelo minorista entrópico, Poston y Wilson (1977) lo hizo por el tamaño del centro comercial. 9 Isard (1977) inició el estudio de los efectos de la aglomeración provocando el surgimiento repentino de ciudades en modelos que equilibran las áreas urbanas y rurales utilizando la catástrofe de la cúspide, con Casetti (1980) y Dendrinos (1980) siguiente. Dendrinos (1978, 1979) utilizaron modelos de catástrofes de orden superior para estudiar la dinámica industrial-residencial y la formación de barrios marginales en las ciudades. Puu (1979, 1981) lo hizo también para estudiar los cambios estructurales en los patrones comerciales regionales. Nijkamp y Reggiani (1988) mostró cómo un modelo de control óptimo de interacción espacial dinámica no lineal puede generar una interpretación teórica de catástrofes.

La aplicación de la teoría del caos al estudio de dinámicas urbanas y regionales complejas fue iniciada por Beaumont, Clarke y Wilson (1981) para la dinámica residencial y comercial intraurbana, de nuevo basándose en el modelo intraurbano entrópico. Blanco (1985) combinó este modelo con ideas de sinergética (Haken, 1983) 10 para mostrar la autoorganización que surge de fluctuaciones caóticas cerca de puntos de bifurcación. Una serie de artículos y libros enfatizaron la migración interregional o la dinámica de la población más general (Rogerson,1985; Day et al.1987; Dendrinos,mil novecientos ochenta y dos; Dendrinos y Sonis,1990). Otra área de estudio fue la dinámica caótica en los modelos de ciclo económico interregional (Puu,1989, 1990). También se han realizado estudios de dinámicas caóticas en versiones extendidas de los nuevos modelos de geografía económica centro-periferia basados \hspace{0pt}\hspace{0pt}en la competencia monopolística (Currie y Kubin,2006; Commendatore y col.2007).

Finalmente, resulta que el inicio mismo de los modelos de complejidad basados \hspace{0pt}\hspace{0pt}en agentes surgió de los esfuerzos por modelar el surgimiento de la segregación racial en las ciudades por Schelling (1971, 1978). Estos cambios se pueden medir mediante métodos entrópicos. Curiosamente, Schelling no usó modelos analíticos ni simulación por computadora, sino que jugó un juego en un tablero Go de 19 por 19 con piedras blancas y negras, simplemente asumiendo pequeñas diferencias locales en los deseos de vivir al lado de personas como una o no. Un comienzo de integración de alta entropía termina con la aparición de un patrón segregado de baja entropía. El modelo de Schelling se ha estudiado desde entonces en muchas variaciones y contextos y se ha encontrado que es muy robusto. Zhang (2004) lo consideró como un juego evolutivo sobre un toro de celosía, mientras que Fagiolo et al.~(2007) como modelo de red. Dichos modelos tienen una similitud con los modelos cibernéticos, excepto que se basan más claramente en generar una autoorganización de orden superior que surge de agentes de bajo nivel que interactúan entre sí de acuerdo con efectos estrictamente locales, un enfoque de complejidad fundamental.

\hypertarget{observaciones-adicionales-1}{%
\section*{Observaciones adicionales}\label{observaciones-adicionales-1}}
\addcontentsline{toc}{section}{Observaciones adicionales}

Es completamente natural que tanto la entropía como la complejidad estén profundamente involucradas en la dinámica y las estructuras espaciales de los sistemas urbanos y regionales. La naturaleza espacial de tales sistemas los abre a que los efectos de vecindad local sean muy importantes, lo cual es fundamental para las visiones avanzadas de la complejidad y la ubicuidad de los efectos de aglomeración externos subyacen a las no linealidades que, además, conducen a complejidades dinámicas de diversos tipos, incluidas las discontinuidades catastróficas y la dinámica caótica. .

Como sistemas abiertos, la complejidad se ve reforzada por la naturaleza disipativa de los sistemas urbanos y regionales. Están sujetos a la competencia entre fuerzas entrópicas y antientrópicas que interactúan para estimular dinámicas complejas. Este es especialmente el caso de la termodinámica ontológica de los sistemas urbanos y regionales que operan como ecosistemas.

Sin embargo, las medidas de entropía metafórica basadas en la entropía de la información de Shannon han demostrado ser útiles para comprender y modelar una variedad de aspectos de los sistemas urbanos y regionales. Esto incluye tanto patrones espaciales como estructuras sociológicas como la segregación racial, que también se ha encontrado que exhiben dinámicas complejas como en el modelo de Schelling. Pocas áreas de la economía o las ciencias sociales más amplias exhiben tantos ejemplos de dinámicas complejas reforzadas por fuerzas entrópicas como los sistemas urbanos y regionales.

Esta interacción requiere una nueva visión del mundo. Como Jeremy Rifkin (1981, pag. 256) dice: ``Al final, nuestro presente individual descansa para siempre en el alma colectiva del proceso de desarrollo mismo. Conservar lo mejor que podamos la dotación fija que nos quedó, y respetar lo mejor que podamos el ritmo natural que gobierna el proceso de devenir, es expresar nuestro amor último por toda la vida que nos precedió y toda la vida que vendrá después.''

\hypertarget{notas-al-pie-2}{%
\section*{Notas al pie}\label{notas-al-pie-2}}
\addcontentsline{toc}{section}{Notas al pie}

\begin{enumerate}
\def\labelenumi{\arabic{enumi}.}
\tightlist
\item
  Purvis y col.~(2019) abogan por un tercer tipo de entropía, ``figurativa'', que sugiere un creciente desorden o aleatoriedad. Sin embargo, aquí esta forma se considerará subsumida en las otras dos, especialmente en el primer Samuelson (1972) ofrece una crítica de algunos usos de la entropía en modelos económicos, así como Kovalev (2016).
\item
  El modelado formal de la aglomeración en sistemas urbanos se debe a Fujita (1988) y Krugman (1991) basado en Dixit y Stiglitz (1977). Véase también Fujita et al.~(1999).
\item
  Durante mucho tiempo fue difícil encontrar una copia del libro de Carnot y durante mucho tiempo tuvo poca influencia directa en el desarrollo de las máquinas de vapor, aunque finalmente se actuó sobre su implicación de que tener una mayor diferencia de temperatura entre la fuente y el fregadero podría aumentar la eficiencia de tales máquinas. por personas como Joseph Diesel en el desarrollo de máquinas de vapor mejoradas (Georgescu-Roegen, 1971).
\end{enumerate}

4 .
La distinción entre entropía ontológica y metafórica se debe a Rosser Jr.~(2016b) y discutido en el capítulo anterior. Lotka (1922) argumentó que la evolución está impulsada fundamentalmente por un proceso termodinámico ontológico basado en la ley de la entropía. Brooks y col.~(1989) ven la entropía de información metafórica como útil para comprender la evolución biológica.

5 .
Para sistemas alternativos que ofrecen posibilidades similares, consulte Allen y Sanglier (1979) y Nijkamp y Reggiani (1988), con Rosser Jr.~(2011a) proporcionando una amplia descripción.

6 .
A veces también se conoce como negentropía , por ``entropía negativa''.

7 .
Irónicamente, Marchettini y los coautores están en el mismo instituto en la Universidad de Siena que Balocco y los coautores, pero ninguno de los grupos cita el trabajo del otro.

8 .
Gabaix y Ioannides (2004) argumentan que Kuznets (1955) primero proporcionó una forma formal de estimar las distribuciones de la ley de potencia para tamaños urbanos.

9 .
Wilson (1981) proporciona una descripción general temprana de muchos de estos modelos. Dendrinos y Rosser Jr.~(1992) muestran cuántos están vinculados.

10 .
Las extensiones de esto a los modelos sinérgicos fractales de autoorganización de jerarquías urbanas se deben a Fotheringham et al.~(1989) y Rosser Jr.~(1994).

\hypertarget{part-sistemas-ecoluxf3gicos-complejos}{%
\part{Sistemas ecológicos complejos}\label{part-sistemas-ecoluxf3gicos-complejos}}

\hypertarget{los-sistemas-ecoluxf3gico-econuxf3micos-complejos-y-sus-problemas-de-gobernanza}{%
\chapter{Los sistemas ecológico-económicos complejos y sus problemas de gobernanza}\label{los-sistemas-ecoluxf3gico-econuxf3micos-complejos-y-sus-problemas-de-gobernanza}}

La difunta Elinor Ostrom fue la persona que vio con mayor claridad el supuesto dilema llamado la ``tragedia de los comunes'' (Hardin 1968; Ostrom 1990). Se argumentó ampliamente que administrar los recursos de propiedad común era una propuesta imposible, que o la propiedad común se privatiza de alguna manera o de lo contrario habrá una tendencia inevitable a que el recurso se sobreexplote, posiblemente para completar la destrucción o el agotamiento. Tales resultados fueron vistos como resultados inevitables de los juegos de dilemas de los prisioneros donde los agentes que utilizan recursos de propiedad común no cooperarán entre sí y, en cambio, buscarán obtener la mayor cantidad de recursos para sí mismos lo antes posible. Sin embargo, ella entendió desde el principio de su trabajo (Ostrom 1976) que la gente busca trabajar en arreglos para administrar los recursos de propiedad común. A medida que estudiaba este fenómeno a lo largo del tiempo, se dio cuenta de que diferentes grupos persiguen diferentes soluciones. Esto la llevó a plantear el concepto de policentrismo y la importancia de la diversidad institucional en todo el mundo, en función de las circunstancias y culturas locales (Ostrom 2005, 2012).

\hypertarget{introducciuxf3n-ostrom-complejidad-y-gobernanza}{%
\section*{Introducción: Ostrom, complejidad y gobernanza}\label{introducciuxf3n-ostrom-complejidad-y-gobernanza}}
\addcontentsline{toc}{section}{Introducción: Ostrom, complejidad y gobernanza}

La difunta Elinor Ostrom fue la persona que vio con mayor claridad el supuesto dilema llamado la ``tragedia de los comunes'' (Hardin 1968; Ostrom1990). Se argumentó ampliamente que administrar los recursos de propiedad común era una propuesta imposible, que o la propiedad común se privatiza de alguna manera o de lo contrario habrá una tendencia inevitable a que el recurso se sobreexplote, posiblemente para completar la destrucción o el agotamiento. Tales resultados fueron vistos como resultados inevitables de los juegos de dilemas de los prisioneros donde los agentes que utilizan recursos de propiedad común no cooperarán entre sí y, en cambio, buscarán obtener la mayor cantidad de recursos para sí mismos lo antes posible. Sin embargo, ella entendió desde el principio de su trabajo (Ostrom1976) que las personas buscan elaborar arreglos para administrar los recursos de propiedad común. A medida que estudiaba este fenómeno a lo largo del tiempo, se dio cuenta de que diferentes grupos persiguen diferentes soluciones. Esto la llevó a plantear el concepto de policentrismo y la importancia de la diversidad institucional en todo el mundo, con base en las circunstancias y culturas locales (Ostrom2005, 2012).

También con el tiempo llegó a comprender que el desafío de administrar los recursos de propiedad común se vuelve más difícil cuando el sistema de gobernanza se convierte inevitablemente en parte de un complejo sistema ecológico-económico (Ostrom 2010a, B). De hecho, a menudo es la intervención humana en un sistema natural lo que introduce la complejidad en el sistema, el sistema ecológico-económico. Esta complejidad inducida hace que quienes la gestionan sean más responsables de lo que hacen.

\hypertarget{dinuxe1mica-pesquera-compleja}{%
\section*{Dinámica pesquera compleja}\label{dinuxe1mica-pesquera-compleja}}
\addcontentsline{toc}{section}{Dinámica pesquera compleja}

La clásica tragedia de los bienes comunes para la pesca fue planteada por primera vez por Gordon (1954), quienes lo identificaron incorrectamente como un problema de propiedad común, al tiempo que identificaron la sobreexplotación ineficiente que puede ocurrir en una pesquería de acceso abierto. Sin embargo, incluso cuando se gestiona de manera eficiente, las pesquerías pueden exhibir una dinámica compleja, particularmente cuando las tasas de descuento son suficientemente altas. Así como las especies pueden extinguirse bajo una gestión óptima cuando los agentes no valoran suficientemente las poblaciones futuras de la especie, lo mismo ocurre en las pesquerías, dado que las poblaciones futuras de peces se valoran cada vez menos, la gestión de la pesquería puede llegar a parecerse a una pesquería de acceso abierto. De hecho, en el límite, a medida que la tasa de descuento llega al infinito, momento en el que el futuro se valora en cero, la ordenación de la pesquería converge con la del caso de acceso abierto. Pero mucho antes de que se alcance ese límite,

Ahora presentaremos un modelo general basado en la optimización intertemporal para ver cómo pueden surgir estos resultados a medida que varían las tasas de descuento, siguiendo a Hommes y Barkley Rosser Jr (2001). 1 Comenzaremos considerando estados estables óptimos donde la cantidad de peces capturados es igual a la tasa de crecimiento natural de los peces dada por Schaeffer (1957) función de rendimiento.

\[h(x)=f(x)=r x(1-x / k)\]

donde las respectivas variables son las mismas que en el Cap. 2 : x es la biomasa de los peces, h es la captura, f ( x ) es la función de rendimiento biológico, r es la tasa natural de crecimiento de la población de peces sin restricciones de capacidad, yk es la capacidad de carga de la pesquería, el cantidad máxima de peces que pueden vivir en él en situación de no captura, que es también el equilibrio bionómico a largo plazo de la pesquería.

Especificamos más completamente el lado humano del sistema al introducir un coeficiente de capturabilidad, q , junto con el esfuerzo, E , para dar que la cosecha en estado estable, Y , también está dada por

\[h(x)=q E x=Y\]

Continuamos asumiendo un costo marginal constante, c , de modo que el costo total, C está dado por

\[C(E)=c E\]

Con p el precio del pescado, esto conduce a una renta, R , es decir

\[R(Y)=p q E x-C(E)\]

Hasta ahora esto ha sido un ejercicio estático, pero ahora pongamos esto más directamente en el marco de optimización intertemporal, asumiendo que la tasa de descuento en el tiempo es δ . Todas las ecuaciones anteriores ahora estarán indexadas en el tiempo por t, y también debemos permitir al menos en principio resultados de estado no estacionario. Por lo tanto

\[d x / d t=f(x)--h(x)\]

con h ( x ) ahora dado por ( 6.2 ) y no necesariamente igual af ( x ). Dejando que los costos unitarios de cosecha en diferentes momentos estén dados por c {[} x ( t ){]}, que será igual a c / qx , y con una constante δ \textgreater{} 0, el problema de control óptimo sobre h ( t ) mientras se sustituye en ( 6.5 ) se convierte en

\[\max \int_{0}^{\infty} e^{-\delta t}(p-c[x(t)])(f(x)--d x / d t) d t\]

sujeto ax ( t ) ≥ 0 y h ( t ) ≥ 0, observando que h ( t ) = f ( x ) - dx / dt en ( 6.6 ). La aplicación de las condiciones de Euler ′ da

\[f(x) / d t=\delta=\left[c^{\prime}(x) f(x)\right] /[p-c(x)]\]

A partir de esto, la curva óptima de oferta de pescado con descuento vendrá dada por

\[x(p, \delta)=k / 4\left\{1+(c / p q k)-(\delta / r)+\left[(1+(c / p q k)-(\delta / r))^{2}+(8 c \delta / p q k r)\right]^{1 / 2}\right\}\]

Todo este sistema se muestra en la Fig. 6.1 (Rosser Jr2001b, pag. 27) como el modelo de pesquería de Gordon-Schaefer-Clark.

\textbf{Figura 6.1} Modelo de pesquería Gordon-Schaefer-Clark

El aspecto más dramático de este modelo es la curva de oferta que se dobla hacia atrás que surge, con Copes (1970) siendo el primero en explicar esta posibilidad para la pesca, fuertemente apoyado por Clark (1990). Se puede ver que un aumento gradual de la demanda en esta situación puede provocar un aumento repentino del precio y un colapso catastrófico de la producción.

Observamos que cuando δ = 0, la curva de oferta en el cuadrante superior derecho de la figura 6.1 no se dobla hacia atrás. Más bien se acercará asintóticamente a la línea vertical que viene del punto de rendimiento máximo sostenido en el punto más alejado a la derecha de la curva de rendimiento en el cuadrante inferior derecho. A medida que aumente δ, esta curva de oferta comenzará a doblarse hacia atrás y, de hecho, lo hará muy por debajo de δ = 2\%. La curva hacia atrás continuará haciéndose más extrema hasta que en δ = ∞ la curva de oferta convergerá en la curva de oferta de acceso abierto de

\[x(p, \infty)=(r c / p q)(1-c / p q k) .\]

Debe quedar claro que la posibilidad de colapsos catastróficos aumentará a medida que esta curva de oferta se doble más hacia atrás y aumente la posibilidad de equilibrios múltiples, de modo que un aumento suave de la demanda puede conducir a un aumento catastrófico del precio y al colapso de la cantidad. Entonces, incluso si las personas se comportan de manera óptima, a medida que se vuelven más miopes, las posibilidades de resultados catastróficos aumentarán.

En cuanto a la naturaleza de la dinámica óptima, Hommes y Rosser Jr (2001) muestran que para las zonas en las que hay equilibrios múltiples en el caso de la curva de oferta que se dobla hacia atrás, hay aproximadamente tres zonas en términos de la naturaleza de los resultados óptimos. Con tasas de descuento suficientemente bajas, el resultado óptimo será simplemente el precio más bajo / la cantidad más alta de los dos resultados de equilibrio estable. A un nivel mucho más alto, el resultado óptimo será simplemente el precio más alto / la cantidad más baja de los dos equilibrios estables. Sin embargo, para las zonas intermedias, el resultado óptimo puede implicar un patrón complejo de rebote entre los dos equilibrios, con la posibilidad de que este patrón sea matemáticamente caótico. 2

Para estudiar su sistema, Hommes y Rosser Jr (2001) suponen una curva de demanda de la forma

\[D(p(t))=A-B p(t)\]

con la curva de oferta dada por ( 6.8 ). La compensación del mercado viene dada por

\[p(t)=[A-S(p(t), \delta] / B\]

Esto se puede convertir en un modelo de dinámica de ajuste de telaraña indexando la p en la función de oferta para que esté un período detrás de la p que se está determinando, con Chiarella (1988) y Matsumoto (1997) que muestra una dinámica caótica en modelos de telaraña generalizados.

Basándose en datos de Clark (1985, págs.25, 45, 48), Hommes y Barkley Rosser Jr (2001) asumió los siguientes valores para los parámetros: A = 5241, B = 0.28, r = 0.05, c = 5000, k = 400,000, q = 0.000014 (con el número para A proveniente de A = kr / ( c - c 2 / qk )). Para estos valores, encontraron que cuando δ subió de cero al principio, la solución era un equilibrio de precios bajos, pero comenzando alrededor de δ = 2\% comenzaron a aparecer bifurcaciones que duplicaban el período, con una dinámica caótica en toda regla que aparecía alrededor de δ = 8\%. Cuando δ subió por encima del 10\% aproximadamente, el sistema pasó al equilibrio de precios altos.

\hypertarget{problemas-de-complejidad-de-la-rotaciuxf3n-uxf3ptima-en-los-bosques}{%
\section*{Problemas de complejidad de la rotación óptima en los bosques}\label{problemas-de-complejidad-de-la-rotaciuxf3n-uxf3ptima-en-los-bosques}}
\addcontentsline{toc}{section}{Problemas de complejidad de la rotación óptima en los bosques}

Algunas complejidades de la dinámica forestal se conocen desde hace mucho tiempo en relación con la dinámica de la picea-gusano de las yemas (Ludwig, Jones y Holling, 1978). 3 Para llegar a tipos de dinámicas relacionadas que surgen de patrones inesperados de beneficios forestales, así como a cuestiones de gestión tales como cómo lidiar con los incendios forestales y el tamaño de las parcelas, así como la cuestión básica de cuándo los bosques deben cortarse de manera óptima, necesitamos para desarrollar un modelo básico (Rosser Jr2005). Comenzaremos con el tipo de modelo más simple en el que el único beneficio de un bosque es la madera que se va a cortar y consideraremos el comportamiento óptimo de un propietario forestal que maximiza las ganancias en tales condiciones.

Irving Fisher (1907) consideró lo que ahora llamamos el problema de la ``rotación óptima'' de cuándo cortar un bosque como parte de su desarrollo de la teoría del capital. Postulando tasas de interés reales positivas, argumentó que sería óptimo cortar el bosque (o un árbol, para ser más precisos) cuando su tasa de crecimiento es igual a la tasa de interés real, la tasa de crecimiento de los árboles que tiende a desacelerarse con el tiempo. Esto fue sencillo: siempre que un árbol crezca más rápidamente que el nivel de la tasa de interés, uno puede aumentar su riqueza más dejando que el árbol crezca. Una vez que se establece que su tasa de crecimiento descienda por debajo de la tasa de interés real, se puede ganar más dinero cortando el árbol y colocando las ganancias de la venta de su madera en un bono que genera la tasa de interés real. Este argumento dominó el pensamiento en la tradición del idioma inglés durante más de media década,1952) y Gaffney (1957).

Sin embargo, como lo argumentó elocuentemente Samuelson (1976), Fisher estaba equivocado. O, para ser más precisos, solo tenía razón en un caso bastante extraño y poco interesante, a saber, aquel en el que el propietario del bosque no replanta un nuevo árbol para reemplazar al viejo, sino que, en efecto, simplemente abandona el bosque y no hace nada con él ( o quizás se lo vende a otra persona). Ciertamente, esta no es la solución al problema de rotación óptima en el que el propietario del bosque tiene la intención de replantar y luego cortar y replantar y cortar y así sucesivamente en el futuro infinito. Curiosamente, la solución a este problema había sido resuelta en 1849 por un forestal alemán, Martin Faustmann (1849), aunque su solución permanecería desconocida en inglés hasta que su obra fuera traducida más de un siglo después.

La solución de Faustmann implica cortar antes que en el caso de Fisher, porque se pueden conseguir árboles más jóvenes de crecimiento más rápido y si se cortan antes, lo que aumenta el valor actual del bosque en comparación con un período de rotación basado en la tala cuando Fisher recomendó.

Sea p el precio de la madera, que se supone constante, 4 f ( t ) la función de crecimiento de la biomasa del árbol a lo largo del tiempo, T el período de rotación óptimo, r la tasa de interés real yc el costo de cortando el árbol, la solución de Fisher viene dada por

\[p f^{\prime}(T)=r p f(T)\]

que al eliminar el precio de ambos lados se puede reducir a

\[f^{\prime}(T)=r f(T)\]

que tiene la interpretación ya dada: corte cuando la tasa de crecimiento sea igual a la tasa de interés real.

Faustmann resolvió esto considerando una suma infinita de ganancias descontadas de los futuros rendimientos descontados de la cosecha y encontró que esto se reducía a

\[p f^{\prime}(T)=r p f(T)+r\left[(p f(T)-c) /\left(e^{r T}--1\right)\right]\]

lo que implica una T menor que en el caso de Fisher debido al término extra en el lado derecho, que es positivo y dado que f ( t ) es cóncava. Hartman (1976) generalizaron esto para permitir que los valores de amenidad no maderables del árbol (o parcela de bosque de árboles de la misma edad se corten simultáneamente), 5 asumiendo que esos valores de amenidad se pueden caracterizar por g ( t ) dado por

\[p f^{\prime}(T)=r p f(T)+r\left[(p f(T)--c) /\left(e^{r T}--1\right)\right]--g(T)\]

Un ejemplo de un valor comercial no maderero que se puede asociar con un bosque de propiedad privada podría ser el pastoreo de animales, que tiende a alcanzar un máximo temprano en la vida de un parche de bosque cuando los árboles aún son jóvenes y bastante pequeños. Swallow y col.~(1990) estimaron los valores de amenidad del pastoreo de ganado en el oeste de Montana para alcanzar un máximo de 12.5 años, con la función dada por

\[g(t)=\beta_{0} \exp \left(-\beta_{1} t\right)\]

Rosser Jr (2005) mostró que este caso alcanzó un máximo global a los 76 años, un poco más largo que los 73 años de la solución de Faustmann, pero de hecho exhibe múltiples óptimos locales, reflejando no linealidades en estas dinámicas forestales (Rosser Jr.~2005; 2011a, B, 2013; Vincent y Potts2005). 6

Con mayor frecuencia, esta función g ( t ) involucra asuntos que no se apropian tan fácilmente por un propietario privado, en resumen, externalidades. Algunos propietarios de bosques del gobierno intentan incorporarlos en los esfuerzos de planificación, y esto es algo que hace mucho tiempo el Servicio Forestal de los Estados Unidos, que utiliza audiencias públicas para medir el sentimiento público con respecto a los usos alternativos de la tierra en su planificación para los bosques nacionales (Johnson et al.1980; Bowes y Krutilla1985). Entre ellas se encuentran la caza y la pesca, que a veces tanto los propietarios públicos como los privados pueden obtener algunos pagos de los usuarios, aunque sea por bosques públicos de forma más indirecta a través de licencias de caza y pesca.

Son más difíciles de captar los problemas de biodiversidad más amplios, especialmente los relacionados con especies en peligro de extinción (Perrings et al.~1995). Este ha sido un tema difícil en muchas naciones en desarrollo, donde se han establecido sistemas para tratar de proporcionar beneficios económicos a las poblaciones locales para la preservación de tales especies, y en algunas naciones el ecoturismo es un método para esto. Esto se vuelve más difícil en situaciones donde los derechos aborígenes han sido violados en el pasado (Kant2000; Gramo2001). 7

La ecuestre de carbono es una externalidad de los bosques que recibe más atención, y la tala menos frecuente tiende a ayudar a esto (Alig et al.~1998), especialmente dado que la extracción de madera estándar a menudo implica la quema de maleza y ramas no utilizadas, sin mencionar que la extracción de madera también puede aumentar la erosión del suelo y las inundaciones (Plantinga y Wu, 2003). Pero los árboles más jóvenes pueden absorber más CO 2 y reemplazar una especie por otras también puede mejorar esto (Alavalapati et al.2002). Todo esto también puede interactuar con los esfuerzos de biodiversidad de diversas maneras (Caparrǿs y Jacquemont2003).

Un buen ejemplo de estas complejidades se ha estudiado para el Bosque Nacional George Washington en Virginia y Virginia Occidental basándose en información en su proceso de planificación (FORPLAN, Johnson et al.~1980). Allí se encuentran múltiples máximos relacionados con la caza vinculados a ciervos que alcanzan un pico 8 años después de una tala rasa, con pavos salvajes y urogallos alcanzando un máximo alrededor de 25 años después de una tala (y este también el máximo para la diversidad vegetativa), y los osos alcanzan un máximo después de aproximadamente 60 años, con esto estableciendo conflictos sobre la tala más frecuente en algunas partes del bosque para complacer a los cazadores de ciervos y mucho menos o incluso ninguna tala en otras partes para favorecer a los cazadores de osos, ambos poderosos grupos de interés presionando a los tomadores de decisiones para ese bosque (Rosser Jr2005, 2011a, 2013).

Si un bosque no es estrictamente de subsistencia y, por lo tanto, tiene al menos un producto vendido en un mercado, entonces, para un área de tierra fija, un bosque puede tener una curva de oferta a largo plazo que se inclina hacia atrás para ese producto, particularmente si es madera. Las observaciones empíricas apoyan la posible existencia de tales situaciones, incluido un estudio de las ventas de madera de los pequeños agricultores en el borde de la selva tropical del Amazonas (Amacher et al.2009). Encontraron elasticidades de oferta de madera fuertemente negativas y estadísticamente significativas en su muestra para parcelas con tenencia segura, aunque para aquellas con tenencia insegura la curva se inclina hacia arriba. Los autores ofrecen pocos argumentos sobre por qué debería ocurrir este resultado, en parte porque están más preocupados por otros temas como el papel del crédito y la presencia o no de la carretera Transamazónica. La poca explicación que brindan invoca el modelo de la curva de oferta del trabajo individual que se inclina hacia atrás en lugar del de la pesca. ``El efecto del precio de la madera se deriva del hecho de que el pequeño productor puede tener objetivos de ingresos predeterminados que las ventas de madera deben ayudar a cumplir'' (Amacher et al.2009, pag. 1796).

Tal como están las cosas, en el pasado se han desarrollado modelos teóricos de la posibilidad de que las curvas de oferta de madera se inclinen hacia atrás, inspirados en particular en el trabajo de Colin Clark sobre tales curvas para la pesca. El primero en hacerlo fue Hyde (1980). Aún más fuertemente inspirado por Clark (1985, 1990), Binkley (1993) desarrolló un modelo formal basado en el modelo Faustmann, 8 presenta también evidencia tentativa en apoyo de la misma desde el suministro a largo plazo de pinos taeda en el sureste de Estados Unidos. Huelga decir que estos casos abren la posibilidad del tipo de dinámica compleja ya discutida para el caso de la pesca.

Utilizando las variables ya definidas, presentamos el modelo de Binkley a continuación, agregando π ( t ) para el valor actual neto de la corriente futura de ingresos de madera, que el propietario del bosque buscará maximizar. En contraste con nuestra discusión anterior, se permitirá que el precio cambie, aunque evitaremos usar la teoría de opciones. Este bosque puede contener árboles o rodales de distintas edades. En un año determinado, algún árbol o rodal alcanzará la edad de rotación óptima, T , y será cosechado. La oferta se expresará en términos de superficie unitaria.

El propietario del bosque busca maximizar

\[\pi(t)=-c+p f(t) e^{-r t}+\pi(t) e^{-r t}\]

La condición de primer orden para resolver esto es encontrar dπ / dt = 0, que viene dado por

\[f^{\prime}(t) /[f(t)--c / p]=r /\left(1--e^{-r t}\right)\]

Esto implica una relación de oferta a largo plazo entre el precio y la edad de rotación óptima, T , dada por

\[S(p)=f(T(p)) / T(p)\]

De esto se obtiene una curva de oferta no monótona en función de T que va de cero a cero a medida que T aumenta, con un rendimiento máximo sostenido (RMS) en un valor intermedio de T dado por

\[1 / T=f^{\prime}(T) / f(T)\]

A partir de esto, es posible derivar la relación entre el precio y la edad de rotación óptima, T , que aparece en ( 6.20 ) dada por

\[p=c /\left\{f(T)--f^{\prime}(t)\left[\left(1-e^{-r t}\right) / r\right]\right\} .\]

Esto se resume en la figura 6.2 . 9

\textbf{Figura 6.2} La curva de oferta de madera que se dobla hacia atrás

Hay paralelos con la curva de oferta de pescado que se inclina hacia atrás presentada anteriormente, pero también algunas diferencias. Para ambos es crucial la suposición de una capacidad de carga máxima. Ambos tienen efectivamente solo tres cifras, con un cuadrante solo una línea de 45 grados, entre la edad de rotación para el bosque y la biomasa de peces para la pesquería. Ambos tienen una función no monótona que se encuentra detrás de la curva hacia atrás de la curva de oferta, la función de rendimiento de Schaefer de la cosecha en estado estacionario y la biomasa de peces para la pesquería y entre la edad de rotación y el suministro de madera para el bosque. En ambos, el punto de suministro máximo está asociado con el punto de RMS.

En ambos, la parte de pendiente positiva de la curva de oferta está asociada con la parte ``exterior'' de la función de rendimiento relevante más allá del punto RMS. Para la pesquería hay muchos peces allí, fácilmente capturados a bajos precios. Para el bosque, estos son los períodos de rotación más largos cuando los árboles son más grandes. En el otro lado del RMS está la parte de la curva de oferta que se dobla hacia atrás. Para la pesquería hay pocos peces, por lo que su captura es costosa. Para el bosque, esto se asocia con un período de rotación mucho más corto en el que los árboles son pequeños cuando se cortan, por lo que producen menos madera con el tiempo.

Binkley resume así la situación en su conclusión (1993, p.~178):

\begin{quote}
``Los altos precios de la madera en pie implican no solo que la producción del bosque tiene un alto valor, sino también que el capital en forma de existencias en formación tiene un alto costo de oportunidad. A precios altos, es óptimo conservar en el uso de capital y, por lo tanto, reducir el inventario de existencias en crecimiento al reducir la edad de rotación''.
\end{quote}

\hypertarget{complejidades-de-los-sistemas-de-economuxeda-climuxe1tica}{%
\section*{Complejidades de los sistemas de economía climática}\label{complejidades-de-los-sistemas-de-economuxeda-climuxe1tica}}
\addcontentsline{toc}{section}{Complejidades de los sistemas de economía climática}

Se ha argumentado durante mucho tiempo que los sistemas climáticos por sí mismos son caóticos, con Lorenz (1963) planteando el efecto mariposa inicialmente específicamente en relación con el modelado del clima, y \hspace{0pt}\hspace{0pt}siendo esta una de las principales razones por las que es difícil hacer predicciones meteorológicas más allá de unos pocos días para una ubicación específica. Sin embargo, incluso si el clima en sí mismo no es caótico y la economía en sí misma no es caótica, un sistema acoplado de los dos bien puede serlo (Rosser Jr2002, 2020d).

En particular, Chen (1997) ha demostrado cómo puede surgir un sistema de este tipo. Asume un modelo económico de dos sectores con agricultura y manufactura que está cerrado por una función de utilidad CES para un agente homogéneo y con el trabajo como único insumo económico. Hay una interacción bidireccional con el clima, basándose en un modelo de Henderson-Sellers y McGuffie (1987). El clima más cálido reduce la producción agrícola, mientras que el aumento de la fabricación calienta el clima debido a la contaminación. Bajo ciertos valores de parámetros de este modelo, surgen dinámicas caóticas, aunque ninguno de los sistemas por sí mismo es caótico.

Rosser Jr (2020d) considera además un modelo que amplía un análisis de los atractores de llamaradas en sistemas económicos, inicialmente utilizados para estudiar reacciones autocatalíticas como las llamaradas en la química física (Rōssler y Hartmann1995). Podría decirse que esto es parte de la econoquímica no tan desarrollada . Las matemáticas subyacentes derivan de los atractores de Milnor (Milnor1985) que son continuos pero no diferenciables en ninguna parte y exhiben ``cuencas acribilladas''. Rōssler1976) utilizó este enfoque para desarrollar su atractor caótico continuo y luego lo extendió en Rōssler et al.~(1995). Hartmann y Rössler (1998) aplicaron este modelo a las actividades empresariales y Rosser Jr.~et al.~(2003a) lo aplicó para examinar la volatilidad de los precios de los activos. Rosser Jr (2020d) aplicó además esto a un sistema climático-económico acoplado que puede proporcionar el tipo de resultados climáticos kurtóticos estudiados por Weitzman (2009, 2011, 2012, 2014) y Rosser Jr (2011a).

En este modelo la parte económica deriva de un modelo de Day (mil novecientos ochenta y dos) que es un modelo de crecimiento de Solow modificado que enfrenta límites a la expansión de capital, posiblemente debido a límites ambientales. Esto lo prepara para una formulación logística que se asemeja al modelo de May (1976) conocido por generar dinámicas caóticas. Este modelo económico se plantea luego en una configuración regional con aportes interactivos al clima que pueden conducir a ``erupciones'' kurtóticas. El modelo económico básico tiene un exponente de trabajo de α , un exponente de capital de β , y es el producto per cápita, k es la relación capital-trabajo, la tasa de crecimiento de la población es λ y m es el ``coeficiente de congestión de capital''. Esta función de producción modificada es

\[F(k)=\beta k^{\beta}(m-k)^{y}\]

Suponiendo una tasa de ahorro constante, el coeficiente de capital implica la siguiente ecuación de crecimiento de diferencias:

\[K_{t+1}=\alpha \beta k_{t}^{\beta}\left(m-k_{t}\right)^{y} /(1+\lambda)\]

Después de mayo (1976), Rosser Jr.~et al.~(2003a) valores asumidos que garantizan una dinámica caótica asumiendo una participación de capital constante, dada por

\[A \beta /(1+\lambda)=3.99=k_{t+1} /\left(1-k_{t}\right)\]

A diferencia de las formulaciones anteriores, las entidades heterogéneas son ubicaciones más que agentes. Están impulsados \hspace{0pt}\hspace{0pt}por una función de reacción B , con parámetros b y un valor crítico de k que es a , más allá del cual habrá un aumento sustancial de temperatura, una ``llamarada''. Un estallido completo depende de un número suficiente de ubicaciones que pasan su valor crítico, con 1\textgreater{} a \textgreater{} 0. Con c \textgreater{} 0 y la ubicación del tipo l de n , s es la demanda general, la forma general de esta función de reacción está dada por :

\[B_{t+1}^{l}=b^{i}{ }_{t}+b_{t}^{l}\left(a^{l}-k_{t}^{l}\right)--c b^{(i) 2}{ }_{t}+c s_{t}\]

En este sistema, el primer término es un componente autorregresivo; el segundo es el término de cambio; el tercero proporciona un componente estabilizador, mientras que el cuarto es el elemento desestabilizador proveniente de la acumulación de tendencias anteriores, con la demanda general dada por:

\[S_{t+1}=b^{1}{ }_{t}+b_{t}^{2}+\cdots+b^{n}{ }_{t}\]

En Rosser Jr.~et al.~(2003a) asumiendo ciertos valores de estos parámetros permiten una simulación que proporciona una secuencia de resultados que exhiben estallidos kurtóticos dispersos consistentes con el escenario de Weitzman para el calentamiento global.

\hypertarget{estabilidad-resiliencia-complejidad-de-los-ecosistemas-revisados-y-poluxedticas}{%
\section*{Estabilidad, resiliencia, complejidad de los ecosistemas revisados \hspace{0pt}\hspace{0pt}y políticas}\label{estabilidad-resiliencia-complejidad-de-los-ecosistemas-revisados-y-poluxedticas}}
\addcontentsline{toc}{section}{Estabilidad, resiliencia, complejidad de los ecosistemas revisados \hspace{0pt}\hspace{0pt}y políticas}

Se ha argumentado que existe una relación entre la diversidad de un ecosistema y su estabilidad, aunque más tarde se descubrió que esto no era cierto en general, y de hecho existen argumentos matemáticos que sugieren exactamente lo contrario. 1973). Luego, algunos sugirieron que la relación aparente entre diversidad y estabilidad en la naturaleza era al revés, que la estabilidad permitía la diversidad. En términos más generales, se argumentó que no existe una relación general, con los detalles de las relaciones dentro de un ecosistema proporcionando la clave para comprender la naturaleza de la estabilidad del sistema, aunque ciertamente la disminución de la biodiversidad es un problema amplio con muchos aspectos (Perrings et al.~.1995).

De esta discusión surgió la fructífera visión de CS Holling (1973) de una profunda relación negativa entre estabilidad y resiliencia. Esta relación puede plantearse como un conflicto entre la estabilidad local y global: que una mayor estabilidad local puede comprarse en cierto sentido a costa de una menor estabilidad o resiliencia global. La palmera no es estable localmente ya que se dobla fácilmente con el viento en comparación con el roble. Sin embargo, a medida que el viento se fortalece, la flexión de la palmera le permite sobrevivir, mientras que el roble se vuelve más susceptible a romperse y no sobrevivir. Incluso se puede argumentar que tal relación se traslada a la economía como en la comparación clásica del capitalismo de mercado y el socialismo de mando. El capitalismo de mercado sufre inestabilidades de precios y macroeconomía, mientras que los precios planificados y los niveles de producción del socialismo de mando estabilizan el nivel de precios, la producción y el empleo.

Este reconocimiento de que los ecosistemas involucran patrones dinámicos y no permanecen fijos en el tiempo, lideró Holling (1992) para ampliar su idea para considerar más ampliamente el papel de tales patrones en el mantenimiento de la resiliencia de tales sistemas, y también para considerar cómo las relaciones entre los patrones variarían a lo largo del tiempo y el espacio dentro de los sistemas jerárquicos (Holling y Gunderson 2002; Holling y col.2002; Gunderson y col.2002a, B). Esto resultó en lo que ha llegado a llamarse el diagrama de los ``ocho perezosos'' de Holling, que se muestra en la figura 6.3 (Holling y Gunderson2002, pag. 34) y muestra una imagen estilizada del paso de un ecosistema típico a través de cuatro funciones básicas a lo largo del tiempo.

\textbf{Figura 6.3} Ciclo de las cuatro funciones del ecosistema

Se puede pensar que esto representa un patrón típico de sucesión ecológica en una parcela de tierra en particular. 10 La ecología convencional se centra en las zonas r y K , correspondientes a adaptadores r y adaptadores K. Entonces, si un ecosistema se ha derrumbado (como en el caso de un bosque después de un incendio total), comienza a tener poblaciones dentro de él creciendo nuevamente desde cero, haciéndolo a un ritmo r durante la fase de explotación. A medida que se llena, se mueve hacia la Ketapa, en la que alcanza la capacidad de carga y entra en la fase de conservación, aunque como se señaló anteriormente, la sucesión puede ocurrir en esta etapa ya que el conjunto preciso de plantas y animales puede cambiar en esta etapa. Luego viene la liberación cuando el sistema sobreconectado ahora se vuelve bajo en resiliencia colapsa en una liberación de biomasa y energía en la etapa Ω, que Gunderson y Holling identifican con la ``destrucción creativa'' de Schumpeter (1950). Finalmente, el sistema entra en la etapa α de reorganización mientras se prepara para permitir la reacumulación de energía y biomasa. En esta etapa, el suelo y otros factores fundamentales se preparan para el regreso a la etapa r , aunque esta es una etapa de crucial importancia ya que es posible que el ecosistema cambie sustancialmente a una forma diferente, dependiendo de cómo se modifique el suelo y qué especies entran en él, con un ejemplo del cambio de pasto búfalo y grama a arbustos de serpientes de cascabel y plantas rodadoras en el suroeste de los EE. UU., una posibilidad como la describe Leopold (1933)

Este patrón básico se puede ver ocurriendo en múltiples escalas de tiempo y espacio dentro de un paisaje más amplio como un conjunto de ciclos anidados (Holling 1986, 1992). Un ejemplo dibujado en el bosque boreal y que también representa los ciclos atmosféricos relevantes se muestra en la Fig. 6.4 (Holling et al.2002, pag. 68). Uno puede pensar en términos del bosque de cada uno de los niveles operando de acuerdo con su propio patrón de ``ocho perezosos'' como se describió anteriormente. Tal patrón se llama panarquía .

\textbf{Figura 6.4} Escalas de tiempo y espacio del bosque boreal y la atmósfera

Cada vez más, los formuladores de políticas llegan a comprender que es la resiliencia más que la estabilidad per se lo que es importante para la sostenibilidad a largo plazo de un sistema. Ante choques exógenos y amenaza de extinción de especies (Solé y Bascompte2006), se deben hacer esfuerzos especiales para abordar las cosas con destreza. Costanza y col.~(1999) proponen siete principios para el caso de la ordenación oceánica: responsabilidad, correspondencia de escalas, precaución, ordenación adaptativa, asignación de costos y participación plena. De estos, Rosser Jr (2001b) sugiere que los más importantes son los Principios de precaución y coincidencia de escalas, con Wilson et al.~(1999) enfatizando especialmente la percepción de la escala y el problema de concordancia como profundamente crucial.

El emparejamiento de escalas significa que los formuladores de políticas operan en el nivel apropiado de la jerarquía del sistema ecológico-económico. Siguiendo a Ostrom (1990) y Bromley (1991), así como Rosser Jr (1995) y Rosser Jr.~y Rosser (2006), la idea es alinear los derechos de propiedad y de control en el nivel apropiado de la jerarquía. Manejar una pesquería a un nivel demasiado alto puede conducir a la destrucción de especies de peces a un nivel más bajo (Wilson et al.1999).

Suponiendo que se haya logrado un ajuste de escala apropiado, y que se haya establecido un sistema operativo de derechos de propiedad y control, el objetivo de administrar para mantener la resiliencia bien puede implicar proporcionar suficiente flexibilidad para que el sistema pueda hacer que sus fluctuaciones locales ocurran sin interferencia mientras se mantienen los límites y límites más amplios que evitan que el sistema colapse. En la difícil situación de la pesca, esto puede implicar el establecimiento de reservas (Lauck et al.1998; Grafton y col.2009 o sistema de uso rotativo (Valderarama y Anderson 2007). Para lograrlo con éxito, es fundamental que el grupo que administra el recurso pueda controlarse y observarse a sí mismo (Sethi y Somanathan1996), siendo ese auto-refuerzo la clave del éxito en la gestión de las pesquerías como en el caso de las bandas de langostas de Maine (Acheson 1988) y las pesquerías de Islandia (Durrenberger y Palsson 1987). No hace falta decir que todo esto es más fácil de decir que de hacer, especialmente en el caso de las pesquerías, donde los grupos locales relevantes son a menudo muy distintos socialmente y de otra manera de los que los rodean y, por lo tanto, tienden a sospechar de los forasteros que intentan que se organicen. ellos mismos para hacer lo que sea necesario (Charles1988).

Los derechos de propiedad y los derechos de control pueden no coincidir (von Ciriacy-Wantrup y Bishop 1975), siendo el control de acceso la clave para gobernar los bienes comunes. Sin control de acceso, los derechos de propiedad son irrelevantes. El trabajo de Ostrom y otros deja en claro que los derechos de propiedad pueden adoptar una variedad de formas. Si bien estos esfuerzos alternativos a menudo tienen éxito, a veces no lo tienen, como demuestra el fracaso de un esfuerzo inicial para establecer derechos de propiedad en la pesquería de salmón de Columbia Británica (Millerd2007). Algunos recursos de propiedad común se han gestionado con éxito durante siglos, como en el caso de los pastos comunes alpinos suizos (Netting1976), cuya existencia ha refutado durante mucho tiempo la versión simple de la ``tragedia de los comunes'' tal como la plantea Garrett Hardin (1968).

Los problemas de política se vuelven más difíciles cuando los diferentes niveles de jerarquía son importantes en la dinámica de un sistema ecológico-económico, especialmente cuando las dinámicas complejas no lineales operan en estos importantes niveles múltiples. Es posible que las políticas deban implementarse en diferentes niveles, pero con coherencia entre sí para que sean eficaces. Este problema se vuelve probablemente más claro al volver a considerar el problema del clima global, que de hecho va desde lo casi minuciosamente local hasta lo completamente global.

Una complicación adicional debido a las complejidades asociadas especialmente con la dinámica caótica es que cuando un sistema se descompone del nivel global al regional o local, puede estar sujeto a efectos severos debido a la dependencia sensible de las condiciones iniciales. Así, Massetti y Lorenzo (2019) han considerado en detalle los pronósticos a nivel regional a partir de simulaciones de modelos climáticos a nivel global utilizando el IPCC de las Naciones Unidas para proyectar posibles resultados climáticos futuros. En particular, ejecutaron simulaciones que variaban ligeramente los valores iniciales iniciales para ciertas variables y, de hecho, encontraron una dependencia sensible sustancial para las predicciones a nivel regional. Por lo tanto, para la parte centro-oeste de los Estados Unidos, algunas proyecciones tendrían un calentamiento sustancial, mientras que otras realmente encontraron que se estaba enfriando, incluso cuando la temperatura promedio global mostró un calentamiento, nuevamente para los valores iniciales solo ligeramente separados. Esto replica el resultado anterior para los modelos climáticos encontrados por Lorenz (1963). No hace falta decir que esto complicó seriamente saber qué hacer a niveles más locales para tales situaciones.

Estas complejidades de múltiples capas implican profundas incertidumbres sobre todos los asuntos mencionados anteriormente y más. Estos incluyen debates en curso sobre cuestiones científicas subyacentes, así como la naturaleza completa de las interacciones entre los aspectos económicos y climatológicos. Los elementos de esto involucran dinámicas caóticas sujetas a una dependencia sensible de las condiciones iniciales, lo que hace que todo el asunto sea mucho más difícil de entender. Todo esto conduce a la incapacidad de cualquier observador o agente para saber de forma fiable cómo funciona el sistema con todo detalle de forma fiable. Esto implica que sería prudente involucrar reglas generales heurísticas basadas en la racionalidad limitada como partes cruciales de la política en situaciones tan complejas (Rosser Jr.~y Rosser2015).

\hypertarget{notas-al-pie-3}{%
\section*{Notas al pie}\label{notas-al-pie-3}}
\addcontentsline{toc}{section}{Notas al pie}

\begin{enumerate}
\def\labelenumi{\arabic{enumi}.}
\tightlist
\item
  Para más información, consulte Rosser Jr (2001b, 2011a, Cap. 9) y Foroni et al.~(2003).
\item
  Esto está por debajo del rango en el que surgen dinámicas caóticas en los modelos de crecimiento de la regla de oro (Nishimura y Yano 1996). Aparecen dinámicas caóticas en el modelo no optimizador de una pesquería de fletán con una curva de oferta que se dobla hacia atrás (Conklin y Kohlberg1994). Doveri y col.~(1993) mostró esto para ecosistemas acuáticos de especies múltiples más generalizados. Zimmer (1999) argumentó que es más probable que aparezcan ciclos caóticos en los laboratorios debido al ruido en ambientes naturales, mientras que Allen et al.~(1993) argumentan que la dinámica caótica en un entorno ruidoso puede ayudar a una especie a sobrevivir.
\item
  Véase también Holling (1965) para un presagio de este argumento. Para enlaces más amplios, Holling (1986) argumentó que estos sistemas de abeto-gusano en los bosques canadienses podrían verse afectados por ``sorpresas locales'' o pequeños eventos en lugares distantes, como el drenaje de pantanos cruciales en el medio oeste de los EE. UU. en las rutas migratorias de las aves que se alimentan de los gusanos de las yemas.
\item
  Esta es una suposición no trivial, con una gran literatura existente sobre el uso de la teoría de opciones para resolver los tiempos de parada óptimos cuando el precio es un proceso estocástico (Reed y Clarke 1990). Flecha y Fisher (1974) sugirió en primer lugar el uso de la teoría de opciones para hacer frente a la pérdida posiblemente irreversible de valores forestales futuros inciertos.
\end{enumerate}

5 .
Un modelo más general basado en Ramsey (1928) La optimización intertemporal que resuelve el perfil óptimo de un bosque fue iniciada por Mitra y Wan Jr.~(1986), Este enfoque tomó en serio la invocación de Ramsey de una rata de descuento cero en la que la gestión de casos converge en la solución de rendimiento máximo sostenido, con Khan y Piazza (2011) estudiando esto desde el punto de vista de la teoría clásica de la autopista de peaje.

6 .
La existencia de estos equilibrios múltiples abre la posibilidad de paradojas de la teoría del capital a medida que varía la tasa de interés real (Rosser Jr 2011b). Prince y Rosser Jr (1985) estudiaron las implicaciones de esto para el análisis de costo-beneficio, y esto potencialmente se aplica al caso del Bosque Nacional George Washington que se analiza en este documento a continuación. Ver Asheim (2008) para una aplicación al caso de la energía nucleoeléctrica.

7 .
Para discusiones más detalladas sobre los problemas especiales de la deforestación tropical y los derechos de los pueblos indígenas, ver Barbier (2001); Kahn y Rivas (2009).

8 .
Yin y Newman (1999) confirmó el modelo básico, aunque también mostró que las curvas de oferta agregada que tienen en cuenta la tierra variable tendrán pendiente ascendente.

9 .
Las variables de la figura son las utilizadas por Binkley, que se traducen en este artículo como v = f , t = T y l = r .

10 .
Observamos aquí la definición que se utiliza a menudo de un ``ecosistema'' como un conjunto de ciclos biogeoquímicos interrelacionados impulsados \hspace{0pt}\hspace{0pt}por la energía. En términos de escala, estos pueden variar desde una sola célula hasta toda la biosfera. Por lo tanto, tenemos un conjunto de ecosistemas anidados que pueden operar en varios niveles de agregación.

\hypertarget{complejidad-y-futuro-de-la-economuxeda}{%
\chapter*{Complejidad y futuro de la economía}\label{complejidad-y-futuro-de-la-economuxeda}}
\addcontentsline{toc}{chapter}{Complejidad y futuro de la economía}

La era neoclásica de la economía ha terminado. Sobre la base de los puntos de vista presentados en este libro, creo que se puede argumentar que ha sido reemplazado por la era de la complejidad . Esta nueva era no ha llegado a través de una revolución. En cambio, ha evolucionado a partir de las muchas corrientes del trabajo neoclásico, junto con el trabajo realizado por economistas heterodoxos y menos ortodoxos. Es la ola del futuro.

\hypertarget{la-evoluciuxf3n-de-la-economuxeda}{%
\section*{La evolución de la economía}\label{la-evoluciuxf3n-de-la-economuxeda}}
\addcontentsline{toc}{section}{La evolución de la economía}

La era neoclásica de la economía ha terminado. Sobre la base de los puntos de vista presentados en este libro, creo que se puede argumentar que ha sido reemplazado por la era de la complejidad 1 . Esta nueva era no ha llegado a través de una revolución. En cambio, ha evolucionado a partir de las muchas corrientes del trabajo neoclásico, junto con el trabajo realizado por economistas heterodoxos y menos ortodoxos. Es la ola del futuro.

Imagine por un momento que uno estuviera mirando la profesión económica en Inglaterra en 1890. Se diría que Alfred Marshall, con su mezcla de economía histórica y analítica, era la economía del futuro; Walras y Edgeworth, quienes adoptaron un enfoque más matemático, serían considerados jugadores menores. Ahora, avanzando rápidamente a la década de 1930, Marshall es visto como un actor secundario, mientras que el enfoque matemático de Walras y Edgeworth se ha convertido en la base de la economía de vanguardia de Samuelson (aunque Marshall ha continuado siendo citado de alguna manera desde entonces). Ahora imagine la economía en 2050. Mucho de lo que se hace actualmente en economía no será citado ni siquiera considerado importante. Algunas partes de la economía, que hoy se consideran menores, serán vistas como las precursoras de lo que se convertirá la economía.

El objetivo de esta comparación es dejar en claro que para juzgar la relevancia de las contribuciones económicas uno debe mirar hacia el futuro. Uno debe tener una visión de lo que será la economía en el futuro y juzgar la investigación en consecuencia. Las métricas de citas y publicaciones de revistas actuales no hacen eso; tienen un sesgo de statu quo porque miran hacia atrás y, por lo tanto, animan a los investigadores a continuar con los métodos y enfoques de investigación del pasado, en lugar de desarrollar enfoques del futuro. Son útiles, obviamente, porque muestran la actividad actual, pero son solo una parte de la imagen. Los artículos que puntean las i y las t cruzadas, incluso los que se citan con relativa frecuencia a corto plazo, son mucho menos importantes que los artículos que apuntan en nuevas direcciones.

Cualquier evaluación de la literatura debe basarse en un juicio sobre la dirección futura de la economía. Si no lo hace, está, por defecto, aceptando el juicio de que continuará el enfoque actual en la profesión. Pero para el futuro de la economía, habrá más aceptación de que la economía es compleja y la profesión, con el tiempo, adoptará ciertos tipos de herramientas técnicas, matemáticas, analíticas y estadísticas para hacer frente a esa complejidad. Los modelos basados \hspace{0pt}\hspace{0pt}en supuestos a priori disminuirán y serán reemplazados por modelos y supuestos impulsados \hspace{0pt}\hspace{0pt}empíricamente. La economía del comportamiento se expandirá; los experimentos se convertirán en parte del conjunto de herramientas del economista, al igual que las herramientas técnicas complejas como el análisis de conglomerados, las ultramétricas y el análisis dimensional. Esta creciente complejidad irá acompañada de una división del trabajo: los teóricos y estadísticos se volverán cada vez más especializados, pero serán complementados por economistas que tienen una visión amplia de hacia dónde se dirige la economía y están capacitados en su aplicación. La economía dejará de intentar responder a grandes preguntas, como si se prefiere el mercado al mando y control, o si el mercado es eficiente, y responderá a preguntas más pequeñas, como qué estructura del mercado logrará los fines que los responsables de la formulación de políticas están tratando de lograr.

Podría decirse que el término ``complejidad'' se ha usado en exceso y se ha exagerado, por lo que esta visión no es de una teoría de gran complejidad que reúne todo. Es una visión que considera la economía tan complicada que los modelos analíticos simples de la economía agregada (modelos que pueden especificarse en un conjunto de ecuaciones que se pueden resolver analíticamente) probablemente no serán útiles para comprender muchos de los problemas que los economistas quieren abordar. . Por lo tanto, la visión neoclásica walrasiana de un conjunto de ecuaciones solubles que capturan las interrelaciones completas de la economía que se pueden utilizar para la planificación y el análisis no va a funcionar. En cambio, el análisis debe basarse en datos experimentales y empíricos. A partir de ahí construimos, utilizando todas las herramientas analíticas que tengamos disponibles. Esto es diferente de la vieja visión en la que los economistas en su mayoría hicieron lo contrario: comenzando desde arriba con grandes teorías matemáticas de tipo axiomático Bourbakista, y luego trabajando hacia abajo.

La visión de la complejidad no solo conecta los diversos hilos de investigación que serán el futuro de la economía; también proporciona la mejor manera de ver la profesión económica en sí misma: la profesión económica como un sistema complejo en evolución que tiene fuerzas en competencia operando en todo momento. Es una profesión que solo puede entenderse como un sistema en constante cambio y cambio.

\hypertarget{muxe1s-sobre-la-naturaleza-de-la-complejidad}{%
\section*{Más sobre la naturaleza de la complejidad}\label{muxe1s-sobre-la-naturaleza-de-la-complejidad}}
\addcontentsline{toc}{section}{Más sobre la naturaleza de la complejidad}

Adoptar una visión de la complejidad no requiere elegir entre las muchas definiciones específicas de complejidad. Sin embargo, una definición general útil de un sistema complejo proviene de Herbert Simon (1962, pag. 267):

\begin{quote}
A grandes rasgos, por sistema complejo me refiero a uno formado por una gran cantidad de partes que interactúan de manera no simple. En tales sistemas, el todo es más que la suma de las partes, no en un sentido metafísico último, sino en el importante sentido pragmático de que, dadas las propiedades de las partes y las leyes de su interacción, no es un asunto trivial inferir las propiedades del todo. Frente a la complejidad, un reduccionista en principio puede ser al mismo tiempo un holista pragmático.
\end{quote}

Simon luego pasa a enfatizar cómo esta definición conduce a un enfoque en la estructura jerárquica de los sistemas y enfatiza que se basa en literatura más antigua, particularmente en la teoría general de sistemas (von Bertalanffy 1974), que considera que incluye el trabajo del economista Kenneth Boulding (1978) con la cibernética (Wiener 1948) y teoría de la información (Shannon y Weaver 1949). De estos, la cibernética puede verse como una forma fundamental de complejidad dinámica, mientras que la teoría de la información puede verse como una forma fundamental de complejidad computacional.

El énfasis en el problema del todo y las partes plantea dos cuestiones centrales en economía y para enfoques más recientes de la complejidad. Uno es el problema de la relación entre micro y macro en economía, que recuerda el viejo problema de la ``falacia de composición'' de Keynes. Los enfoques walrasianos de la macroeconomía han intentado evitar este problema mediante el uso de modelos de agentes representativos. Otros han propuesto abordar este problema mediante la invocación de una zona intermedia entre lo micro y lo macro, el ``meso'', que se considera crucial para la dinámica evolutiva de una economía compleja (Ng1980; Dopfer y col.2004). El mayor desarrollo de este enfoque se ha debido a Potts (2000), Metcalfe y Foster (2004), Dopfer (2005), Shiozawa (2004), Shiozawa y col.~(2019) y Rosser Jr.~(2021), con Hodgson (2006) argumentando que la evolución darwiniana es el más fundamental de todos los sistemas complejos, basándose profundamente en Veblen (1898) quien fue el primero en defender claramente que la economía adopte un enfoque evolutivo.

La definición general de Simon también tiene la virtud de estar cerca del significado original de la palabra ``complejo'' como se encuentra en el Oxford English D ictionary (OED1971, pag. 492) donde primero se define como ``un todo, que comprende en su conjunto varias partes'', del latín ``complectere'', que significa ``abarcar, abrazar, comprender, comprender''. Entre sus sinónimos parciales está ``complicado'', aunque, como Israel (2005) señala, esto proviene de una raíz latina diferente, ``complicare'', que significa ``plegar'' o ``entretejer''. Israel asume la posición firme de que este último es un concepto meramente epistemológico, mientras que el primero es fundamentalmente ontológico, y se queja de que figuras como von Neumann (1966) los confundió con idénticos, aunque podría decirse que esta es una posición demasiado fuerte.

Una última virtud de esta definición general es que abarca una de las áreas actuales de vanguardia de la economía: los enfoques conductual y experimental, que no son idénticos. Algunos que siguen estos enfoques no consideran que el punto de vista de la complejidad sea tan relevante para lo que hacen (Ken Binmore y Matthew Rabin, por ejemplo, incluso cuando estos dos discrepan fuertemente entre sí en ciertos asuntos (Colander et al.2004a)). Sin embargo, en la base de la economía del comportamiento se encuentra el concepto de racionalidad limitada , introducido originalmente por Herbert Simon. No es solo Simon, sino muchos desde entonces quienes han visto la complejidad como implicando que la racionalidad debe estar limitada (Sargent1993; Arthur y col.1997a; Rosser Jr.~y Rosser2015), y por lo tanto se encuentra en la base de la economía del comportamiento, con Sent (1997) discutiendo la relación entre las opiniones de Sargent y Simon.

De cara al futuro, una parte crucial de la economía de la complejidad dinámica es el enfoque de agentes interactuantes heterogéneos. Este enfoque enfatiza agentes heterogéneos dispersos e interactuantes (Arthur et al.1997a; Tesfatsion, 2006; Hommes2021). Para muchos economistas, esto es lo que quieren decir cuando se refieren a ``modelos de complejidad''. Sin embargo, como se discutió anteriormente en este libro, la complejidad dinámica compite con la complejidad computacional como el enfoque más importante de la economía de la complejidad.

Los defensores del enfoque de complejidad computacional (Albin y Foley 1998; Velupillai2000, 2005a, B, 2009; Markose2005) argumentan que su mayor precisión lo convierte en un vehículo superior para la investigación científica en economía. Debe admitirse que hay algo de cierto en esto. Sin embargo, la gran mayoría de la investigación en economía que se identifica con la complejidad tiende a ser más de la variedad dinámica descrita anteriormente. Además, esta definición es ciertamente menos útil cuando consideramos la cuestión de la profesión económica en sí misma como un sistema complejo en evolución. Aquí consideramos que las dos primeras definiciones proporcionan una construcción más útil para el análisis que esta visión de la complejidad, ciertamente desafiante y sustancial, que esperamos tenga el potencial para importantes investigaciones futuras en el área de la complejidad económica. La profesión económica no solo es un conjunto de jerarquías,

\hypertarget{quuxe9-es-el-trabajo-de-complejidad-de-vanguardia}{%
\section*{¿Qué es el trabajo de complejidad de vanguardia?}\label{quuxe9-es-el-trabajo-de-complejidad-de-vanguardia}}
\addcontentsline{toc}{section}{¿Qué es el trabajo de complejidad de vanguardia?}

Las definiciones de complejidad son importantes porque proporcionan una forma de integrar las diferentes corrientes de la economía moderna en un solo tema unificador: el tema de la complejidad. La aceptación por parte de la profesión económica de que la economía es compleja indica una nueva apertura a las ideas de otras disciplinas y la convierte en un campo más transdisciplinario. Algunos trabajos actuales que entran en este enfoque de complejidad amplia de carpas incluyen lo siguiente:

\begin{itemize}
\tightlist
\item
  La teoría de juegos evolutivos está redefiniendo cómo se integran las instituciones en el análisis.
\item
  La economía ecológica está redefiniendo cómo se considera que la naturaleza y la economía se interrelacionan en una formulación transdisciplinaria.
\item
  La economía del comportamiento está redefiniendo cómo se trata la racionalidad.
\item
  El trabajo econométrico que se ocupa de las limitaciones de la estadística clásica está redefiniendo cómo los economistas piensan en la prueba empírica.
\item
  La teoría de la complejidad ofrece una forma de redefinir cómo concebimos el equilibrio general y la dinámica económica de manera más amplia.
\item
  El análisis económico computacional (ACE) basado en agentes proporciona una alternativa al modelado analítico.
\item
  La economía experimental está cambiando la forma en que los economistas piensan sobre el trabajo empírico, siendo este el método principal por el cual se estudia la economía del comportamiento.
\end{itemize}

Estos cambios están en curso y, en diversos grados, se han incorporado a la corriente principal. A medida que eso ha sucedido, ha habido un conjunto más amplio de cambios en la forma en que la economía dominante se ve a sí misma. La economía moderna está más dispuesta a aceptar que la parte formal de la economía tiene una aplicabilidad limitada. También está mucho más dispuesto a cuestionar el estatus especial de la economía sobre los otros campos de investigación e integrar los métodos de otras disciplinas en sus métodos, con Loasby (1989) y Colador (1995) argumentando que esto es más consistente con un enfoque marshalliano que con un enfoque walrasiano.

Cada una de estas diferentes cepas tiene ciertas características que son bastante diferentes de las que se presentan en los libros de texto económicos. En la mayoría de los libros de texto de hoy, uno tiene la impresión de que la economía no ha cambiado mucho durante los últimos 50 años. Esencialmente, uno aprende un paradigma que desarrolla un modelo deductivo analítico simple, a veces llamado modelo Max U. La microeconomía que se enseña en estos textos es una variación del modelo Max U presentado con poco sabor contextual que caracterizó el uso de Marshall. El modelo Max U presentado en el texto estándar se centra casi por completo en la eficiencia y la optimización, asumiendo que los agentes son racionales, egoístas y están operando en un entorno que llega a un equilibrio único.

El modelo MaxU ha sido explorado a muerte y, desde una perspectiva de vanguardia, ya no tiene mucho interés. (Eso no significa que todavía no tenga una relevancia considerable. Todavía hay muchas aplicaciones prácticas que justifican la investigación; sin embargo, desde un punto de vista de vanguardia, hemos hecho todo lo posible por ello). del nuevo trabajo de vanguardia va más allá de estos supuestos. Si bien no niega la utilidad o el conocimiento proporcionado por ese modelo, no considera que un modelo basado solo en estos supuestos sea suficiente y, por lo tanto, está ampliando los límites de cada uno de esos supuestos. Algunos ejemplos de cómo el trabajo de vanguardia está cuestionando estos supuestos neoclásicos serían los siguientes:

\begin{itemize}
\tightlist
\item
  Los investigadores de economía de vanguardia están ampliando el significado de racionalidad para incluir una gama mucho más amplia de acciones de agentes que reflejan acciones reales; en el nuevo enfoque, los individuos tienen un propósito (los incentivos siguen siendo importantes) pero no son necesariamente formalmente racionales. La nueva investigación considera los fundamentos conductuales de las acciones, utilizando experimentos para determinar lo que las personas realmente hacen, en lugar de simplemente basar sus argumentos en lo que las personas deberían hacer racionalmente, con Payne et al.~(1993) integrando la psicología en esto. El trabajo en teoría de juegos de economistas como Peyton Young (1998) está llevando la racionalidad al límite para demostrar la importancia de las expectativas y el entorno de información en las decisiones de las personas. El trabajo de vanguardia que se está haciendo aquí va más allá de la definición tradicional de racionalidad, con versiones extendidas de la racionalidad limitada de Herbert Simon cada vez más aceptadas.
\item
  Los investigadores de vanguardia se están alejando de una visión estrecha del egoísmo. Si bien la economía de los libros de texto generalmente asume que los agentes que solo se preocupan por ellos mismos, el nuevo trabajo está tratando de asimilar el sentido más realista de los individuos que, aunque tienen intereses propios, también son seres sociales, que se preocupan por los demás y obtienen felicidad de los demás. interactuar con los demás.
\item
  Los investigadores de vanguardia se están alejando del supuesto de un equilibrio único y están lidiando con sistemas complejos que tienen equilibrios múltiples, dependencia de la trayectoria y ninguna respuesta clara. Una economía compleja no tiene un equilibrio único; tiene muchas cuencas de atracción. La pregunta que se hacen los investigadores es qué cuenca es sostenible. En este trabajo, el equilibrio no es un estado de la economía; la economía está en constante cambio.
\end{itemize}

Combinados, estos cambios pueden resumirse como un movimiento de una economía de racionalidad, egoísmo y equilibrio a una economía de comportamiento decidido, interés propio ilustrado y sostenibilidad. El trabajo de vanguardia ayuda a impulsar esa transformación.

\hypertarget{cambios-en-los-muxe9todos-de-investigaciuxf3n}{%
\section*{Cambios en los métodos de investigación}\label{cambios-en-los-muxe9todos-de-investigaciuxf3n}}
\addcontentsline{toc}{section}{Cambios en los métodos de investigación}

Otro aspecto del trabajo de vanguardia que es consistente con la era de la complejidad involucra cambios en los métodos de investigación que pueden servir como catalizadores para muchos cambios en la profesión. Por ejemplo, los avances en la tecnología informática han dado lugar a nuevos enfoques, como el modelado basado en agentes. Esto permite a los economistas analizar sistemas complicados, con interacciones más complicadas entre los agentes, de los cuales pueden surgir o autoorganizarse estructuras de orden superior. Además, en lugar de asumir un comportamiento óptimo, los economistas están utilizando experimentos de laboratorio, de campo y naturales para determinar qué es lo que realmente hace la gente. A medida que los economistas han comenzado a utilizar estas nuevas técnicas, se están dando cuenta de las instituciones, ya que los incentivos incorporados en esas instituciones suelen ser fundamentales para comprender el comportamiento de las personas.

Este cambio va acompañado de un cambio en la naturaleza deductiva del razonamiento económico. El nuevo trabajo se basa más en el razonamiento inductivo empírico y mucho menos en el razonamiento deductivo puro. Mientras esto sucede, las matemáticas que se utilizan en el análisis económico se están volviendo menos las matemáticas de Bourbak de ``prueba de teoremas'' y más matemáticas aplicadas, que están diseñadas para dar respuestas sobre cuestiones de política y no solo para hablar de cuestiones generales ( Weintraub2002). La teoría de conjuntos y el cálculo, que llegan a resultados definidos, están siendo reemplazados por la teoría de juegos, que rara vez llega a una conclusión definitiva independiente de la estructura precisa del juego. Por ejemplo, el trabajo actual sobre subastas combina conocimientos de la teoría de juegos con resultados experimentales, que luego se utilizan en la práctica (Banks et al.2003). De manera similar, la economía de la información se utiliza en el diseño de algoritmos eficientes para motores de búsqueda.

\hypertarget{trabajo-de-complejidad-de-vanguardia-y-macroeconomuxeda-moderna}{%
\section*{Trabajo de complejidad de vanguardia y macroeconomía moderna}\label{trabajo-de-complejidad-de-vanguardia-y-macroeconomuxeda-moderna}}
\addcontentsline{toc}{section}{Trabajo de complejidad de vanguardia y macroeconomía moderna}

Curiosamente, estos cambios de vanguardia en la micro teoría hacia el análisis inductivo y un enfoque de complejidad no se han producido en macroeconomía. De hecho, la evolución del pensamiento macroeconómico en Estados Unidos ha ido al revés. Con eso, queremos decir que ha habido un movimiento desde una teoría macro burda y sencilla que caracterizó la macroeconomía de la década de 1960 hacia una teoría macroeconómica teóricamente analítica basada en modelos de agentes abstractos y representativos que se basan en gran medida en los supuestos de equilibrio. Este trabajo macro se denomina Nuevo clásico, ciclo económico real y teoría del equilibrio general estocástico dinámico (DSGE), y se ha convertido en la corriente principal en los EE. UU.

En parte, este desarrollo es comprensible. La teoría macro predominante en la década de 1960 afirmaba tener una base teórica mucho más sólida de lo que se justificaba, y muchas de las conclusiones a las que llegó no estaban respaldadas ni por pruebas empíricas ni teóricas. Sin embargo, si bien los nuevos modelos teóricos han hecho un buen trabajo al eliminar la vieja teoría, es menos claro qué ha agregado el nuevo trabajo teórico a nuestra comprensión de la macroeconomía. En el mejor de los casos, los resultados de los nuevos modelos macro pueden calibrarse aproximadamente con la evidencia empírica, pero a menudo estos nuevos modelos no funcionan mejor que cualquier otro modelo, y la única afirmación que tienen para ser preferidos es estética: tienen micro fundamentos. Sin embargo, se trata de un micro fundamento extraño: un micro fundamento basado en supuestos de interacción no heterogénea entre agentes, cuando, para muchas personas, es precisamente la interacción heterogénea de agentes la que conduce a las características centrales de la macroeconomía. Ésta es la idea esencial de la falacia de composición de Keynes.

Por supuesto, hemos visto esfuerzos para introducir agentes heterogéneos en el contexto DSGE, lo que ha llevado a la aparición de modelos New Keynesianos de agentes heterogéneos (HANK). Sin embargo, a menudo, como en Krusell y Smith Jr.~(1998) estos modelos no implican interacciones directas entre agentes. Más bien, se obtiene un intervalo de un número infinito de agentes que varían en un parámetro particular, con, en efecto, ese intervalo actuando como el agente representativo de otros modelos DSGE. Esto no conduce a un enfoque complejo del modelado macro. Tal enfoque tendrá resultados macro que emergen de un conjunto de agentes heterogéneos que interactúan basados \hspace{0pt}\hspace{0pt}en el comportamiento, siendo un buen ejemplo Delli Gatti et al.~(2008).

El interesante trabajo de vanguardia en macro no está en los desarrollos teóricos organizados en torno a los micro fundamentos del agente representativo, sino en el trabajo que ve la macroeconomía como un sistema complejo. En este trabajo, uno ve la macroeconomía como organizada endógenamente. El problema no es por qué hay fluctuaciones en la macroeconomía, sino por qué hay tan poca inestabilidad donde las interacciones complejas podrían generar caos, aunque las dinámicas caóticas se mantienen dentro de límites consistentes con la idea del ``corredor de estabilidad'' de Leijonhufvud (1973, 2009), que se asemeja a la ``compensación entre resiliencia y estabilidad'' estudiada por Holling (1973) en ecología. La creencia de que se puede desarrollar una base microeconómica para la macroeconomía sin considerar la retroalimentación del macro sistema sobre el individuo es increíble. Si bien aún puede tener sentido impulsar la teoría macro analítica hasta donde sea posible, para ver si proporcionará alguna comprensión, a corto plazo, tales extensiones analíticas de modelos teóricos puros basados \hspace{0pt}\hspace{0pt}en supuestos que están lejos de la realidad ofrecen pocas esperanzas de orientación política. En ausencia de una base teórica pura, la política macroeconómica se basa mejor en modelos estadísticos que extraen la mayor cantidad de información posible de los datos. La macro empírica precede a la macro teórica.

\hypertarget{la-economuxeda-de-la-complejidad-y-el-debate-sobre-la-economuxeda-heterodoxa}{%
\section*{La economía de la complejidad y el debate sobre la economía heterodoxa}\label{la-economuxeda-de-la-complejidad-y-el-debate-sobre-la-economuxeda-heterodoxa}}
\addcontentsline{toc}{section}{La economía de la complejidad y el debate sobre la economía heterodoxa}

El argumento básico de este capítulo de que la economía de la complejidad no solo es una parte crucial de la vanguardia de la investigación económica sino que, de hecho, sustenta sustancialmente el futuro más amplio de la economía fue formulado de manera contundente inicialmente por Colander et al.~(2004a) y Colander et al.~(2004b), siendo el primero de ellos un libro en su mayoría de entrevistas con ``economistas de vanguardia'', todos menos uno de los cuales estaban ubicados en los Estados Unidos, 2 y esto no estaba planeado sino que simplemente surgió por conveniencia dado que todos estamos basados \hspace{0pt}\hspace{0pt}en el EE. UU. A esto le seguiría un libro similar en gran parte de entrevistas centradas en los economistas y la economía europeos (Rosser Jr.~et al.2010), 3 con uno planeado para Asia que nunca sucedió, aunque podría decirse que en Japón existe una tradición que ha llevado a un enfoque de este tipo tan independiente y desarrollado localmente (Morris-Suzuki1989; Ikeo2014; Shiozawa2004; Shiozawa y col.2019; Rosser Jr.2021).

El segundo ítem es un artículo derivado en gran parte del capítulo inicial del libro que estableció el marco que teníamos al entrar en las entrevistas, en el que el tema de la complejidad fue un tema recurrente. Este artículo, publicado en Review of Political Economy , atraería la mayor atención (y las citas) de todos estos trabajos y desencadenaría un debate considerable que se discutirá a continuación, con varios de nuestros trabajos posteriores centrados en gran medida en este debate (Colander et al.~.2007-08, 2010; Rosser Jr.~y col.2013).

Un problema que se remonta a décadas en realidad, como se puede suponer si se ha leído este libro hasta aquí, es que durante gran parte de este tiempo las ideas asociadas con la economía de la complejidad no siempre fueron fácilmente aceptadas por los economistas de la corriente principal. Los artículos a menudo aparecían en revistas extravagantes, con algunas excepciones, o en libros posiblemente extravagantes, aunque en varios casos estos artículos y libros serían más tarde citados y ampliamente respetados e influyentes. Esto nos llevó a pensar seriamente sobre la naturaleza de cómo evoluciona la economía y cómo las nuevas ideas o enfoques se desarrollan y entran en la economía, pasando de ser algo marginal y ridículo a terminar eventualmente en los libros de texto, y uno de nosotros, David Colander, ha usado durante mucho tiempo el sombrero de educador económico (Colander2000b) e historiador del pensamiento económico (Colander 2000c), además de vincular estas preocupaciones a las ideas de la economía de la complejidad e incluso aplicarlas a la economía misma como campo (Colander et al.~2009; Colador2015; Holt y Rosser Jr.2018).

Una pieza central de este proceso y debate involucra el papel de la economía heterodoxa y su relación con la economía no heterodoxa, hasta qué punto surgen nuevas ideas de los economistas heterodoxos y cómo es que cuando son ``exitosas'' se mueven más hacia la corriente principal. Este tema estuvo muy vivo en nuestro primer libro de entrevistas (Colander et al.2004a) en el que, de hecho, los que nos entrevistamos diferían en cómo se veían a sí mismos con respecto a su estatus en la profesión, y algunos se veían a sí mismos como claramente heterodoxos (Duncan Foley) mientras que otros se veían a sí mismos como más en la corriente principal (Ken Binmore). Esto nos empujó a pensar más sobre lo que estaba pasando aquí.

Lo que se nos ocurrió fue bifurcar la cuestión hasta cierto punto y argumentar que tiene un aspecto intelectual y un aspecto sociológico, con tres categorías en consideración: ortodoxia, heterodoxia y corriente principal (aunque confrontando esto uno de nuestros entrevistados, Herb Gintis, bromeó no del todo en serio que le gusta pensar en sí mismo como un ``economista homodoxo''). Decidimos que la ortodoxia es una categoría intelectual, la corriente principal es una categoría sociológica, pero la heterodoxia es ambas, que es donde surgen muchos de los problemas. La economía ortodoxa en su forma pura es la vieja ``economía neoclásica'' que argumentó Colander (Colander2000a) ha muerto, esa economía descrita por la trinidad de racionalidad, codicia y equilibrio. Su manifestación más pura fue en la Universidad de Chicago durante décadas, aunque en un nivel más fundamental sus exponentes más duros se basaron durante mucho tiempo en el ``código postal sagrado'' en Cambridge, Massachusetts en Harvard y especialmente en el MIT, con Paul Samuelson como quizás el supremo. padrino, a quien entrevistamos junto con Ken Arrow para el final de nuestro primer libro después de dejarles ver nuestras otras entrevistas. Tal como están las cosas, incluso en estos bastiones esta vieja ortodoxia ya no domina, y todo tipo de enfoques antes inaceptables, especialmente la economía del comportamiento, ahora infestan los pasillos y las oficinas.

La corriente principal es una categoría sociológica. En realidad, son las personas, los que están a cargo de la profesión económica, los de las mejores escuelas, los que dirigen las principales revistas, controlan la financiación de la investigación, etc. Notamos que incluso poco después del 2000 más o menos había bastantes personas de ese tipo en estos puestos, incluidos los ganadores del Premio Nobel, cuyas ideas no eran estrictamente ortodoxas, con personas como George Akerlof y Vernon Smith destacando como ejemplos, aunque Smith no lo ha hecho. en general, he estado en las mejores escuelas. Esto también incluiría algunos actores anteriores que han sido citados en gran medida en este libro como importantes en el desarrollo de la economía de la complejidad, como Herbert Simon. Todos ellos ganaron premios Nobel y son o fueron muy respetados, pero también durante mucho tiempo se han sentido en desacuerdo con el núcleo duro de ``el establishment,''Incluso cuando veían a forasteros más serios como parte de ese" establishment ``ortodoxo. Son o fueron''convencionales``, pero no''ortodoxos". Esta fue nuestra afirmación clave, y la que provocó muchas críticas sobre nuestras cabezas.

Esta afirmación clave tenía otra parte, la afirmación de que, en contraste con las otras dos categorías principales, la heterodoxia es una categoría tanto intelectual como sociológica. Por lo tanto, los economistas heterodoxos se oponen intelectualmente y son críticos de la vieja economía ortodoxa, y tampoco están en las mejores escuelas y les resulta difícil publicar en las principales revistas, sintiéndose discriminados e incluso oprimidos. En algunos casos, esto les ha llevado a no conseguir la titularidad en varias instituciones debido a sus problemas para publicar lo suficiente en revistas suficientemente prestigiosas y, por lo demás, a sufrir profesionalmente.

Es comprensible que esto haya llevado al resentimiento y la ira de muchos, y algunos de ellos posiblemente estén justificados. Para muchos de estos economistas heterodoxos que se identifican a sí mismos, el enemigo es ``la corriente principal ortodoxa'', y responden a esto identificando a algunos de los economistas principales como ``no ortodoxos''. Para estos economistas heterodoxos más duros, esta corriente dominante no ortodoxa en otro tiempo son, si no totalmente vendidos, entonces las personas que han jugado un juego para hacerse aceptables para los que están a cargo compran sin desafiar con suficiente vigor la ortodoxia (Lavoie2012; Sotavento2012). El hecho de que puedan estar haciendo que sus ideas sean aceptadas hasta cierto punto por la corriente principal e incluso por los viejos ortodoxos simplemente muestra que se están asimilando a la corriente principal y ortodoxa, no que estén logrando que la corriente principal acepte sus ideas e incluso podría decirse que redefinen la naturaleza de la corriente principal. ortodoxia. Tal como están las cosas, incluso entre los críticos de nuestra formulación hay diferencias. Así Marc Lavoie (2012) reconoce un grupo que él llama ``disidentes'' que son, en efecto, nuestro grupo de mainstreamers no ortodoxos, mientras que la línea más dura Fred Lee (2012) básicamente descartó toda esta categoría, argumentando que tomarlos en serio o tratar de ser como ellos era simplemente ceder a la dominación de la ortodoxia y renunciar a la heterodoxia.

Huelga decir que entre los heterodoxos han surgido a lo largo del tiempo muchas escuelas de pensamiento diferentes. Este no es el lugar para entrar en una discusión detallada de todos estos, aunque a lo largo de este libro a veces se han invocado o invocado ideas de uno u otro de ellos, incluidos los marxistas, austriacos, poskeynesianos, evolucionistas, institucionalistas, conductuales, ecológicos, y más, especialmente cuando sus enfoques parecían abiertos o en congruencia con elementos de la economía de la complejidad. De hecho, los orígenes de muchas ideas en la economía de la complejidad surgieron claramente de una u otra de estas escuelas en momentos particulares, y posiblemente los defensores más fuertes de algunas de esas ideas siguen estando firmemente identificados con una u otra de estas escuelas.

Por supuesto, una gran ironía es que cada una de estas escuelas de pensamiento han desarrollado sus propias ortodoxias internas y líderes, revistas y lugares que reclaman autoridad para definir la escuela y quién está o no en ella, con el resultado de que surgen herejías. incluso dentro de estas escuelas, lo que lleva al desarrollo de subescuelas que pueden llegar a ser tan numerosas y diferenciadas unas de otras por debates tan oscuros que a los de afuera les resulta difícil, si no imposible, averiguar qué está sucediendo o quién es qué. Las guerras entre los marxistas se contaban entre las más famosas y, en ocasiones, implicaban literalmente guerras y personas que literalmente se mataban entre sí, como demostró de manera más dramática el asesinato de Trotsky por parte de Stalin. Los austriacos están divididos entre misesianos y hayekianos. Las divisiones entre los poskeynesianos son especialmente numerosas,Journal of Postkeynesian Economics, mientras que grupos rivales con base en Europa, como los sraffianos neo-ricardianos, argumentaron vigorosamente en contra de sus puntos de vista y los de otros. Las diversas escuelas de heterodoxos llegaron a tener sus propios subheterodoxos. En algunas de estas batallas, algunas subescuelas son más amigables con las ideas complejas que otras, con los hayekianos más entre los austriacos y los llamados kaldorianos entre los poskeynesianos también más, solo por dar dos ejemplos.

Estos debates y diferencias de opinión incluso han estado presentes entre los tres coautores que he citado aquí sobre este tema, yo mismo, David Colander y Ric Holt. Dave ha adoptado durante mucho tiempo la línea más dura de criticar a los heterodoxos por no esforzarse más por llevarse bien con los mainstreamers, por no tratar de usar ``más miel'' en lugar de ``más vinagre'', lo que ha tendido a generar más críticas en su cabeza. desde algunos heterodoxos, ya que a menudo ha sido muy público y articulado sobre estos puntos de vista a una forma casi ``en tu cara'' con algunos heterodoxos, para disgusto de estos últimos. Probablemente he sido el que más al otro lado, más comprensivo con las quejas de muchos heterodoxos sobre su rechazo, opresión y discriminación, Ric era el que a menudo hacía las paces diplomáticamente entre Dave y yo cuando trabajábamos juntos. Puede ser que personalmente me sintiera más heterodoxo, estando en una universidad estatal no particularmente prestigiosa y durante mucho tiempo sintiéndome aislado e ignorado.

Pero Dave argumentó que a pesar de todas esas actitudes me convertí en un mainstreamer, especialmente después de la publicación en 1991 de mi primer libro, From Catastrophe to Chaos: A General Theory of Economic Discontinuities , que se convirtió en un éxito después de su publicación , con tres ediciones y recibiendo críticas favorables y muchas citas, a pesar de que había sido rechazado por 13 editores antes de que Kluwer lo aceptara a instancias de Zac Rolnik allí. Mi posición cambió especialmente cuando me convertí en editor en 2001 del Journal of Economic Behavior and Organisation., que durante mucho tiempo ha sido visto como ``heterodoxo pero respetable'', una delgada línea para caminar. Fundada por Dick Day, fue de hecho una salida temprana de muchas ideas complejas, incluida la teoría del caos, así como la teoría de juegos, la economía del comportamiento y la nueva economía institucionalista. Si bien en la década de 1980 gran parte de este trabajo no se podía publicar en las principales revistas, eso ha cambiado, y los líderes de estos campos ganaron premios Nobel y este tipo de material ahora se publica en las principales revistas e incluso ingresa en los libros de texto de posgrado. Esto incluso incluyó, hasta cierto punto, las ideas que expresé en ese libro de 1991, que ahora es visto como un volumen de referencia por muchos. Dave me dijo que me había convertido en la corriente principal, me gustara o no, porque ``la gente superior respeta lo que haces'', y también porque muchas de las ideas en las que he trabajado que eran vistas como heterodoxas se han convertido en, bueno , respetable. De hecho, podría decirse que esto es parte de cómo la economía en general ha entrado en la era de la complejidad.

Cierro esta sección señalando un viejo chiste que escuché de Dave Colander y que escuchó por primera vez de Abba Lerner. ``Pero mire'', protestó la esposa del rabino, ``cuando una de las partes en la disputa presentó su caso, usted dijo `tiene razón' y luego, cuando la otra parte presentó su caso, volvió a decir `tiene toda la razón'. Seguramente ambos no pueden tener razón''. A lo que el rabino respondió: ``Querida, ¡tienes toda la razón!''

\hypertarget{economuxeda-de-la-complejidad-y-poluxedticas-puxfablicas}{%
\section*{Economía de la complejidad y políticas públicas}\label{economuxeda-de-la-complejidad-y-poluxedticas-puxfablicas}}
\addcontentsline{toc}{section}{Economía de la complejidad y políticas públicas}

Si de hecho el futuro de la economía va a estar fuertemente influenciado por ideas de la economía de la complejidad, entonces para muchos la prueba del pudín se reduce a cuán útil es para informar las discusiones y formulaciones de políticas públicas. Este es un tema de controversia y disputa en curso. Mucho de esto ha involucrado especialmente el uso de modelos de agentes heterogéneos del tipo discutido anteriormente en este libro que estaba especialmente asociado con el Instituto Santa Fe, donde posiblemente el enfoque se ha centrado más recientemente en la economía del comportamiento y la teoría de juegos que en ese tipo en particular. de modelado. Por supuesto, como Rosser Jr.~y Rosser (2015) argumentan y se ha argumentado anteriormente en este libro, existen fuertes vínculos entre la economía de la complejidad y la economía del comportamiento, siendo el papel central de Herbert Simon en el desarrollo temprano de ambos un fuerte signo de esto.

También debe reconocerse que gran parte de cada uno no pertenece particularmente al otro. Pero, de hecho, si la antigua ortodoxia se destacó por una trinidad de racionalidad, codicia y equilibrio, tanto la economía del comportamiento estándar como la economía de la complejidad desafían a las tres, por lo que no es sorprendente que haya una superposición considerable, y no es sorprendente de hecho. una vez más, que la revista que edité de 2001 a 2010, la Revista de comportamiento económico y organización (y la que ahora edito, Review of Behavioural Economics ) han sido importantes salidas para ambos enfoques, incluida su superposición.

Un área en la que se ha sentido frustración por parte de muchos economistas orientados a la complejidad ha sido la macroeconomía, discutida anteriormente. Ha habido un gran impulso para adoptar modelos de agentes heterogéneos interactivos en entidades de formulación de políticas tan importantes como los bancos centrales, pero aparte de los estudios que se están realizando en algunos de ellos, estos no han ganado el día ni se han adoptado sustancialmente. Se informa ampliamente que en la Reserva Federal de los EE. UU. Se utilizan tres tipos diferentes de modelos para asesorar a los formuladores de políticas: modelos DSGE, modelos estructurales que son derivaciones esencialmente complicadas del enfoque ISLM y modelos ateóricos basados \hspace{0pt}\hspace{0pt}en métodos autorregresivos vectoriales. Si bien, según se informa, los modelos de agentes heterogéneos que interactúan en toda regla no se han unido a este triunvirato, cada uno de ellos ha absorbido elementos de la economía de la complejidad. Como se señaló anteriormente, Los modelos DSGE han cambiado para incluir múltiples agentes, así como algunas no linealidades e incluso esencialmente correcciones de comportamiento ad hoc. Puede que haya menos de esto sucediendo con los modelos estructurales más antiguos, pero los modelos derivados de VAR han incorporado durante mucho tiempo métodos no lineales de varios tipos, con una larga interacción entre la complejidad y la econometría no lineal y los enfoques de series de tiempo. También ha habido una incorporación en los tres tipos de modelos de factores financieros, y estas partes de los modelos a menudo también involucran varios elementos de complejidad. De hecho, en algunos bancos hay muchos modelos de redes de relaciones financieras (Haldane pero los modelos derivados de VAR han incorporado durante mucho tiempo métodos no lineales de varios tipos, con una larga interacción entre la complejidad y la econometría no lineal y los enfoques de series de tiempo. También ha habido una incorporación en los tres tipos de modelos de factores financieros, y estas partes de los modelos a menudo también involucran varios elementos de complejidad. De hecho, en algunos bancos hay muchos modelos de redes de relaciones financieras (Haldane pero los modelos derivados de VAR han incorporado durante mucho tiempo métodos no lineales de varios tipos, con una larga interacción entre la complejidad y la econometría no lineal y los enfoques de series de tiempo. También ha habido una incorporación en los tres tipos de modelos de factores financieros, y estas partes de los modelos a menudo también involucran varios elementos de complejidad. De hecho, en algunos bancos hay muchos modelos de redes de relaciones financieras (Haldane2013), claramente un enfoque de complejidad, si solo se menciona en este libro.

Más ampliamente, mientras Brock y Colander (2000) hicieron un intento inicial por un enfoque más general, Colander y Kupers (2014) intentan ir más allá de las formulaciones convencionales y ofrecer una postura provocadora, aunque es casi seguro que tiene sus límites. Se basa efectivamente en el énfasis en el surgimiento de la estructura y el orden a partir de enfoques ``de abajo hacia arriba'' en lugar de ``de arriba hacia abajo'', haciendo hincapié en la espontaneidad y la creatividad para buscar soluciones nuevas e innovadoras a problemas arraigados. Se reunieron mientras participaban en una conferencia sobre políticas climáticas. Hubo una división entre quienes defendían políticas orientadas en gran medida al mercado y quienes abogaban principalmente por políticas gubernamentales orientadas a la regulación. No estaban contentos con esta dicotomía simplista y buscaron una alternativa orientada a la complejidad, lo que los llevó a hacer hincapié en políticas de abajo hacia arriba que bien podrían involucrar tanto a los mercados como a los gobiernos.

Su enfoque se resume a continuación (Colander y Kupers 2014, pag. 21):

\begin{quote}
``En el marco de la política de complejidad, uno comienza con el reconocimiento de que no existe una brújula definitiva para la política que no sea un sentido común altamente educado. Los modelos científicos proporcionan, en el mejor de los casos, medias verdades. En nuestra opinión, la educación de ese sentido común incluye en gran medida una apreciación básica de la complejidad, así como de las humanidades, las matemáticas y otras. Las brújulas de política se crean y evolucionan, son productos falibles de un tiempo y lugar en particular, y deben tratarse como tales. La naturaleza de la relación entre el mercado y el gobierno, así como las soluciones de arriba hacia abajo versus las de abajo hacia arriba, así como la propiedad de que la política en sí misma es parte del sistema complejo, se postula con bastante claridad en lo siguiente \ldots{} la dualidad de mercado versus el gobierno es un producto del propio marco de política económica estándar. Dentro de un marco de complejidad, Se considera que tanto la solución de ``gobierno'' de arriba hacia abajo más activa y la menos activa de abajo hacia arriba han evolucionado de abajo hacia arriba. Dentro de este marco, la solución de políticas es un elemento del sistema, no fuera de él ".
\end{quote}

Al invocar la ``metapolítica'', evitan abogar por políticas específicas. Sin embargo, brindan algunos ejemplos de lo que les gusta. Un ejemplo es el sistema de control de tráfico de ``espacio compartido'' en la ciudad de Drachten, Holanda, desarrollado por Hans Monderman. Cuando uno conduce hacia Drachtem, no encuentra señales de alto o luces de la calle o incluso aceras. Sin embargo, el tráfico fluye sin problemas y con pocos accidentes. Ayuda que Drachten no sea una gran ciudad donde tal sistema simplemente no funcione. Esto puede parecer un ``fundamentalismo de mercado sin gobierno'' semi-anarquista, pero ellos argumentan que no es el caso. Esto se debe a que este sistema depende de un marco institucional existente: un sistema preexistente de innumerables reglas y regulaciones, licencias de conducir, estándares de seguridad para automóviles, un marco legal más amplio y más. Por tanto, no es un anarcocapitalismo espontáneo, sino un sistema cuidadosamente enmarcado y acotado que permite el surgimiento del orden. Como también señalan, ``En el marco de la complejidad, un mercado que funciona bien es una consecuencia de una metapolítica gubernamental anterior y exitosa'' (Colander y Kupers2014, pag. 25).

Otro problema relacionado en el que se meten es uno que Rosser Jr.~(2001a, 2020e) también ha abordado, a saber, la relación entre los puntos de vista de Keynes y Hayek y cómo cada uno de ellos se relaciona con la complejidad, con Hayek (1967) habiendo discutido específicamente la complejidad y tomándola en serio en sus últimos años, mientras que Keynes nunca lo abordó específicamente. Para Colander y Kupers, ven cierta superposición de los puntos de vista de los dos, incluso cuando en muchos temas diferían claramente, y de hecho Keynes se ve más como el defensor de la intervención gubernamental de arriba hacia abajo contra Hayek, el defensor de la base de mercado de abajo hacia arriba. orden espontáneo. Es bastante claro que Hayek encaja su enfoque con este enfoque, por lo que la pregunta es ¿dónde encaja Keynes con esto?

Una respuesta que dan es que la pieza más famosa de defensa de arriba hacia abajo de Keynes involucró la Gran Depresión, que él vio como un caso especial ``único''. De lo contrario, generalmente favoreció los enfoques de abajo hacia arriba. Señalan la amistosa carta Keynes (1944) le escribió a Hayek (1944) cuando publicó su The Road to Serfdom en el que expresó su ``simpatía moral y filosófica'' por los argumentos de Hayek. Aun así, la carta misma reconoció sus diferencias, y Keynes argumentó que ``\ldots{} es casi seguro que queremos más {[}planificación{]}. Pero la planificación debe tener lugar en una comunidad en la que la mayor cantidad posible de personas, tanto líderes como seguidores, compartan su posición moral''(Colander y Kupers2014, pag. 40). Afirman que esto muestra a Keynes apoyando soluciones de abajo hacia arriba, pero eso ``minimizaría la intervención del gobierno en el mercado, pero aun así lograría fines socialmente deseables'' (ibid.). Sin embargo, es bastante obvio que otros podrían encontrar que se estiran un poco en este punto.

Como se trata de Keynes y Hayek y su conexión con la complejidad, veo que su superposición proviene de una dirección diferente. Este sería el viejo bugaboo de la incertidumbre fundamental, que se ha discutido anteriormente en este libro. Keynes (1921) primero planteó este argumento de que tal incertidumbre implica la no existencia de una distribución de probabilidad en su Tratado de probabilidad , pero lo trajo más tarde en su Teoría general (1936) y en algunas otras obras. Muchos han visto que esto implica una visión compleja de la economía (Davis1994, 2017).

Hayek no abordó esto específicamente usando la teoría de la probabilidad, pero en su discusión sobre la complejidad (Hayek 1967) está en consonancia con su rechazo de la tendencia a un equilibrio a largo plazo y su preferencia por una economía en constante evolución marcada por la emergencia espontánea del orden. Un argumento más amplio de los austriacos relacionado de manera más general con la incertidumbre es cómo esto abre la puerta al importante papel de los empresarios que operan de manera crucial en un entorno tan profundamente incierto. Cuando se le presiona, Keynes podría estar más inclinado a recurrir al gobierno para controlar y limitar la incertidumbre, mientras que Hayek podría estar más inclinado a confiar en el orden espontáneo que surge de los mercados sin restricciones, pero comparten una comprensión de la naturaleza profunda de los procesos dinámicos de la economía que es compleja.

\hypertarget{la-paradoja-de-la-economuxeda-como-sistema-adaptativo-complejo}{%
\section*{La paradoja de la economía como sistema adaptativo complejo}\label{la-paradoja-de-la-economuxeda-como-sistema-adaptativo-complejo}}
\addcontentsline{toc}{section}{La paradoja de la economía como sistema adaptativo complejo}

La cuestión de si el futuro de la economía será fundamentalmente economía de complejidad o no tiene un aspecto curiosamente paradójico. Un tema entre muchos economistas de la complejidad es que la profesión económica es en sí misma un sistema complejo de adaptación. Se caracteriza por el tipo de no linealidades y retroalimentaciones positivas que Brian Arthur (1994) enfatizado como los elementos centrales de los sistemas complejos. Irónicamente, estas características presentan fuerzas contradictorias, una para la inestabilidad y otra para la estabilidad.

Los efectos de retroalimentación positiva se conocen más comúnmente como socavar el equilibrio. Implican una no convexidad que elimina uno de los supuestos estándar que se hacen cuando se usa un teorema de punto fijo para demostrar la existencia de un equilibrio. En un mercado, si hay rendimientos crecientes, entonces si una empresa se hace más grande que otras, sus costos promedio a largo plazo pueden caer por debajo de los de otras, lo que le permite socavar a sus competidores para que no puedan obtener una ganancia no negativa. , que a su vez, al final, puede conducir a un monopolio natural, ya que los competidores terminan quebrando eventualmente, asumiendo que no hay límite para esas economías de escala.

Pero este resultado nos lleva al aspecto paradójico: si realmente existen estas economías de escala ilimitadas, uno puede terminar en una situación en la que efectivamente existe un monopolio arraigado que no puede ser derrocado por competidores recién ingresados \hspace{0pt}\hspace{0pt}a menos que haya un cambio fundamental en la tecnología. o algún otro elemento del sistema que permita que el participante potencialmente nuevo pueda romper este sistema. Pero el sistema puede volverse profundamente arraigado y difícil de cambiar profundamente. Por lo tanto, un sistema adaptativo complejo podría terminar convirtiéndose en uno esencialmente estancado y conservador, estancado en sus caminos, con todos los cambios simplemente reforzando su estasis, ya que los efectos de retroalimentación positiva simplemente lo llevan más y más profundamente a la condición que ha alcanzado.

Así es que David Colander ve que la profesión económica tiene tendencias a simplemente reforzarse a sí misma en un estado existente a pesar de ser golpeada por fuerzas externas de cambio. Parte de este pesimismo se debe a la evolución de la macroeconomía desde la crisis financiera y la Gran Recesión, cuando el modelo DSGE siguió dominando una posición dominante entre los macroeconomistas responsables de la formulación de políticas en los bancos centrales y el mundo académico, aunque se ha modificado para algunos. grado mediante cambios ad hoc de los tipos mencionados anteriormente. Así argumenta (Colander2015, pag. 230): ``Ahora hay algunas discusiones en los textos de política macroprudencial, límites inferiores cero, estancamiento estructural (aunque gran parte de esa discusión lleva el nombre de estancamiento secular), flexibilización cuantitativa e incluso alguna mención de los momentos de Minsky. Pero en el modelo macro subyacente de un sistema económico estable, la racionalidad agregada compuesta permanece''.

Además, basándose en el trabajo de Piketty (Piketty 2014), las tendencias hacia una desigualdad de ingresos y riqueza cada vez mayor parecen estar profundamente arraigadas y son difíciles de superar o detener, y mucho menos revertir. Obviamente, esta no es una historia simple o directa, y las tendencias en competencia pueden coexistir en diferentes niveles. Por lo tanto, a nivel mundial, vemos una tendencia a aumentar la igualdad agregada debido al aumento de los ingresos en las dos naciones más grandes, China e India, incluso cuando hemos visto una creciente desigualdad dentro de la mayoría de las naciones, lo que socava el optimismo de Simon Kuznets (1955) con respecto a las implicaciones para la desigualdad de ingresos del desarrollo económico a largo plazo. Sin embargo, esto no es inevitable, aparte de la posibilidad de una gran conmoción político-económica revolucionaria como vimos a principios del siglo XX. Por lo tanto, algunas de las naciones más desiguales, en particular algunas de América Latina, han visto algún movimiento hacia una mayor igualdad de ingresos, si no dramático (Rosser Jr y Rosser, 2019, Capítulos 18-19). La tendencia de la desigualdad no es inevitable ni imposible de superar.

Sin embargo, volviendo a la profesión económica en sí, especialmente en los Estados Unidos, que domina cada vez más la profesión económica mundial (Rosser Jr.~et al.~2010), esta tendencia al autorrefuerzo dinámico y al atrincheramiento de una manera dependiente del camino puede estar manifestándose. Colander particularmente ve esto operando a través del sistema educativo, con el conservadurismo del sistema reforzado por lo que él llama ``la regla del 15 por ciento'', la idea de que los libros de texto líderes no pueden cambiar en más del 15 por ciento a la vez debido a la falta de voluntad de los profesores establecidos en un campo para cambiar sus notas de clase con demasiada frecuencia.

Pero en el caso de la profesión económica en particular, en respuesta a la crisis financiera y la Gran Recesión, vimos en efecto un proceso irónicamente peculiar. A pesar de los pedidos generalizados de cambios fundamentales provenientes de muchos sectores, la crisis generó incentivos para que la profesión no cambiara, y estos incentivos reforzaron la autosatisfacción y la inercia. Operó de la siguiente manera, según él: ``Cuanto mayor es la crisis, más estudiantes quieren escuchar lo que la economía tiene que decir, más se apuntan a la economía y más ingresos fluyen hacia la economía, reforzando la estructura institucional. Esto lleva a la profesión a responder: '¿Por qué cambiar lo que estamos haciendo? Lo estamos haciendo bastante bien, gracias''(Colander2015, pag. 234).

Por lo tanto, tenemos la paradoja de que la compleja naturaleza adaptativa de la profesión económica con su dinámica de rendimientos crecientes termina mejorando su tendencia a la estasis y a no cambiar de manera fundamental. El paso a una era de plena complejidad puede continuar, pero es extremadamente difícil volcar el carrito de manzanas y cambiar drásticamente la forma en que se hacen las cosas, para pasar a un tipo de economía fundamentalmente nuevo y diferente. Pero entonces, la naturaleza de los sistemas dinámicamente complejos es generar sorpresas con nuevas formas que surgen inesperadamente cuando uno menos lo espera, incluso como hemos visto en el más grande e importante de todos los sistemas complejos, el proceso evolutivo, que ciertamente opera. en la profesión económica como lo hace en el sistema socioeconómico más amplio y el sistema ecológico-económico aún más grande en el que todos vivimos.

\hypertarget{notas-al-pie-4}{%
\section*{Notas al pie}\label{notas-al-pie-4}}
\addcontentsline{toc}{section}{Notas al pie}

\begin{enumerate}
\def\labelenumi{\arabic{enumi}.}
\tightlist
\item
  Con respecto al ``fin de la economía neoclásica'', ver Colander (2000a), con Veblen (1898) acuñando peyorativamente el término ``neoclásico'' al mismo tiempo, defendía que la economía adoptara un enfoque evolutivo. Para identificar a su sucesor como la ``era de la complejidad'', ver Holt et al.~(2011).
\item
  Los entrevistados en el (Colander et al.~2004a) Libro con sede en EE. UU. Fueron Deirdre McCloskey, Ken Binmore, Herb Gintis, Bob Frank, Mat Rabin, William (``Buz'') Brock, Duncan Foley, Richard Norgaard y Rob Axtell con Peyton Yong, con descripciones ex post de Ken Arrow y Paul Samuelson.
\item
  Los entrevistados en el (Rosser Jr et al.~2010El libro con sede en Europa fueron Alan Kirman, Ernst Fehr, Cars Hommes, Mauro Gallegati con Laura Gardini, Geoff Hodgson, Joan Martinez-Allier y Robert Boyer, con reseñas ex post de János Kornai y Reinhard Selten.
\end{enumerate}

Andresky, S. (1974). The Essential Comte, seleccionado de Cours de Philosophie Positive (1830-1842) . Abington, Nueva York: Routledge.
Google Académico
Bentham, J. (1823 {[}1789{]}). Principios de moral y legislación . Oxford: Clarendon Press.
Google Académico
Berlín, I. (2000). Tres críticos de la Ilustración . Princeton, Oxford: Prensa de la Universidad de Princeton.
Google Académico
Bobbio, N. (1977). Saggi Sulla Scienza Politica en Italia . Bari: Laterza.
Google Académico
Burke, E. (1790). Reflexiones sobre la Revolución en Francia . Londres: Dodsley.
Google Académico
Burke, E. (1800). Reflexiones y detalles sobre la escasez . Londres: Rivington \& Hatchard.
Google Académico
Cristo, C. (1990). Una filosofía de vida. The American Economist, XXXIV .
Google Académico
Clark, C. (2019). Tiempo y poder: visiones de la historia en la política alemana, desde la Guerra de los Treinta Años hasta el Tercer Reich . Princeton, Oxford: Prensa de la Universidad de Princeton.
Google Académico
Croce, B. (1951 {[}1939{]}). Conversazioni Critiche . Bari: Laterza.
Google Académico
De Cecco, M. (1971). Economia e finanza internazionale dal 1890 al 1914 . Bari: Laterza.
Google Académico
de Voltaire. (1770). Cartas sobre la nación inglesa . Glasgow: Robert Urie.
Google Académico
Dobb, M. (1937a). Los requisitos de una teoría del valor. En Economía política y capitalismo: algunos ensayos sobre la tradición económica . Londres: Routledge.
Google Académico
Dobb, M. (1937b). Sobre las fricciones y las expectativas: ciertas tendencias recientes en la teoría económica. En Economía política y capitalismo: algunos ensayos sobre la tradición económica . Londres: Routledge.
Google Académico
Dobb. M. (1937c). Economía política clásica y Marx. En Economía política y capitalismo: algunos ensayos sobre la tradición económica. Londres: Routledge.
Google Académico
Dobb, M. (1937d). Crisis económicas. En Economía política y capitalismo: algunos ensayos sobre la tradición económica. Londres: Routledge.
Google Académico
Dzionek-Kozlowska, J. (2015). Rompecabezas de Alfred Marshall: entre la economía como ciencia positiva y la caballerosidad económica. Documentos de trabajo económico de Lodz , 5.
Google Académico
Einaudi, L. (1967 {[}1944{]}). Lezioni di politica sociale . Turín: Einaudi.
Google Académico
Fawcett, E. (2014). Liberalismo: la vida de una idea . Princeton, Oxford: Prensa de la Universidad de Princeton.
Google Académico
Findlay Shirras, G. (1935). La ley de Pareto y la distribución de la renta. The Economic Journal, 45 (180), 663--681.
Google Académico
Gioja, M. (1831 {[}1797{]}). Dissertazione sul problema quale dei governi liberi meglio convenga alla felicità dell'Italia . G. Ruggia.
Google Académico
Grimmer-Solem, E. (2003). El auge de la economía histórica y la reforma social en Alemania 1864-1894 . Oxford: Clarendon Press.
CrossRefGoogle Académico
Hawes, J. (2014). Ingleses y hunos . Londres, Nueva York: Simon \& Schuster.
Google Académico
Hayek, F. (1955). La contrarrevolución de la ciencia: estudios sobre el abuso de la razón . Nueva York: The Free Press.
Google Académico
Hegel, GF (1899 {[}1837{]}). La Filosofía de la Historia . Nueva York: The colonial Press.
Google Académico
Heilbroner, R. (1988). Detrás del velo de la economía: ensayos en la filosofía mundana . Londres, Nueva York: WW Norton.
Google Académico
Heilbroner, R. y Milberg, W. (1995). La crisis de visión en el pensamiento económico moderno . Cambridge: Cambridge University Press.
Google Académico
Hobsbawm, E. (1997). Sobre la historia . Londres: Weidenfeld y Nicolson.
Google Académico
Hume, D. (1938 {[}1740{]}). Resumen de un tratado de naturaleza humana e. Cambridge: Cambridge University Press.
Google Académico
Hume, D. (sin fecha {[}1770{]}). Discursos políticos . Londres, Nueva York: The Walter Scott Publishing.
Google Académico
Israel, J. (2006). Ilustración impugnada: filosofía, modernidad y emancipación del hombre, 1670-1752 . Oxford, Nueva York: Oxford University Press.
Google Académico
Israel, J. (2010). Una revolución de la mente: Ilustración radical y el origen intelectual de la democracia moderna . Princeton, Oxford: Prensa de la Universidad de Princeton.
Google Académico
Israel, J. (2014). Idea revolucionaria: una historia intelectual de la Revolución francesa desde los derechos del hombre hasta Robespierre . Princeton: Prensa de la Universidad de Princeton.
Google Académico
Jaffé, W. (1977). El sesgo normativo del modelo walrasiano: Walras versus Gossen. The Quarterly Journal of Economics, 91 (3), 371--387.
Google Académico
Jevons, WS (1965 {[}1871-1888{]}). La teoría de la economía política . Nueva York: A. Kelley.
Google Académico
Keynes, JM (1926). El fin del Laissez-Faire . Londres: Hogarth Press.
Google Académico
Lasky, H. (1948). Manifiesto comunista, hito socialista: una nueva apreciación . Londres: George Allen y Unwin.
Google Académico
List, F. (1885 {[}1844{]}). El Sistema Nacional de Economía Política . Londres: Longman, Green \& Co.
Google Académico
Locke, J. (1690). Segundo Tratado de Gobierno ( www.gutemberg.org ).
Lunghini, G. (2001). David Ricardo: La storia come ordine naturale. Rivista di storia economica, 17 (2), 259-269.
Google Académico
Lunghini, G. (2003). Benedetto Croce e l'economia politica. Economia politica, 2, 185-200.
Google Académico
Maddison, A. (2001). La economía mundial: una perspectiva milenaria . París: OCDE.
CrossRefGoogle Académico
Malthus, TR (1926 {[}1798{]}). Primer ensayo sobre población . Londres: Macmillan.
Google Académico
Marshall, A. (1966 {[}1890{]}). Principios de economía . Londres: Macmillan.
Google Académico
Marx, K. (sin fecha {[}1867{]}). El capital: una crítica de la economía política (Vol. I). Moscú: Progress Publishers ( Marxist.org ).
Marx, K. (1956 {[}1885{]}). El capital: una crítica de la economía política (Vol. II). Moscú: Progress Publishers ( libcom.org ).
Marx, K. (sin fecha {[}1894{]}). Capital: una crítica de la economía política (Vol. III). Nueva York: International Publishers ( marxists.org ).
Marx, K. y Engels, F. (1948 {[}1848{]}). Manifiesto del Partido Comunista. En H. Lasky (Eds.), Communist Manifesto, Socialist Landmark, a New Appreciation . Londres: Allen y Unwin.
Google Académico
Mill, JS (2010 {[}1859{]}). Sobre la libertad . Harmondsworth: pingüino.
Google Académico
Mill, JS (2018 {[}1865{]}). Auguste Comte y el positivismo . Cumplimiento de Amazon.
Google Académico
Naldi, N. (2000, octubre). La amistad entre Piero Sraffa y Antonio Gramsci en los años 1919-1927. Revista europea de historia del pensamiento económico, 7, 2.
Google Académico
Norman, J. (2018). Adam Smith: lo que pensó y por qué es importante . Londres: Allen Lane.
Google Académico
Pareto, V. (1919). Manuale di economia politica, con un'introduzione alla scienza sociale . Milán: Societa `Editrice Libraria.
Google Académico
Pareto, V. (1971). Corso di economia politica , Libro III, Sezione 964. Turín: UTET ( Cours d'Économie Politique, Rouge, Pichon, 1897).
Google Académico
Picketty, T. (2014). Capital en el siglo XXI . Cambridge: The Belknap Press de Harvard University Press.
CrossRefGoogle Académico
Pierenkemper, T. y Tilly, R. (2004). La economía alemana durante el siglo XIX . Nueva York: Berghahn Books.
Google Académico
Pigou, AC (2013 {[}1920{]}). La economía del bienestar . Editado por N. Aslanbeigui \& G Oakes. Londres: Palgrave Macmillan.
Google Académico
Plumpe, W. (2016). Historia económica y empresarial alemana en los siglos XIX y XX . Londres: Palgrave Macmillan.
Google Académico
Porter, R. (2000). Ilustración: Gran Bretaña y la creación del mundo moderno . Londres: Allen Lane.
Google Académico
Ricardo, D. (1810). El alto precio de los lingotes: una prueba de la depreciación de los billetes de banco . Londres: John Murray.
Google Académico
Ricardo, D. (2004 {[}1817{]}). Los principios de la economía política y la fiscalidad . Mineola: Dover.
Google Académico
Robinson, J. (1961). Preludio de una crítica de la teoría económica. Documentos económicos de Oxford, 13 (1), 53--58.
Google Académico
Robinson, J. (1974 {[}1962{]}). Filosofía económica . Nueva York: Penguin.
Google Académico
Rousseau, JJ (1913 {[}1762{]}). El contrato social y los discursos . Londres: JM Dent.
Google Académico
Schmoller, G. (1904 {[}1890{]}). Lineamenti di economia nazionale generale . Turín: UTET.
Google Académico
Schmoller, G. (1967 {[}1884{]}). El sistema mercantil y su trascendencia histórica. Nueva York: August Kelley.
Google Académico
Schumpeter, JA (1942). Capitalismo, socialismo, democracia . Nueva York: Harper \& Brothers.
Google Académico
Schumpeter, JA (1954 {[}1912{]}). Doctrina y método económicos: un bosquejo histórico. Nueva York: Oxford University Press.
Google Académico
Schumpeter, JA (1997 {[}1952{]}). Diez grandes economistas: de Marx a Keynes . Londres: Routledge.
Google Académico
Sen, A. (1970). La imposibilidad de un liberal paretiano. Revista de Economía Política, 78 (1), 152-157.
Google Académico
Grillete, GLS (1980). Evolución del pensamiento en economía. Revista trimestral de la Banca Nazionale del Lavoro , 132.
Google Académico
Smith, A. (1811 {[}1775{]}). Una investigación sobre la naturaleza y las causas de la riqueza de las naciones (edición de Playfair). Hartford: Oliver D. Cooke.
Google Académico
Smith, A. (1853 {[}1759{]}). La teoría de los sentimientos morales . Londres: Henry Bohn.
Google Académico
Spencer, H. (1992 {[}1889{]}). El hombre contra el estado . Carmel: Liberty Classics.
Google Académico
Spirito, U. (1939). Economia ed etica nel pensiero di Hegel. En Dall'economia liberale al corporativismo. Critica dell'economia liberale . Milán: Principato.
Google Académico
Sraffa, P. (2017). Lettere editoriali (1947-1975) . Turín: Einaudi.
Google Académico
Stein, H. (1994, 7 de marzo). Recordando a Adam Smith. Wall Street Journal .
Google Académico
Stolper, G. (1967). La economía alemana desde 1870 hasta la actualidad . Londres: Weidenfeld y Nicolson.
Google Académico
Streek, W. (2017). ¿Cómo terminará el capitalismo? Londres, Nueva York: Verso.
Google Académico
Tarascio, VJ (1967). Léon Walras: con motivo de la publicación de su correspondencia y artículos relacionados. Southern Economic Journal, 34, 1.
Google Académico
Trevelyan, GM (1944). Historia social inglesa: una encuesta de seis siglos, Chaucer a Victoria . Londres: Longmans, Green and Co.
Google Académico
Tribu, K. (1995). Estrategias de orden económico: discurso económico alemán 1750-1950 . Cambridge: Cambridge University Press.
Google Académico
Walras, L. (1954 {[}1900{]}). Elementos de la economía pura o teoría de la riqueza social . Londres: Allen \& Unwin.
Google Académico
Zussman, A. (2002). El auge del proteccionismo alemán en la década de 1870: una perspectiva macroeconómica . Instituto de Stanford para la Investigación de Políticas Económicas, documento de debate 01-19.
Google Académico
Baffigi, A. (2009). Luigi Einaudi. Teoria economica e legislazione sociale nel testo delle Lezioni . Quaderni di storia economica della Banca d'Italia, settembre.
Google Académico
Barry, N. (1989). Pensamiento político y económico de los neoliberales alemanes. En W. Peacock (Eds.), Los neoliberales alemanes y la economía social de mercado. Nueva York, NY: Pralgrave Macmillan.
Google Académico
Beveridge, W. (1942) . Informe sobre Seguridad Social y Servicios Afines , HMSO.
Google Académico
Beveridge, W. (1944). Pleno empleo en una sociedad libre . Londres: Allen \& Unwin.
Google Académico
Blackett, B. (1932). Conferencia de Halley Stewart 1931. La crisis económica mundial y la vía de escape . Londres: Allen y Unwin.
Google Académico
Bodei, R. (2003). Il ruolo dell'economia en Croce. Economia politica, 2.
Google Académico
Bonefeld, W. (2012). Libertad, crisis y Estado fuerte: sobre el ordoliberalismo alemán. Nueva Economía Política, 17/5 . ( \url{https://eprints.whiterose.ac.uk} ).
Bonefeld, W. (2016, agosto). Liberalismo autoritario: de Schmitt al ordoliberalismo al euro. Sociología crítica .
Google Académico
Caldwell, PC (2005). Controversias sobre Carl Schmitt: una revisión de la literatura reciente. La Revista de Historia Moderna, 77 (2).
Google Académico
Croce, B. (1941). Materialismo storico ed economia marxista . Bari: Laterza.
Google Académico
Croce, B. (1955 {[}1925{]}). Cultura e vita moral. Intermezzi polemici , serie Nuova, XV, Liberalismo (p.~283). Bari: Laterza.
Google Académico
Croce, B. (1973 {[}1908{]}). Filosofia della pratica. Economia ed etica . Bari: Laterza.
Google Académico
Croce, B. (2015 {[}publicado originalmente como ensayo único, 1927{]}). Liberismo e liberalismo . Etica e politica .
Google Académico
Einaudi, L. (1918). La nuova economia. La Riforma sociale, XXIX.
Google Académico
Einaudi, L. (1964). Lezioni di politica sociale. Turín: Einaudi. Instituto de Nuevo Pensamiento Económico. La tradición italiana . www.hetwebsite.net/net/schools/italian.html .
Einaudi, L. (1972 {[}1924{]}). La bellezza della lotta. En Le lotte del lavoro . Turín: Einaudi.
Google Académico
Einaudi, L., Faucci, R. y Marchionatti, R. (2006). Introducción a Luigi Einaudi, Ensayos económicos seleccionados . Londres: Palgrave Macmillan.
Google Académico
Euken, W. (1948). ¿Qué tipo de sistema económico y social? En Peacock \& Willgerodt (Eds.), Economía social de mercado de Alemania: orígenes y evolución . Londres: Macmillan.
Google Académico
Euken, W. (1951). Esta era sin éxito o los dolores del progreso económico . Edinburg: William Hodge.
Google Académico
Exámenes y expectativas. (2016, 24 de diciembre). The Economist .
Google Académico
Fisher, I. (1935). 100\% dinero . Nueva York: Adelphi.
Google Académico
Friedman, M. (1982 {[}1962{]}). Capitalismo y Libertad . Chicago: Prensa de la Universidad de Chicago.
Google Académico
Hartford, T. (2018, 28 y 29 de abril). Cómo poner precio a las redes sociales. The Financial Times .
Google Académico
Hawtrey, RG (1931). El patrón de oro: teoría y práctica . Londres: Longman, Green and Company.
Google Académico
Hayek, FA (1945). El uso del conocimiento en la sociedad. Revista Económica Estadounidense, XXXV.
Google Académico
Hayek, FA (1955). La contrarrevolución de la ciencia. Estudios sobre el abuso de la razón . London Free Press de Glencoe: Collier-Macmillan.
Google Académico
Hayek, FA (2008 {[}1944{]}). El camino de la servidumbre . Londres: Routledge.
Google Académico
Hayek, FA Introducción a los principios económicos de Carl Menger .
Google Académico
Heilbroner, R. y Milberg, W. (1995). La crisis de visión en el pensamiento económico moderno . Cambridge: Cambridge University Press.
Google Académico
Hoerber, T. (2017). Hayek contra Keynes. Una batalla de ideas . Londres: Reaktion Books.
Google Académico
Jones, T. (2013). ¿'Reacio' o colectivista liberal? El liberalismo social de Keynes y Beveridge. Revista de Historia Liberal, 78 (primavera).
Google Académico
Kaldor, N. Apéndice C de Pleno empleo de Beveridge .
Google Académico
Kelly, D. (2019, 29 de abril). Momentos maltusianos en la obra de John Maynard Keynes. El diario histórico .
Google Académico
Keynes, JM (agosto de 1925). ¿Soy liberal? La nación y el Ateneo.
Google Académico
Keynes, JM (1926). El fin del Laissez-Faire . Londres: The Hogarth Press.
Google Académico
Keynes, JM (junio de 1933). Autosuficiencia Nacional. The Yale Review, 22 (4).
Google Académico
Keynes, JM (1949 {[}1938{]}). Dos memorias . Londres: Rupert Hart ‐ Davis.
Google Académico
Keynes, JM (1964 {[}1936{]}). La teoría general del interés laboral y el dinero . Londres: Macmillan.
Google Académico
Klein, P. Prólogo a los Principios de economía de Carl Menger .
Google Académico
Lerner, A. (1944). Principios de la economía del bienestar . Nueva York: Macmillan.
Google Académico
Lerner, A. (1946). Economía del control . Nueva York: Macmillan.
Google Académico
Lerner, A. (1951). Economía del empleo . Nueva York: McGraw Hill.
Google Académico
Lutz, F. 1935. El funcionamiento del patrón oro. En Peacock \& Willgerodt (Eds.), Economía social de mercado de Alemania: orígenes y evolución . Londres: Macmillan.
Google Académico
Menger, C. (1976 {[}1871{]}). Principios de economía . Auburn: Instituto Ludwig von Mises.
Google Académico
Moggridge, D. (ed.). (1973). Los escritos recopilados de John Maynard Keynes , vol.~XIV. Londres: Macmillan.
Google Académico
Montesano, A. (2003). Croce e la scienza economica. Economia politica , 2.
Google Académico
Müller-Armack, A. (1956). El significado de la economía social de mercado. En Peacock \& Willgerodt (Eds.), Economía social de mercado de Alemania: orígenes y evolución . Londres: Macmillan.
Google Académico
Peacock, A. y Willgerodt, H. (1989). Liberalismo alemán y renacimiento económico. En Peacock \& Willgerodt (Eds.), Economía social de mercado de Alemania: orígenes y evolución . Londres: Macmillan.
Google Académico
Pigou, AC (2013 {[}1920{]}). La economía del bienestar . Londres: Palgrave Macmillan.
Google Académico
Rathenau, W. (1919). L'economia nuova {[}Die Neue Wirtschaft, 1918{]} . Bari: Laterza.
Google Académico
Robbins, L. (1963). Política y Economía. Papeles en Economía Política . Londres: Macmillan.
Google Académico
Robinson, J. (1974 {[}1962{]}). Filosofía económica . Londres: Penguin.
Google Académico
Röpke, W. (1951). Interdependencia de los sistemas económicos nacionales e internacionales. En Peacock \& Willgerodt (Eds.), Economía social de mercado de Alemania: orígenes y evolución . Londres: Macmillan.
Google Académico
Röpke, W. (1957). Welfare Freedom and Inflation, reimpreso en dos ensayos. El problema del orden económico. Welfare Freedom and Inflation , University Press of America, 1987.
Google Académico
Schmidt, K. (1956). El sector público en una economía de mercado. En Peacock \& Willgerodt (Eds.), Economía social de mercado de Alemania: orígenes y evolución . Londres: Macmillan.
Google Académico
Schumpeter, JA (1947). Capitalismo, socialismo y democracia . Nueva York: Harper and Brothers.
Google Académico
Simons, H. (1948). Los requisitos de la libre competencia (1936). En Política Económica.
Google Académico
Simons, H. (1948). Un programa positivo para el Laissez Faire. Algunas propuestas para una política económica liberal (1934). En Política Económica .
Google Académico
Simons, H. (1948). Introducción: un credo político (1945). En Política económica para una sociedad libre . Chicago: Prensa de la Universidad de Chicago.
Google Académico
Simons, H. (1948). El programa Beveridge: una interpretación antipática (1945), en Journal of Political Economy, 53 (3).
Google Académico
Skidelsky, R. (1992). John Maynard Keynes: El economista como salvador , Vol. II de la biografía de Keynes de Skidelsky. Londres: Macmillan.
Google Académico
Skidelsky, R. (2018). Dinero y gobierno. Un desafío para la corriente principal de la economía . Londres: Allen Lane.
Google Académico
Steil, B. (2013). La batalla de Bretton Woods: John Maynard Keynes, Harry Dexter White y la creación de un nuevo orden mundial. Princeton-Oxford: Prensa de la Universidad de Princeton.
Google Académico
Stein, H. (1990). El triunfo de la sociedad adaptativa, en The American Economist, XXXIV , 1.
Google Académico
Streek, W. (2015). Heller, Schmitt y el euro, en European Law Journal, 21 (3).
Google Académico
Streek, W. (4 de mayo de 2017). Jugando a ponerse al día. London Review of Books .
Google Académico
The Economist . (2015, 9 de mayo). De reglas y orden.
Google Académico
Trevelyan, GM (1944). Historia social inglesa: una encuesta de seis siglos, Chaucer a Victoria . Londres, Nueva York y Toronto: Longmans, Green and Co.
Google Académico
Yagi, K. (1997). Carl Menger y el historicismo en economía. En P. Koslowsy (Ed.), Metodología de las Ciencias Sociales. Berlín-Heidelberg: Springer.
Google Académico
Zettelmeier, J. (noviembre de 2017). Ordo alemán y la reforma de la eurozona: una mirada desde las trincheras. En T. Beck \& HH Kotz (Eds.), Ordoliberalism: A German Oddity? Londres: CEPR Press.
Google Académico
Baran, P. y Sweezy, PM (1966). Capital de monopolio. Un ensayo sobre el orden económico y social estadounidense . Nueva York: Monthly Review Press.
Google Académico
Beveridge, W. (1944). Pleno empleo en una sociedad libre . Londres: Allen y Unwin.
Google Académico
Carr, EH (1958). El socialismo en un solo país de 1924 - 1926 . Londres: Macmillan.
Google Académico
Ciocca, P. (2007). Ricchi por semper? Una storia economica d'Italia . Turín: Bollati Boringhieri.
Google Académico
Conti, E. (1986). Dal taccuino di un borghese (pág. 431). Bolonia: Il Mulino.
Google Académico
D'Alfonso, R. (2004). Costruire lo Stato forte. Politica, diritto, economia en Alfredo Rocco . Milán: Franco Angeli.
Google Académico
De Felice, R. (1968). Mussolini il fascista. L'organizzazione dello Stato fascista 1925-1929 . Turín: Einaudi.
Google Académico
Dobb, M. (1937a). La cuestión del derecho económico en una economía socialista, en la economía política y el capitalismo. Algunos ensayos sobre la tradición económica . Londres: Routledge.
Google Académico
Dobb, M. (1937b). Las tendencias de la economía moderna, en economía política y capitalismo. Algunos ensayos sobre la tradición económica . Londres: Routledge
Google Académico
Fenoaltea, S. (2011). La reinterpretación de la historia económica italiana, desde la unificación hasta la Gran Guerra . Cambridge: Cambridge University Press.
Google Académico
Galli, C. (2010). Carl Schmitt nella cultura italiana (1924-1978). Storia, bilancio, prospettive di una presenza problematica. Históricamente, 6 .
Google Académico
Gregor, AJ (2005). Intelectuales de Mussolini. Pensamiento social y político fascista . Princeton y Oxford: Princeton University Press.
Google Académico
Gregory, BS (2012). La reforma involuntaria. Cómo una revolución religiosa secularizó la sociedad . Cambridge: Belknap Press de Harvard University Press.
CrossRefGoogle Académico
Guerin, D. (1939). Fascismo y grandes empresas . Nueva York: Pioneer Publishers.
Google Académico
Harrison, M. (mayo de 2017). La economía soviética: su vida y más allá (Warwick Economics Research Papers, n.~1137).
Google Académico
Harrod, R. (1961). Examen de la producción de productos básicos por medio de productos básicos. The Economic Journal, 71 (284).
Google Académico
Heilbroner, R. y Milberg, W. (1995). La crisis de visión en el pensamiento económico moderno . Cambridge: Cambridge University Press.
Google Académico
Heilbroner, RL (1988). Detrás del velo de la economía. Ensayos de filosofía mundana . Nueva York: WW Norton.
Google Académico
Kalecki, M. (1943). Aspectos políticos del pleno empleo. Political Quarterly, 14 ; reimpreso en Hunt, EK y Schwartz, JG (Eds.). (1972). Una crítica de la teoría económica . Harmondsworth: pingüino.
Google Académico
Kaser, M. (1971). Planificación y Relaciones con el Mercado. Actas de una conferencia celebrada por la Asociación Económica Internacional en Liblice, Checoslovaquia . Londres: Macmillan.
Google Académico
Keynes, JM (diciembre de 1934). Intervención en la charla de Stalin-Wells . El nuevo estadista y nación .
Google Académico
Keynes, JM (1973). Carta a GB Shaw, 1 de enero de 1935. En D. Moggridge (Ed.), The Collected Writings (Vol. XIII). Londres: Macmillan.
Google Académico
Lange, O. (1959). La economía política del socialismo. Ciencia y Sociedad, 23 (1).
Google Académico
Lange, O. (1967). La computadora y el mercado. En CH Feinstein (Ed.), Socialismo, capitalismo y crecimiento económico. Ensayos presentados a Maurice Dobb . Cambridge: Cambridge University Press.
Google Académico
Leijonhufvud, A. (1968). Sobre la economía keynesiana y la economía de Keynes . Nueva York: Oxford University Press.
Google Académico
Maddison, A. (2003). La economía mundial. Estadísticas históricas . París: OCDE.
CrossRefGoogle Académico
Marx, K. (sin fecha {[}1867{]}). Capital. Crítica de la economía política, vol . I. Moscú: Progress Publishers (Marxist.org).
Google Académico
Mazzucato, M. (2018). El valor de todo. Hacer y tomar en la economía global . Londres: Allen Lane.
Google Académico
Meek, R. (1961). Rehabilitación de la economía clásica del Sr.~Sraffa. Revista escocesa de economía política, 8 (1), 119-136.
Google Académico
Meek, R. (1964). Juicios de valor en economía. The British Journal for the Philosophy of Science, XV (18).
Google Académico
Merlini, S. (1995). Il Governno Costituzionale. En R. Romanelli (Ed.), Storia dello Stato italiano dall'unita `a oggi . Donzelli: Roma.
Google Académico
Naldi, N. (2000, octubre). La amistad entre Piero Sraffa y Antonio Gramsci en los años 1919-1927. Revista europea de historia del pensamiento económico, 7 (2).
Google Académico
Napoleoni, C. (1963). Il pensiero economico del '900 . Turín: Einaudi.
Google Académico
Negri Zamagni, V. (2019, 12 de diciembre). Quanto corporativa fu l'economia italiana negli anni 1930? Ponencia presentada en la conferencia Economia ed economisti del periodo fascista, Accademia Nazionale dei Lincei.
Google Académico
Papi, GU (1958). Principii di economia . Padua: Cedam.
Google Académico
Paxton, RO (2004). La anatomía del fascismo . Londres: Allen Lane.
Google Académico
Polanyi, K. (1957 {[}1944{]}). La gran transformación. Los orígenes políticos y económicos de nuestro tiempo . Boston: Beacon Press.
Google Académico
Robinson, J. (1967). Afluencia socialista. En CH Feinstein (Ed.), Socialismo, capitalismo y crecimiento económico. Ensayos presentados a Maurice Dobb . Cambridge: Cambridge University Press.
Google Académico
Robinson, J. (1972). Preludio de una crítica de la teoría económica. En EK Hunt y JG Schwartz (Eds.), Una crítica de la teoría económica . Harmondsworth: pingüino.
Google Académico
Rocco, A., Carli, F. (1914). I principi fondamentali del nazionalismo economico. En Associazione nazionalista italiana: Il nazionalismo economico . Bolonia: Paolo Neri.
Google Académico
Roncaglia, A. (2009). La riqueza de las ideas. Una historia del pensamiento económico . Cambridge: Cambridge University Press.
Google Académico
Russell, B. (1920). La práctica y la teoría del bolchevismo . Londres: Universidad de Libros.
Google Académico
Schlesinger, R. (1947). El espíritu de la Rusia de la posguerra. La ideología soviética 1917 - 1946 . Londres: Dennis Dobson.
Google Académico
Schumpeter, JA (1947). Capitalismo, socialismo y democracia . Nueva York y Londres: Harper and Brothers.
Google Académico
Schumpeter, JA (1954). Historia del análisis económico . Nueva York: Oxford University Press.
Google Académico
Schwartz, H. (1968). Introducción a la economía soviética . Colón: Charles E. Merrill.
Google Académico
Shackle, GLS (1963 {[}1990{]}). Esquemas generales de pensamiento y el economista. En JL Ford (Ed.), Tiempo, expectativas e incertidumbre en economía. Ensayos seleccionados de GLS Shackle . Brookfield: Edward Elgar.
Google Académico
Smith, A. (1853 {[}1759{]}). La teoría de los sentimientos morales . Londres: Henry Bohn.
Google Académico
Spirito, U. (1933). L'economia programmatica corporativa. En Capitalismo e corporativismo. Firenze: Sansoni.
Google Académico
Spirito, U. (1939). Politica ed economia corporativa. En Dall'economia liberale al corporativismo. Critica dell'economia liberale . Mesina-Milán: Principato.
Google Académico
Sraffa, P. (1951-1955 y 1973). David Ricardo. Obras y correspondencia. Cambridge: Cambridge University Press.
Google Académico
Sraffa, P. (1960). Producción de productos básicos por medio de productos básicos. Preludio de una crítica de la teoría económica . Cambridge: Cambridge University Press.
Google Académico
Sraffa, P. (2017). Lettere editoriali (1947-1975) . Turín: Einaudi.
Google Académico
Stalin, JV (1972 {[}1952{]}). Problemas económicos del socialismo en la URSS . Pekín: Prensa extranjera.
Google Académico
Streeck, W. (2016). ¿Cómo terminará el capitalismo? Londres: Verso.
Google Académico
Sylos Labini, P. (1975). Saggio sulle classi sociali . Roma-Bari: Laterza.
Google Académico
Toniolo, G. (1980). L'economia dell'Italia fascista . Roma-Bari: Laterza.
Google Académico
Von Mises, L. (1951). Socialismo. Un análisis económico y social . New Haven: Prensa de la Universidad de Yale.
Google Académico
Webb, SB (1944 {[}1935{]}). Comunismo soviético. Una civilización de noticias. Londres: Longmans, Green and Co.
Google Académico
Weber, M. (1930 {[}1905{]}). La ética protestante y el espíritu del capitalismo. Londres: Allen \& Unwin.
Google Académico
Acharia, V., Philippon, T., Richardson, M. y Roubini, N. (2009). Prólogo: A vista de pájaro. En V. Acharya y M. Richardson (Eds.), Restauración de la estabilidad financiera. Cómo reparar un sistema averiado (pág. 5). Hoboken: Wiley.
Google Académico
Anderson, P. (1978, otoño). Expectativas racionales: qué importancia tiene para el análisis de políticas econométricas. Revisión trimestral del Banco de la Reserva Federal de Minneapolis, 2 .
Google Académico
Axelrod, SH (2011). Dentro de la Fed. Política monetaria y su gestión, de Martin a través de Greenspan a Bernanke . Londres: The MIT Press.
Google Académico
Boettke, P. (2011). Enseñar economía, apreciar el orden espontáneo y la economía como ciencia pública. Journal of Economic Behavior and Organisation, 80, 265-274.
Google Académico
Buchanan, J. (1959, octubre). Economía positiva, economía del bienestar y economía política. Revista de Derecho y Economía, II, 124-138.
Google Académico
Buchanan, J. (1990). El dominio de la economía constitucional. Economía política constitucional, 1 (1), 1-18.
Google Académico
Buchanan, J. (1999 {[}1969{]}). Las obras completas de James Buchanan, vol.~6, Costo y elección (págs. 41--42). Indianápolis: Liberty Fund.
Google Académico
Buchanan, J. (2001). Democracia directa, liberalismo clásico y estrategia constitucional. Kyklos, 54 (fasc 2/3), 235--242.
Google Académico
Buchanan, JM y Tullock, G. (1965 {[}1962{]}). El cálculo del consentimiento. Fundamentos lógicos de la democracia constitucional (págs. 111--112). Ann Arbor: Prensa de la Universidad de Michigan.
Google Académico
Cassidy, J. (2010, 13 de enero). Entrevista a Eugene Fama. The New Yorker .
Google Académico
Collier, P. (2018). El futuro del capitalismo, frente a nuevas ansiedades . Londres: Allen Lane.
Google Académico
Deaton, A. (2020). Desigualdad en Cambridge y Chicago (p.~2). www.project-syndicate.org .
de Voltaire, JF (1937 {[}1759{]}). Cándido . Nueva York: Halcyon House.
Google Académico
Eichengreen, B. (2018). La tentación populista. Reclamación económica y reacción política en la era moderna . Oxford: Prensa de la Universidad de Oxford.
Google Académico
Friedman, M. (1968 {[}1964{]}). Tendencias de posguerra en teoría y política monetarias. En J. Lindauer (Eds.), Lecturas macroeconómicas (págs. 357--359). Nueva York: The Free Press.
Google Académico
Fukuyama, F. (1989, verano). ¿El fin de la historia? El interés nacional .
Google Académico
Glaser, E. (22 de marzo de 2014). Recupera la ideología. The Guardian .
Google Académico
Goodhart, D. (2017). El camino a algún lugar: la revuelta populista y el futuro de la política . Londres: C. Hurst \& Co.
Google Académico
Harvey, D. (2007). Una breve historia del neoliberalismo . Oxford: Prensa de la Universidad de Oxford.
Google Académico
Heilbroner, R. y Milberg, W. (1995). La crisis de visión en el pensamiento económico moderno . Cambridge: Cambridge University Press.
Google Académico
Hobbes, T. (1909 {[}1651{]}). Leviatán . Oxford: Prensa de la Universidad de Oxford.
Google Académico
Israel, J. (2014). Ideas revolucionarias. Una historia intelectual de la Revolución francesa desde 'Los derechos de los hombres' hasta Robespierre (p.~21). Princeton: Prensa de la Universidad de Princeton.
Google Académico
Johnson, HG (1965). Un modelo teórico de nacionalismo económico en los Estados nuevos y en desarrollo. Political Science Quarterly, LXXX (2), 169--185.
Google Académico
Johnson, HG (1971). La revolución keynesiana y la contrarrevolución monetarista. American Economic Review, 61 (2), 1-14.
Google Académico
Kant. I. (1981 {[}1785{]}). Base para la metafísica de la moral (págs. 12-13). Indianápolis: Hackett.
Google Académico
Kelly, D. (31 de julio de 2015). Reseña del libro 'Política en la sociedad comercial: J.-J. Rousseau y A. Smith », de I. Hont. Financial Times .
Google Académico
Keynes, JM (1964 {[}1936{]}). La teoría general del interés laboral y el dinero . Londres: Macmillan.
Google Académico
Keynes, JM (1937). La Teoría General del Empleo. The Quarterly Journal of Economics, 51 (2), 209-223.
Google Académico
Kuper, S. (2020, 15-16 de febrero). La venganza de los antielitistas de clase media. Financial Times .
Google Académico
Leijonhufvud, A. (1983). ¿Qué habría pensado Keynes de las expectativas racionales? (Discusión Beiträge - Serie A, no 177). Universität Konstanz.
Google Académico
Levi-Faur, D. (1997). Nacionalismo económico: de Friedrich List a Robert Reich. Review of International Studies, 23, 359--370.
Google Académico
Lucas, R. y Sargent, T. (1979, primavera). Después de la macroeconomía keynesiana. Revisión trimestral del Banco de la Reserva Federal de Minneapolis, 3 .
Google Académico
Musgrave, RA (1969). Sistemas fiscales . New Haven y Londres: Yale University Press.
Google Académico
Posner, R. (1987, noviembre). La Constitución como documento económico. Revista de leyes de George Washington, 56 .
Google Académico
Posner, R. (2009). Un fracaso del capitalismo. La crisis del 2008 y el descenso a la depresión . Cambridge y Londres: Harvard University Press.
Google Académico
Pringle, R. (2020). El poder del dinero. Cómo las ideas sobre el dinero dieron forma al mundo moderno . Cham: Palgrave Macmillan.
Google Académico
Rachman, G. (2019, 12 al 13 de enero). El enemigo más brillante del liberalismo vuelve a estar de moda. Financial Times .
Google Académico
Rawls, J. (1971). Una teoría de la justicia . Cambridge: The Belknap Press de Harvard University Press.
Google Académico
Rousseau, JJ (1913 {[}1762{]}). El contrato social y los discursos . Londres: JM Dent.
Google Académico
Runciman, D. (2018). Cómo termina la democracia . Londres: Profile Books.
Google Académico
Samuelson, P. (1997, 2 de octubre) ¿Dónde difieren los modelos europeos y estadounidenses? Discurso entregado en la Banca d'Italia, mimeo.
Google Académico
Schmitt, C. (2013 {[}1921{]}). Dictadura . Cambridge: Polity.
Google Académico
Schumpeter, JA (1954). Historia del análisis económico. Oxford: Prensa de la Universidad de Oxford.
Google Académico
Grillete, GLS (1953). ¿Qué hace a un economista? Liverpool: Prensa de la Universidad de Liverpool.
Google Académico
Shearmur, J. (2010). Preferencias, cognitivismo y esfera pública. En C. Favor, G. Gaus y J. Lamont (Eds.), Ensayos sobre filosofía, política y economía . Stanford: Prensa de la Universidad de Stanford.
Google Académico
Skidelsky, R. (2009). Keynes. El regreso del maestro . Londres: Allen Lane.
Google Académico
Stiglitz, JE (2009). Caida libre. Mercados libres y hundimiento de la economía global (p.~238). Londres: Penguin.
Google Académico
Streek, W. (2016). ¿Cómo terminará el capitalismo? Londres: Verso.
Google Académico
Traverso, E. (2016). Fuego y sangre. La Guerra Civil Europea 1914-1945 . Nueva York: Verso.
Google Académico
Volcker, P. (2018). Manteniéndolo. La búsqueda de dinero sólido y buen gobierno . Nueva York: Asuntos Públicos.
Google Académico
Zuckerberg, M. (2019, 17 de octubre). Discurso en la Universidad de Georgetown. Washington Post .
Google Académico
Berlín, I. y Jahanbegloo, R. (1991). Conversaciones con Isaiah Berlin . Hijos de Charles Scribner.
Google Académico
Croce, B. (17 de agosto de 1922). Una visión filosófica de la población. En Manchester Guardian Commercial, Sección 6, Reconstrucción en Europa .
Google Académico
Croce, B. (1949). Liberalismo y Democracia. En mi filosofía: ensayos sobre los problemas morales y políticos de nuestro tiempo . George Allen y Unwin.
Google Académico
Heilbroner, R. (1988). Detrás del velo de la economía. Ensayos de filosofía mundana . Nueva York: WW Norton.
Google Académico
Heilbroner, R., Milberg, W. (1995). La crisis de visión en el pensamiento económico moderno . Cambridge: Cambridge University Press.
Google Académico
Keller, R. (1983). Economía keynesiana e institucional. ¿Compatibilidad o complementariedad? Revista de cuestiones económicas, 17 (4).
Google Académico
Lunghini, G. (2003). Benedetto Croce e l'economia italiana. Economia politica, 2 (8).
Google Académico
Norte, D. (1971). Cambio institucional y crecimiento económico. Revista de Historia Económica, 31 (1).
Google Académico
Popper, K. (2002 {[}1957{]}). La pobreza del historicismo. Abington: Routledge.
Google Académico
Rachman, G. (2020, 12 de mayo). Los liberales deben prepararse para la lucha. Financial Times .
Google Académico
Roberts, DD (1987). Benedetto Croce y los usos del historicismo . Berkeley, Los Ángeles y Londres: University of California Press.
Google Académico
Stein, H. (1990, primavera). El triunfo de la sociedad adaptativa. El economista estadounidense .
Google Académico
Streek, W. (2016). ¿Cómo terminará el capitalismo? Londres y Nueva York.
Google Académico
Weinstein, MM (2007). Bases institucionales para el crecimiento capitalista: introducción y comentarios. En E. Sheshinski, RJ Strom y WJ Baumol (Eds.), Emprendimiento, innovación y el mecanismo de crecimiento de las economías de libre empresa . Princeton: Prensa de la Universidad de Princeton.
Google Académico
Whelan, C. (2012, mayo). Institucionalismo poskeynesiano después de la gran recesión (documento de trabajo núm. 724). Instituto de Economía Levy de Bard College.
Google Académico

\hypertarget{economuxeda-de-la-complejidad}{%
\section*{Economía de la complejidad}\label{economuxeda-de-la-complejidad}}
\addcontentsline{toc}{section}{Economía de la complejidad}

Aaronson, Scott. 2013. Por qué los filósofos deberían preocuparse por la complejidad computacional. En Computability: Turing, Gödel, Church y más allá , ed.~Jack Copeland, Carl J. Posy y Oron Shagrir, 261--328. Cambridge, MA: MIT Press.
Google Académico
Abraham, Ralph H. 1985. Chaostrophes, Intermittency, and Noise. En Chaos, Fractals and Dynamics , ed.~P. Fischer y WR Smith, 3--22. Nueva York: Marcel Dekker.
Google Académico
Abraham, Ralph y Christopher D. Shaw. 1987. Dinámica: una introducción visual. En Self-Organizing Systems: The Emergence of Order , ed.~F. Eugene Yates, 543--597. Nueva York: Plenum Press.
CrossRefGoogle Académico
Abraham, Ralph, Laura Gardini y Christian Mira. 1997. Caos en sistemas dinámicos discretos: una introducción visual en 2 dimensiones . Nueva York: Springer.
CrossRefGoogle Académico
Albin, Peter S. 1982. La Metalogic of Economic Predictions, Calculations and Propositions. Ciencias sociales matemáticas 3: 129-158.
CrossRefGoogle Académico
Albin, Peter S. con Duncan K. Foley. 1998. Barreras y límites a la racionalidad: ensayos sobre la complejidad económica y la dinámica en los sistemas interactivos . Princeton: Prensa de la Universidad de Princeton.
Google Académico
Allen, Timothy FH y Thomas W. Hoekstra. 1990. Hacia una ecología unificada . Nueva York: Columbia University Press.
Google Académico
Álvarez, M. Carrión y Dirk Ehnts. 2016. Samuelson y Davidson sobre Ergodicity: A Reformulation. Revista de economía poskeynesiana 39: 1-16.
CrossRefGoogle Académico
Arthur, W. Brian, Steven N. Durlauf y David A. Lane. 1997a. Introducción. En La economía como un sistema complejo en evolución II , ed.~W. Brian Arthur, Steven N. Durlauf y David A. Lane, 1--14. Reading, MA: Addison-Wesley.
Google Académico
Azariadis, Costas. 1981. Profecías autocumplidas. Revista de teoría económica 25: 380--396.
CrossRefGoogle Académico
Bacharach, Michael y Dale O. Stahl. 2000. Teoría de Marco Variable Nivel-n.~Juegos y comportamiento económico 32, 220--246.
Google Académico
Bak, Per. 1996. Cómo funciona la naturaleza: la ciencia de la criticidad autoorganizada . Nueva York: Copernicus Press para Springer-Verlag.
CrossRefGoogle Académico
Bartholo, Robert S., Carios AN Cosenza, Francisco A. Doria y Carlos TR Lessa. 2009. ¿Se pueden considerar los sistemas económicos como dispositivos informáticos? Revista de comportamiento económico y organización 70: 72--80.
CrossRefGoogle Académico
von Bertalanffy, Ludwig. 1950. Un esquema de la teoría de sistemas generales. Revista británica de filosofía de la ciencia 1: 114-129.
CrossRefGoogle Académico
---------. 1974. Perspectivas sobre la teoría general de sistemas . Nueva York: Braziller.
Google Académico
Binmore, Ken. 1987. Modelado de jugadores racionales, I. Economía y filosofía 3: 9--55.
CrossRefGoogle Académico
Binmore, Ken y Larry Samuelson. 1999. Selección de equilibrio y deriva evolutiva. Review of Economic Studies 66: 363--394.
CrossRefGoogle Académico
Birkhoff, George D. 1931. Prueba del teorema ergódico. Actas de la Academia Nacional de Ciencias 17: 656--660.
CrossRefGoogle Académico
Bishop, Errett A. 1967. Fundamentos del análisis constructivo . Nueva York: McGraw-Hill.
Google Académico
Blum, Lenore, Felipe Cucker, Michael Shub y Steve Smale. 1998. Complejidad y Computación Real . Nueva York: Springer-Verlag.
CrossRefGoogle Académico
Bogdanov, Aleksandr A. 1925-29. Tektologia: Vsobschaya Oranizatsionnay Nauka, Volúmenes 1-III . 3ª ed.~Leningrado-Moscú: Kniga.
Google Académico
Boltzmann, Ludwig. 1884. über die Eigenschaften Monocyklischer und andere damit verwander Systeme. Diario de Crelle für due reine und augwandi Matematik 100: 201--212.
Google Académico
Boulding, Kenneth E. 1978. Ecodinámica: una nueva teoría de la evolución social . Beverly Hills: Sage.
Google Académico
Braudel, Fernand. 1967. Civilization Matérielle et Capitalisme . París: Librairie Armand Colin. (Traducción al inglés: K. Miriam. 1973. Capitalism and Material Life . Nueva York: Harper and Row).
Google Académico
Brock, William A. y Cars H. Hommes. 1997. Una ruta racional hacia la aleatoriedad. Econometrica 65: 1059--1095.
CrossRefGoogle Académico
---------. 1998. Creencias heterogéneas y rutas al caos en un modelo simple de fijación de precios de activos. Journal of Economic Dynamics and Control 22: 1235-1274.
CrossRefGoogle Académico
Brouwer, Luitzen EJ 1908. De Onbetrouwbaarheid der Logische Principes. Tijdschrift voor wijsbegeerte 2: 152-158.
Google Académico
---------. 1952. Una corrección intuicionista del teorema del punto fijo en la esfera. Actas de la Royal Society London 213: 1--2.
Google Académico
Caldwell, Bruce. 2004. El desafío de Hayek: una biografía intelectual de FA Hayek . Chicago: Prensa de la Universidad de Chicago.
Google Académico
---------. 2013. George Soros: ¿Hayekian? Revista de metodología económica 20: 350--356.
CrossRefGoogle Académico
Canning, David. 1992. Racionalidad, computabilidad y equilibrio de Nash. Econometrica 60: 877--888.
CrossRefGoogle Académico
Chaitin, Gregory J. 1966. Sobre la duración de los programas para calcular secuencias binarias finitas. Journal of the ACM 13: 547--569.
CrossRefGoogle Académico
---------. 1987. Teoría algorítmica de la información . Cambridge, Reino Unido: Cambridge University Press.
CrossRefGoogle Académico
Chomsky, Noam. 1959. Sobre ciertas propiedades formales de las gramáticas. Información y control 2: 137-167.
CrossRefGoogle Académico
Iglesia, Alonzo. 1936. Una nota sobre el problema de Entscheidung. Journal of Symbolic Logic 1: 40--41, corrección 101--102.
CrossRefGoogle Académico
van Cleve, J. 1990. Magia o polvo mental: Panpsiquismo vs.~Emergencia. Perspectivas filosóficas 4: 214-226.
CrossRefGoogle Académico
Clower, Robert W. y Peter W. Howitt. 1978. La teoría de las transacciones de la demanda de dinero: una reconsideración. Journal of Political Economy 86: 449--465.
CrossRefGoogle Académico
Cobb, L., P. Koppstein y NH Chen. 1983. Estimación y relaciones de recursividad de momento para distribuciones multimodales de la familia exponencial. Revista de la Asociación Estadounidense de Estadística 78: 124--130.
CrossRefGoogle Académico
Costa, da, CA Newton y Francisco A. Doria. 2005. Computing the Future. En Computability, Complexity and Constructivity in Economic Analysis , ed.~K. Vela Vellupilai, 15--50. Victoria: Blackwell.
Google Académico
---------. 2016. Sobre el algoritmo de O'Donnell para problemas completos NP. Review of Behavioral Economics 3: 221--242.
CrossRefGoogle Académico
Cotogno, Paolo. 2003. La hipercomputación y la tesis física de Church-Turing. Revista británica de filosofía de la ciencia 54: 181-223.
CrossRefGoogle Académico
Crutchfield, James P. 1994. Los cálculos de la emergencia: Computación, dinámica e inducción. Physica D 75: 11--54.
CrossRefGoogle Académico
Davidson, Paul. 1982-83. Expectativas racionales: una base falaz para estudiar los procesos cruciales de toma de decisiones económicas. Revista de economía poskeynesiana 5: 182-198.
CrossRefGoogle Académico
Davidson, Paul. 2015. Una réplica a la crítica de O'Donnell del enfoque ergódico / no ergódico del concepto de incertidumbre de Keynes. Revista de economía poskeynesiana 38: 1--18.
CrossRefGoogle Académico
Davis, John B. 2017. La relevancia continua del pensamiento filosófico de Keynes: reflexividad, complejidad e incertidumbre. Anales de la Fondazione Luigi Einaudi 51: 55--76.
Google Académico
Davis, John B. y Matthias Klaes. 2003. Reflexividad: ¿Maldición o cura? Journal of Economic Methodology 10: 329--352.
CrossRefGoogle Académico
Day, Richard H. 1994. Complex Economic Dynamics, Volumen I: Introducción a los sistemas dinámicos y los mecanismos de mercado . Cambridge, MA: MIT Press.
Google Académico
Dechert, W. Davis, ed.~1996. Teoría del caos en economía: métodos, modelos y evidencia . Edward Elgar: Cheltenham.
Google Académico
Diaconis, Persi y D. Freedman. 1986. Sobre la consistencia de las estimaciones de Bayes. Annals of Statistics 14: 1--26.
Google Académico
Diener, Marc y Tim Poston. 1984. La convención del retraso perfecto. En Caos y orden en la naturaleza , ed.~Hermann Haken, 2ª ed., 249--268. Berlín: Springer-Verlag.
Google Académico
Ehrenfest, Paul y Tatiana Ehrenfest-Afanessjewa. 1911. Begriffte Grundlagen der Statistschen Auffassunf in der Mechanik. En Encyclopädie der Matematischen Wissenschaften, vol.~4 , ed.~F. Klein y C. Müller, 3--90. Leipzig: Teubner. (Traducción al inglés, MJ Moravcsik, 1959. Los fundamentos conceptuales del enfoque estadístico de la mecánica . Ithaca: Cornell University Press.).
Google Académico
Eigen, Manfred y Peter Schuster. 1979. El hiperciclo: un principio natural de autoorganización . Berlín: Springer-Verlag.
CrossRefGoogle Académico
Forrester, Jay W. 1961. Industrial Dynamics . Cambridge, MA: MIT Press.
Google Académico
Gallavotti, G. 1999. Mecánica estadística: un breve tratado . Berlín: Springer-Verlag.
CrossRefGoogle Académico
Garfinkel, Alan. 1987. El Dictostliam Slime Mold como modelo de autoorganización en los sistemas sociales. En Self-Organizing Systems: The Emergence of Order , ed.~F. Eugene Yates, 181--212. Nueva York: Plenum Press.
CrossRefGoogle Académico
Gode, D. y Shyam Sunder. 1993. Eficiencia de asignación de mercados con comerciantes de inteligencia cero: los mercados como un sustituto parcial de la racionalidad individual. Journal of Political Economy 101: 119-137.
CrossRefGoogle Académico
Gödel, Kurt. 1931. Über Formal Unentscheidbare Satze Principia Mathematica und Vergwander Systeme I. Monatshefte für Mathematik und Physik 38: 173-198.
CrossRefGoogle Académico
Goodwin, Richard M. 1947. Acoplamiento dinámico con especial referencia a marcadores que tienen retrasos en la producción. Econometrica 15: 181-204.
CrossRefGoogle Académico
Haken, Hermann. 1983. ``Synergetics''. Una introducción. Transiciones de fase de no equilibrio en física, química y biología . 3ª ed.~Berlín: Springer-Verlag.
Google Académico
---------. 1996. El principio de esclavitud revisado. Physica D 87: 95--103.
CrossRefGoogle Académico
Halmos, Paul R. 1958. Von Neumann sobre Medida y Teoría Ergódica. Boletín de la Sociedad Americana de Matemáticas 64: 86--94.
CrossRefGoogle Académico
Manos, D. Wade. 2001. Reflexión sin reglas: metodología económica y ciencia contemporánea . Cambridge, Reino Unido: Cambridge University Press.
CrossRefGoogle Académico
Hartmann, Georg C. y Otto E. Rössler. 1998. Atractores de bengalas acoplados: un prototipo discreto para el modelado económico. Dinámica discreta en la naturaleza y la sociedad 2: 153-159.
CrossRefGoogle Académico
Hayek, Friedrich A. 1948. Individualismo y orden económico . Chicago: Prensa de la Universidad de Chicago.
Google Académico
---------. 1967. La teoría de los fenómenos complejos. En Estudios de filosofía, política y economía , 22--42. Londres: Routledge y Kegan Paul.
CrossRefGoogle Académico
Hofstadter, Douglas R. 2006. Soy un bucle extraño . Nueva York: Basic Books.
Google Académico
Holden, Lisa y Thomas Erneux. 1993. Comprensión de las oscilaciones explosivas como un paso lento periódico a través de la bifurcación y los puntos límite. Revista de Biología Matemática 31: 351--365.
CrossRefGoogle Académico
Holling, CS 1992. Morfología, geometría y dinámica de ecosistemas a escala cruzada. Monografías ecológicas 62: 447--502.
CrossRefGoogle Académico
Horgan, John. 1997. El fin de la ciencia: afrontando los límites del conocimiento en el ocaso de la era científica . Edición de tapa blanda. Nueva York: Broadway Books.
Google Académico
Horwitz, Steven. 1992. Evolución monetaria, banca libre y orden económico . Boulder: Westfield Press.
Google Académico
Israel, Giorgio. 2005. La ciencia de la complejidad: problemas y perspectivas epistemológicas. La ciencia en contexto 18: 1--31.
CrossRefGoogle Académico
Jantsch, Erich. 1982. De la autorreferencia a la autotrascendencia: la evolución de la dinámica de la autoorganización. En Autoorganización y estructuras disipativas , ed.~William C. Schieve y Peter M. Allen, 344--353. Austin: Prensa de la Universidad de Texas.
Google Académico
Kaufmann, Stuart A. 1993. Los orígenes del orden: autoorganización y selección en la evolución . Oxford: Prensa de la Universidad de Oxford.
Google Académico
Keynes, John Maynard. 1921. Tratado de probabilidad . Londres: Macmillan.
Google Académico
---------. 1936. La teoría general del empleo, el interés y el dinero . Londres: Macmillan.
Google Académico
---------. 1938. Método del profesor Tinbergen. The Economic Journal 49: 558--568.
CrossRefGoogle Académico
Kim, Jaegwon. 1999. Making Sense of Emergence. Estudios filosóficos 95: 3-36.
CrossRefGoogle Académico
Kindleberger, Charles P. 2001. Manías, pánicos y accidentes: una historia de crisis financieras . 4ª ed.~Nueva York: Basic Books.
Google Académico
Kleene, Stephen C. 1967. Lógica matemática . Nueva York: John Wiley \& Sons.
Google Académico
Kleene, Stephen C. y Richard E. Vesley. 1965. Fundamentos de las matemáticas intuicionistas . Amsterdam: Holanda Septentrional.
Google Académico
Knight, Frank H. 1921. Riesgo, incertidumbre y beneficio . Boston: Hart, Schaffer y Marx.
Google Académico
Kolmogorov, Andrei N. 1965. Tres enfoques para la definición cuantitativa de información. Problemas de transmisión de información 1: 4--7.
Google Académico
---------. 1983. Fundamentos combinatorios de la teoría de la información y el cálculo de probabilidades. Encuestas de matemáticas rusas 38 (4): 29--40.
CrossRefGoogle Académico
Koppl, Roger. 2006. Economía austriaca a la vanguardia. Review of Austrian Economics 19: 231--241.
CrossRefGoogle Académico
---------. 2009. Complexity and Austrian Economics. En Handbook of Complexity Research , ed.~J. Barkley Rosser Jr., 393--408. Cheltenham: Edward Elgar.
Google Académico
Koppl, Roger y J. Barkley Rosser Jr.~2002. Todo lo que tengo que decir ya se le ha pasado por la cabeza. Metroeconomica 53: 339--360.
CrossRefGoogle Académico
Lachmann, Ludwig. 1986. El mercado como proceso económico . Oxford: Basil Blackwell.
Google Académico
Landini, Simone, Mauro Gallegati y J. Barkley Rosser Jr.~2020. Consistencia e incompletitud en la teoría del equilibrio general. Revista de economía evolutiva 30: 205-230.
CrossRefGoogle Académico
Lavoie, Don. 1989. ¿Caos económico u orden espontáneo? Implicaciones para la economía política de la nueva visión de la ciencia. Cato Journal 8: 613--635.
Google Académico
Lawson, Tony. 1997. Economía y realidad . Londres: Routledge.
CrossRefGoogle Académico
Leijonufvud, Axel. 1993. Hacia una macroeconomía no demasiado racional. Southern Economic Journal 60: 1--13.
CrossRefGoogle Académico
Lewes, George Henry. 1875. Problemas de la vida y la mente . Londres: Kegan Paul Trench Turbner.
Google Académico
Lewis, Alain A. 1985. Sobre Realizaciones Efectivamente Computables de Funciones de Elección. Ciencias sociales matemáticas 10: 43--80.
CrossRefGoogle Académico
---------. 1992. Sobre los grados de Turing de los modelos walrasianos y un resultado de imposibilidad general en la teoría de la toma de decisiones. Ciencias sociales matemáticas 24: 143-171.
Google Académico
Lewis, Paul. 2012. Propiedades emergentes en la obra de Friedrich Hayek. Journal of Economic Behavior and Organisation 82: 368--378.
CrossRefGoogle Académico
Lipman, Barton L. 1991. Cómo decidir cómo decidir cómo\ldots: Modelar la racionalidad limitada. Econometrica 59: 1105--1125.
CrossRefGoogle Académico
Loasby, Brian J. 1976. Elección, complejidad e ignorancia . Cambridge, Reino Unido: Cambridge University Press.
Google Académico
Lorenz, Edward N. 1963. Deterministic Non-Periodic Flow. Revista de ciencia atmosférica 20: 130-141.
CrossRefGoogle Académico
Lorenz, Hans-Walter. 1992. Atractores múltiples, límites de cuencas complejas y movimiento transitorio en sistemas económicos deterministas. En Modelos económicos dinámicos y control óptimo , ed.~Gustav Feichtinger, 411--430. Amsterdam: Holanda Septentrional.
Google Académico
Lynch, Michael. 2000. Contra la reflexividad como virtud académica y fuente de conocimiento privilegiado. Teoría, cultura y sociedad 17: 26--54.
CrossRefGoogle Académico
Malinvaud, Edmond. 1966. Métodos estadísticos para la econometría . Amsterdam: Holanda Septentrional.
Google Académico
Markose, Sheri M. 2005. Computabilidad y complejidad evolutiva: los mercados como sistemas adaptativos complejos. Economic Journal 115: F159 -- F192.
CrossRefGoogle Académico
Mayo, Robert M. 1976. Modelos matemáticos simples con dinámicas muy complicadas. Nature 261: 459--467.
CrossRefGoogle Académico
Maymin, Philip Z. 2011. Los mercados son eficientes si y solo si P = NP. Finanzas algorítmicas 1: 1--11.
CrossRefGoogle Académico
McCall, John J. 2005. Inducción. En Computability, Complexity and Constructivity in Economic Analysis , ed.~K. Vela Velupillai, 105-131. Victoria: Blackwell.
Google Académico
McCauley, Joseph L. 2004. Dinámica de los mercados: Econofísica y Finanzas . Cambridge, Reino Unido: Cambridge University Press.
CrossRefGoogle Académico
McCauley, Joseph I. 2005. Haciendo que las matemáticas sean efectivas en economía. En Computability, Complexity and Constructivity in Economic Analysis , ed.~K. Vela Veulupillai, 51--84. Victoria: Blackwell.
Google Académico
Meadows, Donella H., Dennis L. Meadows, Jorgen Randers y William W. Behrens III. 1972. Los límites del crecimiento . Nueva York: Universe.
Google Académico
Menger, Carl. 1871/1981. Principios de economia. Traducido al inglés por James Dingwall y Bert F. Hoselitz . Nueva York: New York University Press.
Google Académico
Menger, Carl. 1883/1985. Investigaciones sobre el método de las ciencias sociales con especial referencia a la economía. Traducido al inglés por Francis J. Nock . Nueva York: New York University Press.
Google Académico
Menger, Carl. 1892. Sobre el origen del dinero. Economic Journal 2: 239-255.
CrossRefGoogle Académico
Merton, Robert K. 1938. La ciencia y el orden social. Filosofía de la ciencia 5: 523--537.
CrossRefGoogle Académico
---------. 1948. La profecía autocumplida. Antioch Review 8: 183--210.
CrossRefGoogle Académico
Mill, John Stuart. 1843. Un sistema de lógica: razonado e inductivo . Londres: Longmans Green.
Google Académico
Minsky, Hyman P. 1972. Revisión de la inestabilidad financiera: La economía de los desastres. Reevaluación del mecanismo de descuento de la Reserva Federal 3: 97-136.
Google Académico
Mirowski, Philip. 2002. Machine Dreams: Economics Becomes a Cyborg Science . Cambridge, Reino Unido: Cambridge University Press.
Google Académico
---------. 2007. Los mercados llegan a los bits: evolución, computación y Markomata en ciencias económicas. Journal of Economic Behavior and Organisation 63: 209--242.
CrossRefGoogle Académico
Mirowski, Philip y Edward Nik-Kah. 2017. Conocimiento que hemos perdido en la información: la historia de la información en la economía moderna . Nueva York: Oxford University Press.
CrossRefGoogle Académico
Moore, Christopher. 1990. Indecidibilidad e imprevisibilidad en sistemas dinámicos. Physical Review Letters 64: 2354--2357.
CrossRefGoogle Académico
---------. 1991a. Cambios generalizados: indecidibilidad e imprevisibilidad en sistemas dinámicos. No linealidad 4: 199-230.
CrossRefGoogle Académico
---------. 1991b. Desplazamientos unilaterales generalizados y mapas del intervalo. No linealidad 4: 737--745.
Google Académico
Morgan, C. Lloyd. 1923. Evolución emergente . Londres: Williams y Norgate.
Google Académico
Morgenstern, Oskar. 1935. Voltommene Vorastlicht und Wirtschafrliches Gleichgewicht. Zeitschrift für Nationalökonome 6: 337--357.
CrossRefGoogle Académico
Nash, John F., Jr.~1955. ``Carta a la Agencia de Seguridad Nacional''. nsa.gov/Portals/70/documents/news-features/declassified-documents/nash-letters/nash\_letters1.pdf .
Google Académico
von Neumann, John. 1932. Prueba de la hipótesis cuasi-ergódica. Actas de la Academia Nacional de Ciencias 18: 263--266.
CrossRefGoogle Académico
---------. 1966. Theory of Self-Reproducing Automata, editado y compilado por Arthur W. Burks . Urbana: Prensa de la Universidad de Illinois.
Google Académico
Nicolis, John S. 1986. Dinámica de sistemas jerárquicos: un enfoque evolutivo . Berlín: Springer-Verlag.
CrossRefGoogle Académico
Nicolis, Grégoire e Ilya Prigogine. 1977. Autoorganización en sistemas de desequilibrio: de estructuras disipativas al orden a través de fluctuaciones . Nueva York: Wiley-Interscience.
Google Académico
Nyarko, Yaw. 1991. Aprendizaje en modelos mal especificados y posibilidad de ciclos. Journal of Economic Theory 55: 416--427.
CrossRefGoogle Académico
O'Donnell, M. 1979. Un teorema del lenguaje de programación que es independiente de la aritmética de Peano. Actas del XI Simposio Anual de ACM sobre Teoría de la Computación : 179--188.
Google Académico
O'Donnell, Rod M. 2014-15. Una crítica del enfoque ergódico / no ergódico de la incertidumbre. Revista de economía poskeynesiana 37: 187-209.
Google Académico
O'Driscoll, Gerald P. y Mario J. Rizzo. 1985. La economía del tiempo y la ignorancia . Oxford: Basil Blackwell.
Google Académico
Poincaré, Henri. 1890. Sur les Équations de la Dynamique et le Problème de Trois Corps. Acta Mathematica 13: 1--270.
Google Académico
Popper, Karl. 1959. La lógica del descubrimiento científico . Londres: Hutchinson Verlag von Julius Springer.
Google Académico
Pour-El, Marian Boykan e Ian Richards. 1979. Una ecuación diferencial ordinaria computable que no posee solución computable. Annals of Mathematical Logic 17: 61--90.
CrossRefGoogle Académico
Prasad, Kislaya. 1991. Computabilidad y aleatoriedad de los equilibrios de Nash en juegos infinitos. Journal of Mathematical Economics 20: 429--442.
CrossRefGoogle Académico
---------. 2005. Modelos constructivos y clásicos de resultados en economía y teoría de juegos. En Computability, Complexity and Constructivity in Economic Analysis , ed.~K. Vela Velupillai, 132-147. Victoria: Blackwell.
Google Académico
Pryor, Frederic L. 1995. Evolución y estructura económicas: El impacto de la complejidad en el sistema económico de los Estados Unidos . Nueva York: Cambridge University Press.
CrossRefGoogle Académico
Puu, Tönu. 1990. Un modelo caótico del ciclo económico. Serie de artículos ocasionales sobre dinámica socioespacial 1: 1--19.
Google Académico
Radner, Roy S. 1992. Jerarquía: La economía de la gestión. Revista de literatura económica 30: 1382-1415.
Google Académico
Richter, MK y KV Wong. 1999. No computabilidad del equilibrio competitivo. Teoría económica 14: 1--28.
CrossRefGoogle Académico
Rissanen, Jorma. 1978. Modelado por la descripción de datos más corta. Automatica 14: 465--471.
CrossRefGoogle Académico
---------. 1986. Modelado y Complejidad Estocástica. Annals of Stochastics 14: 1080-1100.
Google Académico
---------. 1989. Complejidad estocástica en la investigación estadística . Singapur: World Scientific.
Google Académico
---------. 2005. Complejidad e información en el modelado. En Computability, Complexity and Constructivity in Economic Analysis , ed.~K. Vela Velupillai, 85-104. Victoria: Blackwell.
Google Académico
Robinson, Joan. 1953-54. La función de producción y la teoría del capital. Revisión de estudios económicos 21: 81--106.
CrossRefGoogle Académico
Robinson, Abraham. 1966. Análisis no estándar . Amsterdam: Holanda Septentrional.
Google Académico
Rosser, J. Barkley, Jr.~1991. De la catástrofe al caos: una teoría general de las discontinuidades económicas . Boston: Kluwer.
CrossRefGoogle Académico
---------. 1994. Dinámica de la jerarquía urbana emergente. Caos, solitones y fractales 4: 553--562.
CrossRefGoogle Académico
---------. 1999. Sobre las complejidades de la dinámica económica compleja. Journal of Economic Perspectives 13 (4): 169-182.
CrossRefGoogle Académico
---------. 2000a. De la catástrofe al caos: una teoría general de las discontinuidades económicas: matemáticas, microeconomía, macroeconomía y finanzas, volumen II . Boston: Kluwer.
Google Académico
---------., Ed. 2000b. Complexity in Economics, Volúmenes I-III: Biblioteca Internacional de Escritos Críticos de Economía, 174 . Cheltenham: Edward Elgar.
Google Académico
---------. 2001a. Perspectivas alternativas keynesianas y poskeynesianas sobre la incertidumbre y las expectativas. Revista de economía poskeynesiana 23: 545--566.
CrossRefGoogle Académico
---------. 2001b. Sistemas Ecológico-Económicos Complejos y Política Ambiental. Economía ecológica 17: 23--37.
CrossRefGoogle Académico
---------. 2004. Implicaciones epistemológicas de la complejidad económica. Anales de la Asociación Japonesa de Filosofía de la Ciencia 31 (2): 3-18.
Google Académico
---------. 2007. El auge y la caída de las aplicaciones de la teoría de la catástrofe en economía: ¿se arrojó al bebé con el agua del baño? Journal of Economic Dynamics and Control 31: 3255--3280.
CrossRefGoogle Académico
---------. 2009a. Complejidad Computacional y Dinámica en Economía. En Handbook of Complexity Research , ed.~J. Barkley Rosser Jr., 22--35. Cheltenham: Edward Elgar.
CrossRefGoogle Académico
---------. 2010a. Lógica constructivista y evolución emergente en la complejidad económica. En Computabilidad, dinámica económica constructiva y conductual: ensayos en honor a Kumaraswamy (Vela) Velupillai , ed.~Stefano Zambelli, 184-197. Londres: Routledge.
Google Académico
---------. 2012a. Emergencia y complejidad en la economía austriaca. Revista de comportamiento económico y organización 81: 122--128.
CrossRefGoogle Académico
---------. 2012b. Sobre los fundamentos de la economía matemática. Nuevas matemáticas y computación natural 8: 53--72.
CrossRefGoogle Académico
---------. 2014a. Los fundamentos de la complejidad económica en la racionalidad del comportamiento en expectativas heterogéneas. Revista de metodología económica 21: 308--312.
CrossRefGoogle Académico
---------. 2016a. Reconsideración de la ergodicidad y la incertidumbre fundamental. Revista de economía poskeynesiana 38: 331--354.
CrossRefGoogle Académico
---------. 2020a. Incomplejidad y complejidad en la teoría económica. En Unraveling Complexity: The Life and Work of Gregory Chaitin , ed.~Shyam Wuppuluri y Francisco Antonio Doria, 345--367. Singapur: World Scientific.
CrossRefGoogle Académico
---------. 2020b. Reflexiones sobre la reflexividad y la complejidad. En Historia, metodología e identidad para una economía social XXI , ed.~C. Wade Hands Wilfred Dolfsma y Robert McMaster, 67--86. Londres: Routledge.
Google Académico
---------. 2020c. El momento de Minsky y la venganza de la entropía. Dinámica macroeconómica 24: 7--23.
CrossRefGoogle Académico
Rosser, J. Barkley, Jr., Marina V. Rosser, Steven J. Guastello y Robert W. Bond. 2001. Histéresis caótica en la transformación económica sistémica. Dinámica no lineal, psicología y ciencias de la vida 5: 345--368.
CrossRefGoogle Académico
Rosser, J. Barkley, Jr., Ehsan Ahmed y Georg C. Hartmann. 2003a. Volatilidad vía Social Flaring. Revista de comportamiento y organización económicos 50: 77--87.
CrossRefGoogle Académico
Rosser, J. Barkley, Jr., Marina V. Rosser y Mauro Gallegati. 2012. Una perspectiva de Minsky-Kindleberger sobre la crisis financiera. Journal of Economic Issues 45: 449--458.
CrossRefGoogle Académico
Rosser, J. Barkley, Sr.~1936. Extensiones de algunos teoremas de Gödel e Church. Journal of Symbolic Logic 1: 87--91.
CrossRefGoogle Académico
Samuelson, Paul A. 1969. Teoría clásica y neoclásica. En Teoría Monetaria: Lecturas , ed.~Robert W. Clower, 182-194. Hammondsworth: pingüino.
Google Académico
Scarf, Herbert E. 1973. El cálculo de los equilibrios económicos . New Haven: Prensa de la Universidad de Yale.
Google Académico
Schelling, Thomas C. 1971. Modelos dinámicos de segregación. Revista de Sociología Matemática 1: 143-186.
CrossRefGoogle Académico
Shackle, George LS 1972. Epistemics and Economics: A Critique of Economic Doctrines . Cambridge, Reino Unido: Cambridge University Press.
Google Académico
Shannon, Claude E. 1948. Una teoría matemática de la comunicación. Revista técnica Bell System 27 (379-423): 623--656.
CrossRefGoogle Académico
Shinkai, S. e Y. Aizawa. 2006. La complejidad de Lempel-Zev del caos no estacionario en casos ergódicos infinitos. Progreso de la física teórica 116: 503--515.
CrossRefGoogle Académico
Simon, Herbert A. 1962. La arquitectura de la complejidad. Actas de la American Philosophical Society 106: 467--482.
Google Académico
Solomonoff, RJ 1964. Una teoría formal de la inferencia inductiva Partes I y II. Información y control 7 (1-22): 224-154.
CrossRefGoogle Académico
Soros, George. 1987. La alquimia de las finanzas . Hoboken: Wiley \& Sons.
Google Académico
---------. 2013. Falibilidad, reflexividad y el principio de incertidumbre humana. Revista de metodología económica 20: 309--329.
CrossRefGoogle Académico
Specker, EP 1953. El axioma de la elección en los nuevos fundamentos de la lógica matemática de Quine. Actas de la Academia Nacional de Ciencias de EE . UU. 39: 972--975.
CrossRefGoogle Académico
Spencer, Herbert. 1867-1874. Sociología descriptiva: enciclopedia de hechos sociales, que representa la constitución de cada tipo y grado de la sociedad humana, pasada y presente, estacionaria y progresiva, clasificada y tabulada para una fácil comparación y estudios convenientes de las relaciones de los fenómenos sociales . Londres: Williams y Norgate.
Google Académico
Stadler, Barbel, Peter Statler, Gunter Wagner y Walter Fontana. 2001. La topología de lo posible: espacios formales que subyacen a los patrones de cambio evolutivo. Revista de biología teórica 21: 241-274.
CrossRefGoogle Académico
Stodder, James P. 1995. La evolución de la complejidad en las economías primitivas: teoría. Revista de economía comparada 20: 1--31.
CrossRefGoogle Académico
Tarski, Alfred. 1949. Sobre la indecidibilidad esencial (Resumen). Journal of Symbolic Logic 14: 75--76.
Google Académico
Thom, René. 1975. Estabilidad estructural y morfogénesis: un esbozo de una teoría de modelos . Lectura: Benjamin.
Google Académico
Tinbergen. 1937. Un enfoque econométrico de los ciclos económicos . París: Hermann.
Google Académico
---------. 1940. Sobre un método de investigación estadística empresarial: una respuesta. Economic Journal 50: 41--54.
CrossRefGoogle Académico
Tsuji, Marcelo, Newton CA da Costa y Francisco A. Doria. 1998. Las teorías de la incompletitud de los juegos. Journal of Philosophical Logic 27: 553--568.
CrossRefGoogle Académico
Turing, Alan M. 1936. Sobre números computables, con una aplicación al problema de Entscheidung . Actas de la London Mathematical Society, Serie 2 (42): 239-265.
Google Académico
---------. 1937. Computabilidad y λ-definibilidad. Journal of Symbolic Logic 2: 153-163.
CrossRefGoogle Académico
---------. 1952. The Chemical Basis of Morphogenesis. Transacciones filosóficas de la Royal Society B 237: 37--72.
Google Académico
Uffink, Jos. 2006. Compendio de los fundamentos de la física estadística clásica. Institución de Historia y Fundamentos de la Ciencia, Universidad de Utrecht.
Google Académico
Vaughn, Karen I. 1999. El pensamiento y el orden del mercado de Hayek como una instancia de sistemas adaptativos complejos. Journal des économistes et des études humaines 9: 241--246.
CrossRefGoogle Académico
Velupillai, Kumaraswamy. 2000. Economía Computable . Oxford: Prensa de la Universidad de Oxford.
CrossRefGoogle Académico
Velupillai, K. Vela. 2002. Efectividad y constructividad en teoría económica. Journal of Economic Behavior and Organization 49: 307--325.
CrossRefGoogle Académico
---------. 2005a. Introducción. En Computability, Complexity and Constructivity in Economic Analysis , ed.~K. Vela Velupillai, 1--14. Victoria: Blackwell.
Google Académico
---------. 2005b. Una introducción a las herramientas y conceptos de la economía computable. En Computability, Complexity and Constructivity in Economic Analysis , ed.~K. Vela Velupillai, 148-197. Victoria: Blackwell.
Google Académico
---------. 2006. Fundamentos algorítmicos de la teoría del equilibrio general computable. Matemáticas aplicadas y computación 179: 360--369.
CrossRefGoogle Académico
---------. 2008. Incomputabilidad e indecidibilidad en teoría económica. Documento de trabajo 806 del Departamento de Economía, Universidad de Trento, Italia.
Google Académico
---------. 2009. Perspectiva de un economista computable sobre la complejidad computacional. En Handbook of Complexity Research , ed.~J. Barkley Rosser Jr., 36--83. Cheltenham: Edward Elgar.
Google Académico
---------. 2011. Dinámica no lineal, complejidad y aleatoriedad: fundamentos algorítmicos. Journal of Economic Surveys 25: 547--568.
CrossRefGoogle Académico
---------. 2012. Domando lo Incomputable, Reconstruyendo lo No Constructivo y Decidiendo lo Indecidible en Economía Matemática. Nuevas matemáticas y computación natural 8: 5--51.
CrossRefGoogle Académico
---------. 2013. Preceptos poskeynesianos para teorías no lineales, endógenas, no estocásticas y del ciclo económico. En Handbook of Postkeynesian Economics , ed.~Geoffrey C. Harcourt y Peter Kreisler, 415--442. Oxford: Prensa de la Universidad de Oxford.
Google Académico
Vriend, Nicolaas J. 2002. ¿Fue Hayek un as? Southern Economic Journal 68: 811--840.
CrossRefGoogle Académico
Wagner, Richard E. 2010. Mente, sociedad y acción humana . Nueva York: Routledge.
CrossRefGoogle Académico
Weintraub, E. Roy. 2002. Cómo la economía se convirtió en una ciencia matemática . Durham: Prensa de la Universidad de Duke.
CrossRefGoogle Académico
Whitehead, Alfred North y Bertrand Russell. 1910-13. Principia Mathematica, volúmenes I-III . Londres: Cambridge University Press.
Google Académico
Wiener, Norbert. 1948. Cibernética: o control y comunicación en el animal y la máquina . Cambridge, MA: MIT Press.
Google Académico
Wigner, Eugene. 1960. La efectividad irrazonable de las matemáticas en las ciencias naturales. Comunicaciones en matemáticas puras y aplicadas 13: 1--14.
CrossRefGoogle Académico
Wolfram, Stephen. 1984. Universalidad y complejidad en autómatas celulares. Physica D 10: 1--35.
CrossRefGoogle Académico
Woolgar, Steve. 1991. El giro de la tecnología en los estudios sociales de la ciencia. Ciencia, tecnología y valores humanos 16: 20--50.
CrossRefGoogle Académico
Zeeman, E. Christopher. 1974. Sobre el comportamiento inestable de las bolsas de valores. Revista de Economía Matemática 1: 39--44.
CrossRefGoogle Académico
Ahmed, Ehsan, J. Barkley Rosser Jr.~y Jamshed Y. Uppal. 2014. ¿Existen burbujas especulativas no lineales en los precios de las materias primas? Revista de economía poskeynesiana 36: 415--438.
CrossRefGoogle Académico
Akerlof, George A. 2002. Behavioral Macroeconomics and Macroeconomic Behavior. American Economic Review 92: 411--433.
CrossRefGoogle Académico
Alchian, Armen A. 1950. Incertidumbre, evolución y teoría económica. Journal of Political Economy 58: 211-222.
CrossRefGoogle Académico
Allen, Peter M., KA Richardson y JA Goldstein, eds.~2011. Emergence: Complexity \& Organization-2009 Annual, Volumen 11 . Naples, FL: Publicaciones ICSE.
Google Académico
Arthur, W. Brian. 1989. Tecnologías en competencia, rendimientos crecientes y el bloqueo por eventos históricos. The Economic Journal 99: 116-131.
CrossRefGoogle Académico
---------. 1994. Rendimientos crecientes y dependencia de la trayectoria en la economía . Ann Arbor: Prensa de la Universidad de Michigan.
CrossRefGoogle Académico
Arthur, W. Brian, Yuri Ermoliev y Yuri Kaniovski. 1987. Procesos de crecimiento adaptativo modelados por esquemas de urnas. Cybernetics 23: 776--789.
Google Académico
Arthur, W. Brian, John H. Holland, Blake LeBaron, B. Palmer y P. Tayler. 1997b. Fijación de precios de activos bajo expectativas endógenas en un mercado de valores artificial. En La economía como un sistema complejo en evolución II , ed.~W. Brian Arthur, Steven N. Durlauf y David A. Lane, 15--44. Reading, MA: Addison-Wesley.
Google Académico
Baumol, William J. 1957. Especulación, rentabilidad y estabilidad. Review of Economics and Statistics 39: 263--271.
CrossRefGoogle Académico
von Bertalanffy, Ludwig. 1950. Un esquema de la teoría de sistemas generales. Revista británica de filosofía de la ciencia 1: 114-129.
CrossRefGoogle Académico
Boerlijst, MC y P. Hogeweg. 1991. Estructura de ondas en espiral en la evolución prebiótica. Physica D 48: 17-28.
CrossRefGoogle Académico
Bogdanov, Aleksandr A. 1925-29. Tektologia: Vsobschaya Oranizatsionnay Nauka, Volúmenes 1-III . 3ª ed.~Leningrado-Moscú: Kniga.
Google Académico
Broadbent, DE 1950. ``La prueba de veinte diales en condiciones silenciosas''. Informe, Unidad de Psicología Aplicada No.~130/50. Londres: HMSO.
Google Académico
Brock, William A. y Cars H. Hommes. 1997. Una ruta racional hacia la aleatoriedad. Econometrica 65: 1059--1095.
CrossRefGoogle Académico
---------. 1998. Creencias heterogéneas y rutas al caos en un modelo simple de fijación de precios de activos. Journal of Economic Dynamics and Control 22: 1235-1274.
CrossRefGoogle Académico
Clark, Colin W. 1990. Bioeconomía matemática . 2ª ed.~Nueva York: Wiley-Interscience.
Google Académico
Coase, Ronald H. 1937. La naturaleza de la empresa. Economica 4: 386--405.
CrossRefGoogle Académico
Colander, David., Ed. 2006. Economía poswalrasiana: más allá del modelo de equilibrio general estocástico . Nueva York: Cambridge University Press.
Google Académico
Commons, John R. 1924. Los fundamentos legales del capitalismo . Nueva York: Macmillan.
Google Académico
---------. 1934. Economía institucional . Nueva York: Macmillan.
Google Académico
Conlisk, John. 1996. ¿Por qué la racionalidad acotada? Revista de literatura económica 34: 1--64.
Google Académico
Copes, Parzival. 1970. La curva de oferta hacia atrás de la industria pesquera. Revista escocesa de economía política 17: 69--77.
CrossRefGoogle Académico
Crow, James F. 1955. Teoría general de la genética de poblaciones: Síntesis. Simposio cuantitativo sobre biología de Cold Spring Harbor 20: 54--59.
CrossRefGoogle Académico
Crow, James F. y Kenichi J. Aoki. 1984. Selección de grupo para un rasgo de comportamiento poligénico: estimación del grado de subdivisión de la población. Actas de la Academia Nacional de Ciencias de EE . UU. 81: 2628--2631.
CrossRefGoogle Académico
Crutchfield, James P. 1994. Los cálculos de la emergencia: Computación, dinámica e inducción. Physica D 75: 11--54.
CrossRefGoogle Académico
---------. 2003. Cuando la evolución es revolución: orígenes de la innovación. En Evolutionary Dynamics: Explorando la interacción de la selección, el accidente, la neutralidad y la función , ed.~James P. Crutchfield y Peter Schuster, 101-133. Oxford: Prensa de la Universidad de Oxford.
Google Académico
Cyert, Richard M. y James G. March. 1963. Una teoría del comportamiento de la empresa . Prentice-Hall: Acantilados de Englewood.
Google Académico
Darwin, Charles. 1859. Sobre el origen de las especies mediante la selección natural o la conservación de razas favorecidas en la lucha por la vida . Londres: John Murray.
Google Académico
Davidson, Paul. 1982-83. Expectativas racionales: una base falaz para estudiar los procesos cruciales de toma de decisiones económicas. Revista de economía poskeynesiana 5: 182-198.
CrossRefGoogle Académico
Davidson, Paul. 1996. Realidad y teoría económica. Revista de economía poskeynesiana 18: 479--508.
CrossRefGoogle Académico
Dawkins, Richard. 1976. El gen egoísta . Oxford: Prensa de la Universidad de Oxford.
Google Académico
Dechert, W. Davis, ed.~1996. Teoría del caos en economía: métodos, modelos y evidencia . Edward Elgar: Cheltenham.
Google Académico
Dopfer, Kurt, John Foster y Jason Potts. 2004. Micro-Meso-Macro. Revista de economía evolutiva 14: 263-279.
CrossRefGoogle Académico
Easterlin, Richard A. 2017. ¿Paradoja perdida? Review of Behavioral Economics 4: 311--339.
CrossRefGoogle Académico
Eigen, Manfred y Peter Schuster. 1979. El hiperciclo: un principio natural de autoorganización . Berlín: Springer-Verlag.
CrossRefGoogle Académico
Eldredge, Niles y Stephen Jay Gould. 1972. Equilibrios puntuados. En Modelos de Paleobiología , ed.~DJM Schopf, 82-115. San Francisco: Freeman, Cooper.
Google Académico
Fisher, Ronald A. 1930. La teoría general de la selección natural . Oxford: Prensa de la Universidad de Oxford.
Google Académico
Gallegati, Mauro, Antonio Palestrini y J. Barkley Rosser Jr.~2011. El período de angustia financiera en burbujas especulativas: agentes heterogéneos y restricciones financieras. Dinámica macroeconómica 13: 60--79.
CrossRefGoogle Académico
Gordon, H. Scott. 1954. Teoría económica de un recurso de propiedad común: la pesca. Journal of Political Economy 62: 124-142.
CrossRefGoogle Académico
Gould, Steven Jay. 2002. La estructura de la teoría económica . Cambridge, MA: Belknap Press de Harvard University Press.
Google Académico
Gowdy, John, J. Barkley Rosser Jr.~y Loraine Roy. 2013. La evolución del descuento hiperbólico: implicaciones para el futuro verdaderamente social. Revista de organización y comportamiento económico 90: S94 -- S104.
CrossRefGoogle Académico
Grandmont, Jean-Michel. 1998. Formación y estabilidad de expectativas en sistemas socioeconómicos a gran escala. Econometrica 66: 741--781.
CrossRefGoogle Académico
Haldane, JBS 1932. Las causas de la evolución . Londres: Longmans \& Green.
Google Académico
Hamilton, WD 1964. La evolución genética del comportamiento social. Revista de biología teórica 7: 1--52.
CrossRefGoogle Académico
---------. 1972. Altruismo y fenómenos relacionados, principalmente en los insectos sociales. Revisión anual de ecología y sistemática 3: 192--232.
CrossRefGoogle Académico
Harcourt, Geoffrey C. y Peter Kreisler, eds.~2013a. El Manual de Oxford de Economía Postkeynesiana, Volumen 1: Teoría y Orígenes . Oxford: Prensa de la Universidad de Oxford.
Google Académico
---------, eds.~2013b. El Manual de Oxford de Economía Postkeynesiana, Volumen 2: Críticas y Metodología . Oxford: Prensa de la Universidad de Oxford.
Google Académico
Hayek, Friedrich A. 1952. El orden sensorial: una investigación sobre los fundamentos de la psicología teórica . Londres: Routledge y Kegan Paul.
Google Académico
---------. 1967. La teoría de los fenómenos complejos. En Estudios de filosofía, política y economía , 22--42. Londres: Routledge y Kegan Paul.
CrossRefGoogle Académico
---------. 1988. La presunción fatal: los errores del socialismo . Chicago: Universidad de Chicago.
CrossRefGoogle Académico
Heinrich, Joseph. 2004. Selección de grupos culturales, procesos coevolucionarios y cooperación a gran escala. Journal of Economic Behavior and Organisation 53: 3--35.
CrossRefGoogle Académico
Hodgson, Geoffrey M. 1993a. Economía institucional: estudio de lo ``antiguo'' y lo ``nuevo''. Metroeconomica 44: 1--28.
CrossRefGoogle Académico
---------. 1993b. Economía y evolución: devolver la vida a la economía . Ann Arbor: Prensa de la Universidad de Michigan.
CrossRefGoogle Académico
Hodgson, Geoffrey M. y Thorbjørn Knudsen. 2006. Por qué necesitamos un darwinismo generalizado y por qué el darwinismo generalizado no es suficiente. Revista de organización y comportamiento económico 61: 1--19.
CrossRefGoogle Académico
Hommes, Cars H. y J. Barkley Rosser Jr.~2001. Equilibrios de expectativas consistentes en los mercados de recursos renovables. Dinámica macroeconómica 5: 10-203.
CrossRefGoogle Académico
Hommes, Cars H. y Gerhard Sorger. 1998. Equilibrios de expectativas consistentes. Dinámica macroeconómica 2: 287--321.
CrossRefGoogle Académico
Kaldor, Nicolás. 1972. La irrelevancia de la economía del equilibrio. The Economic Journal 82: 1237--1255.
CrossRefGoogle Académico
Kaufmann, Stuart A. 1993. Los orígenes del orden: autoorganización y selección en la evolución . Oxford: Prensa de la Universidad de Oxford.
Google Académico
Keynes, John Maynard. 1936. La teoría general del empleo, el interés y el dinero . Londres: Macmillan.
Google Académico
Kindleberger, Charles P. 2001. Manías, pánicos y accidentes: una historia de crisis financieras . 4ª ed.~Nueva York: Basic Books.
Google Académico
Lewes, George Henry. 1875. Problemas de la vida y la mente . Londres: Kegan Paul Trench Turbner.
Google Académico
Lewis, Paul. 2012. Propiedades emergentes en la obra de Friedrich Hayek. Journal of Economic Behavior and Organisation 82: 368--378.
CrossRefGoogle Académico
MacKay, Charles. 1852. Memorias de ilusiones extraordinarias y la locura de las multitudes . Londres: Oficina de la Biblioteca Nacional Ilustrada.
Google Académico
Mayo, Robert M. 1976. Modelos matemáticos simples con dinámicas muy complicadas. Nature 261: 459--467.
CrossRefGoogle Académico
McCauley, Joseph I. 2005. Haciendo que las matemáticas sean efectivas en economía. En Computability, Complexity and Constructivity in Economic Analysis , ed.~K. Vela Veulupillai, 51--84. Victoria: Blackwell.
Google Académico
McLaughlin, Bryan P. 1992. El ascenso y la caída del emergentismo británico. ¿ Emergencia o reducción? Ensayos sobre la perspectiva del fisicalismo no reductor , ed.~A. Beckerman, H. Flohr y J. Kim, 49--93. Berlín: Walter de Gruyter.
Google Académico
Menger, Carl. 1923. Grundsätze der Volkswirtschaftslehre, Zweiite Auflag . Viena: Holden-Pichler-Tempsky.
Google Académico
Mikami, M. 2011. Aspectos evolutivos de la economía de Coasean. Revista de economía evolutiva e institucional 8: 177--187.
CrossRefGoogle Académico
Mill, John Stuart. 1843. Un sistema de lógica: razonado e inductivo . Londres: Longmans Green.
Google Académico
Minniti, María. 1995. La membresía tiene sus privilegios: organizaciones mafiosas antiguas y nuevas. Estudios económicos comparados 37: 31--47.
CrossRefGoogle Académico
Minsky, Hyman P. 1972. Revisión de la inestabilidad financiera: La economía de los desastres. Reevaluación del mecanismo de descuento de la Reserva Federal 3: 97-136.
Google Académico
Moore, Christopher. 1990. Indecidibilidad e imprevisibilidad en sistemas dinámicos. Physical Review Letters 64: 2354--2357.
CrossRefGoogle Académico
Morgan, C. Lloyd. 1923. Evolución emergente . Londres: Williams y Norgate.
Google Académico
Mosekilde, Erik, Steen Rsmussen y Torben S. Sorenson. 1983. Autoorganización y re-casualización estocástica en modelos de dinámica de sistemas. En Actas de la Conferencia de dinámica de sistemas de 1981 , ed.~JDW Morecroft, DF Anderson y JD Sterman, 128-160. Universidad de Pine Manor: Chestnut Hill, MA.
Google Académico
Mueller, Bernardo. 2015. Creencias y persistencia de instituciones ineficientes . Documento de trabajo de la Universidad de Brasilia. Extranet.sioe.org/uploads/isnie2015/mueller.pdf .
Google Académico
Muth, John F. 1961. Expectativas racionales y teoría de los movimientos de precios. Econometrica 29: 315--335.
CrossRefGoogle Académico
Myrdal, Gunnar. 1957. Teoría económica y regiones subdesarrolladas . Londres: Methuen.
Google Académico
Nelson, Richard R. y Sydney G. Winter. 1982. Una teoría evolutiva del cambio económico . Cambridge, MA: Harvard University Press.
Google Académico
Ng, Yew-Kwang. 1986. Mesoeconomics: A Micro-Macro Analysis . Nueva York: St.~Martins.
Google Académico
North, Douglass C. 1990. Instituciones, cambio institucional y progreso económico . Cambridge: Cambridge University Press.
Google Académico
Ostrom, Elinor. 1990. Governing the Commons: The Evolution of Institutions for Collective Action . Cambridge, Reino Unido: Cambridge University Press.
CrossRefGoogle Académico
Paley, William. 1802. Teología natural: o evidencia de la existencia y atributos de la Deidad, recopilada de las apariencias de la naturaleza . Londres: R. Faulder.
Google Académico
Papageorgiou, T., I. Katselides y PG Michaelides. 2013. Schumpeter, Commons y Veblen sobre instituciones. Revista Estadounidense de Economía y Sociología 72: 1232--1254.
CrossRefGoogle Académico
Pingle, Mark y Richard H. Day. 1996. Modes of Economizing Behavior: Experimental Evidence. Journal of Economic Behavior and Organisation 29: 191-209.
CrossRefGoogle Académico
Polanyi, Karl. 1944. La gran transformación . Nueva York: Farrar \& Rinehart.
Google Académico
Price, George R. 1970. Selección y covarianza. Nature 227: 520-521.
CrossRefGoogle Académico
Radzicki, Michael J. 1990. Dinámica institucional, caos determinista y sistemas autoorganizados. Journal of Economic Issues 24: 57--102.
CrossRefGoogle Académico
Rosser, J. Barkley, Jr.~1991. De la catástrofe al caos: una teoría general de las discontinuidades económicas . Boston: Kluwer.
CrossRefGoogle Académico
---------. 1992. El diálogo entre las teorías ecológica y económica de la evolución. Journal of Economic Behavior and Organisation 17: 195--215.
CrossRefGoogle Académico
---------. 1997. Especulaciones sobre burbujas especulativas no lineales. Dinámica no lineal, psicología y ciencias de la vida 1: 275--300.
CrossRefGoogle Académico
---------. 1998. Dinámica compleja en modelos neokeynesianos y poskeynesianos. En New Keynesian Economics / Post Keynesian Alternatives , ed.~Roy J. Rotheim, 288-302. Londres: Routledge.
Google Académico
---------. 2000a. De la catástrofe al caos: una teoría general de las discontinuidades económicas: matemáticas, microeconomía, macroeconomía y finanzas, volumen II . Boston: Kluwer.
Google Académico
---------., Ed. 2000b. Complexity in Economics, Volúmenes I-III: Biblioteca Internacional de Escritos Críticos de Economía, 174 . Cheltenham: Edward Elgar.
Google Académico
---------. 2001a. Perspectivas alternativas keynesianas y poskeynesianas sobre la incertidumbre y las expectativas. Revista de economía poskeynesiana 23: 545--566.
CrossRefGoogle Académico
---------. 2001b. Sistemas Ecológico-Económicos Complejos y Política Ambiental. Economía ecológica 17: 23--37.
CrossRefGoogle Académico
---------. 2006. Dinámica compleja y economía poskeynesiana. En Complexity, Endógenous Money, and Macroeconomics: Essays in Honor of Basil J. Moore , ed.~Mark Setterfield, 74--98. Cheltenham: Edward Elgar.
Google Académico
---------. 2009b. Teoría del caos antes de Lorenz. Dinámica no lineal, psicología y ciencias de la vida 12: 255-267.
Google Académico
---------. 2010b. ¿Es posible una perspectiva transdisciplinaria sobre la complejidad económica? Journal of Economic Behavior and Organisation 75: 3-11.
CrossRefGoogle Académico
---------. 2011a. Dinámicas evolutivas complejas en sistemas urbano-regionales y ecológico-económicos: de la catástrofe al caos y más allá . Nueva York: Springer.
CrossRefGoogle Académico
---------. 2011b. Perspectivas poskeynesianas y dinámicas ecológico-económicas complejas. Metroeconomica 62: 96--121.
CrossRefGoogle Académico
Rosser, J. Barkley, Jr.~y Marina V. Rosser. 2006. Evolución institucional de la gestión ambiental. Journal of Economic Issues 40: 421--429.
CrossRefGoogle Académico
Rosser, J. Barkley, Jr.~y Marina V. Rosser. 2017. Complejidad y evolución institucional. Revista de economía evolutiva e institucional 14: 415--430.
CrossRefGoogle Académico
Rosser, J. Barkley, Jr.~y Marina V. Rosser. 2018. Economía comparada en una economía mundial en transformación . 3ª ed.~Cambridge, MA: MIT Press.
Google Académico
Rosser, J. Barkley, Jr., Carl Folke, Folke Günther, Heikki Isomäki, Charles Perrings y Tönu Puu. 1994. Cambio discontinuo en sistemas jerárquicos multinivel. Investigación de sistemas 11: 7-94.
Google Académico
Rosser, J. Barkley, Jr., Marina V. Rosser y Ehsan Ahmed. 2003b. Equilibrios múltiples de economías no oficiales y dinámica de distribución de la renta en la transición sistémica. Revista de economía poskeynesiana 25: 425--447.
Google Académico
Rosser, J. Barkley, Jr., Marina V. Rosser y Mauro Gallegati. 2012. Una perspectiva de Minsky-Kindleberger sobre la crisis financiera. Journal of Economic Issues 45: 449--458.
CrossRefGoogle Académico
Schaeffer, Milner B. 1957. Algunas consideraciones sobre la dinámica y la economía de la población en relación con la ordenación de la pesca marina. Revista de la Junta de Investigación Pesquera de Canadá 14: 669--681.
CrossRefGoogle Académico
Schumpeter, Joseph A. 1934. La teoría del desarrollo económico . Cambridge, MA: Harvard University Press.
Google Académico
Schumpeter, Joseph A. 1954. Historia de las ideas económicas . Oxford: Prensa de la Universidad de Oxford.
Google Académico
Sethi, Rajiv y Eswaran Somanathan. 1996. La evolución de las normas sociales en el uso común de la propiedad. American Economic Review 80: 766--788.
Google Académico
Setterfield, Mark. 1997. Crecimiento rápido y declive relativo: modelado de la dinámica macroeconómica con histéresis . Nueva York: St.~Martins.
CrossRefGoogle Académico
Shiozawa, Yoshinori, Masahisa Morioka y Kauzhisa Taniguchi. 2019. Microfundamentos de la economía evolutiva . Tokio: Springer Nature.
CrossRefGoogle Académico
Silverberg, Gerald, Giovanni Dosi y Luigi Orsenigo. 1988. Innovación, diversidad y difusión: un modelo de autoorganización. The Economic Journal 98: 1032--1054.
CrossRefGoogle Académico
Simon, Herbert A. 1947. Comportamiento administrativo . Nueva York: Macmillan.
Google Académico
---------. 1955a. Un modelo del comportamiento de elección racional. Quarterly Journal of Economics 60: 99-118.
CrossRefGoogle Académico
---------. 1955b. En una clase de distribuciones sesgadas. Biometrika 42: 425--440.
CrossRefGoogle Académico
---------. 1957. Modelos de hombre: social y racional . Nueva York: John Wiley.
Google Académico
---------. 1962. La arquitectura de la complejidad. Actas de la American Philosophical Society 106: 467--482.
Google Académico
---------. 1969. Las ciencias de lo artificial . Cambridge, MA: MIT Press.
Google Académico
---------. 1976. De la racionalidad sustantiva a la procedimental. En Method and Appraisal in Economics , ed.~SJ Latsis, 129-148. Cambridge, Reino Unido: Cambridge University Press.
CrossRefGoogle Académico
Skinner, BF 1938. El comportamiento de los organismos: un análisis experimental . Cambridge, MA: Fundación BF Skinner.
Google Académico
Smith, Adam. 1759. La teoría de los sentimientos morales . Londres: Miller, Kincaid \& Bell.
CrossRefGoogle Académico
Smith, Vernon L. 1962. Un estudio experimental del comportamiento del mercado competitivo. Journal of Political Economy 70: 111-137.
CrossRefGoogle Académico
Smith, Vernon L., Gerry L. Suchanek y Arlington W. Williams. 1988. Burbujas, choques y expectativas endógenas. Econometrica 56: 1119-1151.
CrossRefGoogle Académico
Stigler, George J. 1961. La economía de la información. Journal of Political Economy 69: 213-225.
CrossRefGoogle Académico
Stokes, Kenneth M. 1995. Paradigma perdido: una crítica teórica cultural y de sistemas de la economía política . Armonk: YO Sharpe.
Google Académico
Stoppard, Tom. 1967. Rosencrantz y Guildenstern están muertos . Londres: Faber \& Faber.
Google Académico
Taleb, Nassim Nicholas. 2010. El cisne negro . 2ª ed.~Nueva York: Random House.
Google Académico
Thompson, D'Arcy W. 1917. Sobre crecimiento y forma . Cambridge, Reino Unido: Cambridge University Press.
Google Académico
Torgler, Benno. 2016. ¿Puede la investigación sobre cumplimiento tributario beneficiarse de la biología? Revisión de Economía del comportamiento 3: 113--144.
CrossRefGoogle Académico
Turing, Alan M. 1952. The Chemical Basis of Morphogenesis. Transacciones filosóficas de la Royal Society B 237: 37--72.
Google Académico
Vanberg, Victor J. 1997. Evolución institucional a través de la selección intencionada: la economía política constitucional de John R. Commons. Economía política constitucional 8: 105-122.
CrossRefGoogle Académico
Varela, Francisco G., Humberto R. Maturana y R. Uribe. 1974. Autopoesis: La organización de los sistemas vivos. Biosystems 5: 187-196.
CrossRefGoogle Académico
Veblen, Thorstein B. 1898. ¿Por qué la economía no es una ciencia evolutiva? Quarterly Journal of Economics 12: 373--397.
CrossRefGoogle Académico
---------. 1899. La teoría de la clase de ocio: un estudio económico de las instituciones . Nueva York: Penguin.
Google Académico
---------. 1919. El lugar de la ciencia en la civilización moderna y otros ensayos . Nueva York: Huebsch.
Google Académico
Velupillai, K. Vela. 2019. Teoría clásica de las finanzas conductuales. Revisión de Economía del comportamiento 6: 1--18.
CrossRefGoogle Académico
Wiener, Norbert. 1948. Cibernética: o control y comunicación en el animal y la máquina . Cambridge, MA: MIT Press.
Google Académico
Williams, George C. 1966. Adaptación y selección natural: una crítica de algún pensamiento evolutivo actual . Princeton: Prensa de la Universidad de Princeton.
Google Académico
Williamson, Oliver E. 1985. Reflexiones sobre la nueva economía institucional. Revista de economía institucional y teórica 3, 187-195.
Google Académico
Williamson, Oliver E. 2000. La nueva economía institucional: balance, mirando hacia el futuro. Journal of Economic Literature 38: 596--613.
CrossRefGoogle Académico
Wilson, Edward O. 2012. La conquista social de la Tierra . Nueva York: WW Norton.
Google Académico
Wright, Sewall. 1931. Evolución en poblaciones mendelianas. Genética 16: 97-159.
CrossRefGoogle Académico
---------. 1932. Los roles de la mutación, la consanguinidad, el cruzamiento y la selección en la evolución. Actas del Congreso Internacional de Genética 1: 356--366.
Google Académico
---------. 1951. La teoría genética de las poblaciones. Annals of Eugenics 15: 323--354.
CrossRefGoogle Académico
---------. 1988. Superficies de valores selectivos revisados. American Naturalist 131: 115-123.
CrossRefGoogle Académico
Young, Allyn A. 1928. Rendimientos crecientes y progreso económico. Economic Journal 38: 527--542.
CrossRefGoogle Académico
Zeeman, E. Christopher. 1974. Sobre el comportamiento inestable de las bolsas de valores. Revista de Economía Matemática 1: 39--44.
CrossRefGoogle Académico
Anderson, Lisa R., Jennifer M. Mellor y Jeffrey Milyo. 2004. Desigualdad y provisión de bienes públicos: un análisis experimental . Mimeo: Colegio de William y Mary.
Google Académico
Arthur, W. Brian. 1994. Rendimientos crecientes y dependencia de la trayectoria en la economía . Ann Arbor: Prensa de la Universidad de Michigan.
CrossRefGoogle Académico
Arthur, W. Brian, Yuri Ermoliev y Yuri Kaniovski. 1987. Procesos de crecimiento adaptativo modelados por esquemas de urnas. Cybernetics 23: 776--789.
Google Académico
Asea, PK 1996. El sector informal: ¿Bebé o agua de baño? Un comentario. Serie de conferencias Carnegie-Rochesrter sobre políticas públicas 44: 163-171.
CrossRefGoogle Académico
Bartlett, B. 1990. La economía subterránea: ¿Talón de Aquiles del Estado? Asuntos económicos 10: 24-27.
CrossRefGoogle Académico
Bhattacharya, DK 1999. Sobre el fundamento económico de la estimación de la economía oculta. Economic Journal 109: 348--359.
CrossRefGoogle Académico
Bjørnskov, Christian. 2006. Las múltiples facetas del capital social. Revista europea de economía política 22: 22--40.
CrossRefGoogle Académico
Blades, Derek y David Roberts. 2002. Midiendo la economía no observada. Resumen estadístico de la OCDE 5: 1--8.
Google Académico
Bourdieu, Pierre. 1977. Esquema de una teoría de la práctica . Cambridge, Reino Unido: Cambridge University Press.
CrossRefGoogle Académico
Breusch, Trevor. 2005. La economía subterránea canadiense: un examen de Giles y Tedds. Canadian Tax Journal 53: 367--391.
Google Académico
Calzaroni, M. y S. Rononi. 1999. ``Introducción a la economía no observada: el marco conceptual y los principales métodos de estimación''. Documento de taller de Chisinau NOE / 02.
Google Académico
Coleman, James. 1990. Fundamentos de la teoría social . Cambridge, MA: Belknap Press de la Universidad de Harvard.
Google Académico
Crane, J. 1993. La teoría epidémica de los guetos y los efectos del vecindario sobre la deserción escolar y la maternidad adolescente. American Journal of Sociology 96: 1226-1259.
CrossRefGoogle Académico
Dasgupta, Partha. 2000. El progreso económico y la idea de capital social. En Social Capital: A Multifacted Perspective , ed.~Partha Dasgupta e Ismail Serageldin, 325--424. Washington: Banco Mundial.
Google Académico
Davis, Lewis S. 2007. Explicación de la evidencia sobre desigualdad y crecimiento: informalidad y redistribución. BE Journal of Macroeconomics 7: 1.
Google Académico
Durlauf, Steven N. y Marcel Fafchamps. 2005. Capital social. En Handbook of Economic Growth, vol.~1 , ed.~Philippe Aghion y Steven Durlauf, 1639--1699. Amsterdam: Elsevier.
CrossRefGoogle Académico
Feige, Edgar L. 1979. ¿Qué tamaño tiene la economía irregular? Desafío 22: 5--13.
CrossRefGoogle Académico
Frey, Bruno S. y Werner Pommerehne. 1984. La economía oculta: Estado y perspectivas de medición. Revisión de ingresos y riqueza 30: 1--23.
CrossRefGoogle Académico
Friedman, Eric, Simon Johnson, Daniel Kaufmann y Pablo Zoido-Lobatón. 2000. Esquivando la mano que agarra: Los determinantes de la actividad no oficial en 69 países. Journal of Public Economics 76: 459--493.
CrossRefGoogle Académico
Goldberger, Arthur S. 1972. Métodos de ecuaciones estructurales en las ciencias sociales . Amsterdam: Holanda Septentrional.
CrossRefGoogle Académico
Granovetter, Mark. 1978. Modelos de umbral de comportamiento colectivo. American Journal of Sociology 83: 1420-1443.
CrossRefGoogle Académico
Guttman, PS 1977. The Subterranean Economy. Diario de analistas financieros 34: 24--34.
Google Académico
Inglehart, Ronald, M. Basaňez y A. Moreno. 1998. Valores y creencias humanos: un libro de consulta transcultural: normas políticas, religiosas, sexuales y económicas en 43 sociedades. Hallazgos de la Encuesta Mundial de Valores 1990-93 . Prensa de la Universidad de Michigan: Ann Arbor.
Google Académico
Johnson, Simon, Daniel Kaufmann y Andrei Shleifer. 1997. La economía no oficial en transición. Brookings Papers on Economic Activity 1997 (2): 159-221.
CrossRefGoogle Académico
Kaufmann, Daniel y Aleksander Kaliberda. 1996. Integración de la economía no oficial en la dinámica de las economías postsocialistas. En Economic Transition in the Newly Independent States , ed.~Bartlomiej Kaminsky, 81--120. Armonk: YO Sharpe.
Google Académico
Knack, Steven y Philip Keefer. 1997. ¿Tiene el capital social una recompensa económica? Una investigación a campo traviesa. Quarterly Journal of Economics 112: 1251--1288.
CrossRefGoogle Académico
Lackó, María. 2000. Economía oculta: ¿una cantidad desconocida? Análisis comparativo de economías ocultas en transición, 1989-90. Economía de la transición 8: 117-149.
CrossRefGoogle Académico
Lassen, David Dreyer. 2007. División étnica, confianza y tamaño del sector informal. Journal of Economic Behavior and Organisation 63: 423--478.
CrossRefGoogle Académico
Lizzera, C. 1979. Mezzogiorno in Controluce {[}Sur de Italia en Eclipse{]} . Nápoles: Exel.
Google Académico
Loayza, NV 1996. La economía del sector informal: un modelo simple y algunas evidencias empíricas de América Latina. Serie de conferencias Carnegie-Rochester sobre políticas públicas 45: 129--162.
CrossRefGoogle Académico
Loury, Glenn. 1977. Una teoría dinámica de las diferencias de ingresos raciales. En Mujeres, minorías y discriminación laboral , ed.~P. Wallace y A. LeMund, 153--186. Lexington: Libros de Lexington.
Google Académico
McCloskey, Deirdre N. y Stephen T. Ziliak. 1996. El error estándar de las regresiones. Journal of Economic Literature 34: 97-114.
Google Académico
Minniti, María. 1995. La membresía tiene sus privilegios: organizaciones mafiosas antiguas y nuevas. Estudios económicos comparados 37: 31--47.
CrossRefGoogle Académico
Norgaard, Julia R. 2020. Superar el anonimato: cómo los mercados de Darknet proporcionan un gobierno privado para habilitar el intercambio cifrado y anónimo. Review of Behavioral Economics 7: 271--298.
CrossRefGoogle Académico
O'Driscoll, Gerald P., Jr., Kim R. Holmes y Melanie Kirkpatrick. 2001. 2002 Índice de libertad económica . Washington y Nueva York: The Heritage Foundation y Wall Street Journal.
Google Académico
Pettinati, Paolo. 1979. Empleo ilegal y no registrado en Italia. Notas económicas 8: 13-30.
Google Académico
Putnam, Robert D. 2000. Bolos solos: el colapso y la renovación de la comunidad estadounidense . Nueva York: Simon \& Schuster.
Google Académico
Putnam, Robert D. con R. Leonardi y E. Nanetti. 1993. Making Democracy Work: Civic Traditions in Italy . Princeton: Prensa de la Universidad de Princeton.
Google Académico
Rauch, James E. 1993. Desarrollo económico, subempleo urbano y desigualdad de ingresos. Canadian Journal of Economics 26: 901--918.
CrossRefGoogle Académico
Rosser, J. Barkley, Jr.~y Marina V. Rosser. 2001. Otro fracaso del Consenso de Washington sobre países en transición: desigualdad y economías subterráneas. Desafío 44 (2): 39--50.
CrossRefGoogle Académico
Rosser, J. Barkley, Jr., Marina V. Rosser y Ehsan Ahmed. 2000. Desigualdad de ingresos y economía informal en economías en transición. Journal of Comparative Economics 28: 156-171.
CrossRefGoogle Académico
---------. 2003b. Equilibrios múltiples de economías no oficiales y dinámica de distribución de la renta en la transición sistémica. Revista de economía poskeynesiana 25: 425--447.
Google Académico
---------. 2007. Desigualdad de ingresos, corrupción y economía no observada: una perspectiva global. En Sugerencias de complejidad para la política económica , ed.~Massimo Salzano y David Colander, 235-254. Milán: Springer.
Google Académico
Schelling, Thomas C. 1978. Micromotives and Macrobehavior . Nueva York: WW Norton.
Google Académico
Schneider, Friedrich. 1997. Resultados empíricos para el tamaño de la economía sumergida de los países de Europa occidental a lo largo del tiempo. Institut für Volkswirtschaftslehre Documento de trabajo núm. 9710. Johannes Kepler Universität Linz.
Google Académico
---------. 2002. El tamaño y desarrollo de las economías en la sombra de 22 países en transición y 21 países de la OCDE durante la década de 1990. Documento de debate de IZA núm. 514, Bonn.
Google Académico
Schneider, Friedrich y Dominik H. Enste. 2000. Economías en la sombra: tamaño, causas y consecuencias. Journal of Economic Literature 38: 77-114.
CrossRefGoogle Académico
Schneider, Friedrich y Dominik H. Enste. 2002. Escondiéndose en las sombras: El crecimiento de la economía subterránea. Documento de trabajo del FMI 227.
Google Académico
Schneider, Friedrich y Robert Klinglmair. 2004. Economías en la sombra en todo el mundo: ¿qué sabemos ?. Documento de trabajo CESifo No.~167, Munich.
Google Académico
Schneider, Friedrich y Reinhard Neck. 1993. El desarrollo de la economía sumergida en los sistemas y estructuras fiscales cambiantes. Finanzarchiv FN 50: 344--369.
Google Académico
Scholz, John T. y Mark Lubell. 1998. Confianza y pago de impuestos: prueba del enfoque heurístico. American Journal of Political Science 42: 398--417.
CrossRefGoogle Académico
Slemrod, Joel. 1998. Sobre Cumplimiento Voluntario, Impuestos Voluntarios y Capital Social. National Tax Journal 51: 485--491.
CrossRefGoogle Académico
de Soto, Hernando. 1989. El Otro Camino: La Revolución Invisible en el Tercer Mundo . Nueva York: Harper and Row.
Google Académico
Svendsen, Gert Tinggard. 2002. Capital social, corrupción y crecimiento económico: Europa oriental y occidental. Documento de trabajo del Departamento de Economía núm. 03-21. Escuela de Negocios de Aarhus.
Google Académico
Svendsen, Gunnar Lind Haase y Gert Tinggard Svendsen. 2004. La Creación y Destrucción del Emprendimiento de Capital Social . En Movimientos e Instituciones Cooperativas . Edward Elgar: Cheltenham.
Google Académico
Tanzi, Vito. 1980. La economía subterránea en los Estados Unidos: evidencia e implicaciones. Revista trimestral de Banca Nazionale Lavoro 135: 427--453.
Google Académico
Transparencia Internacional. 1998. (y años posteriores). Índice de percepción de la corrupción . Berlín: Transparencia Internacional.
Google Académico
Williams, CC y J. Windbeck. 1995. Trabajo en el mercado negro en la Comunidad Europea: ¿Trabajo periférico para localidades periféricas? Revista internacional de investigación urbana y regional 19: 23--39.
CrossRefGoogle Académico
Woolcock, Michael. 1998. Capital social y desarrollo económico: hacia una síntesis y un marco de políticas. Teoría y sociedad 27: 151-208.
CrossRefGoogle Académico
Zak, Paul J. y Yi Feng. 2003. Una teoría dinámica de la transición a la democracia. Revista de comportamiento económico y organización 52: 1--25.
CrossRefGoogle Académico
Zak, Paul J. y Stephen Knack. 2001. Confianza y crecimiento. Economic Journal 111: 295--321.
CrossRefGoogle Académico
Zellner, Arnold. 1970. Estimación de relaciones de regresión que contienen variables no observables. Revista económica internacional 11: 441--454.
CrossRefGoogle Académico
Anderson, Philip W., Kenneth J. Arrow y David Pines, eds.~1988. La economía como un sistema complejo en evolución . Redwood City: Addison-Wesley.
Google Académico
Ángulo, John. 1986. La teoría del excedente de estratificación social y la distribución de la riqueza personal. Fuerzas sociales 65: 293--326.
CrossRefGoogle Académico
Arrow, Kenneth J. 1974. Ensayos sobre la teoría de la asunción de riesgos . Amsterdam: Holanda Septentrional.
Google Académico
Arthur, W. Brian, Steven N. Durlauf y David A. Lane. 1997a. Introducción. En La economía como un sistema complejo en evolución II , ed.~W. Brian Arthur, Steven N. Durlauf y David A. Lane, 1--14. Reading, MA: Addison-Wesley.
Google Académico
Atkinson, Anthony B. 1970. Sobre la medición de la desigualdad. Revista de teoría económica 2: 244-263.
CrossRefGoogle Académico
Auerbach, Felix. 1913. Das Gesetz der Bevōlkerungkonzentration. Petermans Mittelungen 59: 74--76.
Google Académico
Axtell, Robert L. 2001. Distribución Zipf de tamaños de empresas. Science 293: 1818--1820.
CrossRefGoogle Académico
Soltero, Louis. 1900. ``Théorie de la Spéculation''. Annales Sxientifique de l; École Normale Supérieure III-17, 21-86, * traducción al inglés en Cootner, Paul H. ed., 1964. El carácter aleatorio de los precios del mercado de valores , 17--78. Cambridge: MIT Press.
Google Académico
Bak, Per. 1996. Cómo funciona la naturaleza: la ciencia de la criticidad autoorganizada . Nueva York: Copernicus Press para Springer-Verlag.
CrossRefGoogle Académico
Bak, Per, Kan Chen, José Scheinkman y Michael Woodford. 1993. Fluctuaciones agregadas de choques sectoriales independientes: criticidad autoorganizada en un modelo de dinámica de producción e inventario. Ricerche Economiche 47: 3--30.
CrossRefGoogle Académico
Baye, MR, D. Kovenock y CG de Vries. 2012. La paradoja de Herodoto. Juegos y comportamiento económico 74: 399--406.
CrossRefGoogle Académico
Black, Fischer y Myron Scholes. 1973. La fijación de precios de opciones y pasivos corporativos. Journal of Political Economy 81: 637--654.
CrossRefGoogle Académico
Blume, Lawrence E. 1993. La mecánica estadística de la interacción estratégica. Juegos y comportamiento económico 5: 387--424.
CrossRefGoogle Académico
Boltzmann, Ludwig. 1884. Über die Eigenschaften Monocyklischer und andere damit verwander Systeme. Diario de Crelle für due reine und augwandi Matematik 100: 201--212.
Google Académico
Botazzi, G. y A. Secchi. 2003. Un modelo estocástico de crecimiento empresarial. Physica A 324: 213--219.
CrossRefGoogle Académico
Bouchaud, Jean-Philippe y Rama Cont. 1998. Un enfoque de Langevian para las fluctuaciones y caídas del mercado de valores. European Physical Journal B 6: 542--550.
CrossRefGoogle Académico
Bouchaud, Jean-Philippe y Mark Mézard. 2000. Condensación de la riqueza en un modelo simple de economía. Physica A 282: 536--545.
CrossRefGoogle Académico
Bourguignon, François. 1979. Medidas de desigualdad de ingresos descomponibles. Econometrica 47: 901--920.
CrossRefGoogle Académico
Brock, William A. 1993. Pathways to Randomness in the Economy. Estudios Económicos 8: 2-55.
Google Académico
Brock, William A. y Steven N. Durlauf. 2001. Elección discreta con interacciones sociales. Review of Economic Studies 68: 235--260.
CrossRefGoogle Académico
Brock, William A. y Cars H. Hommes. 1997. Una ruta racional hacia la aleatoriedad. Econometrica 65: 1059--1095.
CrossRefGoogle Académico
Canard, Nicholas-François. 1969 {[}1801{]}. Principes d'Économie Politique . Edizioni Bizzarri: Roma.
Google Académico
Canning, D., IAN Amaral, Y. Lee, M. Meyer y HE Stanley. 1998. Una ley de potencia para escalar la volatilidad del PIB g = Tasas de crecimiento con el tamaño del país. Economics Letters 60: 335--341.
CrossRefGoogle Académico
Carnot, Sadi. 1824. Réflexions sur la Puissance Matrice de Feu et sur les Machines Propres a Déveloper cette Puissance . París: Vrin.
Google Académico
Chakrabarti, Bikas K. 2005. Econophys-Kolkata: Una historia corta. En Econophysics of Wealth Distributions , ed.~A. Chatterjee, S. Yarlagadda y BK Chakrabarti, 225--228. Milán: Springer.
CrossRefGoogle Académico
Chakrabarti CG e I. Chakraborty. 2006. ``Entropía de Boltzmann-Shannon: generalización y aplicación''. arXiv: quant-ph / 0610177v1 (20 de octubre de 2006).
Google Académico
Chakrabarti, Bikas K., Anindya Chakraborti y Arnab Chatterjee, eds.~2008. Econofísica y sociofísica: tendencias y perspectivas . Wiley-VCH: Weinhelm.
Google Académico
Chakraborti, Anindya S. y Bikas K. Chakrabarti. 2000. Mecánica estadística del dinero: cómo las propensiones al ahorro afectan su distribución. European Physical Journal B 17: 167-170.
CrossRefGoogle Académico
Chatterjee, Arnab y Bikas K. Chakrabarti, eds.~2006. Econofísica de acciones y otros mercados . Milán: Springer.
Google Académico
Chatterjee, Arnab, Sudhakar Yarlagadda y Bikas K. Chakrabarti, eds.~2007. Econofísica de la distribución de la riqueza . Milán: Springer.
Google Académico
Christensen, Paul P. 1989. Raíces históricas de la economía ecológica: enfoques biofísicos frente a enfoques de asignación. Economía ecológica 1: 17-30.
CrossRefGoogle Académico
Clausius, Rudolf. 1867. La teoría mecánica del calor, con aplicaciones a la máquina de vapor ya las propiedades físicas de los cuerpos . Londres: John van Voorst.
Google Académico
Clementi, Fabio y Mauro Gallegati. 2005. Colas de la ley de energía en la distribución de la renta personal italiana. Physica A 350: 427--438.
CrossRefGoogle Académico
Cockshott, W. Paul, Allin F. Cottrill, Greg J. Michaelson, Ian F. Wright y Victor M. Yakovenko. 2009. Econofísica clásica . Londres: Routledge.
Google Académico
Cowell, FA y K. Kugal. 1981. Aditividad y el concepto de entropía: un enfoque axiomático de la medición de la desigualdad. Journal of Economic Theory 25: 131--143.
CrossRefGoogle Académico
Cozzolino, JM y MI Zahneri. 1973. Distribución máxima de entropía de la distribución futura del precio de mercado futuro de una acción. Investigación de operaciones 21: 1200-1211.
CrossRefGoogle Académico
Davidson, Julius. 1919. Uno de los fundamentos físicos de la economía. Quarterly Journal of Economics 33: 717--724.
CrossRefGoogle Académico
Davis, Harold J. 1941. The Theory of Econometrics . Bloomington: Prensa de la Universidad de Indiana.
Google Académico
Dionosio, A., R. Menezes y D. Mendes. 2009. Un enfoque económico para analizar la incertidumbre en los mercados financieros: una aplicación al mercado de valores portugués. European Physical Journal B 60: 161-164.
Google Académico
Drăgulescu, Adrian A. y Victor M. Yakovenko. 2001. Distribuciones de probabilidad exponencial y de la ley de potencias de la riqueza y la renta en el Reino Unido y los Estados Unidos. Physica A 299: 213-221.
CrossRefGoogle Académico
Durlauf, Steven N. 1993. Crecimiento económico no ergódico. Review of Economic Studies 60: 340--366.
CrossRefGoogle Académico
---------. 1997. Enfoques de la mecánica estadística al comportamiento socioeconómico. En La economía como un sistema complejo en evolución II , ed.~W. Brian Arthur, Steven N. Durlauf y David A. Lane, 81--104. Redwood City: Addison-Wesley.
Google Académico
Ehrenfest, Paul y Tatiana Ehrenfest-Afanessjewa. 1911. Begriffte Grundlagen der Statistschen Auffassunf in der Mechanik. En Encyclopädie der Matematischen Wissenschaften, vol.~4 , ed.~F. Klein y C. Müller, 3--90. Leipzig: Teubner. (Traducción al inglés, MJ Moravcsik, 1959. Los fundamentos conceptuales del enfoque estadístico de la mecánica . Ithaca: Cornell University Press.).
Google Académico
Einstein, Albert. 1905. Über die von der molekularkinetischen Theorie der Wärme geforrderte Bewegung von der ruhenden Flūsstgkeiten suspendierten Teichen. Annalen der Physik 17: 549--560.
CrossRefGoogle Académico
Farmer, J. Doyne y Shareen Joshi. 2002. La dinámica de precios de las estrategias comerciales comunes. Journal of Economic Behavior and Organisation 49: 149-171.
CrossRefGoogle Académico
Fisher, Irving. 1920. Investigaciones matemáticas sobre la teoría del valor y los precios . New Haven: Prensa de la Universidad de Yale. {[}1892{]}.
Google Académico
Foley, Duncan K. 1994. Una teoría del equilibrio estadístico de los mercados. Journal of Economic Theory 62: 321--345.
CrossRefGoogle Académico
Foley, Duncan K. y Eric Smith. 2008. Termodinámica clásica y teoría del equilibrio económico general. Journal of Economic Dynamics and Control 32: 7-65.
CrossRefGoogle Académico
Föllmer, Hans. 1974. Economías aleatorias con muchos agentes que interactúan. Revista de economía matemática 1: 51--62.
CrossRefGoogle Académico
Gabaix, Xavier. 1999. Ley de Zipf para las ciudades: una explicación. Quarterly Journal of Economics 114: 739--767.
CrossRefGoogle Académico
Galam, Serge, Yuval Gefen (Feigenblat) y Yonathan Shapir. 1982. Sociofísica: un nuevo enfoque de la conducta colectiva sociológica. I. Descripción del comportamiento medio de una huelga. Revista de Sociología Matemática 9: 1-13.
CrossRefGoogle Académico
Gallegati, Mauro, Steve Keen, Thomas Lux y Paul Ormerod. 2006. Worrying Trends in Econophysics. Physica A 370: 1--6.
CrossRefGoogle Académico
Gallegati, Mauro, Antonio Palestrini y J. Barkley Rosser Jr.~2011. El período de angustia financiera en burbujas especulativas: agentes heterogéneos y restricciones financieras. Dinámica macroeconómica 13: 60--79.
CrossRefGoogle Académico
Georgescu-Roegen, Nicolás. 1971. La ley de la entropía y el proceso económico . Cambridge: Prensa de la Universidad de Harvard.
CrossRefGoogle Académico
Gerelli, E. 1985. Entropy and the `End of the World'. Ricerche Economiche 34: 435--438.
Google Académico
Gibbs, J. Willard. 1902. Principios elementales de mecánica estadística . New Haven: Prensa de la Universidad de Yale.
Google Académico
Gibrat, Robert. 1931. Les Inégalités Économiques . París: Sirey.
Google Académico
Gopakrishnan, P., V. Plerou, LAN Amaral, M. Meyer y HE Stanley. 1999. Escalado de las distribuciones de fluctuaciones de los índices del mercado financiero. Revisión física E 60: 5305--5316.
CrossRefGoogle Académico
Hartmann, Georg C. y Otto E. Rössler. 1998. Atractores de bengalas acoplados: un prototipo discreto para el modelado económico. Dinámica discreta en la naturaleza y la sociedad 2: 153-159.
CrossRefGoogle Académico
Helm, G. 1887. Die Lehre von der Energie . Felix: Leipzig.
Google Académico
Gallinas, Thorsten. 2000. Teoría de la cartera evolutiva. Informe especial del almanaque de asignación de activos 84 . \url{http://www.evolutionaryfinance.ch/upleads/media/Merrilll.ynch.pdf} .
Hodgson, Geoffrey M. 1993a. Economía institucional: estudio de lo ``antiguo'' y lo ``nuevo''. Metroeconomica 44: 1--28.
CrossRefGoogle Académico
---------. 1993b. Economía y evolución: devolver la vida a la economía . Ann Arbor: Prensa de la Universidad de Michigan.
CrossRefGoogle Académico
Huang, DW 2004. Acumulación de riqueza con redistribución aleatoria. Repaso físico E 69: 57--103.
Google Académico
Ijiri, Yuji y Herbert A. Simon. 1977. Distribuciones sesgadas y el tamaño de las empresas comerciales . Amsterdam: Holanda Septentrional.
Google Académico
Jacobo, Juan Esteban y Anwar Shaikh. 2020. Arbitraje económico y econofísica de la desigualdad de ingresos. Review of Behavioral Economics 7: 299--315.
CrossRefGoogle Académico
Kindleberger, Charles P. 2001. Manías, pánicos y accidentes: una historia de crisis financieras . 4ª ed.~Nueva York: Basic Books.
Google Académico
Lee, Julian y Steve Pressé. 2012. Una derivación de la ecuación maestra a partir de la maximización de la entropía de la ruta. arXiv.org/abs/1206.1416 .
Lévy, Paul. 1925. Calcul des Probabilités . París: Gauthier-Villars.
Google Académico
Levy, Moshe y Sorin Solomon. 1997. Nueva evidencia para la distribución de la riqueza por ley de poder. Physica A 242: 90--94.
CrossRefGoogle Académico
Li, Hongangg y J. Barkley Rosser, Jr.~2004. Dinámica del mercado y volatilidad del precio de las acciones. The European Physical Journal B 39, 409--413.
Google Académico
Lisman, JHC 1949. Econometría y termodinámica: una observación sobre la teoría de presupuestos de Davis. Econometrica 17: 56--62.
CrossRefGoogle Académico
Lotka, Alfred J. 1925. Elementos de Biología Física . Baltimore: Williams y Wilkins. (reimpreso en 1945 como Elementos de Biología Matemática ).
Google Académico
---------. 1926. La distribución de frecuencia de la productividad científica. Revista de la Academia de Ciencias de Washington 12: 317--323.
Google Académico
Lux, Thomas. 2009. Aplicaciones de la física estadística en finanzas y economía (Cap. 9). En Handbook of Complexity Research , ed.~J. Barkley Rosser Jr.~Cheltenham: Edward Elgar.
Google Académico
Lux, Thomas y Michele Marchesi. 1999. Escalado y criticidad en un modelo estocástico de múltiples agentes de un mercado financiero. Nature 397: 498--500.
CrossRefGoogle Académico
Majorana, Ettore. 1942. Il Valore delle Leggi Statistiche nelle Fisica e nelle Scienze Sociali. Scientia 36: 58--66.
Google Académico
Mandelbrot, Benoit B. 1963. La variación de ciertos precios especulativos. Journal of Business 36: 394--419.
CrossRefGoogle Académico
---------. 1997. Fractales y escalamiento en finanzas . Nueva York: Springer-Verlag.
CrossRefGoogle Académico
Mantegna, Rosario N. 1991. Lévy Walks and Enhanced Diffusion in Milan Stock Exchange. Physica A 179, 232--242.
Google Académico
Mantegna, Rosario N. y H. Eugene Stanley. 2000. Introducción a la econofísica: correlaciones y complejidad en finanzas . Cambridge: Cambridge University Press.
Google Académico
Marshall, Alfred. 1920. Principios de economía . 8ª ed.~Londres: Macmillan.
Google Académico
Martínez-Allier, Juan. 1987. Economía ecológica: energía, medio ambiente y escasez . Oxford: Blackwell.
Google Académico
McCauley, Joseph L. 2004. Dinámica de los mercados: Econofísica y Finanzas . Cambridge, Reino Unido: Cambridge University Press.
CrossRefGoogle Académico
McCauley, Joseph I. 2008. Respuesta a ``Tendencias preocupantes en Econofísica''. Physica A 371: 601--609.
CrossRefGoogle Académico
Mimkes, Juergen. 2008. Una formulación termodinámica de la economía. En Econophysics and Sociophysics: Trends and Perspectives , ed.~Bikas K. Chakrabarti, Anindya Chakraborti y Arnab Chatterjee, 1--33. Weinhelm: Wiley-VCH.
Google Académico
Minsky, Hyman P. 1972. Revisión de la inestabilidad financiera: La economía de los desastres. Reevaluación del mecanismo de descuento de la Reserva Federal 3: 97-136.
Google Académico
---------. 1982. ¿Puede ``volver a suceder''? Ensayos sobre inestabilidad y finanzas . Armonk: YO Sharpe.
Google Académico
Mirowski, Philip. 1989a. Más calor que luz: economía como física social: física como economía de la naturaleza . Cambridge: Cambridge University Press.
CrossRefGoogle Académico
---------. 1989b. Cómo no hacer cosas con metáforas: Paul Samuelson y la ciencia de la economía neoclásica. Estudios de Historia y Filosofía de la Ciencia Parte A 20: 175-191.
CrossRefGoogle Académico
Montroll, EW y MF Schlesinger. 1983. Formalismo de máxima entropía, fractales, fenómenos de escala y ruido 1 / f: una historia de colas. Revista de física estadística 32: 209-230.
CrossRefGoogle Académico
Nordhaus, William D. 1992. Modelo letal 2: Revisión de los límites del crecimiento. Documentos de Brookings sobre actividad económica : 1--59.
Google Académico
Osborne, MFM 1959. Movimiento browniano en los mercados de valores. Investigación de operaciones 7: 134-173.
Google Académico
Ostwald, Wilhelm. 1908. Die Energie . Leipzig: JA Barth.
Google Académico
Padgett, JF, D. Lee y N. Collier. 2003. Producción económica como química. Cambio industrial y corporativo 12: 843--877.
CrossRefGoogle Académico
Pareto, Vilfredo. 1897. Cours d'Économie Politique . Lausana: R. Rouge. {[}Traducción al inglés de Ann Schwier. 1971. Manual de Economía Política . Nueva York: Kelly{]}.
Google Académico
Piketty, Thomas. 2014. Capital en el siglo XXI . Cambridge, MA: The Belknap Press de Harvard University Press.
CrossRefGoogle Académico
Plerou, V., LAN Amaral, P. Gopakrishnan, M. Meyer y HE Stanley. 1999. Similitudes entre la dinámica de crecimiento de la investigación universitaria y las actividades económicas competitivas. Nature 400: 433--437.
CrossRefGoogle Académico
Renyi, Alfred. 1961. Sobre las medidas de entropía e información. En Actas del Cuarto Simposio de Berkeley sobre Matemáticas, Estadística y Probabilidad, 1960, Volumen 1: Contribuciones a la teoría de la estadística , ed.~Jerzy Neyman, 547--561. Berkeley: Prensa de la Universidad de California.
Google Académico
Rosser, J. Barkley, Jr.~1991. De la catástrofe al caos: una teoría general de las discontinuidades económicas . Boston: Kluwer.
CrossRefGoogle Académico
---------. 1994. Dinámica de la jerarquía urbana emergente. Caos, solitones y fractales 4: 553--562.
CrossRefGoogle Académico
---------. 2008a. Debate sobre el papel de la econofísica. Dinámica no lineal, psicología y ciencias de la vida 12: 311--323.
Google Académico
---------. 2008b. Econofísica y Complejidad Económica. Avances en sistemas complejos 11: 745--761.
CrossRefGoogle Académico
---------. 2010b. ¿Es posible una perspectiva transdisciplinaria sobre la complejidad económica? Journal of Economic Behavior and Organisation 75: 3-11.
CrossRefGoogle Académico
---------. 2016a. Reconsideración de la ergodicidad y la incertidumbre fundamental. Revista de economía poskeynesiana 38: 331--354.
CrossRefGoogle Académico
---------. 2016b. Entropía y Econofísica. European Physical Journal --- Temas especiales 225: 3091--3104.
CrossRefGoogle Académico
---------. 2020c. El momento de Minsky y la venganza de la entropía. Dinámica macroeconómica 24: 7--23.
CrossRefGoogle Académico
Rosser, J. Barkley, Jr., Marina V. Rosser y Mauro Gallegati. 2012. Una perspectiva de Minsky-Kindleberger sobre la crisis financiera. Journal of Economic Issues 45: 449--458.
CrossRefGoogle Académico
Samuelson, Paul A. 1947. Fundamentos del análisis económico . Cambridge: Prensa de la Universidad de Harvard.
Google Académico
---------. 1972. Principios máximos en economía analítica. American Economic Review 62: 2-17.
Google Académico
---------. 1990. Gibbs en Economía. En Proceedings of the Gibbs Symposium , ed.~G. Caldi y GD Mostow, 255--267. Providencia: Sociedad Matemática Estadounidense.
Google Académico
Schinkus, Christoph. 2009. Incertidumbre económica y econofísica. Physica A 388: 4415--4423.
CrossRefGoogle Académico
Schrōdinger, Erwin. 1945. ¿Qué es la vida? Los aspectos físicos de la célula viva . Londres: Cambridge University Press.
Google Académico
Shaikh, Anwar. 2016. Capitalismo: competencia, conflicto, crisis . Nueva York: Oxford University Press.
Google Académico
Shannon, Claude E. y Warren Weaver. 1949. Teoría matemática de la comunicación . Urbana: Prensa de la Universidad de Illinois.
Google Académico
Simon, Julian L. 1981. The Ultimate Resource . Princeton: Prensa de la Universidad de Princeton.
Google Académico
Smeeding, Timothy. 2012. Renta, riqueza y deuda y la gran recesión . Stanford: Centro Stanford sobre Pobreza y Desigualdad.
Google Académico
Solomon, Sorin y Peter Richmond. 2002. Leyes de energía estable en economías variables: Lotka-Volterra implica Pareto-Zipf. European Physical Journal B 27: 257--261.
Google Académico
Sornette, Didier. 2003. Por qué se desploman los mercados de valores: eventos críticos en sistemas financieros complejos . Princeton: Prensa de la Universidad de Princeton.
Google Académico
Sornette, Didier y A. Johansen. 2001. Importancia de los precursores logarítmicos periódicos de las caídas financieras. Finanzas cuantitativas 1: 42--471.
CrossRefGoogle Académico
Sornette, Didier y D. Zajdenweber. 1999. Rendimientos económicos de la investigación: la ley de Pareto y sus implicaciones. European Physical Journal B 8: 653--664.
CrossRefGoogle Académico
Spitzer, Frank. 1971. Campos aleatorios y sistemas de partículas interactivas . Providencia: Sociedad Matemática Estadounidense.
Google Académico
Stanley, MHR, LAN Amaral, SV Buldyrev, S. Havlin, H. Leschhorn, P. Maass, MA Salinger y HE Stanley. 1996b. Comportamiento de escala en el crecimiento de las empresas. Nature 379: 804--806.
CrossRefGoogle Académico
Stutzer, Michael J. 1994. La mecánica estadística de los precios de los activos. En Ecuaciones diferenciales, sistemas dinámicos y ciencia de control: un Festschrift en honor a Lawrence Markuss, volumen 152 , ed.~KD Elsworthy, W. Norris Everett y E. Bruce Lee, 321--342. Nueva York: Marcel Dekker.
Google Académico
---------. 2000. Derivación entrópica simple de un modelo de Black-Scholes generalizado. Entropía 2: 70--77.
CrossRefGoogle Académico
Takayasu, H. y K. Okuyama. 1998. Dependencia del país de la distribución del tamaño de la empresa y un modelo numérico basado en la competencia y la cooperación. Fractales 6: 67--79.
CrossRefGoogle Académico
Thurner, S. y R. Hanel. 2012. La entropía de sistemas complejos no ergódicos: una derivación de los primeros principios. Serie de conferencias de la Revista Internacional de Física Moderna 16: 105-115.
CrossRefGoogle Académico
Tinbergen. 1937. Un enfoque econométrico de los ciclos económicos . París: Hermann.
Google Académico
Tsallis, Constantino. 1988. Posibles generalizaciones de las estadísticas de Boltzmann-Gibbs. Revista de física estadística 52: 470--487.
CrossRefGoogle Académico
Uffink, Jos. 2014. El trabajo de Boltzmann en física estadística. Enciclopedia de Filosofía de Stanford . Centro para el Estudio del Lenguaje y la Información, Universidad de Stanford, (plato.stanford.edu/entries/statphys-Boltzmann, 2014).
Google Académico
Weidlich, Wolfgang y Günter Haag. 1980. Comportamiento migratorio de poblaciones mixtas en una ciudad. Fenómenos colectivos 3: 89-102.
Google Académico
---------. 1983. Conceptos y modelos de una sociología cuantitativa: la dinámica de las poblaciones de interacción . Berlín: Springer-Verlag.
CrossRefGoogle Académico
Winiarski, L. 1900. Essai sur la Mécanique Sociale: L'Éneegie Sociale et ses Mensurations. Revue Philosophique 49: 265-287.
Google Académico
Yakovenko, Victor M. 2013. Aplicaciones de la mecánica estadística a la economía: origen entrópico de las distribuciones de probabilidad de dinero, ingresos y consumo de energía. En Social Fairness and Economics: Economic Essays in the Spirit of Duncan Foley , ed.~Lance Taylor, Armon Rezai y Thomas Michl, 53--82. Londres: Routledge.
Google Académico
Yakovenko, Victor M. y J. Barkley Rosser Jr.~2009. Coloquio: Mecánica estadística del dinero, la riqueza y la renta. Reseñas de la física moderna 81: 1704-1725.
CrossRefGoogle Académico
Young, Jeffrey T. 1994. Entropía y escasez de recursos naturales: una respuesta a los críticos. Revista de Economía y Gestión Ambiental 26: 210--213.
CrossRefGoogle Académico
Zipf, George K. 1941. Unidad nacional y desunión . Bloomington: Principia Press.
Google Académico
Acheson, James M. 1988. The Lobster Gangs of Maine . Durham: University Press de Nueva Inglaterra.
Google Académico
Alavalapati, JRR, GA Stainback y DR Carter. 2002. Restauración del ecosistema de pino de hoja larga en tierras privadas en el sur de los Estados Unidos. Ecological Economics 40: 411--419.
CrossRefGoogle Académico
Alchian, Armen A. 1952. Política de sustitución económica . Santa Mónica: Corporación RAND.
Google Académico
Alig, RJ, DM Adams y BA McCarl. 1998. Impactos ecológicos y económicos de las políticas forestales: interacciones entre la silvicultura y la agricultura. Economía ecológica 27: 63--78.
CrossRefGoogle Académico
Allen, JC, WM Schaffer y D. Rosko. 1993. El caos reduce la extinción de especies simplificando el ruido de la población local. Nature 364: 229-232.
CrossRefGoogle Académico
Amacher, Gregory S., Fred D. Murray y Mary S. Bowman. 2009. Ventas de madera para pequeños agricultores en la frontera del Amazonas. Ecological Economics 68: 1787-1796.
CrossRefGoogle Académico
Arrow, Kenneth J. y Anthony C. Fisher. 1974. Preservación, incertidumbre e irreversibilidad. Quarterly Journal of Economics 87: 312--319.
CrossRefGoogle Académico
Asheim, Geir B. 2008. La ocurrencia de un comportamiento paradójico en un modelo donde la actividad económica tiene efectos ambientales. Journal of Economic Behavior and Organisation 65: 529--546.
CrossRefGoogle Académico
Barbier, Edward B. 2001. La economía de la deforestación tropical y el uso de la tierra: Introducción al número especial. Land Economics 77: 155-171.
CrossRefGoogle Académico
Bowes, Michael y John V. Krutilla. 1985. Manejo de usos múltiples de las tierras forestales públicas. En Handbook of Natural Resources and Energy Economics , ed.~AV Kneese y JL Sweeny, 531--569. Amsterdam: Elsevier.
Google Académico
Bromley, Daniel W. 1991. Medio ambiente y economía: derechos de propiedad y políticas públicas . Oxford: Basil Blackwell.
Google Académico
Caparrǿs, A. y F. Jacquemont. 2003. Conflictos entre Biodiversidad y Secuestro de Caron. Economía ecológica 46: 143-157.
CrossRefGoogle Académico
Charles, Anthony T. 1988. Socioeconomía pesquera: una encuesta. Land Economics 64: 276--295.
CrossRefGoogle Académico
Chen, Zhiqi. 1997. ¿Puede la actividad económica conducir al caos climático? Revista Canadiense de Economía 30: 349--366.
CrossRefGoogle Académico
Chiarella, Carl. 1988. El modelo de telaraña y su inestabilidad. Economic Modelling 5: 377--384.
CrossRefGoogle Académico
von Ciriacy-Wantrup, Siegfried y Richard C. Bishop. 1975. `Propiedad común' como concepto en la política de recursos naturales. Economía de los recursos naturales 15: 36--45.
Google Académico
Clark, Colin W. 1985. Modelización bioeconómica y ordenación pesquera . Nueva York: Wiley Interscience.
Google Académico
---------. 1990. Bioeconomía matemática . 2ª ed.~Nueva York: Wiley-Interscience.
Google Académico
Conklin, James E. y William C. Kohlberg. 1994. ¿Caos para el halibut? Economía de los recursos marinos 9: 153--182.
CrossRefGoogle Académico
Copes, Parzival. 1970. La curva de oferta hacia atrás de la industria pesquera. Revista escocesa de economía política 17: 69--77.
CrossRefGoogle Académico
Costanza, Robert, F. Andrade, P. Antunes, M. van den Belt, D. Boesch, D. Boersma, F. Catarino, S. Hanna, K. Limburg, B. Low, M. Molitor, JG Pereira, S Rayner, R. Santos, J. Wilson y M. Young. 1999. Economía ecológica y gobernanza sostenible de los océanos. Economía ecológica 31: 171-187.
CrossRefGoogle Académico
Day, Richard H. 1982. Ciclos de crecimiento irregular. American Economic Review 72: 406--414.
Google Académico
Doveri, F., M. Scheffer, S. Rinaldi, S. Muratori y Yu.A. Kuznetsov. 1993. Estacionalidad y caos en un modelo de plancton-pez. Biología teórica de poblaciones 43: 159-183.
CrossRefGoogle Académico
Durrenberger, E.Paul y Gisli Palsson. 1987. Propiedad en el mar: territorios pesqueros y acceso a los recursos marinos. Etnólogo estadounidense 14: 508--522.
CrossRefGoogle Académico
Faustmann, Martin. 1849. Berechnung des Werthes, weichen Waldboden, Sowie noch nicht Haubare Holzhestande fūr die Waldwirtschaft Besitzen. Allgemeine Forst und Jagd-Zeitung 25: 441--455.
Google Académico
Fisher, Irving. 1907. La tasa de interés . Nueva York: Macmillan.
Google Académico
Foroni, Ilaria, Laura Gardini y J. Barkley Rosser Jr.~2003. Expectativas estadísticas adaptativas en un mercado de recursos renovables. Matemáticas y computadoras en simulación 63: 541--567.
CrossRefGoogle Académico
Gaffney, Mason. 1957. Conceptos de madurez financiera de la madera y otros activos , Serie de información sobre economía agrícola , 62. Raleigh: North Carolina State College.
Google Académico
Gordon, H. Scott. 1954. Teoría económica de un recurso de propiedad común: la pesca. Journal of Political Economy 62: 124-142.
CrossRefGoogle Académico
Grafton, R. Quentin, Tim Kampas y Hu Phan Van. 2009. Bacalao hoy y ninguno mañana: el valor económico de una reserva marina. Land Economics 85: 454--469.
CrossRefGoogle Académico
Gram, S. 2001. Valoración económica de productos forestales especiales. Economía ecológica 26: 109-117.
CrossRefGoogle Académico
Gunderson, Lance H., CS Holling y Garry Peterson. 2002a. Sorpresas y sostenibilidad: ciclos de renovación en los Everglades. En Panarchy: Understanding Transformations in Human and Natural Systems , ed.~Lance H. Gunderson y CS Holling, 293--313. Washington: Island Press.
Google Académico
Gunderson, Lance H., CS Holling, Lowell Pritchard Jr.~y Garry D. Peterson. 2002b. Comprensión de la resiliencia: teoría, metáforas y marcos. En Resilience and the Behavior of Large-Scale Systems , ed.~Lance H. Gunderson y Lowell Pritchard Jr., 3--20. Washington: Island Press.
Google Académico
Hardin, Garrett. 1968. La tragedia de los comunes. Science 162: 1243-1248.
CrossRefGoogle Académico
Hartman, Richard. 1976. La decisión de aprovechamiento cuando un bosque en pie tiene valor. Consulta económica 14: 52--58.
CrossRefGoogle Académico
Hartmann, Georg C. y Otto E. Rössler. 1998. Atractores de bengalas acoplados: un prototipo discreto para el modelado económico. Dinámica discreta en la naturaleza y la sociedad 2: 153-159.
CrossRefGoogle Académico
Henderson-Sellers, Ann y K. McGuffie. 1987. A Climate Modeling Primer . Nueva York: Wiley.
Google Académico
Holling, CS 1965. La respuesta funcional de los depredadores a la densidad de presas y su papel en el mimetismo y la regulación de la población. Memoriales de la Sociedad Entomológica de Canadá 45: 1-60.
Google Académico
---------. 1973. Resiliencia y estabilidad de sistemas ecológicos. Revisión anual de ecología y sistemática 4: 1--24.
CrossRefGoogle Académico
---------. 1986. La resiliencia de los ecosistemas terrestres: sorpresa local y cambio global. En Desarrollo sostenible de la biosfera , ed.~WC Clark y RE Munn, 292--317. Cambridge, Reino Unido: Cambridge University Press.
Google Académico
---------. 1992. Morfología, geometría y dinámica de ecosistemas a escala cruzada. Monografías ecológicas 62: 447--502.
CrossRefGoogle Académico
Holling, CS y Lance H. Gunderson. 2002. Resiliencia y ciclos adaptativos. En Panarchy: Understanding Transformations in Human and Natural Systems , ed.~Lance H. Gunderson y CS Holling, 25--62. Washington: Island Press.
Google Académico
Holling, CS, Lance H. Gunderson y Garry D. Peterson. 2002. Sostenibilidad y Panarquías. En Panarchy: Understanding Transformations in Human and Natural Systems , ed.~Lance H. Gunderson y CS Holling, 63--112. Washington: Island Press.
Google Académico
Hommes, Cars H. y J. Barkley Rosser Jr.~2001. Equilibrios de expectativas consistentes en los mercados de recursos renovables. Dinámica macroeconómica 5: 10-203.
CrossRefGoogle Académico
Hyde, William F. 1980. Suministro de madera, asignación de tierras y eficiencia económica . Baltimore: Prensa de la Universidad Johns Hopkins.
Google Académico
Johnson, KN, DB Jones y BM Kent. 1980. Guía del usuario del modelo de planificación forestal (FORPLAN) . Planificación del Manejo de Tierras del Servicio Forestal del USDA: Fort Collins.
Google Académico
Kahn, James y Alexandre Rivas. 2009. El desarrollo económico sostenible de los pueblos tradicionales. En Economía ecológica y poskeynesiana: Enfrentando los problemas ambientales , ed.~Richard PF Holt, Steven Pressman y Clive L. Spash, 254--278. Edward Elgar: Cheltenham.
Google Académico
Kant, Shashi. 2000. Un enfoque dinámico de los regímenes forestales en los países en desarrollo. Ecological Economics 32: 287--300.
CrossRefGoogle Académico
Khan, M. Ali y Adrianna Piazza. 2011. Ciclicidad óptima y caos en el modelo RSS de 2 sectores: todo vale la construcción. Journal of Economic Behavior and Organisation 80: 387--417.
Google Académico
Lauck, Tim, Colin W. Clark, Marc Mangel y Gordon R. Munro. 1998. Implementación del principio de precaución en la ordenación pesquera a través de los recursos marinos. Aplicaciones ecológicas 8: S72 -- S78.
CrossRefGoogle Académico
Leopold, Aldo. 1933. La ética de la conservación. Journal of Forestry 33: 636--637.
Google Académico
Lorenz, Edward N. 1963. Deterministic Non-Periodic Flow. Revista de ciencia atmosférica 20: 130-141.
CrossRefGoogle Académico
Ludwig, Donald, Dixon D. Jones y CS Holling. 1978. Análisis de calidad de los sistemas de brotes de insectos: el gusano del cogollo del abeto y el bosque. Journal of Applied Ecology 47, 315--332.
Google Académico
Massetti, Emanuele y Emanuele Di Lorenzo. 2019. Caos en las estimaciones del impacto del cambio climático. Documento de trabajo, Instituto de Tecnología de Georgia.
Google Académico
Matsumoto, Akio. 1997. Ergodic Cobweb Chaos. Dinámica discreta en la naturaleza y la sociedad 1: 135-146.
CrossRefGoogle Académico
Mayo, Robert M. 1973. Estabilidad y complejidad en ecosistemas modelo . Princeton: Prensa de la Universidad de Princeton.
Google Académico
---------. 1976. Modelos matemáticos simples con dinámica muy complicada. Nature 261: 459--467.
CrossRefGoogle Académico
Millerd, Frank. 2007. Primeros intentos de establecer derechos exclusivos en la pesquería de salmón de Columbia Británica. Land Economics 83: 23--40.
CrossRefGoogle Académico
Milnor, John. 1985. Sobre el concepto de atractor. Comunicaciones en física matemática 102: 517--519.
CrossRefGoogle Académico
Mitra, Tapan y HJ Wan Jr.~1986. Sobre la solución de Faustmann al problema del manejo forestal. Journal of Economic Theory 40: 229--249.
CrossRefGoogle Académico
Red, Richard McC. 1976. Lo que los campesinos alpinos tienen en común: observaciones sobre la tenencia comunal en una aldea suiza. Ecología humana 4: 134-146.
CrossRefGoogle Académico
Nishimura, Kazuo y M. Yano. 1996. Sobre el límite mínimo superior de los factores de descuento que son compatibles con el período óptimo de tres ciclos. Journal of Economic Theory 69: 306--333.
CrossRefGoogle Académico
Ostrom, Elinor, ed.~1976. La prestación de servicios urbanos: resultados del cambio , revisión anual de asuntos urbanos. Vol. 10. Beverly Hills: Sage.
Google Académico
---------. 1990. Governing the Commons: The Evolution of Institutions for Collective Action . Cambridge, Reino Unido: Cambridge University Press.
CrossRefGoogle Académico
---------. 2005. Comprendiendo la diversidad institucional . Princeton: Prensa de la Universidad de Princeton.
Google Académico
---------. 2010a. Más allá de los mercados y los estados: gobernanza policéntrica de sistemas económicos complejos. American Economic Review 100: 641--672.
CrossRefGoogle Académico
---------. 2010b. El desafío del autogobierno en entornos contemporáneos complejos. Journal of Speculative Philosophy 24: 316--332.
CrossRefGoogle Académico
---------. 2012. Coevolucionando las relaciones entre ciencia política y economía. Racionalidad, mercados y moral 3: 51--65.
Google Académico
Perrings, Charles, Karl-Gōran Mäler, Carl Folke, CS Holling y Bengt-Owe Jansson, eds.~1995. Pérdida de biodiversidad: cuestiones económicas y ecológicas . Cambridge, Reino Unido: Cambridge University Press.
Google Académico
Prince, Raymond y J. Barkley Rosser Jr.~1985. Algunas implicaciones de los costos ambientales demorados para el análisis de costos y beneficios. Crecimiento y cambio 16: 18-25.
CrossRefGoogle Académico
Ramsey, Frank P. 1928. Una teoría matemática del ahorro. Economic Journal 38: 543--549.
CrossRefGoogle Académico
Reed, WJ y HR Clarke. 1990. Decisión de cosecha y valoración de activos para recursos biológicos que exhiben un crecimiento estocástico dependiente del tamaño. Revista económica internacional 31: 147--169.
CrossRefGoogle Académico
Rosser, J. Barkley, Jr.~1995. Crisis sistémicas en economías ecológicas jerárquicas. Land Economics 71: 163-172.
CrossRefGoogle Académico
---------., Ed. 2001b. Complexity in Economics, Volúmenes I-III: Biblioteca Internacional de Escritos Críticos de Economía, 174 . Cheltenham: Edward Elgar.
Google Académico
---------. 2001a. Perspectivas alternativas keynesianas y poskeynesianas sobre la incertidumbre y las expectativas. Revista de economía poskeynesiana 23: 545--566.
CrossRefGoogle Académico
---------. 2002. Dinámica de sistemas acoplados complejos y el problema del calentamiento global. Dinámica discreta en la naturaleza y la sociedad 7: 93-100.
CrossRefGoogle Académico
---------. 2005. Complejidades de las políticas de gestión forestal dinámica. En Economía, recursos naturales y sostenibilidad: economía de la ordenación forestal sostenible , ed.~Shashi Kant y R. Albert Berry, 191-206. Dordrecht: Springer.
CrossRefGoogle Académico
---------. 2011a. Dinámicas evolutivas complejas en sistemas urbano-regionales y ecológico-económicos: de la catástrofe al caos y más allá . Nueva York: Springer.
CrossRefGoogle Académico
---------. 2011b. Perspectivas poskeynesianas y dinámicas ecológico-económicas complejas. Metroeconomica 62: 96--121.
CrossRefGoogle Académico
---------. 2013. Problemas especiales de los bosques como sistemas ecológico-económicos. Política y economía forestal 35: 31--38.
CrossRefGoogle Académico
---------. 2020d. Sistemas caóticos acoplados y resultados ecológico-económicos extremos. En Juegos y dinámica: ensayos en honor a Akio Matsumoto , ed.~Ferenc Szidarovsky, 3--15. Singapur: Springer Nature.
CrossRefGoogle Académico
Rosser, J. Barkley, Jr.~y Marina V. Rosser. 2006. Evolución institucional de la gestión ambiental. Journal of Economic Issues 40: 421--429.
CrossRefGoogle Académico
Rosser, J. Barkley, Jr.~y Marina V. Rosser. 2015. Complexity and Behavioral Economics. Dinámica no lineal, psicología y ciencias de la vida 19: 67--92.
Google Académico
Rosser, J. Barkley, Jr., Ehsan Ahmed y Georg C. Hartmann. 2003a. Volatilidad vía Social Flaring. Revista de comportamiento y organización económicos 50: 77--87.
CrossRefGoogle Académico
Rosser, J. Barkley, Jr., Marina V. Rosser y Ehsan Ahmed. 2003b. Equilibrios múltiples de economías no oficiales y dinámica de distribución de la renta en la transición sistémica. Revista de economía poskeynesiana 25: 425--447.
Google Académico
Rōssler, Otto E. 1976. Una ecuación para el caos continuo. Physics Letters A 57: 397--398.
CrossRefGoogle Académico
Rōssler, Otto E. y Georg Hartmann. 1995. Atractores con bengalas. Fractales 3: 285-286.
CrossRefGoogle Académico
Rōssler, Otto E., Carsten Knudsen, John L. Hudson e Ichiro Tsuda. 1995. Atractores diferenciables en ninguna parte. Revista internacional de sistemas inteligentes 10: 15--23.
CrossRefGoogle Académico
Samuelson, Paul A. 1976. Economía de la silvicultura en una sociedad en evolución. Economic Inquiry 14, 466--491.
Google Académico
Schaeffer, Milner B. 1957. Algunas consideraciones sobre la dinámica y la economía de la población en relación con la ordenación de la pesca marina. Revista de la Junta de Investigación Pesquera de Canadá 14: 669--681.
CrossRefGoogle Académico
Schumpeter, Joseph A. 1950. Capitalismo, socialismo y democracia . 3ª ed.~Nueva York: Harper and Row.
Google Académico
Sethi, Rajiv y Eswaran Somanathan. 1996. La evolución de las normas sociales en el uso común de la propiedad. American Economic Review 80: 766--788.
Google Académico
Solé, Richard V. y Jordi Bascompte. 2006. Autoorganización en ecosistemas complejos . Princeton: Prensa de la Universidad de Princeton.
CrossRefGoogle Académico
Swallow, SK, PJ Parks y DN Wear. 1990. No convexidades relevantes para las políticas en la producción de múltiples beneficios forestales. Revista de Economía y Gestión Ambiental 19: 264--280.
CrossRefGoogle Académico
Valderarama, Diego y James L. Anderson. 2007. Mejora de la utilización del recurso de vieira del Atlántico: un análisis de la gestión rotacional de caladeros. Land Economics 83: 86--103.
CrossRefGoogle Académico
Vincent, Jeffrey R. y Matthew D. Potts. 2005. No linealidades, conservación de la biodiversidad y ordenación forestal sostenible. En Economic Sustainability and Natural Resources: Economics of Sustainable Forest Management , ed.~Shashi Kant y R. Albert Berry, 207--222. Dordrecht: Springer.
CrossRefGoogle Académico
Weitzman, Martin L. 2009. Sobre modelado e interpretación de la economía del cambio climático catastrófico. Revisión de Economía y Estadística 91: 1--19.
CrossRefGoogle Académico
---------. 2011. Incertidumbre de cola gruesa en la economía del cambio climático catastrófico. Review of Environmental Economics and Policy 5: 275--292.
CrossRefGoogle Académico
---------. 2012. Metas de GEI como seguro contra el cambio climático catastrófico. Revista de teoría económica pública 14: 221--244.
CrossRefGoogle Académico
---------. 2014. Fat Tails y el costo social del carbono. American Economic Review: Papers and Proceedings 104: 544--546.
CrossRefGoogle Académico
Wilson, James, Bobbi Low, Robert Costanza y Elinor Ostrom. 1999. Percepción errónea de escalas y dinámica espacial de un sistema socioecológico. Economía ecológica 31: 243-257.
CrossRefGoogle Académico
Yin, R. y DH Newman. 1999. Suministro de madera a largo plazo y economía de la producción de madera. Forest Science 43: 113--120.
Google Académico
Zimmer, Carl. 1999. Life after Chaos. Science 284: 83--86.
CrossRefGoogle Académico
Albin, Peter S. con Duncan K. Foley. 1998. Barreras y límites a la racionalidad: ensayos sobre la complejidad económica y la dinámica en los sistemas interactivos . Princeton: Prensa de la Universidad de Princeton.
Google Académico
Arthur, W. Brian. 1994. Rendimientos crecientes y dependencia de la trayectoria en la economía . Ann Arbor: Prensa de la Universidad de Michigan.
CrossRefGoogle Académico
Arthur, W. Brian, Steven N. Durlauf y David A. Lane. 1997a. Introducción. En La economía como un sistema complejo en evolución II , ed.~W. Brian Arthur, Steven N. Durlauf y David A. Lane, 1--14. Reading, MA: Addison-Wesley.
Google Académico
Banks, Jeffrey, Mark Olson, David Porter, Stephen Rassenti y Vernon Smith. 2003. Teoría, Experimento y Subastas de Espectro de la Comisión Federal de Comunicaciones. Revista de comportamiento económico y organización 51: 303--350.
CrossRefGoogle Académico
von Bertalanffy, Ludwig. 1974. Perspectivas sobre la teoría general de sistemas . Nueva York: Braziller.
Google Académico
Boulding, Kenneth E. 1978. Ecodinámica: una nueva teoría de la evolución social . Beverly Hills: Sage.
Google Académico
Brock, William A. y David Colander. 2000. Complejidad y política. En La visión de la complejidad y la enseñanza de la economía , ed.~David Colander. Edward Elgar: Cheltenham.
Google Académico
Colador, David. 1995. Análisis Marshalliano de Equilibrio General. Eastern Economic Journal 21: 281-293.
Google Académico
---------. 2000a. La muerte de la economía neoclásica. Revista de Historia del Pensamiento Económico 22: 127-143.
CrossRefGoogle Académico
---------., Ed. 2000b. La visión de la complejidad y la enseñanza de la economía . Edward Elgar: Cheltenham.
Google Académico
---------., Ed. 2000c. Complejidad e historia del pensamiento económico . Londres: Routledge.
Google Académico
---------. 2015. Por qué los libros de texto de economía deberían cambiar, pero no lo hacen y no cambiarán. Revista Europea de Economía y Políticas Económicas: Intervención 12: 229--235.
Google Académico
Colander, David y Richard PF Holt. 2004b. La cara cambiante de la economía convencional. Review of Political Economy 16: 485--499.
CrossRefGoogle Académico
Colander, David y Roland Kupers. 2014. La complejidad y el arte de las políticas públicas: resolver los problemas de la sociedad desde abajo . Princeton: Prensa de la Universidad de Princeton.
CrossRefGoogle Académico
Colander, David, Richard PF Holt y J. Barkley Rosser Jr.~2004a. La cara cambiante de la economía . Ann Arbor: Prensa de la Universidad de Michigan.
CrossRefGoogle Académico
Colander, David, Richard PF Holt y J. Barkley Rosser Jr.~2007--08. Temas vivos y muertos en la metodología de la economía. Revista de economía poskeynesiana 30: 303--312.
CrossRefGoogle Académico
Colander, David, M. Goldberg, A. Haas, K. Juselius, Alan Kirman, Thomas Lux y B. Soth. 2009. La crisis financiera y el fracaso sistémico de la profesión económica. Critical Review 21: 249--267.
CrossRefGoogle Académico
Colander, David, Richard PF Holt y J. Barkley Rosser Jr.~2010. Cómo ganar amigos y (posiblemente) influir en la economía convencional. Revista de economía poskeynesiana 32: 397--408.
CrossRefGoogle Académico
Davis, John B. 1994. Pensamiento filosófico de Keynes . Dordrecht: Kluwer.
CrossRefGoogle Académico
---------. 2017. La relevancia continua del pensamiento filosófico de Keynes: reflexividad, complejidad e incertidumbre. Anales de la Fondazione Luigi Einaudi 51: 55--76.
Google Académico
Dopfer, Kurt, ed.~2005. Los fundamentos evolutivos de la economía . Cambridge, Reino Unido: Cambridge University Press.
Google Académico
Dopfer, Kurt, John Foster y Jason Potts. 2004. Micro-Meso-Macro. Revista de economía evolutiva 14: 263-279.
CrossRefGoogle Académico
Gatti, Delli, Antonio Palestrini Domenico, Edoardo Gaffeo, Gianfranco Giulioni y Mauro Gallegati. 2008. Macroeconomía emergente: un enfoque basado en agentes para las fluctuaciones comerciales . Berlín: Springer.
Google Académico
Haldane, Andrew. 2013. Repensar la red financiera . Wiesbaden: Springer.
CrossRefGoogle Académico
Hayek, Friedrich A. 1944. El camino de la servidumbre . Chicago: Prensa de la Universidad de Chicago.
Google Académico
---------. 1967. La teoría de los fenómenos complejos. En Estudios de filosofía, política y economía , 22--42. Londres: Routledge y Kegan Paul.
CrossRefGoogle Académico
Hodgson, Geoffrey M. 2006. Economía a la sombra de Darwin y Marx . Edward Elgar: Cheltenham.
CrossRefGoogle Académico
Holling, CS 1973. Resiliencia y estabilidad de sistemas ecológicos. Revisión anual de ecología y sistemática 4: 1--24.
CrossRefGoogle Académico
Holt, Richard PF y J. Barkley Rosser Jr.~2018. The Economist Watcher: Las contribuciones económicas de David Colander. Review of Political Economy 30: 534--555.
CrossRefGoogle Académico
Holt, Richard PF, J. Barkley Rosser Jr.~y David Colander. 2011. La Era de la Complejidad en Economía. Review of Political Economy 23: 357--369.
CrossRefGoogle Académico
Hommes, Cars H. 2021. Modelos de agentes heterogéneos en economía y finanzas. En Handbook of Computational Economics: Agent-Based Computational Economics, Volumen 2 , ed.~Leigh Tesfatsion y Kenneth L. Judd, 1109-1188. Amsterdam: Holanda Septentrional.
Google Académico
Ikeo, Aiko. 2014. Una historia de la ciencia económica en Japón: la internacionalización de la economía en el siglo XX . Londres: Routledge.
CrossRefGoogle Académico
Israel, Giorgio. 2005. La ciencia de la complejidad: problemas y perspectivas epistemológicas. La ciencia en contexto 18: 1--31.
CrossRefGoogle Académico
Keynes, John Maynard. 1921. Tratado de probabilidad . Londres: Macmillan.
Google Académico
---------. 1944. ``Carta a Hayek, 28 de junio de 1944''. Reimpreso en Donald Moggridge, ed.~1980. The Collected Writings of John Maynard Keynes, vol.~27 . Cambridge, Reino Unido: Cambridge University Press.
Google Académico
Krusell, Per y AA Smith Jr.~1998. Heterogeneidad de ingresos y riqueza en la macroeconomía. Journal of Political Economy 106: 867--896.
CrossRefGoogle Académico
Kuznets, Simon. 1955. Crecimiento económico y desigualdad de ingresos. American Economic Review 45: 1--28.
Google Académico
Lavoie, Marc. 2012. Perspectivas de la economía poskeynesiana. Review of Political Economy 24: 321--335.
CrossRefGoogle Académico
Lee, Fred S. 2012. La economía heterodoxa y sus críticos. Review of Political Economy 24: 337--351.
CrossRefGoogle Académico
Leijonhufvud, Axel. 1973. Fallos de demanda efectiva. Revista Sueca de Economía 75: 27--48.
CrossRefGoogle Académico
Leijonhufvud, Axel. 2009. Fuera del corredor: Keynes y la crisis. Cambridge Journal of Economics 33, 741--757.
Google Académico
Loasby, Brian J. 1989. La mente y el método del economista: una evaluación crítica de los principales economistas del siglo XX . Edward Elgar: Cheltenham.
Google Académico
Markose, Sheri M. 2005. Computabilidad y complejidad evolutiva: los mercados como sistemas adaptativos complejos. Economic Journal 115: F159 -- F192.
CrossRefGoogle Académico
Metcalfe, JS y John Foster, eds.~2004. Evolución y complejidad económica . Cheltenham: Edward Elgar.
Google Académico
Morris-Suzuki, Teresa. 1989. Historia del pensamiento económico japonés . Londres: Routledge.
Google Académico
von Neumann, John. 1966. Theory of Self-Reproducing Automata, editado y compilado por Arthur W. Burks . Urbana: Prensa de la Universidad de Illinois.
Google Académico
Ng, Yew-Kwang. 1980. Macroeconomía con competencia no perfecta. Economic Journal 90: 598--610.
CrossRefGoogle Académico
Diccionario de ingles Oxford. 1971. The Compact Edition del Oxford Dictionary, Volume AO . Oxford: Prensa de la Universidad de Oxford.
Google Académico
Payne, John, J. Bettman y F. Johnson. 1993. The Adaptive Decision Maker . Nueva York: Cambridge University Press.
CrossRefGoogle Académico
Piketty, Thomas. 2014. Capital en el siglo XXI . Cambridge, MA: The Belknap Press de Harvard University Press.
CrossRefGoogle Académico
Potts, Jason. 2000. La nueva microeconomía evolutiva: complejidad, competencia y comportamiento adaptativo . Cheltenham: Edward Elgar.
Google Académico
Rosser, J. Barkley, Jr.~2001a. Perspectivas alternativas keynesianas y poskeynesianas sobre la incertidumbre y las expectativas. Revista de economía poskeynesiana 23: 545--566.
CrossRefGoogle Académico
---------. 2020e. Temas austriacos y las controversias de Cambridge en la teoría del capital. Review of Austrian Economics 33: 415--431.
CrossRefGoogle Académico
---------. 2021. Un enfoque alternativo a la economía evolutiva. Metroeconomica , de próxima publicación .
Google Académico
Rosser, J. Barkley, Jr.~y Marina V. Rosser. 2015. Complexity and Behavioral Economics. Dinámica no lineal, psicología y ciencias de la vida 19: 67--92.
Google Académico
Rosser, J. Barkley, Jr., Richard PF Holt y David Colander. 2010. La economía europea en una encrucijada . Cheltenham: Edward Elgar.
CrossRefGoogle Académico
---------. 2013. ¿Cómo puede algo tan correcto como la economía heterodoxa tener tan poca influencia? Documento de trabajo: Universidad James Madison.
Google Académico
Sargent, Thomas J. 1993. Racionalidad limitada en macroeconomía . Oxford: Clarendon Press.
Google Académico
Shannon, Claude E. y Warren Weaver. 1949. Teoría matemática de la comunicación . Urbana: Prensa de la Universidad de Illinois.
Google Académico
Enviado, Esther-Mirjam. 1997. Sargent y Simon: Racionalidad limitada sin consolidar. Cambridge Journal of Economics 21, 323--330.
Google Académico
Shiozawa, Yoshinori. 2004. Economía evolutiva en el siglo XXI: un manifiesto. Revisión económica evolutiva e institucional 1: 5--47.
CrossRefGoogle Académico
Shiozawa, Yoshinori, Masahisa Morioka y Kauzhisa Taniguchi. 2019. Microfundamentos de la economía evolutiva . Tokio: Springer Nature.
CrossRefGoogle Académico
Simon, Herbert A. 1962. La arquitectura de la complejidad. Actas de la American Philosophical Society 106: 467--482.
Google Académico
Veblen, Thorstein B. 1898. ¿Por qué la economía no es una ciencia evolutiva? Quarterly Journal of Economics 12: 373--397.
CrossRefGoogle Académico
Velupillai, Kumaraswamy. 2000. Economía Computable . Oxford: Prensa de la Universidad de Oxford.
CrossRefGoogle Académico
Velupillai, K. Vela. 2005a. Introducción. En Computability, Complexity and Constructivity in Economic Analysis , ed.~K. Vela Velupillai, 1--14. Victoria: Blackwell.
Google Académico
---------. 2005b. Una introducción a las herramientas y conceptos de la economía computable. En Computability, Complexity and Constructivity in Economic Analysis , ed.~K. Vela Velupillai, 148-197. Victoria: Blackwell.
Google Académico
---------. 2009. Perspectiva de un economista computable sobre la complejidad computacional. En Handbook of Complexity Research , ed.~J. Barkley Rosser Jr., 36--83. Cheltenham: Edward Elgar.
Google Académico
Weintraub, E. Roy. 2002. Cómo la economía se convirtió en una ciencia matemática . Durham: Prensa de la Universidad de Duke.
CrossRefGoogle Académico
Wiener, Norbert. 1948. Cibernética: o control y comunicación en el animal y la máquina . Cambridge, MA: MIT Press.
Google Académico
Young, H. Peyton. 1998. Estrategia individual y estructura social: una teoría evolutiva de las instituciones . Princeton: Prensa de la Universidad de Princeton.
CrossRefGoogle Académico

  \bibliography{book.bib,packages.bib}

\end{document}
